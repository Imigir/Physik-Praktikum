\section{Durchführung}
\label{sec:Durchführung}
Die magnetische Flussdichte $B$ wird durch Einbringen einer Hallsonde in das Magnetfeld des Elektromagneten bestimmt.
Eine Probe wird in das Magnetfeld eingebracht und ein Interferenzfilter installiert.
Am Oszilloskop wird die Differenzspannung bei maximalem Magnetfeld über Drehung des Eingangspolarisators minimiert und der Winkel $\theta_.1$ notiert. Das Magnetfeld wird umgepolt und der Winkel $\theta_.2$ gesucht, der die Spannung minimiert.
Der Drehwinkel $\theta$ lässt sich dann bestimmen zu
\begin{equation}
\theta = \frac{1}{2} |\theta_.2 - \theta_.1|\label{eq:dth}
\end{equation}
Dies wird für eine hochreine und zwei unterschiedlich dotierte Proben und neun verschiedene Interferenzfilter durchgeführt.

%☺