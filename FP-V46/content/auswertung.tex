\section{Auswertung}
\label{sec:Auswertung}

Die Graphen werden sowohl mit Matplotlib \cite{matplotlib} als auch NumPy \cite{numpy} erstellt. Die Fehlerrechnung wird mithilfe von Uncertainties \cite{uncertainties} durchgeführt. Die Konstanten stammen aus SciPy \cite{scipy}

\subsection{Vermessung des B-Feldes}

\[
B_.{max}=\SI{213}{\milli\tesla}
\]

\begin{figure}
	\centering
	\includegraphics[width=\linewidth-60pt,height=\textheight,keepaspectratio]{build/magnetfeld.pdf}
	\caption{Das Magnetfeld in Abhängigkeit der $z$-Position.}
	\label{fig:B}
\end{figure}

\begin{table}
	\centering
	\caption{Die Messwerte für die B-Feld Vermessung.}
	\label{tab:magnetfeld}
	\sisetup{table-format=1.2}
	\begin{tabular}{S[table-format=2.0]S[table-format=3.0]}
		\toprule
		{$z/\si{\milli\metre}$} & {$B/\si{\milli\tesla}$} \\
		\midrule
		-25 &   2 \\
		-20 &   8 \\
		-15 &  37 \\
		-10 & 112 \\
		-5 & 188 \\
		-4 & 198 \\
		-3 & 203 \\
		-2 & 208 \\
		-1 & 211 \\
		 0 & 212 \\
		\bottomrule
	\end{tabular}

	\label{tab:magnetfeld2}
	\sisetup{table-format=1.2}
	\begin{tabular}{S[table-format=2.0]S[table-format=3.0]}
		\toprule
		{$z/\si{\milli\metre}$} & {$B/\si{\milli\tesla}$} \\
		\midrule
		 0 & 212 \\
		 1 & 213 \\
		 2 & 212 \\
		 3 & 209 \\
		 4 & 206 \\
		 5 & 201 \\
		10 & 142 \\
		15 &  54 \\
		20 &  13 \\
		25 &   3 \\
		\bottomrule
	\end{tabular}

	\label{tab:B}
\end{table}

\subsection{Untersuchung der Undotierten Probe}

\begin{figure}
	\centering
	\includegraphics[width=\linewidth-60pt,height=\textheight,keepaspectratio]{build/reinprobe.pdf}
	\caption{Die Winkeldifferenz $\Delta\theta$ in Abhängig der quadratischen Wellenlänge für die undotierte Probe.}
	\label{fig:undot}
\end{figure}

\begin{table}
	\centering
	\caption{Die Messwerte für die Messreihe mit der undotierten Probe.}
	\label{tab:undotiert}
	\sisetup{table-format=1.2}
	\begin{tabular}{S[table-format=1.2]S[table-format=3.0]S[table-format=3.0]S[table-format=2.0]}
		\toprule
		{$\lambda/\si{\micro\metre}$} & {$\theta_1/\si{\degree}$} & {$\theta_2/\si{\degree}$} & {$\Delta\theta_.{norm}/\si{.{rad}\per\metre}$} \\
		\midrule
		1.06 & 218 & 194 & 41 \\
		1.29 & 214 & 197 & 29 \\
		1.45 & 212 & 200 & 20 \\
		1.72 & 210 & 202 & 15 \\
		1.96 & 209 & 202 & 12 \\
		2.16 & 208 & 203 &  9 \\
		2.34 & 208 & 203 &  8 \\
		2.51 & 208 & 204 &  6 \\
		2.65 & 208 & 207 &  2 \\
		\bottomrule
	\end{tabular}

	\label{tab:undot}
\end{table}

\subsection{Untersuchung der Undotierten Probe}

\begin{figure}
	\centering
	\includegraphics[width=\linewidth-60pt,height=\textheight,keepaspectratio]{build/dotiert1.pdf}
	\caption{Die normierte Winkeldifferenz $\Delta\theta$ in Abhängig der quadratischen Wellenlänge für die dotierte Probe mit $N=\SI{1.2e18}{\per\cubic\centi\metre}$.}
	\label{fig:undot}
\end{figure}

\begin{figure}
	\centering
	\includegraphics[width=\linewidth-60pt,height=\textheight,keepaspectratio]{build/dotiert2.pdf}
	\caption{Die normierte Winkeldifferenz $\Delta\theta$ in Abhängig der quadratischen Wellenlänge für die dotierte Probe mit $N=\SI{2.8e18}{\per\cubic\centi\metre}$..}
	\label{fig:undot}
\end{figure}

\begin{table}
	\centering
	\caption{Die Messwerte für die Messreihen mit den dotierten Proben ($N=~\SI{1.2e18}{\per\cubic\centi\metre}$ links, $N=\SI{2.8e18}{\per\cubic\centi\metre}$ rechts).}
	\label{tab:dotiert1}
	\sisetup{table-format=1.2}
	\begin{tabular}{S[table-format=1.2]S[table-format=3.0]S[table-format=3.0]S[table-format=2.0]}
		\toprule
		{$\lambda/\si{\micro\metre}$} & {$\theta_1/\si{\degree}$} & {$\theta_2/\si{\degree}$} & {$\Delta\theta_.{norm}/\si{.{rad}\per\metre}$} \\
		\midrule
		1.06 & 209 & 196 & 83 \\
		1.29 & 206 & 200 & 44 \\
		1.45 & 207 & 200 & 46 \\
		1.72 & 207 & 200 & 41 \\
		1.96 & 209 & 201 & 50 \\
		2.16 & 210 & 204 & 35 \\
		2.34 & 210 & 203 & 45 \\
		2.51 & 211 & 203 & 54 \\
		2.65 & 211 & 201 & 66 \\
		\bottomrule
	\end{tabular}

	\label{tab:dotiert2}
	\sisetup{table-format=1.2}
	\begin{tabular}{S[table-format=1.2]S[table-format=3.0]S[table-format=3.0]S[table-format=2.0]}
		\toprule
		{$\lambda/\si{\micro\metre}$} & {$\theta_1/\si{\degree}$} & {$\theta_2/\si{\degree}$} & {$\Delta\theta_.{norm}/\si{.{rad}\per\metre}$} \\
		\midrule
		1.06 & 211 & 201 & 67 \\
		1.29 & 210 & 202 & 52 \\
		1.45 & 211 & 202 & 61 \\
		1.72 & 211 & 202 & 65 \\
		1.96 & 211 & 201 & 65 \\
		2.16 & 212 & 201 & 68 \\
		2.34 & 211 & 199 & 80 \\
		2.51 & 211 & 198 & 85 \\
		2.65 & 212 & 198 & 95 \\
		\bottomrule
	\end{tabular}

	\label{tab:undot}
\end{table}

\begin{figure}
	\centering
	\includegraphics[width=\linewidth-60pt,height=\textheight,keepaspectratio]{build/dotiert_dif.pdf}
	\caption{Die normierte Winkeldifferenz $\Delta\theta$ in Abhängig der quadratischen Wellenlänge für die mit $N=\SI{1.2e18}{\per\cubic\centi\metre}$ (1) und $N=\SI{2.8e18}{\per\cubic\centi\metre}$ (2) dotierten Proben abzüglich den Werten der undotierten Probe.}
	\label{fig:undot}
\end{figure}