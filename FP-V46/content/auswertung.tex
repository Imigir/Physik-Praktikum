\section{Auswertung}
\label{sec:Auswertung}

Die Graphen werden sowohl mit Matplotlib \cite{matplotlib} als auch NumPy \cite{numpy} erstellt. Die Fehlerrechnung wird mithilfe von Uncertainties \cite{uncertainties} durchgeführt. Die Werte der Konstanten werden SciPy \cite{scipy} entnommen.

\subsection{Vermessung des B-Feldes}
Es wird das Magnetfeld innerhalb der Spule vermessen. Die Werte sind in Tabelle \ref{tab:B} eingetragen und in Abbildung \ref{fig:B} grafisch dargestellt. Es lässt sich das maximale Magnetfeld ablesen zu:
\[
B_.{max}=\SI{213}{\milli\tesla}\text{.}
\]

\begin{figure}
	\centering
	\includegraphics[width=\linewidth-60pt,height=\textheight,keepaspectratio]{build/magnetfeld.pdf}
	\caption{Das Magnetfeld in Abhängigkeit der $z$-Position.}
	\label{fig:B}
\end{figure}

\begin{table}
	\centering
	\caption{Die Messwerte für die B-Feld Vermessung.}
	\label{tab:magnetfeld}
	\sisetup{table-format=1.2}
	\begin{tabular}{S[table-format=2.0]S[table-format=3.0]}
		\toprule
		{$z/\si{\milli\metre}$} & {$B/\si{\milli\tesla}$} \\
		\midrule
		-25 &   2 \\
		-20 &   8 \\
		-15 &  37 \\
		-10 & 112 \\
		-5 & 188 \\
		-4 & 198 \\
		-3 & 203 \\
		-2 & 208 \\
		-1 & 211 \\
		 0 & 212 \\
		\bottomrule
	\end{tabular}

	\label{tab:magnetfeld2}
	\sisetup{table-format=1.2}
	\begin{tabular}{S[table-format=2.0]S[table-format=3.0]}
		\toprule
		{$z/\si{\milli\metre}$} & {$B/\si{\milli\tesla}$} \\
		\midrule
		 0 & 212 \\
		 1 & 213 \\
		 2 & 212 \\
		 3 & 209 \\
		 4 & 206 \\
		 5 & 201 \\
		10 & 142 \\
		15 &  54 \\
		20 &  13 \\
		25 &   3 \\
		\bottomrule
	\end{tabular}

	\label{tab:B}
\end{table}

\subsection{Untersuchung der Undotierten Probe}
Es wird die Wellenlängenabhängigkeit des Drehwinkels $\Delta\theta$ der undotierten Probe untersucht. Der Drehwinkel wird dabei nach Formel \eqref{eq:} bestimmt und auf die Dicke $d=\SI{5.11}{\milli\metre}$ normiert. Die Werte sind in Tabelle \ref{tab:undot} eingetragen. In Abbildung \ref{fig:undot} ist $\Delta\theta$ gegen die quadratische Wellenlänge $\lambda^2$ aufgetragen. Es wird eine Ausgleichsrechnung der Form 
\[
\Delta\theta = a*\frac{1}{\lambda^2} 
\]
durchgeführt. Es ergibt sich für den Parameter $a$:
\[
a = \SI{45.3(15)}{\pico\metre}\text{.}
\]
\begin{figure}
	\centering
	\includegraphics[width=\linewidth-60pt,height=\textheight,keepaspectratio]{build/reinprobe.pdf}
	\caption{Die Winkeldifferenz $\Delta\theta$ in Abhängig der quadratischen Wellenlänge für die undotierte Probe.}
	\label{fig:undot}
\end{figure}

\begin{table}
	\centering
	\caption{Die Messwerte für die Messreihe mit der undotierten Probe.}
	\label{tab:undotiert}
	\sisetup{table-format=1.2}
	\begin{tabular}{S[table-format=1.2]S[table-format=3.0]S[table-format=3.0]S[table-format=2.0]}
		\toprule
		{$\lambda/\si{\micro\metre}$} & {$\theta_1/\si{\degree}$} & {$\theta_2/\si{\degree}$} & {$\Delta\theta_.{norm}/\si{.{rad}\per\metre}$} \\
		\midrule
		1.06 & 218 & 194 & 41 \\
		1.29 & 214 & 197 & 29 \\
		1.45 & 212 & 200 & 20 \\
		1.72 & 210 & 202 & 15 \\
		1.96 & 209 & 202 & 12 \\
		2.16 & 208 & 203 &  9 \\
		2.34 & 208 & 203 &  8 \\
		2.51 & 208 & 204 &  6 \\
		2.65 & 208 & 207 &  2 \\
		\bottomrule
	\end{tabular}

	\label{tab:undot}
\end{table}

\subsection{Untersuchung der dotierten Probe}
Es wird die Wellenlängenabhängigkeit des Drehwinkels $\Delta\theta$ der dotierten Proben untersucht. Die erste Probe besitzt eine Dotierung von $N=\SI{1.2e18}{\per\cubic\centi\metre}$ und eine Dicke $d=\SI{1.36}{\milli\metre}$. Die zweite Probe besitzt eine Dotierung von $N=\SI{2.8e18}{\per\cubic\centi\metre}$ und eine Dicke $d=\SI{1.296}{\milli\metre}$. Der Drehwinkel wird dabei nach Formel \eqref{eq:} bestimmt und auf die Dicke normiert. Die Werte sind in Tabelle \ref{tab:dot} eingetragen. In den Abbildungen \ref{fig:dot1} und \ref{fig:dot2} sind die Drehwinkel $\Delta\theta$ jeweils gegen die quadratische Wellenlänge $\lambda^2$ aufgetragen.

\begin{figure}
	\centering
	\includegraphics[width=\linewidth-60pt,height=\textheight,keepaspectratio]{build/dotiert1.pdf}
	\caption{Die normierte Winkeldifferenz $\Delta\theta$ in Abhängig der quadratischen Wellenlänge für die dotierte Probe mit $N=\SI{1.2e18}{\per\cubic\centi\metre}$.}
	\label{fig:dot1}
\end{figure}

\begin{figure}
	\centering
	\includegraphics[width=\linewidth-60pt,height=\textheight,keepaspectratio]{build/dotiert2.pdf}
	\caption{Die normierte Winkeldifferenz $\Delta\theta$ in Abhängig der quadratischen Wellenlänge für die dotierte Probe mit $N=\SI{2.8e18}{\per\cubic\centi\metre}$..}
	\label{fig:dot2}
\end{figure}

\begin{table}
	\centering
	\caption{Die Messwerte für die Messreihen mit den dotierten Proben ($N=~\SI{1.2e18}{\per\cubic\centi\metre}$ links, $N=\SI{2.8e18}{\per\cubic\centi\metre}$ rechts).}
	\label{tab:dotiert1}
	\sisetup{table-format=1.2}
	\begin{tabular}{S[table-format=1.2]S[table-format=3.0]S[table-format=3.0]S[table-format=2.0]}
		\toprule
		{$\lambda/\si{\micro\metre}$} & {$\theta_1/\si{\degree}$} & {$\theta_2/\si{\degree}$} & {$\Delta\theta_.{norm}/\si{.{rad}\per\metre}$} \\
		\midrule
		1.06 & 209 & 196 & 83 \\
		1.29 & 206 & 200 & 44 \\
		1.45 & 207 & 200 & 46 \\
		1.72 & 207 & 200 & 41 \\
		1.96 & 209 & 201 & 50 \\
		2.16 & 210 & 204 & 35 \\
		2.34 & 210 & 203 & 45 \\
		2.51 & 211 & 203 & 54 \\
		2.65 & 211 & 201 & 66 \\
		\bottomrule
	\end{tabular}

	\label{tab:dotiert2}
	\sisetup{table-format=1.2}
	\begin{tabular}{S[table-format=1.2]S[table-format=3.0]S[table-format=3.0]S[table-format=2.0]}
		\toprule
		{$\lambda/\si{\micro\metre}$} & {$\theta_1/\si{\degree}$} & {$\theta_2/\si{\degree}$} & {$\Delta\theta_.{norm}/\si{.{rad}\per\metre}$} \\
		\midrule
		1.06 & 211 & 201 & 67 \\
		1.29 & 210 & 202 & 52 \\
		1.45 & 211 & 202 & 61 \\
		1.72 & 211 & 202 & 65 \\
		1.96 & 211 & 201 & 65 \\
		2.16 & 212 & 201 & 68 \\
		2.34 & 211 & 199 & 80 \\
		2.51 & 211 & 198 & 85 \\
		2.65 & 212 & 198 & 95 \\
		\bottomrule
	\end{tabular}

	\label{tab:dot}
\end{table}

Nun wird die Differenz der normierten Drehwinkel der dotierten mit der undotierten Probe gebildet. Die Ergebnisse sind in Tabelle \ref{tab:dot_dif} eingetragen und in Abbildung \ref{fig:dot_dif} grafisch dargestellt.
Für beide Proben werden Ausgleichsrechnungen der Form
\[
\Delta\theta_.{dif}=a\lambda^2
\] 
durchgeführt.
Es ergeben sich die Parameter:
\begin{align*}
a_1 &= \SI{8.0(6)e12}{\per\cubic\metre}\\
a_2 &= \SI{13.4(5)e12}{\per\cubic\metre}\text{.}
\end{align*}
Aus der Steigung und dem Brechungsindex $n=3,4$ \cite{} lässt sich die effektive Masse $m^*$ nach Formel \eqref{eq:} bestimmen:
\begin{align*}
m^*_1 &= \SI{4.52(16)e-32}{\kilogram} = \SI{0.0496(18)}{m_e}\\
m^*_2 &= \SI{5.35(10)e-32}{\kilogram} = \SI{0.0587(11)}{m_e}\text{.}
\end{align*}

\begin{figure}
	\centering
	\includegraphics[width=\linewidth-60pt,height=\textheight,keepaspectratio]{build/dotiert_dif.pdf}
	\caption{Die normierte Winkeldifferenz $\Delta\theta$ in Abhängig der quadratischen Wellenlänge für die mit $N=\SI{1.2e18}{\per\cubic\centi\metre}$ (1) und $N=\SI{2.8e18}{\per\cubic\centi\metre}$ (2) dotierten Proben abzüglich den Werten der undotierten Probe.}
	\label{fig:dot_dif}
\end{figure}

\begin{table}
	\centering
	\caption{Die normierten Drehwinkel $\Delta\theta$ für die dotierten Proben $N=~\SI{1.2e18}{\per\cubic\centi\metre}$ (dot1) und $N=\SI{2.8e18}{\per\cubic\centi\metre}$ (dot2), die undotierte Probe (rein), sowie die berechneten Differenzen (dif1 und dif2).}
	\label{tab:dotiertdif}
	\sisetup{table-format=1.2}
	\begin{tabular}{S[table-format=1.2]S[table-format=2.0]S[table-format=2.0]S[table-format=2.0]S[table-format=2.0]S[table-format=2.0]}
		\toprule
		{$\lambda/\si{\micro\metre}$} & {$\Delta\theta_.{dot1}/\si{.{rad}\per\metre}$} & {$\Delta\theta_.{dot2}/\si{.{rad}\per\metre}$} & {$\Delta\theta_.{rein}/\si{.{rad}\per\metre}$} & {$\Delta\theta_.{dif1}/\si{.{rad}\per\metre}$} & {$\Delta\theta_.{dif2}/\si{.{rad}\per\metre}$} \\
		\midrule
		1.06 & 83 & 67 & 41 & 42 & 26 \\
		1.29 & 44 & 52 & 29 & 15 & 23 \\
		1.45 & 46 & 61 & 20 & 26 & 40 \\
		1.72 & 41 & 65 & 15 & 27 & 50 \\
		1.96 & 50 & 65 & 12 & 38 & 53 \\
		2.16 & 35 & 68 &  9 & 26 & 59 \\
		2.34 & 45 & 80 &  8 & 37 & 71 \\
		2.51 & 54 & 85 &  6 & 48 & 79 \\
		2.65 & 66 & 95 &  2 & 64 & 93 \\
		\bottomrule
	\end{tabular}

	\label{tab:dot_dif}
\end{table}