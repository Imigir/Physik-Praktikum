
\section{Diskussion}
\label{sec:Diskussion}
Bei der Untersuchung der undotierten Probe wurde wie theoretisch erwartet ein $\propto\frac{1}{\lambda^2}$ Verlauf beobachtet. Die Messwerte zeigen nur eine geringe Abweichung von der Kurve, welches durch den geringen Fehler auf den Parameter $a$  bestätigt wird.\\
Bei der Untersuchung der dotierten Proben ist ebenfalls der theoretische Verlauf $\propto\lambda^2$ erkennbar. Allerdings zeigen die Ausgleichsgeraden beide einen (in etwa gleich großen) Achsenabschnitt, obwohl dieser verschwindend sein sollte. Dies könnte auf einen systematischen Fehler zurückzuführen sein, der durch eine im Verlauf des Experiments erhitzte Spule zustande kommt.\\
Die ermittelten Werte der effektiven Masse sind mit ihren Abweichungen zum Literaturwert in Tabelle \ref{tab:fehler} eingetragen. Die Abweichungen sind mit $\SI{-3.0}{\%}$ und $\SI{-3.3}{\%}$ gering. Dabei liefert die stärker dotierte Probe das Ergebnis mit der geringeren Unsicherheit und scheint damit geeigneter zur Bestimmung der effektiven Masse..

\begin{table}
	\centering
	\caption{Die Ergebnisse der Messungen für die effektive Masse verglichen mit dem Literaturwert \cite{}.}
		\sisetup{table-format=1.2}
	\begin{tabular}{cS[table-format=1.4]@{${}\pm{}$}S[table-format=1.4]S[table-format=1.3]S[table-format=2.1]}
		\toprule
		{Messung} & \multicolumn{2}{c}{$m^*_.{gemessen}/m_e$} & {$m^*_.{theorie}/m_e$} & {Abweichung/\%} \\
		\midrule
		N=\SI{1.2e18}{\per\cubic\centi\metre} & 0.065 & 0.014 & 0.067 & -3.0\\
		N=\SI{2.8e18}{\per\cubic\centi\metre} & 0.0648 & 0.0022 & 0.067 & -3,3 \\
		\bottomrule
	\end{tabular}

	\label{tab:fehler}
\end{table}