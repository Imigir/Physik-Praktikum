\section{Theorie}
\label{sec:Theorie}

\subsection{Die Faraday-Rotation}

Wird ein optisch inaktives Medium parallel von einem Magnetfeld durchdrungen und trifft ein linear polarisierter Lichtstrahl parallel zur Feldrichtung auf das Material, kommt es zu einer Drehung der Polarisationsebene, welche als Faraday-Rotation bezeichnet wird.\\
Linear polarisiertes Licht kann als eine Kombination aus links- und rechtshändig zirkular polarisierten Komponenten beschrieben werden. Ist der Brechungsindex im Medium für diese Komponenten und damit ihre Phasengeschwindigkeit unterschiedlich dreht sich so die Polarisationsebene. Dieser Effekt wird auch zirkulare Doppelbrechung genannt.\\
Wird ein Medium von einem Magnetfeld durchflossen bilden Gitteratome und Valenzelektronen Dipole, sodass eine Polarisation $\vec{P}$ entsteht:
\[
\vec{P} = \varepsilon_.0 \chi \vec{E}
\]
mit der Dielektrizitätskonstante $\varepsilon_.0$, der Suszeptibilität $\chi$ und dem elektrischen Feld $\vec{E}$. Handelt es sich um ein anisotropes Material, etwa einen dotierten Halbleiter, wird $\chi$ ein Tensor.\\
Nebendiagonalelemente entstehen nur bei eingeschaltetem Magnetfeld und beschreiben die optische Aktivität. In diesem Fall lässt sich der Drehwinkel der Polarisationsebene $\theta$ genähert darstellen als
\begin{equation}
\theta \approx \frac{L \omega}{2 c n}\chi_.{xy}\text{.}\label{eq:theta1}
\end{equation}
Dabei ist $L$ die Länge der Probe, $\omega$ die Winkelgeschwindigkeit des Lichts, $c$
die Lichtgeschwindigkeit im Medium mit Brechungsindex $n$ und $\chi_.{xy}$ ein Nebendiagonalelement des Suszeptibilitätstensors.

\subsection{Bandstruktur in Festkörpern}
Anhand der Bandstruktur lassen sich Festkörper in drei Kategorien einteilen.
Bei Metallen überlappen Valenz- und Leitungsband, sodass dieses halb besetzt und eine hohe elektrische Leitfähigkeit gegeben ist.\\
In Halbleitern befindet sich zwischen den Bändern eine Energielücke der Ordnung $\mathcal{O}(\si{\eV})$. Diese kann durch Energiezufuhr überwunden und das Material leitend werden.\\
Im Fall von Isolatoren hingegen ist die Bandlücke so groß, dass sie kaum überwunden werden kann, sodass auch keine Leitungselektronen entstehen.

\subsection{Dotierung von Halbleitern}
Durch Einbringen von Fremdatomen in einen Halbleiterkristall kann dessen Leitfähigkeit beeinflusst. Wird ein Donator, also ein Atom mit mehr Valenzelektronen als dem Halbleiter eingeführt entsteht eine ortsfeste positive Ladung, während das überschüssige Elektronen das Leitungsband erreicht und zu einer frei im Kristall beweglichen negativen Ladung wird. Dieser Prozess wird als n-Dotierung bezeichnet. Beim Gegenteil, der p-Dotierung, wird ein niedrigerwertiges Fremdatom eingebracht, sodass eine ortsfeste negative Ladung und ein frei bewegliches positives Defektelektron entstehen. Durch die Dotierung verändert sich das Gitterpotential an den betroffenen Stellen, sodass dessen Isotropie verloren geht.

\subsection{Die effektive Masse $m^*$}
Aufgrund der Komplexität der Bandstruktur in Festkörpern können für einfache Berechnungen Näherungen betrachtet werden.
Im Halbleiter lässt sich die Bandkante des Leitungsbandes näherungsweise über eine Taylorentwicklung der Energiedispersion
\begin{equation*}
\epsilon(k) = \epsilon(0) + \sum\sum\frac{1}{2}\frac{\partial^2\epsilon}{\partial k_.i\partial k_.j}k^2 + \mathcal{O}(k^3)
\end{equation*}
Der Vergleich mit
\[
\epsilon(k) = \frac{\hbar^2k^2}{2m}
\]
liefert für die effektive Masse somit
\begin{equation}
m^*_.i = \frac{\hbar}{\frac{\partial^2\epsilon}{\partial k^2_.i}}\label{eq:mstar}
\end{equation}
Bei symmetrischen Kristallen sind die beschriebenen Energieflächen kugelförmig und die Ladungsträger können als freie Teilchen mit Masse $m^*$ interpretiert werden.
Damit lässt sich der bereits beschriebene Drehwinkel $\theta$ berechnen als
\begin{equation}
\theta_.{frei} = \frac{e^3_.0}{8\pi^2\epsilon_.0c^3}\frac{1}{(m^*)^2}\frac{NBL}{n}\lambda^2\text{.}\label{eq:theta2}
\end{equation}
Dabei ist $e_.0$ die Elementarladung, $N$ die Anzahl der freien Ladungsträger pro Volumen, $B$ die magnetische Flussdichte und $\lambda$ die Wellenlänge des Lichts.
Es zeigt sich also die Abhängigkeit $\theta\propto \lambda^2$ wenn viele freie Ladungsträger vorhanden sind, also für hochdotierte Halbleiter. Ist der Halbleiter undotiert und hat somit ausschließlich gebundene Elektronen zeigt sich hingegen eine Proportionalität von
\begin{equation}
\theta\propto\frac{1}{\lambda^2}\text{.}\label{eq:theta3}
\end{equation}



%☻