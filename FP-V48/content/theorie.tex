\section{Theorie}
\label{sec:Theorie}

Werden in einen Ionenkristall, der aus einfach geladenen Ionen besteht, ein zweifach geladenes Kation eingebaut, so entsteht an einem Gitterplatz, der normalerweise von einem einfach geladenen Kation besetzt wäre eine Leerstelle.
Zwischen dem zwischen dieser und dem zweiwertigen Kation bildet sich ein Dipolmoment $\vec{p}$ aus, deren Richtung zunächst willkürlich verteilt ist. Da sich beide Enden auf Gitterplätzen befinden müssen, sind nur diskrete Änderung der Dipolrichtung möglich. Bei Temperaturen $T<\SI{500}{\kelvin}$ können sich nur die Leerstellen auf den Gitterplätzen bewegen, wenn sie die nötige Aktivierungsenergie $W$ besitzen. Nach der Stefan-Boltzmann-Verteilung kann, mit der Boltzmannkonstante $k_.B$, ein Anteil von $\mathrm{e}^{\frac{W}{k_.BT}}$ der Leerstellen bereits durch thermische Bewegung den Gitterplatz wechseln.
Die mittlere Zeit zwischen zwei Dipolrichtungsänderung wird Relaxationszeit genannt und ist gegeben durch
\begin{equation}
\tau=\tau_.0\mathrm{e}^{\frac{W}{k_.BT}}\text{.}\label{eq:tau}
\end{equation}
Wird an den Kristall ein elektrisches Feld $E$ angeschlossen, richten sich gestört durch die bereits vorhandene thermische Energie ein Teil $y$ der Dipole entlang der Feldrichtung aus. Für hohe Temperaturen, also wenn mit dem Betrag des Dipolmoments $p$ und der Boltzmann-Konstante $k_.B$ gilt
\[
pE\ll k_.BT,
\]
kann dieser Teil näherungsweise beschrieben werden als
\begin{equation}
y=\frac{pE}{3k_.BT}\text{.}\label{eq:y}
\end{equation}
Durch schnelles Abkühlen auf eine Temperatur $T_.0$ steigt nach \eqref{eq:tau} die Relaxationszeit stark an, sodass $y=y(T_.p)$ mit der Polarisationstemperatur $T_.p$ als konstant angenommen werden kann.
Durch Erhitzen mit einer konstanten Heizrate $b$ steigt die thermische Energie, sodass die Dipole wieder eine statistische Richtungsverteilung annehmen. Der resultierende Depolarisationsstrom $i(T)$ folgt gemäß Abbildung \ref{fig:i} einer Gaußverteilung, jedoch lässt wird diese in der Realität bei höheren Temperaturen von einer weiteren Gaußverteilung überlagert.
Es gilt mit der Rate der pro Volumen- und Zeiteinheit relaxierenden Dipole $\frac{\mathrm{d}N}{\mathrm{d}t}$ und dem Probenquerschnitt $F$ für den Strom
\begin{equation}
i(T)=F y(T_.p)p\frac{\mathrm{d}N}{\mathrm{d}t}\text{.}\label{eq:i1}
\end{equation}
Das Lösen der DGL
\[
\frac{\mathrm{d}N}{\mathrm{d}t}=-\frac{N}{\tau(T)}
\]
liefert mit der Zahl der zu Beginn bei $T=T_.0$ orientierten Dipole $N_.p$
\begin{equation}
N=N_.p\mathrm{e}^{-\frac{1}{b}\int_{T_.0}^T\frac{\mathrm{d}T'}{\tau(T')}}\text{.}\label{eq:N}
\end{equation}
Mit den Gleichungen \eqref{eq:tau}, \eqref{eq:y} und \eqref{eq:i1} ergibt sich so
\begin{equation}
i(T)=F\frac{p^2E}{3k_.BT_.p}\frac{N_.p}{\tau_.0}\mathrm{e}^{-\frac{1}{b\tau_.0}\int_{T_.0}^T\mathrm{e}^{-\frac{W}{k_.BT'}}\mathrm{d}T'}\cdot e^{-\frac{W}{k_.BT}}\text{.}\label{eq:i2}
\end{equation}
Für kleine Temperaturen verschwindet das Integral im Exponent, sodass sich $i$ im Anfangsbereich der Stromkurve nähern lässt zu
\begin{equation}
i_.{kl}(T)\approx F\frac{p^2E}{3k_.BT_.p}\frac{N_.p}{\tau_.0}e^{-\frac{W}{k_.BT}}\text{.}\label{eq:i_kl}
\end{equation}
Logarithmieren liefert 
\begin{equation}
\ln\left(\frac{i}{i_.0}\right)=-\frac{W}{k_.B}\cdot\frac{1}{T}+const\text{.}\label{eq:ln1}
\end{equation}
Für den gesamten Kurvenverlauf lässt sich die Polarisation $P$ des Kristalls betrachten, die ebenfalls der DGL
\begin{equation}
\frac{\mathrm{d}P}{\mathrm{d}t}=-\frac{P}{\tau(T)}\label{eq:DGL}
\end{equation}
folgt und deren Änderung den Strom
\[
i(t)=F\frac{\mathrm{d}P}{\mathrm{d}t}
\]
bewirkt. Da sich für $t\rightarrow\inf$ alle Dipolrichtungen wieder statistisch verteilt haben und somit gilt $P(\inf)=0$, liefert eine Integration
\[
\int_{t(T)}^{\infty}i(t')\mathrm{d}t'=-F P(t),
\]
also eingesetzt in \eqref{eq:DGL}, unter Berücksichtigung, dass $T$ eine lineare Funktion von $t$ mit Steigung $b$ ist,mit einer hinreichend großen Temperatur $T*$, bei der $i$ bereits auf $i\approx\SI{0}{\pico\ampere}$ abgefallen ist und mit Gleichung \eqref{eq:tau}
\begin{equation}
\mathrm{e}^{\frac{W}{k_.BT}}=\frac{\int_{T}^{T*}i(T')\mathrm{d}T'}{b\tau_.0 i(T)}=\frac{I(T)}{b\tau_.0 i(T)}\text{.}\label{eq:i3}
\end{equation}
Logarithmieren liefert
\begin{equation}
\ln\left(\frac{I(T)}{i(T)\cdot const}\right)=\frac{W}{k_.B}\cdot\frac{1}{T}\text{.}\label{eq:ln2}
\end{equation}
Trägt man die linke Seite den Gleichungen \eqref{eq:ln1} und \eqref{eq:ln2} gegen $\frac{1}{T}$ auf, lässt sich $W$ aus der Steigung der Gerade bestimmen.
Wenn $\frac{\mathrm{d}}{\mathrm{d}T}\left(\eqref{eq:i2}\right)=0$ ist, ist bei $T=T_.max$ das Maximum des Polarisationsstroms.
Das liefert eine Beziehung zur Berechnung von $\tau_.0$:
\begin{equation}
\tau_.0=\frac{k_.BT^2_.{max}}{W b}e^{-\frac{W}{k_.BT_.{max}}}\text{.}\label{eq:tau0}
\end{equation}


%♥♥☺♥
