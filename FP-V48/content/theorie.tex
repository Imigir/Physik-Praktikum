\section{Theorie}
\label{sec:Theorie}

Werden in einen Ionenkristall, der aus einfach geladenen Ionen besteht, ein zweifach geladenes Kation eingebaut, so entsteht an einem Gitterplatz, der normalerweise von einem einfach geladenen Kation besetzt wäre eine Leerstelle.
Zwischen dem zwischen dieser und dem zweiwertigen Kation bildet sich ein Dipolmoment $\vec{p}$ aus, deren Richtung zunächst willkürlich verteilt ist. Da sich beide Enden auf Gitterplätzen befinden müssen, sind nur diskrete Änderung der Dipolrichtung möglich. Bei Temperaturen $T<\SI{500}{\kelvin}$ können sich nur die Leerstellen auf den Gitterplätzen bewegen, wenn sie die nötige Aktivierungsenergie $W$ besitzen. Nach der Stefan-Boltzmann-Verteilung kann, mit der Boltzmannkonstante $k_.B$, ein Anteil von $\e^{\frac{W}{k_.BT}}$ der Leerstellen bereits durch thermische Bewegung den Gitterplatz wechseln.
Die mittlere Zeit zwischen zwei Dipolrichtungsänderung wird Relaxationszeit genann und ist gegeben durch
\begin{equation}
\tau=\tau_.0\e^{\frac{W}{k_.BT}}\label{eq:tau}
\end{equation}

