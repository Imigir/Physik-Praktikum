
\section{Diskussion}
\label{sec:Diskussion}
<<<<<<< HEAD

Der Anstieg der Flanke des zweiten Peaks kann bei beiden Heizraten gut durch eine Exponentialfunktion approximiert werden. Wird jedoch die Abbildung \ref{fig:W1_1} zur Bestimmung von $W_.{1,1}$ betrachtet, so lässt sich erkennen, dass die ersten vier Werte nicht auf der erwarteten Gerade liegen, weshalb sie in der Ausgleichsrechnung nicht mit berücksichtigt werden. Die schlechten Werte sind in dem zu geringen Strom begründet, der hier noch nicht wie erwartet ansteigt, sondern nahe Null bleibt. Das die Gerade im hinteren Bereich abflacht, war zu erwarten, da die genutzte Näherung nur für geringe Temperaturen gilt. Lässt man diese Werte jedoch weg, so ist der Fehler auf $W_{1,1}=\SI{}{}$ wie auch bei der zweiten Heizrate auf $W_{2,1}=\SI{}{}$ gering.\\
Bei der Bestimmung von $W_{1,2}=\SI{}{}$ und $W_{2,2}=\SI{}{}$ über den gesamten Kurvenverlauf werden bei der ersten Heizrate (vergleiche Abbildung \ref{fig:W_1,2}) die ersten vier Werte abermals vernachlässigt, da diese bereits zuvor als ungeeignet identifiziert wurden. Zudem ist bei beiden Heizraten eine Abweichung in den höheren Temperaturbereichen der abfallenden Flanke des ersten Peaks zu erkennen. Dies liegt daran, dass hier die Überlagerung durch den zweiten Peak bereits zu stark ist, sodass sie nicht geeignet durch die Approximation der Exponentialfunktion herausgerechnet werden kann. Davon abgesehen liefert die Ausgleichsrechnung geringe Fehler auf die $W$.\\
Wird der Mittelwert der $W$ gebildet, so ist $W=\SI{}{}$ und der Fehler ist gering, was auf eine gute Bestimmung dieses Wertes schließen lässt. Die Heizraten sind mit $b1=\SI{}{}$ und $b2=\SI{}{}$ ebenfalls sehr konstant gehalten worden.\\
Die charakteristischen Relaxationszeiten $\tau_1\SI{}{}$ und $\tau_2=\SI{}{}$ haben jedoch trotz der geringen Fehler der vorherigen Größen einen relativ großen Fehler. Dies ist in der Exponentialfunktion in Formel \eqref{eq:tau0} begründet, wodurch eine geringe Abweichung in $W$ zu einer großen Abweichung im Ergebnis führt.
=======
\label{tab:tabFehler}
	\sisetup{table-format=1.2}
	\begin{tabular}{c ccc}
		\toprule
		{Wert}&{gemessen}&{Referenzwert}&{Abweichung} \\
		\midrule
		$W_.{1,1}$ & \SI{0,744(19)}{\eV} & \SI{0,6}{\eV} & 
		$W_.{1,2}$ & \SI{0,767(17)}{\eV} & \SI{0,6}{\eV} & 
		$W_.{2,1}$ & \SI{0,714(8)}{\eV} & \SI{0,6}{\eV} & 
		$W_.{2,2}$ & \SI{0,772(9)}{\eV} & \SI{0,6}{\eV} & 
		$W_.{avg}$ & \SI{0,749(13)}{\eV} & \SI{0,6}{\eV} & 
		$\tau_.{0,1}$ & \SI{0,8(5)e-12}{\second} & - & - \\
		$\tau_.{0,2}$ & \SI{1,0(6)e-12}{\second} & - & - \\ 
		\bottomrule
	\end{tabular}
	
	
	
%☻♠
>>>>>>> 7dbe2d23f0d2841e9b672f8399a38706038543db
