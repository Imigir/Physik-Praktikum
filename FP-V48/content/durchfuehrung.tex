\section{Durchführung}
\label{sec:Durchführung}

Der Kristall wird in einem evakuierten Behälter über eine Konstantstromquelle auf eine Temperatur von $T=\SI{320}{\kelvin}$ erhitzt und für $t=\SI{1200}{\second}$ eine Spannung von $U=\SI{900}{\volt}$ an den Kondensator angelegt.
Bei eingeschalteter Spannung wird der Kühlfinger in flüssigen Stickstoff eingetaucht und die Probe so auf $T=\SI{220}{\kelvin}$ abgekühlt.
Nach Abschalten der Spannung wird der Kondensator über ein Kabel entladen und an das Picoamperemeter angeschlossen.
Für eine Heizrate von $b=2$ und $\SI{1,4}{\kelvin\per\minute}$ wird jede Minute der Strom $i$ in Abhängigkeit von der Temperatur bis $T=\SI{320}{\kelvin}$ aufgenommen.
Während der Messung wird der Heizstrom so reguliert, dass die Heizrate möglichst konstant bleibt.
