\section{Durchführung}
\label{sec:Durchführung}

Zu Beginn wird die Intensität des Lichtstrahles durch justieren der Elemente im Strahlengang maximiert. Anschließend wird der Versuchsaufbau mit einer schwarzen Decke abgedeckt, um die Photozelle vor externen Lichtquellen und Streulicht zu schützen.
Es wird der Aufbau möglichst genau am Erdmagnetfeld ausgerichtet und das vertikale Erdmagnetfeld wird kompensiert. Dazu wird das Signal des Sweep-Feldes auf den X-Eingang des Oszilloskops gelegt, sodass die Intensität in Abhängigkeit dieses Feldes auf dem Schirm erscheint. Zu Beginn ist bei der Nullstelle eine breite Resonanz zu erkennen, welche durch abwechselndes justieren von dem Vertikalfeld und der Ausrichtung des Versuchsaufbaus minimiert wird. Anhand des an der vertikalen Spule angelegten Stromes wird das vertikale Erdmagnetfeld bestimmt.\\
Es werden die Resonanzen der Rb-Isotope in Abhängigkeit von der Frequenz des RF-Feldes untersucht. Das RF-Feld wird mit einem $\SI{4}{\volt}$ Sinus-Signal betrieben. Die Frequenz wird zwischen $\SI{100}{\kilo\hertz}$ und $\SI{1000}{\kilo\hertz}$ variiert. Es werden das horizontale Erdmagnetfeld, die Landé-Faktoren und die Kernspins bestimmt.
Es wird ein Signalbild bei $\SI{100}{\kilo\hertz}$ aufgenommen und aus der Tiefe der Resonanzen das Verhältnis der Rb-Isotope in der Probe bestimmt.\\
Es wird eine weitere Rechteckspannung mit $\SI{5}{\hertz}$ und $\SI{4}{\volt}$ als modulation des RF-Feldes hizugefügt. Das Oszilloskop wird auf YT-Betrieb umgestellt. Die Ansteigende Flanke der Transparenzkurven wird für beide Isotope gespeichert und untersucht.\\
Zuletzt wird die Amplitude der RF-Spannung von $\SI{2}{\volt}$ bis $\SI{10}{\volt}$ variiert und die Periodendauer der Rabi-Oszillationen wird in Abhängigkeit der Amplitude vermessen. Aus den Messwerten wird das Verhältnis der Landé-Faktoren bestimmt. 