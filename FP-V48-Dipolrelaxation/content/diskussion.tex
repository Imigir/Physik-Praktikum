
\section{Diskussion}
\label{sec:Diskussion}

Der Anstieg der Flanke des zweiten Peaks kann bei beiden Heizraten gut durch eine Exponentialfunktion approximiert werden. Wird jedoch die Abbildung \ref{fig:W1_1} zur Bestimmung von $W_.{1,1}$ betrachtet, so lässt sich erkennen, dass die ersten vier Werte nicht auf der erwarteten Gerade liegen, weshalb sie in der Ausgleichsrechnung nicht mit berücksichtigt werden. Die schlechten Werte sind in dem zu geringen Strom begründet, der hier noch nicht wie erwartet ansteigt, sondern nahe Null bleibt. Das die Gerade im hinteren Bereich abflacht, war zu erwarten, da die genutzte Näherung nur für geringe Temperaturen gilt. Lässt man diese Werte jedoch weg, so ist der Fehler auf $W_{1,1}=\SI{0,744(19)}{\electronvolt}$ wie auch bei der zweiten Heizrate auf $W_{2,1}=\SI{0,714(8)}{\electronvolt}$ gering.\\
Bei der Bestimmung von $W_{1,2}=\SI{0,767(17)}{\electronvolt}$ und $W_{2,2}=\SI{0,772(9)}{\electronvolt}$ über den gesamten Kurvenverlauf werden bei der ersten Heizrate (vergleiche Abbildung \ref{fig:W1_2}) die ersten vier Werte abermals vernachlässigt, da diese bereits zuvor als ungeeignet identifiziert wurden. Zudem ist bei beiden Heizraten eine Abweichung in den höheren Temperaturbereichen der abfallenden Flanke des ersten Peaks zu erkennen. Dies liegt daran, dass hier die Überlagerung durch den zweiten Peak bereits zu stark ist, sodass sie nicht geeignet durch die Approximation der Exponentialfunktion herausgerechnet werden kann. Davon abgesehen liefert die Ausgleichsrechnung geringe Fehler auf die $W$.\\
Wird der Mittelwert der Aktivierungsenergien gebildet, so ist $W=\SI{0,749(13)}{\electronvolt}$ und der Fehler auf den Mittelwert ist gering, was auf eine akzeptable Bestimmung dieses Wertes schließen lässt. Wird der Wert jedoch mit dem Literaturwert $W_.{lit}=\SI{0,65}{\eV}$ \cite{DipolW} verglichen (vergleiche Tabelle \ref{tab:Diskussion}), liegt der Fehler bei $15\%$. Dies könnte auf einen systematischen Fehler hindeuten. Die Heizraten sind, wie am geringen Fehler auf den Mittelwert zu erkennen, mit $b1=\SI{1,95(7)}{\kelvin\per\minute}$ und $b2=\SI{1,39(3)}{\kelvin\per\minute}$ ebenfalls sehr konstant gehalten worden.\\
Die charakteristischen Relaxationszeiten $\tau_.{0,1}$ und $\tau_.{0,2}$ aus Tabelle \ref{tab:Diskussion} haben jedoch trotz der geringen Fehler der vorherigen Größen einen relativ großen Fehler. Dies ist in der Exponentialfunktion in Formel \eqref{eq:tau0} begründet, wodurch eine geringe Abweichung in $W$ zu einer großen Abweichung im Ergebnis führt. Die große Abweichung vom Literaturwert lässt sich ebenfalls durch die Exponentialfunktion begründen.
Beim Vergleich der beiden Messreihen bestätigt sich, dass die Aktivierungsenergie $W$ und die charakteristische Relaxationszeit $\tau_.0$ konstant und unabhängig von der Heizrate sind. An den Graphen zeigt sich, dass für größere Heizraten das Maximum des Depolarisationsstroms leicht zu höheren Temperaturen verschoben und der gesamte Peak stärker lokalisiert ist. Die Verschiebung lässt sich dadurch erklären, dass $\tau_.0$ nach Formel \ref{eq:tau0} invers proportional zur Heizrate $b$ und proportional zu $T^2_.{max}$ ist. Da $\tau_.0$ konstant sein muss, muss sich bei einer höheren Heizraten auch $T_.{max}$ erhöhen. Die Schärfe des Peaks lässt sich darauf zurückführen, dass auf Grund der schnellen Temperaturerhöhung mehr Dipole auf einmal die nötige Aktivierungsenergie besitzen.
\begin{table}
\centering
	\caption{Die Werte für die Aktivierungsenergien und die Relaxationszeiten im Vergleich mit den Literaturwerten.}
	\label{tab:Diskussion}
	\sisetup{table-format=1.2}
	\begin{tabular}{c ccc}
		\toprule
		{Wert}&{gemessen}&{Referenzwert\cite{DipolW}\cite{t_dis}}&{Abweichung} \\
		\midrule
		$W_.{1,1}$ & \SI{0,744(19)}{\eV} & \SI{0,65}{\eV} & 14,46\%\\
		$W_.{1,2}$ & \SI{0,767(17)}{\eV} & \SI{0,65}{\eV} & 18,00\%\\
		$W_.{2,1}$ & \SI{0,714(8)}{\eV} & \SI{0,65}{\eV} & 9,85\%\\
		$W_.{2,2}$ & \SI{0,772(9)}{\eV} & \SI{0,65}{\eV} & 18,77\%\\
		$W_.{avg}$ & \SI{0,749(13)}{\eV} & \SI{0,65}{\eV} & 15,23\%\\
		$\tau_.{0,1}$ & \SI{0,8(5)e-12}{\second} & \SI{4e-14}{\second} & 1900\%\\
		$\tau_.{0,2}$ & \SI{1,0(6)e-12}{\second} & \SI{4e-14}{\second} & 2400\%\\ 
		\bottomrule
	\end{tabular}
\end{table}

%☻