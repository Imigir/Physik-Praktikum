
\section{Durchführung}
\label{sec:Durchführung}

Es werden die theoretischen Fourier-Koeffizienten einer geraden beziehungsweise ungeraden Rechteck-, Dreieck- und Sägezahnspannung berechnet.
Mithilfe des ersten Versuchsaufbaus werden die ersten neun von null verschiedenen Fourier-Koeffizienten der Spannungen experimentell bestimmt und mit den theoretischen Werten verglichen.
Die Einhaltung des Abtasttheorems wird überprüft.\newline
Mithilfe des zweiten Versuchsaufbaus und den theoretisch berechneten Werten für die Fourier-Koeffizienten werden die Spannungen aus ihren Schwingungsanteilen bis zur achten Oberwelle zusammengesetzt. Um die Phasen der Schwingungen richtig einzustellen, werden mit dem XY-Eingang des Oszilloskops die Lissajous-Figuren der Grundschwingung mit den $n$ten Oberschwingungen betrachtet. Bei ungeradem $n$ hat man die Phase $0$ oder $\pi$ eingestellt, wenn die Lissajous-Figur eine Linie mit Anfangs- und Endpunkt bildet. Bei geradem $n$ hat man die Phase $0$ oder $\pi$ eingestellt, wenn die Lissajous-Figur eine achsensymmetrische geschlossene Kurve bildet.
Es werden Thermodrucke der Schwingungsbilder erstellt. 