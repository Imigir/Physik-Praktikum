\section{Diskussion}
\label{sec:Diskussion}
Die bei der Fourier-Analyse auftretenden Fehler lassen sich Tabelle \ref{tab:tabd} entnehmen.
\begin{table}
	\centering
	\caption{Einstellungen zur Synthese einer Dreieckspannung.}
	\sisetup{table-format=1.2}
	\begin{tabular}{c S[table-format=1.0] S[table-format=1.2] S[table-format=1.1]}
		\toprule
		{$Art der Spannung$}&{$b_.{theorie}$}&{$b_.{mess}$}&{$\Delta b$}\\
		\midrule
		 Rechteck & 1 & 1,01\pm 0,01 & 1\%  \\
		 Sägezahn & 1 & 1,00\pm 0,02 & 0\% \\
		 Dreieck & 2 & 2,05\pm 0,01 & 0,5\% \\
		
		\bottomrule
	\end{tabular}
	\label{tab:tabd}
\end{table}
Hieran wird deutlich, dass dieser Teil des Versuchs die Theoriewerte nahezu exakt belegen konnte.
Bei allen drei Fourier-Synthesen sind annähernd periodische Nebenschwingungen zu erkennen, die darauf zurückzuführen sind, dass nicht unendlich viele, sondern nur wenige Oberwellen verwendet wurden. 
Die besonders starken Störungen, die bei der Sägezahnspannung auftreten, obwohl hier die meisten Oberwellen zur Synthese verwendet wurden, sind womöglich auf fehlerhafte Justierung der Lissajous-Figuren zurückzuführen. 
Die fehlende Punkt- oder Achsensymmetrie der synthetisierten Dreieckspannung lässt darauf schließen, dass dort ein Schalter zur Verschiebung der Phase um $\SI{90}{\degree}$ umgelegt war.
