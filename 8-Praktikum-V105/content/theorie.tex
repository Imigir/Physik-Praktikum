\section{Theorie}
\label{sec:Theorie}
\subsection{Das Trägheitsmoment einer Vollkugel}
Das Trägheitsmoment einer Vollkugel bestimmt sich als \cite{V101}
\begin{equation}
J_.K = \frac{2}{5} m r^2\label{eq:J}
\end{equation}
\subsection{Das magnetische Moment}
Die kleinste Form des Magnetismus ist ein Dipol.
Dieser besitzt ein magnetisches Moment $\vec{\mu}$ , das sich im Falle einer stromdurchflossenen Leiterschleife ergibt als
\[\vec{\mu} = I\vec{A},\]
wobei $I$ den fließenden Strom und $A$ die gerichtete Querschnittsfläche der Schleife ist.
Das magnetische Moment eines Permanentmagnet lässt sich so allerdings nicht berechnen.
Es wird zur Messung ein externes homogenes magnetisches Feld benötigt,
das hier durch ein Helmholtz-Spulenpaar erzeugt wird, da diese in ihrem Zentrum ein nahezu homogenes Magnetfeld erzeugen.
Die Feldstärke $B$ ergibt sich dabei mit dem vereinfachten Biot-Savart-Gesetz für eine stromdurchflossene Spule für das Zentrum \cite{V105}
\begin{equation}
B(0) = \frac{\mu_.0IR^2}{(R^2+x^2)^{\frac{3}{2}}}\label{eq:B}
\end{equation}
wobei $R$ der Radius der Spulen ist. Die Hälfte des Abstands $d$ zu den Spulen
ist dabei
\[x=\frac{d}{2}\]
und $\mu_.0 = \SI{4\pi e-7}{\newton\per\ampere\squared}$ die magnetische Feldkonstante ist.

\subsubsection{Messung mithilfe der Gravitation}
In einem externen Magnetfeld wirkt auf den magnetischen Dipol im Inneren der Kugel ein Drehmoment
\[\vec{M}_.B = \vec{\mu}_.{Dipol}\times\vec{B}\]
Die Gravitation bewirkt wegen der außerhalb der Kugel im Abstand $r$ befestigten Masse $m$ das Drehmoment
\[\vec{M}_.G = m\vec{r}\times\vec{g}\]
Gleichsetzen und umschreiben ergibt
\[m r g sin(\theta)= \mu_.{Dipol} B sin(\theta)\]
Auf Grund der Parallelität von Gewichtskraft und Magnetfeld kürzen sich die Sinus-Therme weg\cite{V105} und es ergibt sich
\begin{equation}
\mu_.{Dipol} = \frac{m r g}{B}\label{eq:mu}
\end{equation}
\subsubsection{Messung mithilfe der Periodendauer}
Durch Auslenkung der Kugel im homogenen $\vec{B}$-Feld entsteht eine harmonisch oszillierende Schwingung, die sich durch
\[-|\vec{\mu}_.{Dipol}\times\vec{B}| = J_.K \frac{\mathrm{d}^2\theta}{\mathrm{d}t^2}\]
beschreiben lässt\cite{V105}.\newline
Eine Lösung dieser Gleichung ist die Periodendauer $T$ der Schwingung,
weshalb sich für das magnetische Moment ergibt:
\begin{equation}
\mu_.{Dipol} = \frac{4\pi^2J_.K}{T^2B}\label{eq:mu2}
\end{equation}

\subsubsection{Messung mithilfe der Präzessionsperiodendauer}
Wirkt auf die rotierende aus der aufrechten Lage ausgelenkte Kugel ein externes Magnetfeld, so lässt sich die Bewegung der Kugel mit dem Drehimpuls 
\[|\vec{L}_.K| = 2\pi J_.K \nu\], mit der Stroboskop-Frequenz $\nu$,
beschreiben als
\[\vec{\mu}_.{Dipol}\times\vec{B} = \frac{\mathrm{d}\vec{L}_.K}{\mathrm{d}t}\]
dessen Lösung durch die Präzessionsfrequenz
\[\Omega_.P = \frac{\mu B}{|\vec{L}_.K|}\]
gegeben ist\cite{V105}.
Dadurch ergibt sich für das magnetische Moment
\begin{equation}
\mu_.{Dipol} = \frac{2\pi\nu\Omega_.PJ_.K}{B}\label{eq:mu3}
\end{equation}