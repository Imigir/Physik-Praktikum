
\section{Diskussion}
\label{sec:Diskussion}

\begin{table}
	\centering
	\caption{Die Werte der Dichten der Stäbe und deren Abweichungen vom Literaturwert}
	\label{tab:tabFehler}
	\sisetup{table-format=1.2}
	\begin{tabular}{cS[table-format=1.2]S[table-format=1.1]}
		\toprule
		{} & {$\rho/10^3\si{\kilogram\per\cubic\metre}$} & {$\Delta\rho/\%$}\\
		\midrule
		Stab,rund & 2.81 & 4,1\\
		Stab,quad & 2.78 & 3.0\\
		\bottomrule
	\end{tabular}

	\label{tab:Fehler1}
\end{table}
\noindent Vergleicht man die ermittelten Dichten der Stäbe aus Tabelle \ref{tab:Fehler1} mit dem Literaturwert von Aluminium ($\SI{2,70e3}{\kilogram\per\cubic\metre}$\cite{Dichte}), so wird bestätigt, dass es sich bei den Stäben um Aluminiumstäbe handelt.

\begin{table}
	\centering
	\caption{Die Werte der bestimmten Elastizitätsmodule der Stäbe und deren Abweichungen vom Literaturwert}
	
%/10^{10}\si{\pascal}
\label{tab:tabFehler}
	\sisetup{table-format=1.2}
	\begin{tabular}{cS[table-format=1.2]S[table-format=2.1]}
		\toprule
		{} & {$E/10^{10}\si{\pascal}$} & {$\Delta E/\%$}\\
		\midrule
		Stab,rund,einseitig  & 7.41 & 5.9\\
		Stab,quad,einseitig  & 6.13 & -12.4\\
		Stab,quad,beidseitig & 7.01 & 0.1\\
		\bottomrule
	\end{tabular}
	\label{tab:Fehler2}
\end{table}

\noindent Die Werte für E aus Tabelle \ref{tab:Fehler2} liegen bis auf den zweiten einigermaßen in der Nähe des Literaturwertes von $\SI{7e10}{\pascal}$\cite{ElastizitaetsmodulAlu}.
Dabei ist der durch beidseitige Auflage des quadratischen Stabs bestimmte Wert beinahe identisch mit dem Literaturwert. Das der zweite Wert etwas mehr abweicht, könnte an einem falsch bestimmten $L$ liegen, da diese Länge nachträglich gemessen wurde. Eine erneute Messreihe könnte diesen Fehler beheben. 
Die Empfindlichkeit der Messuhren, die selbst bei geringeren Erschütterung zu veränderten Messsergebnissen führt, könnte ebenfalls eine Ursache darstellen.
