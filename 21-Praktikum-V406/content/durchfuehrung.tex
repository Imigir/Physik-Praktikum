
\section{Durchführung}
\label{sec:Durchführung}

Der Abstand $L$ zwischen Blende und Photoelement, sowie die Wellenlänge des Laser werden notiert. Der Dunkelstrom des Photoelements wird notiert, indem der Strom bei nicht eingeschaltetem Laser gemessen wird. Für einen korrekten Dunkelstrom sollten sich die Lichtverhältnisse nach der Dunkelstrommessung nicht mehr ändern. Für einen Einzelspalt und zwei Doppelspalte werden die Lichtintensitäten der Beugungsfigur bestimmt, indem mit dem Photoelement an mindestens 50 stellen die Intensität aufgenommen wird. Bei dem Einzelspalt soll dabei im Bereich vom ersten linken Maximum bis zum ersten rechten Maximum gemessen werden. Bei den Doppelspalten sollen die Intensitäten im Bereich zwischen den zweiten Maxima gemessen werden. 