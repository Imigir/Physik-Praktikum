
\section{Diskussion}
\label{sec:Diskussion}

Die bestimmten Messwerte und ihre Abweichungen von den bekannten Referenzwerten sind in Tabelle \ref{tab:Fehler} zu finden.
Die Interferenzmuster wurden so ausgerichtet, dass das Hauptmaximum jeweils bei $\Delta x =\SI{0}{\metre}$ liegt. Das es dennoch zu Aweichungen der $x_.0$ vom Nullpunkt kommt, lässt sich damit erklären, dass das Hauptmaximum nur per Augenmaß ausgerichtet wurde.
Die Abweichungen der Spaltbreiten $b$ liegen im Rahmen der Messungenauigkeit und sind darauf zurückzuführen, dass bei der Messung der Ströme mehrere Skalenwechsel erfolgten, die die Ergebnisse veränderten. 
Der Spaltabstand $s$ der beiden Doppelspalte konnte hingegen äußerst genau bestimmt werden.
Beim Vergleich der Interferenzmuster \ref{fig:Einzel} bis \ref{fig:Doppel2} fällt zunächst auf, dass die Intensitäten der Doppelspalte um einen Faktor 10 größer sind als die des Einzelspalts. Dies ist auch sinnvoll, da deren Spaltbreite $b_.D=\SI{1,5e-4}{\metre}$ doppelt so groß sind wie die des Einzelspalts $b_.E=\SI{7,5e-5}{\metre}$. Außerdem lässt sich erkennen, dass das Muster der Doppelspalte durch den Intensitätsverlauf eines Einzelspalts eingehüllt werden kann.
Die Maxima des 2. Doppelspalts liegen wesentlich näher zusammen, als die des 1., was sich auf den größeren Spaltabstand $s$ zurückführen lässt.

\begin{table}
\centering
\caption{Die in der Auswertung bestimmten Messwerte mit den zugehörigen Referenzwerten und Abweichungen}
\label{tab:Ergebnisse}
	\sisetup{table-format=1.2}
	\begin{tabular}{c ccc}
		\toprule
		{Wert}&{gemessen}&{Referenzwert}&{Abweichung} \\
		\midrule
		$\alpha_{max}$ & \SI{28,1}\,\si{\degree} & \SI6{28}\,\si{\degree} & \SI{0,3}\,\si{\percent} \\
		$\theta_{gr}$ & \SI{5}\,\si{\degree} & \SI{5,04}\,\si{\degree} & \SI{-0,9}\,\si{\percent} \\
		$\lambda_{min}$ & \SI{35,1}\,\si{\pico\metre} & \SI{35,4}\,\si{\pico\metre}  & \SI{-1,13}\,\si{\percent} \\
		$E_{kin,max}$ & \SI{35316}\,\si{\eV} & \SI{35000}\,\si{\eV} & \SI{0,9}\,\si{\percent} \\
		$\Delta E_{alpha}$ & \SI{130,7}\,\si{\eV} & - & - \\
		$\Delta E_{beta}$ & \SI{155,0}\,\si{\eV} & - & - \\
		$\sigma_{Cu_K}$ & \SI{3,28} & \SI{3,31} & \SI{-0,76}\,\si{\percent}  \\
		$\sigma_{Cu_{L}}$ & \SI{13,16} & \SI{20,72} & \SI{-36,48}\,\si{\percent} \\
		$\sigma_{Cu_{M}}$ & \SI{29} & \SI{26,64} & \SI{8,87}\,\si{\percent} \\
		$E_{K_{Br}}$ & \SI{13282}\,\si{\eV} & \SI{13470}\,\si{\eV} & \SI{-1,40}\,\si{\percent} \\
		$E_{K_{Sr}}$ & \SI{15988}\,\si{\eV} & \SI{16090}\,\si{\eV} & \SI{-0,64}\,\si{\percent} \\
		$E_{K_{Zn}}$ & \SI{9650}\,\si{\eV} & \SI{9650}\,\si{\eV} & \SI{0,0}\,\si{\percent} \\
		$E_{K_{Zr}}$ & \SI{17903}\,\si{\eV} & \SI{17970}\,\si{\eV} & \SI{-0,37}\,\si{\percent} \\
		$\sigma_{K_{Br}}$ & \SI{3,75} & \SI{3,53} & \SI{6,23}\,\si{\percent} \\
		$\sigma_{K_{Sr}}$ & \SI{3,71} & \SI{3,66} & \SI{1,37}\,\si{\percent} \\
		$\sigma_{K_{Zn}}$ & \SI{3,36} & \SI{3,36} & \SI{0,0}\,\si{\percent} \\
		$\sigma_{K_{Zr}}$ & \SI{3,72} & \SI{3,65} & \SI{1,92}\,\si{\percent} \\
		$R_{\infty}$ & \SI{16,87\pm 0,32}\,\si{\eV} & \SI{13,6}\,\si{\eV} & \SI{24,04}\,\si{\percent} \\
		$\sigma_{L_{Bi}}$ & \SI{3,31} & \SI{3,58} & \SI{-7,54}\,\si{\percent} \\
		\bottomrule
	\end{tabular}

\label{tab:Fehler}
\end{table}
