\documentclass[
  bibliography=totoc,     % Literatur im Inhaltsverzeichnis
  captions=tableheading,  % Tabellenüberschriften
  titlepage=firstiscover, % Titelseite ist Deckblatt
]{scrartcl}

\usepackage[ngerman]{babel} 
\usepackage[utf8]{inputenc}
\usepackage[T1]{fontenc} 
\usepackage{lmodern}
\usepackage{setspace}
\usepackage{wrapfig} %Für Bilder
\usepackage{xcolor} %Farbe

%\usepackage{scrpage2}

\usepackage{amssymb} % Mathe Symbole
\usepackage{amsmath} % Mathe
\usepackage{mathtools} % Erweiterungen für amsmath
\usepackage{esvect} %Für vektoren

\usepackage{setspace} % Fuer eine Liste mit Abk.
\usepackage{listings} % Um Code schoener aussehen zu lassen

% Zahlen und Einheiten
\usepackage[
  locale=DE,                   % deutsche Einstellungen
  separate-uncertainty=true,   % immer Fehler mit \pm
  per-mode=symbol-or-fraction, % / in inline math, fraction in display math
]{siunitx}

% chemische Formeln
\usepackage[
  version=4,
  math-greek=default, % ┐ mit unicode-math zusammenarbeiten
  text-greek=default, % ┘
]{mhchem}

% richtige Anführungszeichen
\usepackage[autostyle]{csquotes}

% schöne Brüche im Text
\usepackage{xfrac}

% Standardplatzierung für Floats einstellen
\usepackage{float}
\floatplacement{figure}{htbp}
\floatplacement{table}{htbp}


%keine Floats in anderen Sectionen
\usepackage{placeins}

\let\Oldsection\section
\renewcommand{\section}{\FloatBarrier\Oldsection}

\let\Oldsubsection\subsection
\renewcommand{\subsection}{\FloatBarrier\Oldsubsection}

\let\Oldsubsubsection\subsubsection
\renewcommand{\subsubsection}{\FloatBarrier\Oldsubsubsection}

% Seite drehen für breite Tabellen: landscape Umgebung
\usepackage{pdflscape}

% Captions schöner machen.
\usepackage[
  labelfont=bf,        % Tabelle x: Abbildung y: ist jetzt fett
  font=small,          % Schrift etwas kleiner als Dokument
  width=0.9\textwidth, % maximale Breite einer Caption schmaler
]{caption}
% subfigure, subtable, subref
\usepackage{subcaption}

% Grafiken können eingebunden werden
\usepackage{graphicx}
% größere Variation von Dateinamen möglich
\usepackage{grffile}

% schöne Tabellen
\usepackage{booktabs}

% Verbesserungen am Schriftbild (scheint zu Fehlern zu führen)
%\usepackage{microtype}

% Literaturverzeichnis
\usepackage[
  backend=biber,
]{biblatex}
% Quellendatenbank
\addbibresource{lit.bib}
\addbibresource{programme.bib}

% Hyperlinks im Dokument
\usepackage[
  unicode,        % Unicode in PDF-Attributen erlauben
  pdfusetitle,    % Titel, Autoren und Datum als PDF-Attribute
  pdfcreator={},  % ┐ PDF-Attribute säubern
  pdfproducer={}, % ┘
]{hyperref}

% erweiterte Bookmarks im PDF
\usepackage{bookmark}

%erweiterte Aufzählungen
\usepackage{paralist}

% Trennung von Wörtern mit Strichen
\usepackage[shortcuts]{extdash}

\author{%
  David Rolf%
  \texorpdfstring{%
    \\%
    \href{mailto:david.rolf@tu-dortmund.de}{david.rolf@tu-dortmund.de}
  }{}%
  \texorpdfstring{\and}{, }%
  Jonah Blank%
  \texorpdfstring{%
    \\%
    \href{mailto:jonah.blank@tu-dortmund.de}{jonah.blank@tu-dortmund.de}
  }{}%
}
%\publishers{TU Dortmund – Fakultät Physik}

% Ableitungen mit \diff
\newcommand{\diff}{\mathop{}\!\mathrm{d}}

% mache . zu einem aktiven Zeichen im Mathemodus 
\mathcode`\.="8000 
% Dann kommt die Umdefinition des Dezimalpunktes 
\begingroup\lccode`~=`. 
  \lowercase{\endgroup\def~}#1{\mathrm{#1}} 