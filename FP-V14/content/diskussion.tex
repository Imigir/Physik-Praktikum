
\section{Diskussion}
\label{sec:Diskussion}

Bei der Messung der Energieverteilung konnte das erwartete Energiespektrum beobachtet werden.
Es wurden drei Referenzmessungen gemacht, eine gerade durch den Würfel, eine durch die Hauptdiagonale und eine durch die erste Nebendiagonale. Dies hat den Grund, dass bei den diagonalen Messungen das Aluminiumgehäuse effektiv dicker ist. Es wird zwischen Haupt- und Nebendiagonale unterschieden, da die Verarbeitung des Würfels an den Kanten nicht optimal ist und dadurch Abweichungen in der Intensität auftreten.\\
\newpage
\noindent Die für die Bestimmung der Materialien notwendigen Literaturwerte sind in Tabelle \ref{tab:Literatur} eingetragen.
Dabei berechnen sich die Absorptionskoeffizienten $\mu_.{lit}$ über
\[
\mu_.{lit} = \sigma_.{ges,lit}\cdot\rho_.{lit},
\]
mit der Dichte $\rho$ und dem Massenschwächungskoeffizienten $\sigma$.

\begin{table}
	\centering
	\caption{Die Literaturwerte für $\sigma$ \cite{absorptionskoeffizienten} und $\rho$ \cite{dichten}, sowie die daraus berechneten Absorptionskoeffizienten $\mu$.}
	
	\sisetup{table-format=1.2}
	\begin{tabular}{cS[table-format=2.3]S[table-format=2.2]S[table-format=1.4]}
		\toprule
		{Material} & {$\sigma_.{ges,lit}/10^{-2}\si{\centi\metre\squared\per\gram}$} & {$\rho_.{lit}/\si{\gram\per\cubic\centi\metre}$} & {$\mu_{lit}/\si{\per\centi\metre}$} \\
		\midrule
		{Aluminium}	& 7,466 	& 2,7 	& 0,2016 \\
		{Blei}     	& 11,013 	& 11,34	& 1,2485 \\
		{Eisen}	 	& 7,346 	& 7,86 	& 0,5774 \\
		{Deril}	 	& 8,231 	& 1,41 	& 0,1161 \\
		{Messing}	& 7,282 	& 8,52 	& 0,6204 \\
		\bottomrule
	\end{tabular}

	\label{tab:Literatur}
\end{table}

\noindent Die bestimmten gemittelten Absorptionskoeffizienten für den zweiten und dritten Würfel, sowie die zugeordneten Materialien und Abweichungen sind in Tabelle \ref{tab:Zuordnung1} eingetragen. Es ist zu erkennen, dass der dritte Würfel dem Material Delrin zugeordnet werden kann mit einer Abweichung des Absorptionskoeffizienten von $\SI{-7.0}{\%}$. Das bestimmte Absorptionskoeffizient des zweiten Würfels liegt genau zwischen dem von Messing und Eisen, sodass keine eindeutige Zuordnung möglich ist. Betrachtet man die relative Abweichung, liegt der Wert mit einer Abweichung von $\SI{2.5}{\%}$ näher an dem Absorptionskoeffizienten von Eisen.

\begin{table}
	\centering
	\caption{Die Zuordnung der Materialien der Würfel 2 und 3, sowie die Abweichungen vom Literaturwert.}
	\label{tab:tabWürfel23}
	\sisetup{table-format=1.2}
	\begin{tabular}{cS[table-format=1.3]@{${}\pm{}$}S[table-format=1.3]ccc}
		\toprule
		{Messung} & \multicolumn{2}{c}{$\mu/\si{\per\centi\metre}$} & {Material} & {$\mu_.{lit}/\si{\per\centi\metre}$} & {Abweichung/\si{\%}}\\
		\midrule
		 {Würfel 2}     & 0.592 & 0.022 & {Eisen/Messing} 	& 0.5774/0.6204	& 2.5/-4.6  \\
		 {Würfel 3} 	& 0.108 & 0.003 & {Delrin} 			& 0.1161  		& -7.0  \\
		\bottomrule
	\end{tabular}

	\label{tab:Zuordnung1}
\end{table}

\noindent Da bekannt ist, dass der fünfte Würfel aus einer Kombination der Materialien der Würfel zwei und drei besteht, werden über die bestimmten Werte der Absorptionskoeffizienten die dicht beieinander liegen Mittelwerte gebildet. Sie sind mit der Zuordnung zu einem Material in Tabelle \ref{tab:Zuordnung2} eingetragen. Hier können die Materialien Eisen mit einer Abweichung von $\SI{-1.8}{\%}$ und Delrin mit einer Abweichung von $\SI{18.0}{\%}$ zugeordnet werden. Es ist daher anzunehmen, dass der zweite Würfel wahrscheinlicher aus Eisen, als aus Messing besteht. 

\begin{table}
	\centering
	\caption{Die Zuordnung der gemittelten Materialien des Würfels 5, sowie die Abweichungen vom Literaturwert.}
	\label{tab:tabWürfel5ab}
	\sisetup{table-format=1.2}
	\begin{tabular}{cS[table-format=1.3]@{${}\pm{}$}S[table-format=1.3]cS[table-format=1.4]S[table-format=2.1]}
		\toprule
		{Messung} & \multicolumn{2}{c}{$\mu/\si{\per\centi\metre}$} & {Material} & {$\mu_.{lit}/\si{\per\centi\metre}$} & {Abweichung/\si{\%}}\\
		\midrule
		 {Würfel 5,a}   & 0.567 & 0.014 & {Eisen} 	& 0.5774	& -1.8  \\
		 {Würfel 5,b} 	& 0.137 & 0.013 & {Delrin} 	& 0.1161  	& 18.0  \\
		\bottomrule
	\end{tabular}

	\label{tab:Zuordnung2}
\end{table}

\newpage
\noindent Alle bestimmten Absorptionskoeffizienten der Elementarwürfel des fünften Würfels, sowie die zugeordneten Materialien und Abweichungen sind in Tabelle \ref{tab:Zuordnung3} eingetragen. Die größten Abweichungen besitzen die Elementarwürfel 2 mit einer Abweichung von $\SI{23.2}{\%}$ und 6 mit einer Abweichung von $\SI{37.8}{\%}$. Generell ist die relative Abweichung bei Delrin höher, als bei Eisen. Die absolute Abweichung ist dabei nicht zwingend größer. Da der Literaturwert jedoch geringer ist, haben kleine absolute Abweichungen bereits einen großen Einfluss auf die relativen Abweichungen.\\
Das Messverfahren ist somit in dieser Hinsicht besser für die Untersuchung von Materialien mit einem höheren Absorptionskoeffizienten geeignet. Allerdings kann auch bei dem schwereren Material keine eindeutige Zuordnung zu Eisen oder Messing vorgenommen werden. Das Verfahren liefert also nur eine ungefähre Vorstellung, wie der untersuchte Körper aufgebaut ist. Die Unterscheidung zwischen schwerem und leichten Material konnte nämlich sehr gut vorgenommen werden, sodass mit den Referenzmessungen der Würfel zwei und drei eine recht eindeutige Zuordnung der Materialien vorgenommen werden konnte.        

\begin{table}
	\centering
	\caption{Die Zuordnung der Materialien der Elementarwürfel des Würfels 5, sowie die Abweichungen vom Literaturwert.}
	\label{tab:tabWürfel51-9}
	\sisetup{table-format=1.2}
	\begin{tabular}{cS[table-format=1.3]@{${}\pm{}$}S[table-format=1.3]cS[table-format=1.4]S[table-format=2.1]}
		\toprule
		{Messung} & \multicolumn{2}{c}{$\mu/\si{\per\centi\metre}$} & {Material} & {$\mu_.{lit}/\si{\per\centi\metre}$} & {Abweichung/\si{\%}}\\
		\midrule
		 {Würfel 5,1}   & 0.538 & 0.006 & {Eisen} 	& 0.5774	&  -6.8\\
		 {Würfel 5,2} 	& 0.143 & 0.006 & {Delrin} 	& 0.1161  	&  23.2\\
		 {Würfel 5,3} 	& 0.108 & 0.005 & {Delrin} 	& 0.1161  	&  -7.0\\
		 {Würfel 5,4} 	& 0.572 & 0.006 & {Eisen} 	& 0.5774  	&  -0.9\\
		 {Würfel 5,5} 	& 0.133 & 0.006 & {Delrin} 	& 0.1161  	&  14.6\\
		 {Würfel 5,6} 	& 0.166 & 0.006 & {Delrin} 	& 0.1161  	&  37.8\\
		 {Würfel 5,7} 	& 0.614 & 0.006 & {Eisen} 	& 0.5774  	&  6.3\\
		 {Würfel 5,8} 	& 0.572 & 0.006 & {Eisen} 	& 0.5774  	&  -0.9\\
		 {Würfel 5,9} 	& 0.541 & 0.006 & {Eisen} 	& 0.5774  	&  -6.3\\
		\bottomrule
	\end{tabular}

	\label{tab:Zuordnung3}
\end{table}