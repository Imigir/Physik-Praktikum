\section{Auswertung}
\label{sec:Auswertung}

Die Fehlerrechnung wird mithilfe von Uncertainties \cite{uncertainties} durchgeführt.

\subsection{Bestimmung der Wellenlänge des Lichtes des Lasers}
\begin{table}
	\centering
	\caption{Die gemessene Anzahl von Lichtimpulsen $N$ bei einer Verschiebung des Spiegels um $\frac{\Delta s}{ü}$.}
	\label{tab:tab1}
	\sisetup{table-format=1.2}
	\begin{tabular}{S[table-format=1.2]S[table-format=4.0]}
		\toprule
		{$\Delta s/\si{\milli\meter}$} & {$N$} \\
		\midrule
		5.00 & 3144 \\
		5.00 & 3105 \\
		5.00 & 3076 \\
		5.00 & 2973 \\
		5.00 & 3183 \\
		\bottomrule
	\end{tabular}

	\label{tab:1}
\end{table}
\noindent Aus der gemessenen Anzahl von Lichtimpulsen $N$ bei einer Verschiebung des Spiegels um $\Delta d=\frac{\Delta s}{ü}$ lässt sich nach Formel \eqref{lambda} die Wellenlänge $\lambda$ mit den Werten aus Tabelle \ref{tab:1} berechnen zu:
\[
	\lambda = \frac{2\Delta s}{N ü}=\SI{644(8)}{\nano\meter}\text{.}
\]
Dabei gibt $ü=5,017$ das Übersetzungsverhältnis an.

\subsection{Bestimmung des Brechungsindexes von Luft unter Normalbedingungen}
\begin{table}
	\centering
	\caption{Die gemessene Anzahl von Lichtimpulsen $N$ bei einer Änderung des Druckes in der Messzelle um $\Delta p$ bei einer Temperatur $T$ von ca. $\SI{20}{\degreeCelsius}$.}
	\label{tab:tab2}
	\sisetup{table-format=1.2}
	\begin{tabular}{S[table-format=1.2]S[table-format=2.0]}
		\toprule
		{$\Delta p/\si{\bar}$} & {$N$} \\
		\midrule
		0.80 & 39 \\
		0.80 & 30 \\
		0.80 & 33 \\
		0.80 & 33 \\
		0.80 & 33 \\
		0.80 & 34 \\
		0.80 & 33 \\
		0.80 & 33 \\
		\bottomrule
	\end{tabular}

	\label{tab:2}
\end{table}
\noindent Aus der gemessen Anzahl von Lichtimpulsen $N$ bei einer Änderung des Druckes in der Messzelle der breite $b=\SI{50}{\milli\meter}$ um $\Delta p$ lässt sich nach Formel \eqref{eq:n3} der Brechungsindex $n$ von Luft unter Normalbedingungen mit den Werten aus Tabelle \ref{tab:2} berechnen zu:
\[
	n = 1+ \frac{N \lambda T p_0}{2 b T_0 \Delta p} = \SI{1.000293(8)}{}\text{.}
\]
Dabei sind $T_0=\SI{273,15}{\kelvin}$ und $p_0=\SI{1,0132}{\bar}$. 