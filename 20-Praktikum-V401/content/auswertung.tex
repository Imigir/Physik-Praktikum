\section{Auswertung}
\label{sec:Auswertung}

Die Fehlerrechnung wird mithilfe von Uncertainties \cite{uncertainties} durchgeführt.

\subsection{Bestimmung der Wellenlänge des Lichtes des Lasers}

Aus der gemessenen Anzahl von Lichtimpulsen $N$ bei einer Verschiebung des Spiegels um $\Delta d=\frac{\Delta s}{ü}$ lässt sich nach Formel \eqref{eq:lambda} die Wellenlänge $\lambda$ berechnen zu:
\[
	\lambda = \frac{2\Delta s}{N ü}
\]
Dabei gibt $ü=5,017$ das Übersetzungsverhältnis an.
Mit der Formel für den Mittelwert und den Werten aus Tabelle \ref{tab:1} ergibt sich damit
\[
\bar{\lambda}= \frac{1}{n}\sum_.{i=1}^n \lambda_.i = \SI{644(8)}{\nano\meter}
\]
Der Fehler berechnet sich über die Standardabweichung
\[
\sigma_.{\lambda}=\sqrt{\frac{1}{n^2-n}\sum_.{i=1}^n \left(\lambda_.i-\bar{\lambda}\right)^2}\text{.}
\]

\begin{table}
	\centering
	\caption{Die gemessene Anzahl von Lichtimpulsen $N$ bei einer Verschiebung des Spiegels um $\frac{\Delta s}{ü}$.}
	\label{tab:tab1}
	\sisetup{table-format=1.2}
	\begin{tabular}{S[table-format=5.0] @{${}\pm{}$} S[table-format=3.0]S[table-format=3.0] @{${}\pm{}$} S[table-format=1.0]S[table-format=3.0]S[table-format=1.2]S[table-format=3.1] @{${}\pm{}$} S[table-format=1.1]}
		\toprule
		\multicolumn{2}{c}{$N_.{minute}/\si{1\per\minute}$} & \multicolumn{2}{c}{$N/\si{\becquerel}$} & {$U/\si{\volt}$} & {$I/10^{-6}\si{\ampere}$} & \multicolumn{2}{c}{$\Delta Q/10^{8}\mathrm{e}$} \\
		\midrule
		    0 &   0 &   0 & 0 & 300 & 0.00 & 0.0 & 0.0 \\
		15346 & 124 & 256 & 2 & 310 & 0.05 & 12.2 & 0.1 \\
		16073 & 127 & 268 & 2 & 320 & 0.10 & 23.3 & 0.2 \\
		16196 & 127 & 270 & 2 & 330 & 0.15 & 34.7 & 0.3 \\
		16190 & 127 & 270 & 2 & 340 & 0.20 & 46.3 & 0.4 \\
		16592 & 129 & 277 & 2 & 350 & 0.20 & 45.1 & 0.4 \\
		16365 & 128 & 273 & 2 & 360 & 0.25 & 57.2 & 0.4 \\
		16575 & 129 & 276 & 2 & 370 & 0.30 & 67.8 & 0.5 \\
		16543 & 129 & 276 & 2 & 380 & 0.30 & 67.9 & 0.5 \\
		16877 & 130 & 281 & 2 & 390 & 0.35 & 77.7 & 0.6 \\
		16663 & 129 & 278 & 2 & 400 & 0.40 & 89.9 & 0.7 \\
		16648 & 129 & 277 & 2 & 410 & 0.40 & 90.0 & 0.7 \\
		16491 & 128 & 275 & 2 & 420 & 0.45 & 102.2 & 0.8 \\
		16680 & 129 & 278 & 2 & 430 & 0.45 & 101.0 & 0.8 \\
		16783 & 130 & 280 & 2 & 440 & 0.50 & 111.6 & 0.9 \\
		16932 & 130 & 282 & 2 & 450 & 0.55 & 121.6 & 0.9 \\
		16901 & 130 & 282 & 2 & 460 & 0.60 & 132.9 & 1.0 \\
		16976 & 130 & 283 & 2 & 470 & 0.60 & 132.4 & 1.0 \\
		16856 & 130 & 281 & 2 & 480 & 0.65 & 144.4 & 1.1 \\
		16815 & 130 & 280 & 2 & 490 & 0.65 & 144.8 & 1.1 \\
		16936 & 130 & 282 & 2 & 500 & 0.70 & 154.8 & 1.2 \\
		16970 & 130 & 283 & 2 & 510 & 0.80 & 176.5 & 1.4 \\
		16704 & 129 & 278 & 2 & 520 & 0.80 & 179.4 & 1.4 \\
		16949 & 130 & 282 & 2 & 530 & 0.85 & 187.8 & 1.4 \\
		17096 & 131 & 285 & 2 & 540 & 0.90 & 197.1 & 1.5 \\
		17328 & 132 & 289 & 2 & 550 & 0.95 & 205.3 & 1.6 \\
		16883 & 130 & 281 & 2 & 560 & 0.95 & 210.7 & 1.6 \\
		17153 & 131 & 286 & 2 & 570 & 1.00 & 218.3 & 1.7 \\
		16992 & 130 & 283 & 2 & 580 & 1.05 & 231.4 & 1.8 \\
		17148 & 131 & 286 & 2 & 590 & 1.10 & 240.2 & 1.8 \\
		17111 & 131 & 285 & 2 & 600 & 1.10 & 240.7 & 1.8 \\
		16826 & 130 & 280 & 2 & 610 & 1.10 & 244.8 & 1.9 \\
		17411 & 132 & 290 & 2 & 620 & 1.15 & 247.4 & 1.9 \\
		17223 & 131 & 287 & 2 & 630 & 1.20 & 260.9 & 2.0 \\
		17259 & 131 & 288 & 2 & 640 & 1.25 & 271.2 & 2.1 \\
		17281 & 131 & 288 & 2 & 650 & 1.25 & 270.9 & 2.1 \\
		17549 & 132 & 292 & 2 & 660 & 1.30 & 277.4 & 2.1 \\
		17835 & 134 & 297 & 2 & 670 & 1.40 & 294.0 & 2.2 \\
		17923 & 134 & 299 & 2 & 680 & 1.40 & 292.5 & 2.2 \\
		17801 & 133 & 297 & 2 & 690 & 1.40 & 294.5 & 2.2 \\
		17950 & 134 & 299 & 2 & 700 & 1.50 & 312.9 & 2.3 \\
		\bottomrule
	\end{tabular}

	\label{tab:1}
\end{table}

\subsection{Bestimmung des Brechungsindexes von Luft unter Normalbedingungen}

Aus der gemessen Anzahl von Lichtimpulsen $N$ bei einer Änderung des Druckes in der Messzelle der breite $b=\SI{50}{\milli\meter}$ um $\Delta p$ lässt sich nach Formel \eqref{eq:n3} der Brechungsindex $n$ von Luft unter Normalbedingungen berechnen zu:
\[
	n = 1+ \frac{N \lambda T p_0}{2 b T_0 \Delta p} = \SI{1.000293(8)}{}\text{.}
\]
Dabei sind $T_0=\SI{273,15}{\kelvin}$ und $p_0=\SI{1,0132}{\bar}$. Als $N$ wird der Mittelwert aus der Anzahl der gemessenen Lichtimpulse aus Tabelle \ref{tab:2}
$\bar{N}=33,5 \pm 0,9$ verwendet, der sich wie im vorangegangenen Kapitel berechnet. Für das $\lambda$ wird der errechnete Wert aus dem vorherigen Kapitel genutzt. Für die Temperatur $T$ wird eine Raumtemperatur von $T=\SI{293,15}{\kelvin}$ verwendet.
Der Fehler ergibt sich über die Gaußsche Fehlerfortpflanzung zu
\[
\sigma_.n=\sqrt{\left(\frac{\partial n}{\partial \lambda}\right)^2\cdot\sigma^2_.{\lambda}+\left(\frac{\partial n}{\partial N}\right)^2\cdot\sigma^2_.N}\text{.}
\]
\begin{table}
	\centering
	\caption{Die gemessene Anzahl von Lichtimpulsen $N$ bei einer Änderung des Druckes in der Messzelle um $\Delta p$ bei einer Temperatur $T$ von ca. $\SI{20}{\degreeCelsius}$.}
	\label{tab:tab2}
	\sisetup{table-format=1.2}
	\begin{tabular}{S[table-format=3.0]S[table-format=3.0]S[table-format=2.1]S[table-format=1.2]}
		\toprule
		{$p/\si{\milli\bar}$} & {$N_.2/\si{\becquerel}$} & {$x_.{eff2}/\si{\milli\metre}$} & {$E_2/\si{\mega e\volt}$} \\
		\midrule
		  0 & 3194 & 0.0 & 4.00 \\
		 50 & 3180 & 0.5 & 3.92 \\
		100 & 3169 & 1.0 & 3.86 \\
		150 & 3150 & 1.5 & 3.81 \\
		200 & 3116 & 2.0 & 3.73 \\
		250 & 3104 & 2.5 & 3.69 \\
		300 & 3089 & 3.0 & 3.65 \\
		350 & 3073 & 3.5 & 3.56 \\
		400 & 3061 & 3.9 & 3.49 \\
		450 & 3046 & 4.4 & 3.44 \\
		500 & 3016 & 4.9 & 3.39 \\
		550 & 3008 & 5.4 & 3.34 \\
		600 & 2994 & 5.9 & 3.29 \\
		650 & 2965 & 6.4 & 3.22 \\
		700 & 2930 & 6.9 & 3.16 \\
		750 & 2909 & 7.4 & 3.11 \\
		800 & 2886 & 7.9 & 3.05 \\
		850 & 2844 & 8.4 & 2.99 \\
		900 & 2818 & 8.9 & 2.93 \\
		950 & 2794 & 9.4 & 2.88 \\
		1000 & 2711 & 9.9 & 2.84 \\
		\bottomrule
	\end{tabular}

	\label{tab:2}
\end{table}