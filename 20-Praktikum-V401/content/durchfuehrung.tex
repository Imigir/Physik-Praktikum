\section{Durchführung}
\label{sec:Durchführung}

\subsection{Messung der Wellenlänge eines Helium-Neon-Lasers}
\label{sec:WL}
Als Lichtquelle wird ein Helium-Neon-Laser verwendert.Das Interferenzmuster wird mittels einer Sammellinse ausgeweitet. Anschließend wird der verschiebbare Spiegel mittels eines Synchronmotors in Strahlrichtung bewegt. Die Lichtintensitätsschwankungen, die durch das sich ändernde Interferenzmuster am Photoelement entstehen, werden von diesem in elektrische Impulse umgewandelt und ihre Anzahl von einem Zählwerk gemessen.
\subsection{Messung des Brechungsindex in Luft}
Mittels Vakuumpumpe wird der Druck in der Messzelle um $\Delta p = \SI{0,8}{\bar}$ gesenkt. Anschließend wird durch Öffnen des Ventils der Druck wieder langsam bis zum Normaldruck erhöht und die Anzahl der entstehenden Intensitätsschwankungen gemessen.
%Für die Messung mit einem anderen Gas wird die Messzelle evakuiert und anschließend mittels Füllventil auf einen Druck von etwa $p=\SI{1}{\bar}$ aufgefüllt. Der Vorgang wird mehrmals wiederholt.
%\subsection{Vermessung einer Dublett-Linie}
%Als Lichtquelle wird eine Na-Spektrallampe verwendet. Das Photoelement wird an ein Oszilloskop angeschlossen. Der verschiebbare Spiegel wird wie in Abschnitt \ref{sec:WL} bewegt und der räumliche Abstand zweier Schwebungsminima wird ebenso gemessen wie die mittlere Wellenlänge.
