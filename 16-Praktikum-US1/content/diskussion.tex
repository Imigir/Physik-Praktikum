
\section{Diskussion}
\label{sec:Diskussion}

\begin{table}
	\centering
	\caption{Die in der Auswertung bestimmten Werte mit den zugehörigen Referenzwerten und Abweichungen.}
	\label{tab:Ergebnisse}
	\sisetup{table-format=1.2}
	\begin{tabular}{c ccc}
		\toprule
		{Wert}&{gemessen}&{Referenzwert\cite{cAcryl},\cite{alphaAcryl}}&{Abweichung} \\
		\midrule
		$l_.{Tiefenmessung}$ & \SI{120.04}\,\si{\milli\meter} & \SI{120.5}\,\si{\milli\meter} & \SI{-0.4}\,\si{\percent} \\
		$c_\text{1}$ & \SI{2708\pm17}\,\si{\meter\per\second} & \SI{2730}\,\si{\meter\per\second} & \SI{-0.8}\,\si{\percent} \\
		$c_\text{2}$ & \SI{2700\pm120}\,\si{\meter\per\second} & \SI{2730}\,\si{\meter\per\second} & \SI{-1.1}\,\si{\percent} \\
		$\alpha$ & \SI{61\pm17}\,\si{\per\meter} & \SI{57\pm2}\,\si{\per\meter} & \SI{7}\,\si{\percent} \\
		$\alpha_{I1}$ & \SI{8\pm2}\,\si{\per\meter} & \SI{57\pm3}\,\si{\per\meter} & \SI{-86}\,\si{\percent} \\
		$\alpha_{I2}$ & \SI{61\pm2}\,\si{\per\meter} & \SI{57\pm2}\,\si{\per\meter} & \SI{7}\,\si{\percent} \\
		$D_{1,1}$ & \SI{}\,\si{\milli\meter} & \SI{6.0}\,\si{\milli\meter} & \SI{}\,\si{\percent} \\
		$D_{1,2}$ & \SI{}\,\si{\milli\meter} & \SI{9.8}\,\si{\milli\meter} & \SI{}\,\si{\percent} \\
		$D_{2,1}$ & \SI{}\,\si{\milli\meter} & \SI{6.0}\,\si{\milli\meter} & \SI{}\,\si{\percent} \\
		$D_{2,2}$ & \SI{}\,\si{\milli\meter} & \SI{9.8}\,\si{\milli\meter} & \SI{}\,\si{\percent} \\
		$A_1$ & \SI{7.33}\,\si{\milli\meter} & - & - \\
		$A_2$ & \SI{4.09}\,\si{\milli\meter} & - & - \\
		$A_3$ & \SI{8.90}\,\si{\milli\meter} & - & - \\
		$A_4$ & \SI{35.3}\,\si{\milli\meter} & - & - \\
		\bottomrule
	\end{tabular}

	\label{tab:Ergebnisse}
\end{table}

\noindent In Tabelle \ref{tab:Ergebnisse} ist zu erkennen, dass die bestimmten Geschwindigkeiten $c_1$ und $c_2$ nah am Literaturwert liegen. Dabei ist die Abweichung, sowie der relative Fehler beim Durchschallungs-Verfahren größer als bei dem Impuls-Echo-Verfahren. Dies ist durch die halb so lange Strecke zu erklären, da Abweichungen in der Zeitmessung bei geringerer Zeit eine größere Rolle spielen. Die bei der Tiefenmessung bestimmte Länge des Zylinders stimmt mit der durch die Schieblehre bestimmten Länge überein.\\
Bei der Bestimmung des Dämpfungskonstanten $\alpha$ sind die schlechten Messwerte auffallend. Diese sind wahrscheinlich durch eine falsche Einstellung des Messgerätes zustande gekommen. In Abbildung \ref{fig:Daempfung} sind zwei mögliche Interpretationen der Messergebnisse zu sehen. Werden die Ergebnisse dieser mit dem Literaturwert von $\SI{57(3)}{\per\metre}$ verglichen, scheint die zweite Interpretation mit $\alpha_2=\SI{61(2)}{\per\metre}$ die naheliegende zu sein, obwohl mehr Punkte auf der ersten Ausgleichsgeraden liegen. Dies lässt vermuten, dass die Messung im niedrigen Bereich zu starke Spannungen geliefert hat, was möglicherweise auf eine falsche Einstellung der Verstärkung hindeutet.\\
Die Messung der Dicken der Acrylplatten ist hingegen sehr genau, da hier die berechneten Werte aus dem Cepstrum und dem Spektrum kaum von den gemessenen Werten mit der Schieblehre abweichen. Dabei werden die Werte für die Zeitdifferenzen für das Cepstrum aus Abbildung \ref{fig:Cepstrum} entnommen. Die Abstände für das Auge scheinen ebenfalls in der richtigen Größenordnung zu liegen, sind jedoch ohne Standardwerte nicht weiter zu überprüfen.     