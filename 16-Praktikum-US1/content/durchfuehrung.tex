\section{Durchführung}
\label{sec:Durchführung}

Als Koppelungsmittel wird bei allen Versuchen nach dem Impuls-Echo-Verfahren bidestilliertes Wasser verwendet. Bei der Durchschallungs-Methode wird Koppelgel verwendet.

\subsection{Vorbereitung}
Es werden die Höhen der vorliegenden Acrylzylinder mit einer Schieblehre vermessen. Bei einem Acrylzylinder wird oben eine $\SI{2}{\mega\hertz}$-Sonde gekoppelt.
Aus den gemessenen Daten eines A-Scan wird die Schallgeschwindigkeit $c$ im Material bestimmt, ins Messprogramm eingetragen und eine Tiefenmessung durchgeführt.

\subsection{Messung der Dämpfung mittels Impuls-Echo-Verfahren}
Eine $\SI{2}{\mega\hertz}$-Sonde wird oben an einen Acrylinder gekoppelt und mit einem Echoskop im \textit{REFLEC.}-Modus verbunden, welches die auszuwertenden Daten an einen angeschlossenen Computer übermittelt.
Im Auswertungsprogramm wird die Differenz zwischen den Amplituden des gesendeten und des reflektierten Impulses gemessen. 
Die Messung wird für sechs weitere Zylinder wiederholt.

\subsection{Messung zur Bestimmung der Schallgeschwindigkeit im Medium}
Die Schallgeschwindigkeit wird mit dem Impuls-Echo- und dem Durchschallungs-Verfahren bestimmt und die Ergebnisse werden verglichen.
Bei dem Impuls-Echo-Verfahren wird eine $\SI{2}{\mega\hertz}$-Sonde oben an einen Zylinder gekoppelt und mit einem Echoskop im \textit{REFLEC.}-Modus verbunden.
Es wird ein A-Scan durchgeführt und im Auswertungsprogramm die Zeitdifferenz zwischen dem Sende- und Echoimpuls gemessen. Die Messung wird für sieben weitere Acrylzylinder wiederholt.\\
Im Durchschallungs-Verfahren wird ein Zylinder horizontal in eine Halterung gelegt und an beiden Stirnseiten werden $\SI{2}{\mega\hertz}$-Sonden gekoppelt. Sie werden an das Echoskop im \textit{TRANS.}-Modus angeschlossen und es wird im Auswertungsprogramm die Zeitdifferenz der Impulse gemessen.
Die Messung wird für vier weitere Zylinder wiederholt.

\subsection{Spektrale Analyse und Cepstrum}
Zwei Acrylscheiben und ein $\SI{4e-2}{\metre}$ langer Acrylzylinder werden gekoppelt. Oben an den Zylinder wird eine $\SI{2}{\mega\hertz}$-Sonde gekoppelt.
Die Verstärkung am Echoskop im \textit{REFLEC.}-Modus wird so eingestellt, dass im Auswertungsprogramm drei Mehrfachreflexionen zu sehen sind. Die zeitlichen Abstände zwischen den Echos werden ermittelt. Mit der FFT-Funktion wird das Spektrum und das Cepstrum der Sonde aufgezeichnet.

\subsection{Untersuchung eines Augenmodels}
Eine $\SI{2}{\mega\hertz}$-Sonde wird an die Hornhaut eines Augenmodels so gekoppelt, dass eine Reflexion an der Retina zu sehen ist. Mit einem A-Scan werden die Echos an Hornhaut, Iris, Linse und Retina aufgenommen und die Entfernungen bestimmt