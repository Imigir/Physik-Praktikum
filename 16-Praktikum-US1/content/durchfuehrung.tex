\section{Durchführung}
\label{sec:Durchführung}

Als Koppelgel wird bei allen Versuchen bidestilliertes Wasser verwendet.

\subsection{Messung der Dämpfung mittels Impuls-Echo-Verfahren}
Eine $\SI{2e6}{\hertz}$-Sonde wird oben an einen Acrylinder gekoppelt und mit einem Echoskop im \textit{REFLEC.}-Modus verbunden, welches die auszuwertenden Daten an einen angeschlossenen Computer übermittelt.
Im Auswertungsprogramm werden die Amplituden des gesendeten und des reflektierten Impulses gemessen. 
Die Messung wird für sechs weitere Zylinder wiederholt.

\subsection{Messung zur Bestimmung der Schallgeschwindigkeit im Medium}
\subsubsection{Impuls-Echo-Verfahren}
Die Höhe eines Acrylzylinders wird vermessen. Eine $\SI{2e6}{\hertz}$-Sonde wird oben an den Zylinder gekoppelt und mit einem Echoskop im \textit{REFLEC.}-Modus verbunden.
Es wird ein A-Scan durchgeführt und im Auswertungsprogramm die Zeitdifferenz zwischen dem Sende- und Echoimpuls gemessen.
Die Messung wird für sieben weitere Acrylzylinder wiederholt.

\subsubsection{Durchschallungs-Verfahren}
Der Zylinder wird horizontal in eine Halterung gelegt und an beiden Stirnseiten werden $\SI{2e6}{\hertz}$-Sonden gekoppelt. Sie werden an das Echoskop im \textit{TRANS.}-Modus angeschlossen und es wird im Auswertungsprogramm die Zeitdifferenz der Impulse gemessen.
Die Messung wird für sieben weitere Zylinder wiederholt.

\subsection{Spektrale Analyse und Cepstrum}
Zwei Acrylscheiben und ein $\SI{4e-2}{\metre}$ langer Acrylzylinder werden gekoppelt. Oben an den Zylinder wird eine $\SI{2e6}{\hertz}$-Sonde gekoppelt.
Die Verstärkung am Echoskop im \textit{REFLEC.}-Modus wird so eingestellt, dass im Auswertungsprogramm drei Mehrfachreflexionen zu sehen sind. Mit der FFT-Funktion wird das Spektrum und das Cepstrum der Sonde aufgezeichnet.

\subsection{Untersuchung eines Augenmodels}
Eine $\SI{2e6}{\hertz}$-Sonde wird an die Hornhaut eines Augenmodels so gekoppelt, dass eine Reflexion an der Retina zu sehen ist. Mit einem A-Scan werden die Echos an Iris und Retina aufgenommen.