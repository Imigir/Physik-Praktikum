\section{Theorie}
\label{sec:Theorie}

\subsection{Ultraschall}
\subsubsection{Ausbreitung im Medium}
Als Ultraschall wird der Schall im Frequenzbereich von $\SI{2e4}{\hertz}$ bis $\SI{e9}{\hertz}$.
In Gasen und Flüssigkeiten pflanzen sich Schallwellen longitudinal durch Druckschwankungen fort, welche als 
\[
p(x,t)=p_.0+v_.0 Z \cos(\omega t - kx)
\]
geschrieben werden können. Dabei wird $Z=c\rho$, mit der Schallgeschwindigkeit im Medium $c$ und der Dichte $\rho$ des Mediums, akustische Impedanz genannt, die dazu führt, dass die Ausbreitung des Schalls abhängig vom Medium ist.
Die Schallgeschwindigkeit hängt ebenfalls von der Dichte ab und wird bei Flüssigkeiten mit der Kompressibilität $\kappa$ berechnet über
\begin{equation}
c_.{Fl}=\sqrt{\frac{1}{\kappa\rho}}\label{eq:Fl}\text{.}
\end{equation}
In Festkörpern hingegen können auf Grund von Schubspannungen auch transversale Wellen auftreten, weshalb $c$ hier über den Elastizitätsmodul definiert ist:
\begin{equation}
c_.{Fe}=\sqrt{\frac{E}{\rho}}\label{eq:Fe}
\end{equation}
\subsubsection{Reflexion,Transmission und Absorption}
Die Intensität einer Schallwelle nimmt exponentiell mit dem Abstand ab:
\begin{equation}
I(x)=I_.0\cdot\mathrm{e}^{-\alpha x}\label{eq:I}
\end{equation}
Der Absorptionskoeffizient $\alpha$ ist dabei vom Medium abhängig.
Beim Übergang von einem Medium in ein anderes wird der Teil
\begin{equation}
R=\left(\frac{Z_.1-Z_.2}{Z_.1+Z_.2}\right)^2\label{eq:R}
\end{equation}
der einfallenden Intensität reflektiert. Der Rest, also
\begin{equation}
T=1-R\label{eq:T}\text{,}
\end{equation}
wird transmittiert. $T$ und $R$ werden dabei Transmissions- und Reflexionskoeffizient genannt.

\subsubsection{Erzeugung und Anwendung}
Erzeugt wird Ultraschall durch Einbringen eines piezoelektrischen Kristalls in ein wechselstrombetriebenes E-Feld. Ein Einstellen auf die Resonanzfrequenz führt zu großen Schwingungsamplituden.\newline
Verwendung finden Ultraschall-Untersuchungen in der Medizin um Informationen über das Körperinnere zu erhalten ohne ihn zu beschädigen.\newline
Dabei ist die eine Verfahrensweise die Durchschallungs-Methode bei der Sender und Empfänger auf gegenüberliegenden Seiten der Probe. Der Sender erzeugt einen kurzen Schallimpuls der vom Empfänger erhalten wird. Sind Fehlstellen im Körper ist die Intensität des empfangenen Impulses geringer, eine genau Lokalisierung und Beschaffenheitsanalyse der Fehlstellen ist jedoch nicht möglich.\newline
Die zweite Möglichkeit ist das Impuls-Echo-Verfahren.
Hier ist der Sender auch gleichzeitig der Empfänger. Hier wird der Impuls ausgesendet hinter dem Körper reflektiert und nach dem zweiten Durchlaufen des Körpers empfangen.
Hierbei lassen sich Lage und Größe einer Fehlstelle über 
\begin{equation}
s=\frac{1}{2}c t
\end{equation}
bestimmen, wobei $c$ die Schallgeschwindigkeit und $t$ die Laufzeit ist.