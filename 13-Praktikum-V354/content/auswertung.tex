\section{Auswertung}
\label{sec:Auswertung}


Die Graphen wurden sowohl mit Matplotlib \cite{matplotlib} als auch NumPy \cite{numpy} erstellt. Die
Fehlerrechnung wurde mithilfe von Uncertainties \cite{uncertainties} durchgeführt.

\subsection{Apparatekonstanten}
Für die Messreihen wurde Gerät 2 verwendet.
Die Werte für Induktivität $L$ der Spule, die Kapazität $C$ und den verwendeten Widerstand $R_.1$ befinden sich in Tabelle \ref{tab:tab1}

\begin{table}
	\centering
	\caption{Apparatekonstanten}
\label{tab:tab1}
	\sisetup{table-format=1.2}
	\begin{tabular}{S[table-format=1.2]S[table-format=1.3]S[table-format=2.1]}
		\toprule
		{$L/ 10^3 \si{\henry}$} & {$C/10^{-9}\si{\farad}$} & {$R/\si{\ohm}$} \\
		\midrule
		${10,11\pm 0,03}$ & ${2,098\pm 0,006}$ & ${48,1\pm 0,1}$ \\
		\bottomrule
	\end{tabular}
\end{table}
\subsection{Abklingzeit und effektiver Dämpfungswiderstand}
In Tabelle \ref{tab:tab2} sind die Messwerte des in Abbildung \ref{fig:abb1} sichtbaren Graphen zu sehen.\newline
Als Generatorspannung wurde eine Rechteckspannung von $U_.G=20V$ verwendet.
\begin{table}
	\centering
	\caption{Apparatekonstanten}
\label{tab:tab2}
	\sisetup{table-format=1.2}
	\begin{tabular}{S[table-format=3.0]S[table-format=2.1]}
		\toprule
		{$t/10^{-6}\si{\second}$} & {$U_.C/\si{\volt}$} \\
		\midrule
		0 & 19.4 \\
		30 & 16.0 \\
		60 & 13.4 \\
		90 & 11.2 \\
		120 & 9.6 \\
		150 & 8.0 \\
		180 & 7.0 \\
		210 & 5.8 \\
		240 & 4.8 \\
		270 & 4.0 \\
		300 & 3.6 \\
		330 & 3.0 \\
		360 & 2.6 \\
		390 & 2.2 \\

		\bottomrule
	\end{tabular}
\end{table}
\begin{figure}
\centering
\includegraphics[scale=0.6]{content/images/Grapha.pdf}
\caption{Messdaten eines Abklingvorgangs eines RLC-Schwingkreises}
\label{fig:abb1}
\end{figure}
\newpage
\noindent Die Regression $U(t)=A_.0e^{-2\pi\mu t}$ liefert:
\begin{align}
A_.0 &= \SI{19,11(12)}{\volt} \\
\mu &= \SI{909(9)}{\hertz}
\end{align}
Nach Gleichung \eqref{eq:gamma} und \eqref{eq:tau} ergibt sich mit $\gamma=2\pi\mu$:
\begin{align}
R_.{eff}&=\SI{115(1)}{\ohm} \\
\tau &=\SI{175(1)e-6}{\second}
\end{align}
Der Fehler von $R_.{eff}$ errechnet sich dabei aus der Gaußschen Fehlerfortpflanzung
\[
\sigma_.R=\sqrt{\left(4\pi L\sigma_.{\mu}\right)^2 +\left(4\pi\mu\sigma_.L\right)^2)},
\]
wobei $\sigma_.L$ aus Tabelle \ref{tab:tab1} abzulesen und $\sigma_.{\mu}$ durch die Regression bestimmt wurde.
Analog dazu errechnet sich der Fehler der Abklingzeit $\tau$ zu
\[
\sigma_.{\tau}=\frac{\sigma_.{\mu}}{2\pi\mu^2}
\]
Der berechnete effektive Widerstand $R_.{eff}$ weicht von dem geschalteten Widerstand $R_.1$ um $\Delta R=140\%$ ab.
\subsection{Aperiodischer Grenzfall}
Der Widerstand für den die Schwingung in den aperiodischen Grenzfall übergeht wurde zu
\[
R_.{ap,exp}=\SI{3,5e3}{\ohm}\text{.}
\]
Der aus Gleichung \eqref{eq:} berechnete Wert beträgt
\[
R_.{ap,theo}=\SI{4,390(9)e3}{\ohm},
\]
wobei sich der Fehler aus
\[
\sigma_.{R}=\sqrt{\frac{\sigma_.L^2}{CL}+\frac{\left(L\sigma_.C\right)^2}{C^3L}}
\]
berechnet.\newline
Die Abweichung beträgt somit
\[
\Delta R_.{ap}=25,4\%
\]
\subsection{Frequenzabhängigkeit der Kondensatorspannung}
\subsubsection{Frequenzabhängigkeit der Amplitude}
Die in Tabelle \ref{tab:tab3} festgehaltenen Werte der eingestellten Frequenz $\nu$ und der gemessenen Kondensatorspannung $U_.C$ sowie ihres Verhältnisses zur Generatorspannung $U_.G=\SI{9,8}{\volt}$
wurden in Abbildung \ref{fig:abb2} halblogarithmisch gegeneinander aufgetragen.\newline
In Abbildung \ref{fig:abb3} wurde der Ausschnitt um die Resonanzfrequenz herum linear dargestellt, um die Schärfe $\nu_.+-\nu_.-$ der Frequenz zu bestimmen.
\begin{table}
	\centering
	\caption{Messwerte zur Frequenzabhängigkeit der Kondensatorspannung}
\label{tab:tab3}
	\sisetup{table-format=1.2}
	\begin{tabular}{S[table-format=2.1]S[table-format=3.1]S[table-format=2.2]}
		\toprule
		{$\omega/10^3\si{\second}$} & {$U_.C/\si{\volt}$} & {$\frac{U_.C}{U_.G}/\si{}$}\\
		\midrule
		15 & 12.2 & 1.24 \\
		20 & 15.0 & 1.53 \\
		25 & 21.6 & 2.20 \\
		30 & 47.2 & 4.82 \\
		31 & 63.2 & 6.45 \\
		32 & 95.2 & 9.71 \\
		32.5 & 122 & 12.45 \\
		33 & 150 & 15.31 \\
		33.3 & 157 & 16.02 \\
		33.6 & 150 & 15.31 \\
		34.1 & 122 & 12.45 \\
		34.6 & 96.8 & 9.88 \\
		35.6 & 63.2 & 6.45 \\
		36.6 & 45.6 & 4.65 \\
		41.6 & 17.4 & 1.78 \\
		46.6 & 10.4 & 1.06 \\
		50 & 8.0 & 0.82 \\
		\bottomrule
	\end{tabular}
\end{table}
\begin{figure}
\centering
\includegraphics[scale=0.8]{content/images/Graphc1.pdf}
\caption{halblogarithmische Darstellung des Spannungsverhältnisses $\frac{U_.C}{U_.G}$ in Abhängigkeit von der Frequenz $\nu$}\label{fig:abb2}
\includegraphics[scale=0.8]{content/images/Graphc2.pdf}
\caption{Lineare Darstellung in der Nähe der Resonanzfrequenz}
\label{fig:abb3}
\end{figure}
\newpage
\noindent
Mittels der Regression 
\[
\frac{U_.0}{U_.G}(\nu)=\frac{1}{\sqrt{\left(1-\left(2\pi\nu a\right)^2\right)^2+\left(2\pi\nu b\right)^2}}
\]
ergibt sich für die Parameter
\begin{align*}
a^2&=\SI{2,276(2)e-11}{\henry\farad} \\
b&=\SI{3,01(2)e-7}{\ohm\farad}
\end{align*}
Die gemessene Güte $q_.{exp}$ lässt sich damit aus $q=\frac{a}{b}$ berechnen:
\[
q_.{exp}=15,84\pm 0,08
\]
Der Fehler wird dabei durch
\[
\sigma_.{q,exp}=\frac{\sqrt{\frac{\sigma_.a^2}{4a}+\frac{a\sigma_.b^2}{b^2}}}{b}
\]
berechnet
Aus Gleichung \eqref{eq:} lässt sich die theoretische Güte 
\[
q_.{theo}=19,01\pm 0,20
\] 
berechnen.
Der Fehler ist dabei durch 
\[
\sigma_.{q,theo}=\sqrt{\frac{\sigma_.R^2L}{CR^4}+\frac{\sigma_.C^2L^2}{C^34R^2L}+\frac{\sigma_.L^2}{4R^2CL}}
\]
gegeben.
Die Abweichung zwischen Theorie und Experiment beträgt
\[
\Delta q = 16,7\%
\]
Die experimentelle Breite der Resonanzkurve ergibt sich mit der Gleichung 
\[
\nu_.{\pm}=\pm\frac{b}{2a^2}+\sqrt{\left(\frac{b}{2a^2}\right)^2+\frac{1}{a^2}}
\]
und damit $\nu_.-=\SI{3,2319(13)e4}{\hertz}$ und $\nu_.+=\SI{3,4425(14)e4}{\hertz}$:
\[
\left(\nu_.+-\nu_.-\right)_.{exp}=\SI{2,106(11)e3}{\hertz}
\]
Der Fehler berechnet sich jeweils über
\begin{align*}
\sigma_.{\nu_.+}&=\sqrt{\sigma_.a^2\left(\frac{\frac{2}{a^3}+\frac{b^2}{a^5}}{2\sqrt{\frac{1}{a^2}+\frac{b^2}{4a^2}}}+\frac{b}{a^3}\right)^2+\sigma_.b^2\left(\frac{b}{4a^2\sqrt{\frac{1}{a^2}+\frac{b^2}{4a^2}}}+\frac{1}{2a^2}\right)^2} \\
\sigma_.{\nu_.-}&=\sqrt{\sigma_.a^2\left(-\frac{\frac{2}{a^3}+\frac{b^2}{a^5}}{2\sqrt{\frac{1}{a^2}+\frac{b^2}{4a^2}}}+\frac{b}{a^3}\right)^2+\sigma_.b^2\left(\frac{b}{4a^2\sqrt{\frac{1}{a^2}+\frac{b^2}{4a^2}}}-\frac{1}{2a^2}\right)^2} \\
\sigma_.{\nu_.+-\nu_.-}&=\sqrt{\sigma_.{\nu_.+}^2+\sigma_.{\nu_.-}^2}\text{.}
\end{align*}
Die Fehler $\sigma_.a$ und $\sigma_.b$ wurden durch die Regression bestimmt.
Mit den aus Gleichung \eqref{eq:} berechneten $\nu_.-=\SI{3.366(7)e4}{\hertz}$ und 
$\nu_.+=\SI{3,548(7)e4}{\hertz}$ beträgt
\[
\left(\nu_.+-\nu_.-\right)_.{theo}=\SI{1,818(20)e3}{\hertz}\text{.}
\]
Der Fehler berechnet sich jeweils über

\begin{align*}
\sigma_.{\nu_.+}&=\sqrt{\sigma_.R^2\left(\frac{R}{4L^2\sqrt{\frac{R^2}{4L^2}+\frac{1}{CL}}}+\frac{1}{2L}\right)^2+\sigma_.C^2\left(\frac{1}{C^2\sqrt{\frac{R^2}{4L^2}+\frac{1}{CL}}}\right)^2+\sigma_.L^2\left(\frac{\frac{1}{CL^2}+\frac{R^2}{2L^3}}{2\sqrt{\frac{R^2}{4L^2}+\frac{1}{CL}}}+\frac{R}{2L^2}\right)^2} \\
\sigma_.{\nu_.+}&=\sqrt{\sigma_.R^2\left(\frac{R}{4L^2\sqrt{\frac{R^2}{4L^2}+\frac{1}{CL}}}-\frac{1}{2L}\right)^2+\sigma_.C^2\left(\frac{1}{C^2\sqrt{\frac{R^2}{4L^2}+\frac{1}{CL}}}\right)^2+\sigma_.L^2\left(-\frac{\frac{1}{CL^2}+\frac{R^2}{2L^3}}{2\sqrt{\frac{R^2}{4L^2}+\frac{1}{CL}}}+\frac{R}{2L^2}\right)^2} \\
\sigma_.{\nu_.+-\nu_.-}&=\sqrt{\sigma_.{\nu_.+}^2+\sigma_.{\nu_.-}^2}\text{.}
\end{align*}
Die Fehler $\sigma_.R$, $\sigma_.L$ und $\sigma_.C$ lassen sich dabei aus Tabelle \ref{fig:abb1} entnehmen.
Die Abweichung von experimentellem und theoretischem Wert beträgt
\[
\Delta\left(\nu_.+-\nu_.-\right)=15,8\%
\]

\subsubsection{Frequenzabhängigkeit der Phase}
Die Werte des in Abbildung \ref{fig:abb4} und \ref{fig:abb5} sichtbaren Graphen lassen sich aus Tabelle \ref{tab:tab4} ablesen.
\begin{table}
	\centering
	\caption{Messwerte zur Frequenzabhängigkeit der Kondensatorspannung}
\label{tab:tab4}
	\sisetup{table-format=1.2}
	\begin{tabular}{S[table-format=2.1]S[table-format=2.1]S[table-format=3.2]}
		\toprule
		{$\omega/10^3\si{\second}$} & {$a/10^{-6}\si{\second}$} & {$\phi/\si{\degree}$}\\
		\midrule
		15 & 0,6 & 3,24 \\
		20 & 0,6 & 4,32 \\
		25 & 0,6 & 5,40 \\
		30 & 1,0 & 10,80 \\
		31 & 1,2 & 13,39 \\
		32 & 2,2 & 25,34 \\
		32.5 & 3,0 & 35,10 \\
		33 & 5,0 & 59,40 \\
		33.3 & 7,2 & 86,31 \\
		33.6 & 9,2 & 111,28 \\
		34.1 & 11,2 & 137,49 \\
		34.6 & 12,4 & 154,45 \\
		35.6 & 13,0 & 166,61 \\
		36.6 & 13,0 & 171,29 \\
		41.6 & 11,6 & 173,72 \\
		46.6 & 10,6 & 177,82 \\
		50 & 9,8 & 176,40 \\
		\bottomrule
	\end{tabular}
\end{table}
\begin{figure}
\centering
\includegraphics[scale=0.8]{content/images/Graphd1.pdf}
\caption{Verlauf der Phasenverschiebung in Abhängigkeit von der Frequenz}
\label{fig:abb4}
\includegraphics[scale=0.8]{content/images/Graphd2.pdf}
\caption{Verlauf der Phasenverschiebung in der Nähe der Resonanzfrequenz}
\label{fig:abb5}
\end{figure}
Die Regression 
\[
\phi(\nu)=arctan\left(\frac{a}{1-b^2x^2}\right)
\]
liefert
\begin{align*}
a&=\SI{1,77(5)e-7}{\ohm\farad} \\
b^2&=\SI{2,277(2)e-11}{\henry\farad}\text{.}
\end{align*}
Mit der Formel
\[
\nu_.{res}=\sqrt{\frac{1}{b^2}-\frac{a}{4\pi b^4}}
\]
ergibt sich 
\[
\nu_.{res,exp}=\SI{3,3342(11)e4}{\hertz},
\]
wobei sich mit den durch die Regression bestimmten Fehlern $\sigma_.a$ und $\sigma_.b$ der Fehler aus
\[
\sigma_.{\nu_.{res},exp}=\sqrt{\sigma_.a^2\left(\frac{1}{8\pi b^4\sqrt{\frac{1}{b^2}-\frac{a}{4\pi b^4}}}\right)^2+\sigma_.b^2\left(\frac{\frac{a}{\pi b^5}-\frac{2}{b^3}}{2\sqrt{\frac{1}{b^2}-\frac{a}{4\pi b^4}}}\right)^2}
\]
berechnet.
Verglichen mit dem Theoriewert, der sich nach Gleichung \eqref{eq:} ergibt
\[
\nu_.{res,theo}=\SI{3,453(7)e4}{\hertz}
\]
mit dem Fehler
\[
\sigma_.{\nu_.{res},theo}=\sqrt{\sigma_.R^2\left(\frac{R}{4\pi L^2\sqrt{\frac{1}{CL}-\frac{R^2}{4\pi L^2}}}\right)^2+\sigma_.C^2\left(\frac{1}{2C^2\sqrt{\frac{1}{CL}-\frac{R^2}{4\pi L^2}}}\right)^2+\sigma_.L\left(\frac{\frac{R^2}{2\pi L^3}-\frac{1}{CL^2}}{2\sqrt{\frac{1}{CL}-\frac{R^2}{4\pi L^2}}}\right)^2}
\]
eine Abweichung von
\[
\Delta\nu_.{res}=3,45\%
\]
Die Werte für $\nu_.1$ und $\nu_.2$ lassen sich analog berechnen zu
\begin{table}
	\centering
	\caption{Frequenzen}
\label{tab:tab6}
	\sisetup{table-format=1.2}
	\begin{tabular}{cS[table-format=2.2] @{${}\pm{}$} S[table-format=1.2]S[table-format=2.2] @{${}\pm{}$} S[table-format=1.2]S[table-format=1.2]}
		\toprule
		{} & \multicolumn{2}{c}{$\nu_\text{exp}/ 10^3 \si{\hertz}$} & \multicolumn{2}{c}{$\nu_\text{theo}/10^3\si{\hertz}$} & {$\Delta\nu$} \\
		\midrule
		$\nu_.1$ & 32,74 & 0,02 & 33,66 & 0,07 & 2,73\% \\
		$\nu_.2$ & 33,98 & 0,02 & 35,48 & 0,07 & 4,23\% \\
		\bottomrule
	\end{tabular}
\end{table}

%\begin{table}
%	\centering
%	\caption{Frequenzen}
%	\label{tab:tab6}
%	\sisetup{table-format=1.2}
%\end{table}