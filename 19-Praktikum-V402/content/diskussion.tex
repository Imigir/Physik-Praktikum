
\section{Diskussion}
\label{sec:Diskussion}
Der brechende Winkel $\phi$ eines Prismas ist $phi=\SI{60}{\degree}$. Davon weicht der aus den Messungen ermittelte Mittelwert $\bar{\phi}=\SI{59,99(2)}{\degree}$ um $\delta\phi=0,16\%$ ab. Der wenn auch geringe Fehler kommt daher das per Augenmaß gemessen wurde. Der Brechungsindex im Bereich $1,75-1,81$liegt und die Abbe-Zahl $\nu=24\pm 11$, weisen darauf hin, dass das verwendete Prisma wahrscheinlich aus Schwerflint N-SF10 besteht, dessen Abbe-Zahl bei $\nu_.{SF10}=28$ um $\delta\nu\approx 15\%$ vom berechneten abweichen.
Anhand der Graphen \ref{fig:reg2} bis \ref{fig:reg'} lässt sich erkennen das normale Dispersion stattfindet, da der Brechungsindex $n$ mit zunehmender Wellenlänge $\lambda$ abnimmt. Allerdings wird ebenfalls an diesen Graphen deutlich, dass Gleichung \eqref{eq:n^2 für l<<l1} den Abfall wesentlich schlechter charakterisiert, als Gleichung \eqref{eq:n^2 für l>>l1}. Dies wird auch an den Abweichungsquadraten deutlich, da
$s^2_.{n_.2}=1,541\cdot 10^{-6}$ bzw. $s^2_.{n_.4}=1,536\cdot 10^{-5}$ um einen Faktor 50 kleiner sind, als $s^2_.{n'}=5,041\cdot 10^{-4}$.
Die berechnete Absorptionsstelle $\lambda_.1=\SI{176,5(15)e-9}{\metre}$ liegt im kurzwelligen ultravioletten Bereich und ist somit weit genug vom sichtbaren Spektrum entfernt, um dort nicht durch anormale Dispersion aufzufallen.
