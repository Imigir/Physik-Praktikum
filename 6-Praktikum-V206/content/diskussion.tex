
\section{Diskussion}
\label{sec:Diskussion}

Die realen Werte der Güteziffer weichen von den berechneten idealen Werten stark nach unten ab. Die Abweichungen liegen zwischen $-74 \%$ und $-90 \%$. Dabei weichen die Werte bei geringerer Zeit und somit bei niedriger Temperaturdifferenz stärker ab, als bei hoher Temperaturdifferenz. Dies lässt sich mit Formel \eqref{eq:nyideal} erklären, da die ideale Güte bei hohen Temperaturdifferenzen stark absinkt. Das es überhaupt zu den Abweichungen kommt liegt an der Annahme, dass keine äußere Wärmemenge $\Delta Q$ dem System hinzugefügt oder entnommen wird. Durch die mangelnde beziehungsweise nicht optimale Dichtung ist dies jedoch nicht gegeben. Die Massendurchsätze fallen ebenfalls gering aus. Dies könnte an einem schlechten Wirkungsgrad des Kompressors, oder an einem geringen L liegen.
Der schlechte Wirkungsgrad des Kompressors wird anhand der bestimmten Leistungen bestätigt.  Sie betragen knapp $3 \%$ der abgelesenen Leistungen. Eine weitere Fehlerquelle ist das Ablesen, da dies insgesamt etwa 10 Sekunden gedauert hat und somit die Messdaten ungenau sind. 