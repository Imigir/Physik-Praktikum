
\section{Diskussion}
\label{sec:Diskussion}

Vergleicht man die ermittelten Dichten der Stäbe mit dem Literaturwert von Aluminium ($\SI{2,7e3}{\kilogram\per\cubic\metre}$), so wird bestätigt, dass es sich bei den Stäben um Aluminiumstäbe handelt.
Mittelt man die Ergebnisse für $E$, so kommt man auf einen Wert von:
\[
	E=\SI{6.85(3)e10}{\pascal}
\]
Dies entspricht in etwa dem Literaturwert von $\SI{7e10}{\pascal}$.
Dabei ist der durch beidseitige Auflage des quadratischen Stabs bestimmte Wert mit $E = \SI{7,01(4)e10}{\pascal}$ beinahe identisch mit diesem Literaturwert, was darauf schließen lässt, das dieser Versuchsaufbau der stabilere ist.\newline
Die während des Versuchs festgestellte Empfindlichkeit der Messuhren, die selbst bei geringeren Erschütterung zu veränderten Messsergebnissen führte, kam hier nicht zur Geltung.