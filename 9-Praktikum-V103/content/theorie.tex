\section{Theorie}
\label{sec:Theorie}

\subsection{Das Elastizitätsmodul}

Wirkt auf die Oberfläche eines elastischen Körpers eine Kraft $F$ so kommt es zu Verformungen $\Delta L$.
Das Verhältnis von Verformung zu Ausgangslänge $\frac{\Delta L}{L}$ wird als relative Längenänderung bezeichnet.
Die Kraft lässt sich dann in die Normalspannung $\sigma$, die senkrecht zur Oberfläche angreift, und die Tangentialspannung, die entlang der Oberfläche wirkt, aufteilen.
Nach dem Hookeschen Gesetz lässt sich die Normalspannung beschreiben als:
\begin{equation}
\sigma = E \frac{\Delta L}{L} \label{eq:Hooke}
\end{equation}
Dabei wird der Proportionalitätsfaktor $E$ zwischen der Spannung und der relativen Längenänderung Elastizitätsmodul genannt.

\subsection{Bestimmung des Elastizitätsmoduls eines Stabes durch einseitige Einspannung}

Wird an das freie Ende eines stabförmigen Probekörpers, der auf der einen Seite eingespannt ist, ein Gewicht angehängt, dann bewirkt die Kraft $F_.G$ ein Drehmoment $M_.F$ auf einen Querschnitt $Q$ im Abstand $x$ vom Einspannungspunkt.
Da das Gewicht am Ende des Stabs hängt, hat dieses Drehmoment einen Hebelarm von $L-x$ und damit den Betrag:
\begin{equation}
M_.F = F(L-x) \label{eq:M_F}
\end{equation}
Auf Grund der Elastizität des Stabs bewirken in seinem Inneren die entgegengesetzt gleich gerichteten Normal- und Tangentialspannungen ebenfalls ein Drehmoment $M_.{\sigma}$, dass der Deformierung entgegenwirkt und sich berechnen lässt als:
\begin{equation}
M_.{\sigma} = \int_Q y \sigma (y) \mathrm{d}q, \label{M_s}
\end{equation}
wobei $y$ den Abstand vom Angriffspunkt der Spannung zur neutralen Faser, also dem Bereich des Stabes in dem sich beide Spannungen aufheben, bezeichnet.
Für
\begin{equation*}
M_.F = M_.{\sigma}
\end{equation*}
stellt sich so eine Durchbiegung des Stabs $D(x)$ ein, also wenn mit Gleichung \eqref{eq:M_F} und \eqref{M_s} gilt:
\begin{equation}
F(L-x) = \int_Q y \sigma (y) \mathrm{d}q \label{eq:Mom}
\end{equation}
Da nun die Längenänderung der Änderung in $x$-Richtung entspricht, kann das Hookesche Gesetz aus Gleichung \eqref{eq:Hooke} auch als 
\[
\sigma (y) = E \frac{\delta x}{\Delta x} 
\]
mit der Längenänderung $\delta x$ und der Ausgangslänge $\Delta x$ geschrieben werden.
Für kleine $\delta x$ ergibt sich der Zusammenhang 
\[
\delta x = y \Delta\phi = \frac{y\Delta x}{R} \text{.}
\]
Ist der Krümmungsradius $R$ sehr groß, so lässt sich $\frac{1}{R}$ nähern als
\[
\frac{1}{R}\approx \frac{\mathrm{d}^2D}{\mathrm{d}x^2}\text{.}
\]
Damit ergibt sich:
\[
\sigm (y) = E y \frac{\mathrm{d}^2D}{\mathrm{d}x^2}
\]
und mit Gleichung \eqref{eq:Mom}
\begin{equation}
E \frac{\mathrm{d}^2D}{\mathrm{d}x^2} \int_Q y^2 \mathrm{d}q = F(L-x)\text{.} \label{eq:Mom2}
\end{equation}
Das Integral wird als Flächenträgheitsmoment $I$ genannt:
\[
I \coloneqq \int_Q y^2 \mathrm{d}q
\]
Nach Integration ergibt sich für $0 \leq x \leq L$  damit für die Durchbiegung:
\begin{equation*}
D(x) = \frac{F}{2 E I}\left(L x^2 -\frac{x^3}{3}\right)
\end{equation*}
Dabei fallen die Integrationskonstanten weg, da für die Auslenkung an der Einspannungsstelle $D(0) \equiv 0$ gilt.
Damit ergibt sich für das Elastiziätsmodul:
\begin{equation}
E = \frac{F}{2 D(x) I}\left(L x^2 -\frac{x^3}{3}\right)\label{eq:E}
\end{equation}

\subsection{Bestimmung des Elastizitätsmoduls eines Stabes durch beidseitige Lagerung}

Wird ein Gewicht an die Mitte eines auf beiden Seiten gelagerten Stabes gehängt, so gibt es für das Drehmoment $M_.F$ und damit mit Gleichung \eqref{eq:Mom2} eine Fallunterscheidung:
Für $0 \leq x \leq \frac{L}{2}$ greift die Kraft $\frac{F}{2}$ mit dem Hebelarm $x$ und es gilt:
\[
\frac{\mathrm{d}^2D}{\mathrm{d}x^2} = -\frac{F}{E I}\frac{x}{2}
\]
Für $\frac{L}{2} \leq x \leq L$ greift dieselbe Kraft mit dem Hebelarm $L-x$:
\[
\frac{\mathrm{d}^2D}{\mathrm{d}x^2} = -\frac{1}{2}\frac{F}{E I} (L-x)
\]
Integration bringt:
\[
\frac{\mathrm{d}D}{\mathrm{d}x} = -\frac{F}{E I}\frac{x^2}{4} + C_.1  \left(0 \leq x \leq \frac{L}{2}\right)
\]
bzw.
\[
\frac{\mathrm{d}D}{\mathrm{d}x} = -\frac{1}{2}\frac{F}{E I}\left(L x - \frac{x^2}{2}\right) + C_.2  \left(\frac{L}{2} \leq x \leq L\right) \text{.}
\]
Hier fallen die Konstanten nicht weg, da für $x = \frac{L}{2}$ die Durchbiegung ein Maximum und damit gelten muss:
\[
\frac{\mathrm{d}D(x)}{\mathrm{d}x}|_{x = \frac{L}{2}} = 0
\]
Daraus ergibt sich für $C_.1$ und $C_.2$:
\begin{align*}
C_.1 &= \frac{F}{E I}\frac{L^2}{16}
C_.2 &= \frac{3}{16}\frac{F}{E I} L^2
\end{align*}
und mit Integration für $D(x)$:
\begin{equation*}
D(x) = \frac{F}{48 E I}\left(3 L^2 x - 4 x^3\right)
\end{equation*}
für $0 \leq x \leq \frac{L}{2}$ und
\begin{equation*}
D(x) = \frac{F}{48 E I}\left(4 x^3 - 12 L x^2 + 9 L^2 x - L^3\right)
\end{equation*}
für $\frac{L}{2} \leq x \leq L$.
Da für die Durchbiegung an den Stabenden wieder \[D(0)=D(L)\eqiv 0 \] gilt, lässt sich das Elastizitätsmodul $E$ also über
\begin{equation}
E = \frac{F}{48 D I}\left(3 L^2 x - 4 x^3\right)   \left(0 \leq x \leq \frac{L}{2}\right)
\end{equation}
und
\begin{equation}
D(x) = \frac{F}{48 E I}\left(4 x^3 - 12 L x^2 + 9 L^2 x - L^3\right)  \left(\frac{L}{2} \leq x \leq L\right)
\end{equation}
berechnen.