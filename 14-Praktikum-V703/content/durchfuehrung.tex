\section{Durchführung}
\label{sec:Durchführung}

\subsection{Charakteristik des GMZ}
Die $\beta$-Quelle wird vor das Endfenster gestellt und die Zählrate $N$ in Abhängigkeit von $U$ für Werte zwischen $300$ und $\SI{700}{\volt}$ gemessen, um Schäden durch die selbstständige Gasentladung zu vermeiden. Als Messzeit werden $t=\SI{60}{\second}$ gewählt, damit die Zählrate groß genug ist, sodass der statistische Fehler weniger als 1\% beträgt.
Mit dem in Abbildung \ref{fig:Aufbau} zu sehenden Strommessgerät wird für die verschiedenen Zählrohrspannungen $U$ der mittlere Strom im Zählrohr gemessen.
%\subsection{Nachentladungen}
%Die Strahlintensität der Quelle wird so weit abgesenkt, dass bei einer angelegten Spannung $U=\SI{350}{\volt}$ auf dem Oszilloskop nur der Impuls eines Teilchens zusehen ist. Der Abstand des nach der Erhöhung der Spannung auf $U=\SI{710}{\volt}$ zusehenden Nachladeimpulses zum Primärimpuls wird gemessen.

\subsection{Tot- und Erholungszeit des GMZ}
Zur Bestimmung von Tot- und Erholungszeit werden zwei Methoden angewandt:
Für die Messung mittels Oszilloskop wird die Spannung erhöht, sodass ein Abbildung \ref{fig:TZ} ähnlicher Graph entsteht, aus dem Tot- und Erholungszeit abgelesen werden können.
\newline\newline
Zur Totzeit-Messung über zwei Quellen wird die Zählrate $N_.1$ einer Quelle gemessen.
Eine zweite Quelle wird auf das Zählrohr gerichtet und $N_.{1+2}$ gemessen.
Nach Entfernen der ersten Quelle wird $N_.2$ gemessen.
