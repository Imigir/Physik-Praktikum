\section{Durchführung}
\label{sec:Durchführung}

\subsection{Charakteristik des GMZ}
Die $\beta$-Quelle wird vor das Endfenster gestellt und die Zählrate $N$ in Abhängigkeit von $U$ gemessen, wobei nur $U\leqq \SI{700}{\volt}$ betrachtet wird um Schäden durch die selbstständige Gasentladung zu vermeiden. Die Messzeit wird ausreichend lang gewählt, damit der statistische Fehler weniger als 1\% beträgt.

\subsection{Nachentladungen}
Die Strahlintensität der Quelle wird so weit abgesenkt, dass bei einer angelegten Spannung $U=\SI{350}{\volt}$ auf dem Oszilloskop nur der Impuls eines Teilchens zusehen ist. Der Abstand des nach der Erhöhung der Spannung auf $U=\SI{710}{\volt}$ zusehenden Nachladeimpulses zum Primärimpuls wird gemessen.

\subsection{Totzeit des GMZ}
\subsubsection{Totzeit-Messung über das Oszilloskop}
Die Strahlintensität wird erhöht, sodass auf dem Oszilloskop ein Abbildung \ref{fig:TZ} ähnlicher Graph entsteht aus dem Tot- und Erholungszeit abgelesen werden können.
\subsubsection{Totzeit-Messung mit zwei Quellen}
Es wird die Zählrate $N_.1$ einer Quelle gemessen.
Eine zweite Quelle wird auf das Zählrohr gerichtet und $N_.{1+2}$ gemessen.
Nach Entfernen der ersten Quelle wird $N_.2$ gemessen.
\subsection{Freigesetzte Ladungsmenge pro eindringendem Teilchen}
Mit dem in Abbildung \ref{fig:Aufbau} zu sehenden Strommessgerät wird für verschiedene Zählrohrspannungen $U$ der mittlere Strom im Zählrohr gemessen.