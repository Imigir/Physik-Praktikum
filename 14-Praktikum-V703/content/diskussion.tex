
\section{Diskussion}
\label{sec:Diskussion}

Die Plateausteigung $m_.{rel}$ liegt mit $\SI{2.2(2)}{\%\per100\volt}$ unterhalb von $\SI{5}{\%\per100\volt}$. Da auch der Fehler gering ist und das Plateau, wie in Abbildung \ref{fig:Graph1.1} zu sehen, frühzeitig auf einen Wert größer als $\SI{250}{\volt}$ ansteigt und diesen bis zum Ende der Messung nahezu hält, kann auf eine gute Qualität des Zählrohrs geschlossen werden. Dabei ist in Abbildung \ref{fig:Graph1.2} zu sehen, dass einzelne Werte leicht von der Ausgleichsgeraden abweichen. Dies kann an der statistischen Ungenauigkeit liegen, da die Fehlerbalken nur die Standartabweichung angeben und Abweichungen durchaus plausibel sind.\\
Die Bestimmte Totzeit von $T_2=\SI{120(310)e-6}{\second}$  ist mit einer Abweichung von $\SI{20}{\%} $ in der Nähe von dem abgelesenen Wert $T_1=\SI{100(25)e-6}{\second}$. Allerdings ist der Wert aufgrund seines großen Fehlers nicht sehr aussagekräftig. Der große Fehler kommt zustande, da die Totzeit sich im Vergleich zu der Impulsrate in einer sehr kleinen Größenordnung befindet. Die Erholungszeit $T_.E=\SI{225e-6}{\second}$ ist nur ungenügend genau am Oszilloskop abgeschätzt. Sie kann jedoch in dieser Größenordnung angenommen werden.\\
Wird die freigesetzte Ladungsmenge wie in Abbildung \ref{fig:Graph2} in Abhängigkeit von der Spannung betrachtet, so lässt sich ein linearer Zusammenhang zwischen den Größen feststellen. Die Abweichungen von der Ausgleichsgeraden sind wie bei der Plateausteigung gering, was die gute Qualität des Zählrohrs unterstützt.  