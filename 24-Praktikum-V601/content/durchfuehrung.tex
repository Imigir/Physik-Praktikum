\section{Durchführung}
\label{sec:Durchführung}
Für jede Messung muss der XY-Schreiber zunächst kalibriert werden. Dazu wird nach Einlegen des Millimeterpapiers der Nullpunkt der Achsen mit den "zero"-Reglern in die linke untere Ecke des Blattes gelegt und die Empfindlichkeiten der Messreihe angepasst.
\subsection{Die integrale Energieverteilung}
Zur Messung der integralen Energieverteilung werden die Bremsspannung der regelbaren Spannungsquelle und das Ampermeter an den XY-Schreiber angeschlossen. $U_.B$ wird auf $\SI{11}{\volt}$ eingestellt und der Anodenstrom $I$ in Abhängigkeit von $U_.A$ bei Zimmertemperatur und $T\approx \SI{140}{\degree\celsius}$ aufgenommen.
\subsection{Die Franck-Hertz-Kurve}
Zur Aufnahme der Franck-Hertz-Kurve wird die Bremsspannung an den XY-Schreiber angeschlossen und bei $\SI{160}{\degree\celsius}<T<\SI{200}{\degree\celsius}$ und $U_.A=\SI{1}{\volt}$
der ankommende Strom in Abhängigkeit von $U_.B$ aufgenommen.
\subsection{Die Ionisierungsenergie}
Zur Bestimmung der Ionisierungsenergie des Quecksilbers, wird die Konstantspannungsquelle als Lieferant der Bremsspannung verwendet und der ankommende Strom bei $T\approx\SI{110}{\degree\celsius}$ in Abhängigkeit von der Beschleunigungsspannung aufgenommen.