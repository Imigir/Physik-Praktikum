
\section{Diskussion}
\label{sec:Diskussion}

Die gemessenen Werte und ihre Abweichungen vom Literaturwert sind in Tabelle \ref{tab:Ergebnisse} zu finden.
Die 1. Anregungsenergie $E_.1$ und damit die Wellenlänge $\lambda$ des im Fall eines in den Grundzustand zurückfallenden Elektrons ausgesendeten Lichts konnten ziemlich exakt bestimmt werden.\newline
Die sehr großen Abweichungen bei der Ionisierungsenergie von $\delta_.{Io1}=\SI{50}{\percent}$ und $\delta_.{Io2}=\SI{24}{\percent}$ hingegen, sind darauf zurückzuführen, dass es bei dieser Messung zunächst zu einer Fehlfunktion kam. Außerdem gab es Wackelkontakte an den Eingängen des XY-Schreiber, wodurch der Graph \ref{fig:c} möglicherweise beeinflusst wurde.\newline Der Unterschied der Graphen \ref{fig:a1} und \ref{fig:a2} entsteht dadurch, dass bei höheren Temperaturen die Elektronen auf Grund des erhöhten Dampfdrucks vermehrt mit den Hg-Atomen stoßen. Auch wenn viele Elektronen nur elastische Stöße durchführen, werden sie, wenn diese zwischen Gitter und Anode stattfinden aus ihrer Bahn abgelenkt und verfehlen entweder die Anode oder haben eine zu geringe Geschwindigkeit in Feldrichtung um die Gegenspannung $U_.A$ zu überwinden.
\begin{table}
\centering
\caption{Die Werte der Spannung $U_.B$ des $n$-ten Peaks abgelesen aus Abbildung \ref{fig:b}.}
\label{tab:Ergebnisse}
	\sisetup{table-format=1.2}
	\begin{tabular}{c ccc}
		\toprule
		{Wert}&{gemessen}&{Referenzwert}&{Abweichung} \\
		\midrule
		$\alpha_{max}$ & \SI{28,1}\,\si{\degree} & \SI6{28}\,\si{\degree} & \SI{0,3}\,\si{\percent} \\
		$\theta_{gr}$ & \SI{5}\,\si{\degree} & \SI{5,04}\,\si{\degree} & \SI{-0,9}\,\si{\percent} \\
		$\lambda_{min}$ & \SI{35,1}\,\si{\pico\metre} & \SI{35,4}\,\si{\pico\metre}  & \SI{-1,13}\,\si{\percent} \\
		$E_{kin,max}$ & \SI{35316}\,\si{\eV} & \SI{35000}\,\si{\eV} & \SI{0,9}\,\si{\percent} \\
		$\Delta E_{alpha}$ & \SI{130,7}\,\si{\eV} & - & - \\
		$\Delta E_{beta}$ & \SI{155,0}\,\si{\eV} & - & - \\
		$\sigma_{Cu_K}$ & \SI{3,28} & \SI{3,31} & \SI{-0,76}\,\si{\percent}  \\
		$\sigma_{Cu_{L}}$ & \SI{13,16} & \SI{20,72} & \SI{-36,48}\,\si{\percent} \\
		$\sigma_{Cu_{M}}$ & \SI{29} & \SI{26,64} & \SI{8,87}\,\si{\percent} \\
		$E_{K_{Br}}$ & \SI{13282}\,\si{\eV} & \SI{13470}\,\si{\eV} & \SI{-1,40}\,\si{\percent} \\
		$E_{K_{Sr}}$ & \SI{15988}\,\si{\eV} & \SI{16090}\,\si{\eV} & \SI{-0,64}\,\si{\percent} \\
		$E_{K_{Zn}}$ & \SI{9650}\,\si{\eV} & \SI{9650}\,\si{\eV} & \SI{0,0}\,\si{\percent} \\
		$E_{K_{Zr}}$ & \SI{17903}\,\si{\eV} & \SI{17970}\,\si{\eV} & \SI{-0,37}\,\si{\percent} \\
		$\sigma_{K_{Br}}$ & \SI{3,75} & \SI{3,53} & \SI{6,23}\,\si{\percent} \\
		$\sigma_{K_{Sr}}$ & \SI{3,71} & \SI{3,66} & \SI{1,37}\,\si{\percent} \\
		$\sigma_{K_{Zn}}$ & \SI{3,36} & \SI{3,36} & \SI{0,0}\,\si{\percent} \\
		$\sigma_{K_{Zr}}$ & \SI{3,72} & \SI{3,65} & \SI{1,92}\,\si{\percent} \\
		$R_{\infty}$ & \SI{16,87\pm 0,32}\,\si{\eV} & \SI{13,6}\,\si{\eV} & \SI{24,04}\,\si{\percent} \\
		$\sigma_{L_{Bi}}$ & \SI{3,31} & \SI{3,58} & \SI{-7,54}\,\si{\percent} \\
		\bottomrule
	\end{tabular}

\end{table}