
\section{Theorie}
\label{sec:Theorie}

\section{Das Franck-Hertz-Experiment}

Beim Franck-Hertz-Versuch wird in einem evakuierten Gefäß ein Tropfen Quecksilber erhitzt, sodass es ausschließlich mit Hg-Dampf gefüllt ist. Der Sättigungsdampfdruck lässt sich dabei über die Temperatur $T$ regeln.
Durch eine Heizspannung wird eine in das Gefäß eingelassenen Glühkathode erhitzt, sodass Elektronen austreten. Um die Elektronenausbeute zu erhöhen wird die üblicherweise aus Wolfram bestehende Kathode mit einem Material niedrigerer Austrittsarbeit beschichtet.
Das Gefäß wird durch ein Gitter in zwei Bereiche unterteilt:
Zwischen Kathode und Gitter herrscht eine Beschleunigungsspannung $U_.B$.
Es gilt für die kinetische Energie $E$ der Elektronen in diesem Bereich mit der Elektronenmasse $m_.e$ und der Elementarladung $e_.0$:
\begin{equation*}
E=\frac{m_.ev_.{vor^2}}{2}=e_.0U_.B
\end{equation*}
und damit für ihre Geschwindigkeit
\begin{equation}
v_.{vor}=\sqrt{\frac{2e_.0U_.B}{m_.e}\label{eq:v}}
\end{equation}
Da im gesamten Gefäß Hg-Dampf enthalten ist, kommt es auf dem Weg der Elektronen zu elastischen und inelastischen Stößen mit den Quecksilberatomen. Während erstere nur für eine Richtungsänderung der Bahn der Elektronen sorgen, verlieren die Elektronen bei letzteren an Energie und werden somit langsamer.
Die Energiedifferenz $\Delta E$, die vom Atom aufgenommen wird berechnet sich demnach über
\[
\Delta E = \frac{m}{2}\left(v^2_.{vor}-v^2_.{nach}\right)\text{.}\label{eq:DeltaE}
\]
Sie bezeichnet die Energie, die das Atom benötigt um vom Grundzustand $E_.0$ in den nächsthöheren $E_.1$ zu wechseln.
Dahinter herrscht zwischen Gitter und einer Auffängerelektrode ein Gegenfeld $U_.A$.
Um dies zu überwinden muss für die in Feldrichtung zeigende Komponente der Geschwindigkeit gelten
\[
v
\]