\section{Diskussion}
\label{sec:Diskussion}
Auffallend ist zunächst der Unterschied der Amplituden mit und ohne Rausch-Schaltung.
Die Frequenz $f$ hingegen wurde sehr exakt bestimmt.
Da sich die Bilder der Spannungsverläufe, die vor Zuschalten des Tiefpassfilters entstanden sind, sowohl ohne, als auch mit Rauschgenerator sehr ähneln, liegt die Vermutung nahe, dass der Verstärker des Tief-Pass-Filters unterschiedlich eingestellt war.
Eine andere Möglichkeit besteht darin, dass es sich um einen Anzeigefehler handelt, da bei der Durchführung nur durch Hinzufügen des Rauschgenerators die Skalierung des Oszilloskops etwa um einen Faktor 100 kleiner wurde.
Das würde auch erklären, dass bei der Messung mit Rauschgenerator der Fehler der Amplitude nur
\[
\Delta U_.{max}= 3,20\%
\]
beträgt.
Die Phasenverschiebung $\Delta\phi$ von
\[
\Delta\phi = \SI{84(2)}{\degree}
\]
ohne Rauschen und
\[
\Delta\phi = \SI{80(3)}{\degree}
\]
mit Rauschen sind auf die Grundeinstellung des Steuergeräts zurückzuführen.
Bei der Messung mit der Leuchtdiode konnte kein maximaler Abstand $r_.{max}$ ermittelt werden, da selbst bei maximal einstellbarer Entfernung eine Spannung gemessen werden konnte.
Dies könnte daran liegen, dass der Gain des Vorverstärkers zu stark eingestellt wurde.
Weiterhin kann es durch die äußeren Lichtverhältnisse auch zu einer Verfälschung der Messergebnisse gekommen sein. Die $\frac{1}{r^2}$-Abnahme der Intensität konnte indes bestätigt werden.