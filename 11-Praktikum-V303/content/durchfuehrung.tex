
\section{Durchführung}
\label{sec:Durchführung}

Im ersten Versuchsteil werden die Signale des Funktionsgenerators mit dem Oszilloskop abgegriffen und es wird bestimmt, bei welchem Ausgang es sich um die Referenzspannung $U_.{ref}$ und die Signalspannung $U_.{sig}$ handelt. Die Amplitude $U_.1$ der Referenzspannung wird notiert.\newline
Die Schaltung in \ref{fig:Aufbau1} wird ohne den Rauschgenerator aufgebaut. Für $U_.{sig}$ wird ein sinusförmiges Signal mit einer Amplitude von $U_.0=\SI{10e-3}{\volt}$ und einer Frequenz von $\omega_.0=\SI{1000}{\hertz}$ angelegt. Das Referenzsignal besitzt die selbe Frequenz.
Das Ausgangssignal nach dem Detektor wird für fünf verschiedene Phasen skizziert. Es werden zehn Messwerte für die Ausgangsspannung $U_.{out}$ nach dem Tiefpass notiert und mit den theoretischen Werten verglichen.
Die Messungen werden mit Rauschgenerator wiederholt und die Unterschiede diskutiert.\newline
Im zweiten Versuchsteil wird die Schaltung nach Abbildung \ref{fig:Aufbau2} aufgebaut. Die Leuchtdiode wird mittels einer Rechteckspannung mit einer Frequenz von $\SI{200}{\hertz}$ zum blinken gebracht. Die Lichtintensität wird im Verhältnis zum Abstand $r$ aufgetragen und mit der Theorie verglichen. Falls möglich, wird der maximale Abstand $r_.{max}$ ermittelt, bei dem noch ein Signal am Photodetektor nachgewiesen werden kann.        