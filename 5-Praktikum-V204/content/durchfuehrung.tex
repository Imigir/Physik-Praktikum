
\section{Durchführung}
\label{sec:Durchführung}

Bei jeder Messung werden alle Acht Thermoelemente wärmeisoliert und ausgemessen.
Nach jeder Messung müssen die Stäbe unter $\SI{30}{\grad\celsius}$ abkühlen, bevor die nächste Messung beginnt. Der Abstand zwischen den Thermoelementen der jeweiligen Materialien wird notiert. 

\subsection{Statische Methode}

Die Wärmeleitfähigkeit wird über den zeitlichen Temperaturverlauf an zwei Messstellen bestimmt.

Mit dem Datenlogger wird in einem Abstand von $\SI{5}{\seconds}$ bei einer Spannung von $U_.P = \SI{5}{\volt}$ und maximalem Strom der Temperaturverlauf der Acht Thermoelemente gemessen, bis $\text{T}7$ ungefähr eine Temperatur von $\SI{45}{\grad\celsius}$ erreicht.

Die ermittelten Temperaturverläufe der Thermoelemente $\text{T}1$, $\text{T}4$, $\text{T}5$ und $\text{T}8$ werden graphisch dargestellt und nach $\SI{700}{\seconds}$ werden die Temperaturen notiert. Gemeinsamkeiten und Unterschiede werden diskutiert. 

Es werden für fünf verschiedene Messzeiten der Wärmestrom berechnet. 
Die Temperaturdifferenz von $\text{T}2$ und $\text{T}1$, sowie von $\text{T}7$ und $\text{T}8$ wird in Abhängigkeit von der Zeit in ein Diagramm eingetragen.
 
\subsection{Dynamische Methode}

Die Wärmeleitfähigkeit wird aus der Ausbreitungsgeschwindigkeit der Temperaturwelle bestimmt, die durch periodisches heizen der Stäbe entsteht.

Mit dem Datenlogger wird in einem Abstand von $\SI{2}{\seconds}$ bei einer Spannung von $U_.P = \SI{8}{\volt}$ und maximalem Strom zweimal der Temperaturverlauf der Acht Thermoelemente gemessen.

Bei der ersten Messung werden die Stäbe mit einer Periode von $\SI{80}{\seconds}$ erhitzt und mindestens zehn Perioden durchgeführt.

Die Temperaturverläufe für den breiten Messingstab und für Aluminium werden graphisch dargestellt und die jeweiligen Amplituden $A_.{nah}$ und $A_.{fern}$, sowie die Phasendifferenz $\Delta t$ bestimmt. Aus den ermittelten Werten wird die Wärmeleitfähigkeit $\kappa$ der Materialien errechnet. 

Bei der zweiten Messung werden die Stäbe mit einer Periode von $\SI{200}{\seconds}$ erhitzt, bis eins der Thermoelemente über $\SI{80}{\grad\celsius}$ erreicht.

Die Temperaturverläufe für Edelstahl werden graphisch dargestellt und die Wärmeleitfähigkeit wird analog zu Messing und Aluminium berechnet.