
\section{Durchführung}
\label{sec:Durchführung}

Bei jeder Messung werden alle acht Thermoelemente wärmeisoliert und ausgemessen.
Nach jeder Messung müssen die Stäbe unter $\SI{30}{\celsius}$ abkühlen, bevor die nächste Messung beginnt. Der Abstand zwischen den Thermoelementen der jeweiligen Materialien wird notiert. 

\subsection{Statische Methode}

Die Wärmeleitfähigkeit wird über den zeitlichen Temperaturverlauf an zwei Messstellen bestimmt.
Mit dem Datenlogger wird in einem Abstand von $\SI{5}{\second}$ bei einer Spannung von $U_.P = \SI{5}{\volt}$ und maximalem Strom der Temperaturverlauf der Acht Thermoelemente gemessen, bis $T_.7$ ungefähr eine Temperatur von $\SI{45}{\celsius}$ erreicht.\newline
Die ermittelten Temperaturverläufe der Thermoelemente $T_.1$, $T_.4$, $T_.5$ und $T_.8$ werden graphisch dargestellt und nach $\SI{700}{\second}$ werden die Temperaturen notiert. Gemeinsamkeiten und Unterschiede werden diskutiert. 
Es wird für fünf verschiedene Messzeiten der Wärmestrom berechnet. 
Die Temperaturdifferenz von $T_.2$ und $T_.1$, sowie von $T_.7$ und $T_.8$ wird in Abhängigkeit von der Zeit in ein Diagramm eingetragen.
 
\subsection{Dynamische Methode}

Die Wärmeleitfähigkeit wird aus der Ausbreitungsgeschwindigkeit der Temperaturwelle bestimmt, die durch periodisches heizen der Stäbe entsteht.
Mit dem Datenlogger wird in einem Abstand von $\SI{2}{\second}$ bei einer Spannung von $U_.P = \SI{8}{\volt}$ und maximalem Strom zweimal der Temperaturverlauf der Acht Thermoelemente gemessen.\newline
Bei der ersten Messung werden die Stäbe mit einer Periode von $\SI{80}{\second}$ erhitzt und mindestens zehn Perioden durchgeführt.
Die Temperaturverläufe für den breiten Messingstab und für Aluminium werden graphisch dargestellt und die jeweiligen Amplituden $A_.{nah}$ und $A_.{fern}$, sowie die Phasendifferenz $\Delta t$ bestimmt. Aus den ermittelten Werten wird die Wärmeleitfähigkeit $\kappa$ der Materialien errechnet. \newline
Bei der zweiten Messung werden die Stäbe mit einer Periode von $\SI{200}{\second}$ erhitzt, bis eins der Thermoelemente $\SI{80}{\celsius}$ erreicht.
Die Temperaturverläufe für Edelstahl werden graphisch dargestellt und die Wärmeleitfähigkeit wird analog zu Messing und Aluminium berechnet.