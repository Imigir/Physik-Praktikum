
\section{Auswertung}
\label{sec:Auswertung}

Alle Mittelwerte werden berechnet mit:
\begin{equation}
\bar{x} = \sum_{i=1}^{n}x_i/n \label{eq:quer}
\end{equation}
Die Abweichungen ergeben sich nach:
\begin{equation}
\bar{\sigma} = \sqrt{\frac{1}{n(n-1)}\sum_{i=1}^{n}(\bar{x}-x_i)^2}  \label{eq:sigma}
\end{equation}

\subsection{Statische Methode}

Messwerte für $T_.1$, $T_.4$, $T_.5$ und $T_.8$ nach $\SI{700}{\second}$
\begin{align*}
T_\text{1}=\SI{316.46}{\kelvin}\\
T_\text{4}=\SI{314.43}{\kelvin}\\
T_\text{5}=\SI{320.15}{\kelvin}\\
T_\text{8}=\SI{305.88}{\kelvin}
\end{align*}
Der Wärmestrom pro Zeit $\frac{\mathrm{d}Q}{\mathrm{d}t}$
lässt sich nach Formel\eqref{eq:dQ/dt} berechnen. Die gemessenen Werte lassen sich gemeinsam mit diesen Werten in Tabelle \ref{tab:tab1} ablesen. Die Graphen der Temperaturdifferenzen $T_\text{2}-T_\text{1}$ und $T_\text{7}-T_\text{8}$ sind in Abbildung (3) und (4) im Anhang zu sehen.
Es werden die Wärmeleitungsfähigkeit $\kappa$ aus der Literatur\cite{kappa} und die Querschnitsfläche $A$ aus der Anleitung\cite{V204} entnommen. Dabei wird für $\kappa_\text{Messing}$ der Mittelwert des angegeben Bereichs genommen.
Die Distanz zwischen den Thermoelementen eines Stabes ist $\Delta x$.
\begin{align*}
\Delta x 			&= \SI{0.03}{\metre}\\
\kappa_\text{Messing}	&= \SI{93}{\watt\per\metre\per\kelvin}\\
\kappa_\text{Edelstahl}	&= \SI{20}{\watt\per\metre\per\kelvin}\\
A_\text{Messing, breit} 	&= \SI{ 48e-6}{\metre\squared} \\
A_\text{Edelstahl} 		&= \SI{ 48e-6}{\metre\squared}
\end{align*}

\begin{table}
	\centering
	\caption{Die gemessenen Daten für Temperaturdifferenzen und den Wärmestrom pro Zeit zum Zeitpunkt $t$.}
		\sisetup{table-format=1.2}
	\begin{tabular}{S[table-format=3.2] S[table-format=3.2]S[table-format=3.2]S[table-format=3.2]S[table-format=3.2]}
		\toprule
		{$t/\si[per-mode=reciprocal]{\second}$}&{$T_\text{2-1}/\si[per-mode=reciprocal]{\kelvin}$}&{$\frac{\mathrm{d}Q_\text{2-1}}{\mathrm{d}t}/\si[per-mode=reciprocal]{\milli\watt}$}&{$T_\text{7-8}/\si[per-mode=reciprocal]{\kelvin}$}&{$\frac{\mathrm{d}Q_\text{7-8}}{\mathrm{d}t}/\si[per-mode=reciprocal]{\milli\watt}$}\\
		\midrule
		5     & 0.35 & -5.34   & -0.08  & 0.26 \\
		200 & 4.49 & -66.81 & 11.90 & -38.08 \\
		400 & 3.07 & -45.68 & 10.30 & -32.96 \\
		550 & 2.81 & -41.81 & 9.65   & -30.88 \\
		950 & 2.69 & -40.03 & 8.91   & -28.51 \\
		\bottomrule
	\end{tabular}
	\label{tab:tab1}
\end{table}

\subsection{Dynamische Methode}

Für die dynamischen Methode wird $\kappa$ über Formel \eqref{eq:k} berechnet.
Die gemessenen Werte für die Periodendauer $\Delta t$ und für die beiden Amplituden $A_\text{nah}$ und $A_\text{fern}$ des näheren und des weiter entfernteren Thermoelements lassen sich aus den Tabellen \ref{tab:tab2}, \ref{tab:tab3} und \ref{tab:tab4} ablesen. Die Materialkonstanten $\rho$ und $c$ werden aus der Anleitung entnommen\cite{V204}. Die Distanz $\Delta x$ ist dabei dieselbe, wie bei der statischen Methode.\newline
Für Messing ergibt sich damit: \[\kappa = \SI{120(11)}{\watt\per\metre\per\kelvin}\]\newline
Für Aluminium ergibt sich damit: \[\kappa = \SI{205(14)}{\watt\per\metre\per\kelvin}\]\newline
Für Edelstahl ergibt sich damit: \[\kappa = \SI{30(2)}{\watt\per\metre\per\kelvin}\]

\begin{table}

	\centering

	\caption{Temperatur des breiten Messingstabs mit Periodendauer 80 s.}

	\sisetup{table-format=1.2}

	\begin{tabular}{S[table-format=3.2] S[table-format=3.2]S[table-format=3.2]}

		\toprule

		{$\Delta t/\si[per-mode=reciprocal]{\second}$}&{$A_\text{nah}/\si[per-mode=reciprocal]{\kelvin}$}&{$A_\text{fern}/\si[per-mode=reciprocal]{\kelvin}$} \\

		\midrule

		13.33 & 7.27 & 3.41 \\

		11.11 & 6.36 & 3.18 \\

		13.33 & 6.14 & 2.73 \\

		13.33 & 5.91 & 2.50 \\

		13.33 & 5.68 & 2.27 \\

		13.33 & 5.45 & 2.27 \\

		13.33 & 5.45 & 2.05 \\

		17.78 & 5.45 & 1.82 \\

		17.78 & 5.23 & 1.82 \\

		17.78 & 5.23 & 1.82 \\

		\bottomrule

	\end{tabular}

	\label{tab:tab2}

\end{table}

\begin{table}

	\centering

	\caption{Temperatur des Aluminiumstabs mit Periodendauer 80 s.}

	\sisetup{table-format=1.2}

	\begin{tabular}{S[table-format=3.2] S[table-format=3.2]S[table-format=3.2]}

		\toprule

		{$\Delta t/\si[per-mode=reciprocal]{\second}$}&{$A_\text{nah}/\si[per-mode=reciprocal]{\kelvin}$}&{$A_\text{fern}/\si[per-mode=reciprocal]{\kelvin}$} \\

		\midrule

		8.89 & 10.20 & 6.60 \\

		8.89 & 8.80 & 5.40 \\

		8.89 & 8.20 & 4.80 \\

		6.67 & 8.00 & 4.40 \\

		11.11 & 7.80 & 4.00 \\

		8.89 & 7.60 & 4.00 \\

		6.67 & 7.60 & 3.80 \\

		8.89 & 7.40 & 3.80 \\

		8.89 & 7.40 & 3.60 \\

		8.89 & 7.20 & 3.60 \\

		\bottomrule

	\end{tabular}

	\label{tab:tab3}

\end{table}

\begin{table}

	\centering

	\caption{Temperatur des Edelstahlstabs mit Periodendauer 200 s.}

	\sisetup{table-format=1.2}

	\begin{tabular}{S[table-format=3.2] S[table-format=3.2]S[table-format=3.2]}

		\toprule

		{$\Delta t/\si[per-mode=reciprocal]{\second}$}&{$A_\text{nah}/\si[per-mode=reciprocal]{\kelvin}$}&{$A_\text{fern}/\si[per-mode=reciprocal]{\kelvin}$} \\

		\midrule

		30.30 & 13.33 & 3.56 \\

		36.36 & 11.33 & 3.11 \\

		30.30 & 10.67 & 2.67 \\

		36.36 & 10.22 & 2.22 \\

		36.36 & 10.00 & 2.00 \\

		\bottomrule

	\end{tabular}

	\label{tab:tab4}

\end{table}
