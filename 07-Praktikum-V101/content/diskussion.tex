\section{Diskussion}
\label{sec:Diskussion}
Alle berechneten Trägheitsmomente der verschiedenen Körper sind negativ.
Die Differenzen zwischen den theoretischen und experimentellen Werten liegen bei:
\begin{align*}
I_.{Kugel,Theorie}-I_.{Kugel}		&=\SI{2,9(2)e-3}{\kilo\gram\metre\squared}\\
I_.{Zylinder,Theorie}-I_.{Zylinder}	&=\SI{3,12(2)e-3}{\kilo\gram\metre\squared}\\
I_.{Puppe,aus,theo}-I_.{Puppe,ausg}	&=\SI{3,3(1)e-3}{\kilo\gram\metre\squared}\\
I_.{Puppe,an,theo}-I_.{Puppe,ang}	&=\SI{2,82(1)e-3}{\kilo\gram\metre\squared}\\
\end{align*}
Das Eigenträgheit $I_.D$ wurde bestimmt zu $I_.D=\SI{3,2(1)e-3}{\kilo\gram\metre\squared}$.
Es lässt sich folgern, dass $I_.D$ zu groß bestimmt wurde, da es nach Formel \eqref{eq:I_K} jedes Mal von den durch die Periodendauer bestimmten Werten abgezogen wird. Wird $I_.D$ in der Rechnung vernachlässigt, liegen alle Werte in der Nähe der theoretischen Werte.\newline
Das $I_.D$ zu groß ist könnte daran liegen, dass die Periodendauer $T_0^2$ zu groß bestimmt wurde, oder der Stab an dem die Zylinder befestigt sind doch hätte mit berücksichtigt werden müssen. Letzteres sollte allerdings einen nicht so großen Einfluss auf das Ergebnis haben. Wahrscheinlicher ist, dass die Spiralfeder der Drillachse bei größerer Auslenkung eine stärkere rückwirkende Kraft ausübt, als bei kleinen Auslenkungen. Dadurch weichen die Periodendauern in größeren Abständen nach unten ab, was zu einer Abweichung von $T_.0^2$ nach oben führt.\newline
Außerdem ist bei der Puppe auf Grund des Näherns der Körperteile zu einfachen geometrischen Gebilden und des inhomogenen Materials nur beschränkt ein exakter theoretischer Wert zu ermitteln, was zu Abweichungen vom durch die Periodendauer bestimmten Wert führen kann.
Wenn man den Quotienten der theoretischen Werte \[\frac{I_.{Puppe,aus,theo}}{I_.{Puppe,an,theo}}\approx 9,466\]
mit dem Quotienten der experimentellen Werte \[\frac{I_.{Puppe,ausg}}{I_.{Puppe,ang}}\approx 0,80\]
vergleicht fällt auf, dass das Verhältnis der Theoriewerte sich um einen Faktor 10 von dem der Messwerte unterscheidet, was sich unter anderem darauf zurückführen lässt, dass $I_.{Puppe,ang}$ im Betrag größer als $I_.{Puppe,ausg}$ ist, wodurch der Quotient kleiner als 1 wird. Absolut ist allerdings $I_.{Puppe,ausg}$ größer, was auch physikalisch sinnvoll ist, da mit ausgebreiteten Armen ein Teil der Masse weiter von der Drehachse entfernt und damit das Trägheitsmoment auf Grund der $R^2$-Proportionalität größer ist.
