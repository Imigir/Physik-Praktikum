
\section{Durchführung}
\label{sec:Durchführung}

\subsection{Bestimmung des Eigenträgheitsmomentes und der Winkelrichtgröße der Drillachse}
Es wird die Winkelrichtgröße und das Eigenträgheitsmoment der Drillachse bestimmt. Dazu wird ein Kraftmesser senkrecht zum Radius im Abstand $a$ an der Drillachse befestigt. Diese wird um den Winkel $\phi$ ausgelenkt. Kraft, Abstand und Winkel werden zehn mal gemessen.\newline
Das Eigenträgheitsmoment wird bestimmt, indem zwei Zylinder im Abstand $a$ an einer als masselos angenommenen Stange befestigt werden, welche auf die Drillachse gesteckt wird. Das System wird in Schwingung versetzt und die Periodendauer $T^2$ gegen $a^2$ abgetragen. Die Messung wird für zehn verschiedene Abstände wiederholt und durch lineare Regression wird das Eigenträgheitsmoment bestimmt.   

\subsection{Bestimmung des Drehmomentes von Kugel, Zylinder und Puppe}
Um das Trägheitsmoment von Kugel und Zylinder zu bestimmen, werden diese jeweils an der Drillachse befestigt und fünf mal aus der Ruhelage ausgelenkt. Die Werte für $T$, $a$ und $\phi$ werden notiert und das Trägheitsmoment bestimmt. \newline
Das Trägheitsmoment einer Holzpuppe wird in zwei unterschiedlichen Haltungen bestimmt. Dazu wird analog wie bei der Kugel und dem Zylinder verfahren. Das Ergebnis wird mit einem genäherten theoretischen Wert verglichen.   