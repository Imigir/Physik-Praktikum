\section{Auswertung}
\label{sec:Auswertung}

Die Graphen werden sowohl mit Matplotlib \cite{matplotlib} als auch NumPy \cite{numpy} erstellt. Die Fehlerrechnung wird mithilfe von Uncertainties \cite{uncertainties} durchgeführt. Die Konstanten werden SciPy \cite{scipy} entnommen.

\subsection{Berechnung der theoretischen Werte}

Zu Beginn werden die theoretischen Landefaktoren der Energieniveaus aus den Übergängen der roten Cadmiumlinie ($^3D_2 \rightarrow ^3P_1$) und der blauen Cadmiumlinie ($^3P_1 \rightarrow ^3S_1$) berechnet. 
Anschließend werden das Dispersionsgebiet $\Delta\lambda_D$ und das Auflösungsvermögen $A$ bestimmt.
Es werden dabei die Landefaktoren $g_im_i-g_jm_j$ aus Gleichung \ref{eq:} zu $g_{i,j}$ zusammengefasst.
Für die rote Cadmiumlinie ergeben sich die Landefaktoren
\[
g_i=g_j=1\text{.}
\]
Daraus folgen für die Übergänge:
\begin{align*}
g_{i,j_\pi}&=0\\
g_{i,j_{\sigma1}}&=1\\
g_{i,j_{\sigma2}}&=-1 \text{.}
\end{align*}
Für die blaue Cadmiumlinie ergeben sich die Landefaktoren
\begin{align*}
g_i &= 1.5\\
g_j &= 2 \text{.}
\end{align*}
Daraus folgen für die Übergänge:
\begin{align*}
g_{i,j_\pi}&=(-0.5,0,0.5)\\
g_{i,j_{\sigma1}}&=(1.5,2)\\
g_{i,j_{\sigma2}}&=(-1.5,-2) \text{.}
\end{align*}
Da die $g_{i,j_{\sigma}$ sehr dicht beisammen liegen, werden bei der Messung nur Aufspaltungen mit $g_{i,j_\sigma} = \pm 1.75$ gemessen.
Das Dispersionsgebiet ergibt sich für die rote Linie mit $\lambda_.{rot}=\SI{643.8}{\nano\metre}$ und für die blaue Linie mit $\lambda_.{blau}=\SI{480}{\nano\metre}$ bei einer Lumma-Gehrcke Platte mit den Dimensionen $L={120}{\milli\metre}$, $d={4}{\milli\metre}$ und den Brechungsindizes $n(\SI{480}{\nano\metre})=1.4635$ und $n(\SI{643.8}{\nano\metre})=1.4567$ und Formel \eqref{eq:} zu:
\[
\Delta\lambda_D = \left{
\begin{cases}
\SI{4.891e-11}{\metre}\\
\SI{2.695e-11}{\metre}
\end{cases}
\]

\subsection{Kalibrierung des B-Feldes}

Zunächst wird das B-Feld kalibriert. Hierzu wird die stärke des B-Feldes in Abhängigkeit von der Stromstärke I aufgenommen. Die Werte sind in Tabelle \ref{tab:B} aufgelistet und in Abbildung \ref{fig:B} graphisch dargestellt. Es wird eine lineare Ausgleichsrechnung der Form
\[
B(I) = aI+b
\]
durchgeführt. Dies liefert die Werte:
\begin{align*}
a &= \SI{}{\tesla\per\ampere}\\
b &= \SI{}{\tesla}
\end{align*}

\begin{figure}
	\centering
	\includegraphics[width=\linewidth-130pt,height=\textheight-130pt,keepaspectratio]{build/Bfeldkali.pdf}
	\caption{Die Kalibrationskurve des B-Feldes in Abhängigkeit der Stromstärke I.}
	\label{fig:B}
\end{figure}

\begin{table}
	\centering
	\caption{Die Messwerte für die B-Feldkalibrierung.}
	\label{tab:tabIB1}
	\sisetup{table-format=1.2}
	\begin{tabular}{S[table-format=2.1]S[table-format=1.3]S[table-format=2.1]S[table-format=1.3]}
		\toprule
		{$ I / \si{\ampere}$} & {$ B/ \si{\tesla}$} & {$ I / \si{\ampere}$} & {$ B/ \si{\tesla}$} \\
		\midrule
		0.0 & 0.000 & 11.0 & 0.661 \\
		1.0 & 0.062 & 11.5 & 0.685 \\
		2.0 & 0.122 & 12.0 & 0.730 \\
		3.0 & 0.192 & 12.5 & 0.762 \\
		4.0 & 0.230 & 13.0 & 0.786 \\
		5.0 & 0.300 & 13.5 & 0.813 \\
		6.0 & 0.348 & 14.0 & 0.846 \\
		7.0 & 0.425 & 14.5 & 0.880 \\
		8.0 & 0.480 & 15.0 & 0.906 \\
		9.0 & 0.550 & 16.0 & 0.961 \\
		10.0 & 0.600 & 17.0 & 1.014 \\
		\bottomrule
	\end{tabular}

	/\input{build/tabIB2.tex}
	\label{tab:B}
\end{table}

\subsection{Untersuchung des normalen Zeeman-Effekts}

Zunächst wird der normale Zeeman-Effekt anhand des $\sigma$-Übergangs der roten Cadmiumlinie mit $\lambda=\SI{643.8}{\nano\metre}$ untersucht. 

\subsection{Untersuchung des anormalen Zeeman-Effekts}