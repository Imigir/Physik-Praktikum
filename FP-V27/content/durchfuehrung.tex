
\section{Durchführung}
\label{sec:Durchführung}
Vor Beginn der Messung wird der Elektromagnet geeicht, indem mit Hilfe einer Hall-Sonde das Magnetfeld in Abhängigkeit von der Stromstärke gemessen wird. Das in Abbildung \ref{fig:schema} zu sehende Objektiv und die erste Linse $L_.1$ werden so eingestellt, dass das Spektrum der Cd-Lampe scharf auf den ersten Spalt $S_.1$ abgebildet wird, während die folgende Linse $L_.2$ so kalibriert wird, dass ein möglichst paralleles Strahlenbündel auf das Prisma trifft, dessen Durchmesser jedoch kleiner sein sollte als der des Prismas.
Der Spalt $S_.2$ wird so positioniert, dass nur die roten Spektrallinien hindurch gelassen werden und der Polarisationsfilter wird eingestellt, sodass nur der $\sigma$-Anteil des Lichts hindurch gelassen wird. Mittels Justierung der Linse $L_.4$ wird der Strahl scharf auf die Lummer-Gehrcke-Platte (LGP) abgebildet. Das entstehende Interferenzmuster der wird zweimal aufgenommen.
$S_.2$ wird verschoben, sodass nur der blaue Anteil des Spektrums ihn passiert und der Polarisationsfilter verwendet um nacheinander die $\sigma$- und die $\pi$-Komponente zu betrachten. Nach Justierung von $L_.4$ werden die durch die LGP entstehende Interferenzmuster je zweimal mit der Digitalkamera aufgenommen.
Da die aus der LGP austretenden Strahlen parallel sind, muss die Fokussierung der Kamera auf eine unendlich weit entfernte Quelle gestellt werden.
