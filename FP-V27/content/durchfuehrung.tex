\section{Durchführung}
\label{sec:Durchführung}
Vor Beginn der Messung wird der Elektromagnet geeicht, indem mit Hilfe einer Hall-Sonde das Magnetfeld in Abhängigkeit von der Stromstärke gemessen wird. Das in Abbildung \ref{fig:schema} zu sehende Objektiv und die erste Linse $L_.1$ werden so eingestellt, dass das Spektrum der Cd-Lampe scharf auf den ersten Spalt $S_.1$ abgebildet wird, während die folgende Linse $L_.2$ so kalibriert wird, dass ein möglichst paralleles Strahlenbündel auf das Prisma trifft, dessen Durchmesser jedoch kleiner sein sollte als der des Prismas.
Der Spalt $S_.2$ wird so positioniert, dass nur die rote Spektrallinie hindurchgelassen wird und mittels Justierung der Linse $L_.4$ scharf auf die Lummer-Gehrcke-Platte (LGP) abgebildet. Das entstehende
Interferenzmuster wird zweimal aufgenommen.
$S_.2$ wird verschoben, sodass nur der blaue Anteil des Spektrums ihn passiert und nach Justierung von $L_.4$ das durch die LGP entstehende Interferenzmuster zweimal mit der Digitalkamera aufgenommen.

