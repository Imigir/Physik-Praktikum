
\section{Diskussion}
\label{sec:Diskussion}

In den Abbildungen \ref{fig:TL1} bis \ref{fig:DL4} der Leckratenmessung ist zu erkennen, dass die Werte gut auf einer geraden liegen. Dabei wird bei den ersten drei Messreihen der Turbopumpe zusätzlich eine zweite Ausgleichsrechnung durchgeführt, die zu einer geringeren Abweichung der Werte von den Geraden führen. Der dadurch zustande kommende negative Achsenabschnitt kann dadurch erklärt werden, dass zuerst die Zeit gestartet und danach die Ventile geschlossen wurden, wodurch die Graphen nach rechts verschoben werden.\\
Bei der vierten Messreihe der Turbopumpe (Abbildung \ref{fig:TL4}) und der zweiten der Drehschieberpumpe (Abbildung \ref{fig:DL2}) wird jeweils der letzte Messwert bei der Ausgleichsrechnung vernachlässigt, da es sich hier um Ausreißer zu handeln scheint.\\
Bei den Evakuierungskurven der Pumpen (Abbildungen \ref{fig:TSE} und \ref{fig:TSL} für die Turbopumpe, sowie Abbildungen \ref{fig:DSE} und \ref{fig:DSL} für die Drehschieberpumpe) lässt sich der erwartete exponentielle Abfall beobachten. Bei der Turbopumpe können dabei in der logarithmischen Darstellung drei lineare Bereiche beobachtet werden, während es bei der Drehschieberpumpe zwei sind.\\
Wird das Saugvermögen der Pumpen betrachtet (vergleiche Abbildungen \ref{fig:TGes} und \ref{fig:DGes}), so lässt sich erkennen, dass das die Turbopumpe ihren optimalen Wirkungsbereich zwischen $p=\SI{10e-5}{\milli\bar}$ und $p=\SI{20e-5}{\milli\bar}$ besitzt. Auch sonst zeigt die Abhängigkeit von $S$ zu $p$ den erwarteten Verlauf, wobei das Saugvermögen der Turbopumpe bei höheren Drücken nur etwa halb so groß ist wie das Maximum. Dies könnte daran liegen, dass die Turbopumpe für niedrige Druckbereiche ausgelegt ist.\\
Das maximal bestimmte Saugvermögen der Pumpe von $S_.{TL3,2}=\SI{25(2)}{\litre\per\second}$ ist dabei deutlich geringer, als der vom Hersteller angegebene Wert von $S=\SI{77}{\litre\per\second}$ \cite{V70}. Dies liegt vor allem an der Querschnittverengung B4, da wie in der Theorie beschrieben das Saugvermögen stark vom Rohrdurchmesser abhängt. Somit liegt der ermittelte Wert im erwarteten Bereich.\\
Bei der Drehschieberpumpe ist das Plateau mit $S_.{DE1}=\SI{1(1)}{\litre\per\second}$ gut in der Nähe des vom Hersteller angegebenen Wertes von $S=\SI{1.1}{\litre\per\second}$ \cite{V70}.
Im optimalen Bereich zwischen $p=\SI{0.8}{\milli\bar}$ und $p=\SI{1}{\milli\bar}$ ist der ermittelte Wert mit $S_.{DL2}=\SI{1.4(1)}{\litre\per\second}$ sogar größer als der angegebene Wert. Dies kann dadurch begründet werden, dass zuvor durch die Turbopumpe bereits ein sehr gutes Vakuum geschaffen wurde, bevor die Leckratenmessung der Drehschieberpumpe durchgeführt wurde.   