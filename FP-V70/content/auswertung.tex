\section{Auswertung}
\label{sec:Auswertung}

Die Graphen werden sowohl mit Matplotlib \cite{matplotlib} als auch NumPy \cite{numpy} erstellt. Die Fehlerrechnung wird mithilfe von Uncertainties \cite{uncertainties} durchgeführt.\\
In Tabelle \ref{tab:V} sind die Volumina der Bauteile eingetragen und bei welcher Messung diese verwendet werden. TL bezieht sich auf die Leckratenmessung der Turbopumpe, TE auf die Evakuierungskurve der Turbopumpe, DL auf die Leckratenmessung der Drehschieberpumpe und DE auf die Evakuierungskurve der Drehschieberpumpe. Die Zahl in Klammern hinter diesen Bezeichnungen gibt eine mögliche häufigere Verwendung des Bauteils an. Der Fehler auf die Volumina wird als systematisch angenommen.\\
Es ergeben sich die Volumina der Messungen zu:
\begin{align*}
V_.{TL}	&= \SI{10.2(9)}{\litre}\text{,}\\
V_.{TE}	&= \SI{10.3(9)}{\litre}\text{,}\\
V_.{DL}	&= \SI{11(1)}{\litre}\text{,}\\
V_.{DE}	&= \SI{11(1)}{\litre}\text{.}\\
\end{align*}
Die Fehler der Gesamtvolumina ergeben sich durch Addition der Fehler der Volumina der einzelnen Komponenten. 

\begin{table}
\centering
\caption{Die Werte für die Volumina der Bauteile \cite{V70}.}
\label{tab:tabV}
	\sisetup{table-format=1.2}
	\begin{tabular}{l S[table-format=1.3] @{${}\pm{}$} S[table-format=1.3] l}
		\toprule
		{Bauteil} & \multicolumn{2}{c}{$V/\si{\litre}$} & {Verwendung}\\
		\midrule
		Tank B1 					& 9.5 	& 0.8 	& TL, TE, DL, DE\\
		langer Schlauch S2 			& 0.8 	& 0.1 	& DL, DE\\
		kurzer Schlauch S1 			& 0.087	& 0.011	& DL, DE\\
		T-Stück klein B6 			& 0.013	& 0.002	& DL, DE\\
		T-Stück groß B3		 		& 0.25	& 0.01	& TL(2x), TE(2x), DL(2x), DE(2x)\\
		Kreuzstück klein B5			& 0.016	& 0.002	& DL, DE\\
		Kreuzstück groß B2			& 0.177	& 0.009	& TL, TE ,DL, DE\\
		Kugelventil V2/3/5 offen	& 0.044	& 0.004	& TL, DL(2x), DE\\
		Kugelventil V2/3/5 zu		& 0.005	& 0.001	& TL, TE(2x), DL, DE(2x)\\
		Klappenventil V1 offen		& 0.044	& 0.004	& TE\\
		Klappenventil V1 zu			& 0.022	& 0.002	& TL, DL, DE\\
		Querschnittsverengung B4	& 0.067	& 0.004	& TE\\
		Belüftungsventil D1			& 0		& 0		& TL, DL\\
		\bottomrule
	\end{tabular}

\label{tab:V}
\end{table}

\subsection{Turbomolekularpumpe}

\subsubsection{Bestimmung des Saugvermögens $S$ über die Leckratenmessung}

Die Leckratenmessung wurde bei vier unterschiedlichen Gleichgewichtsdrücken $p_g$ durchgeführt. Dabei werden die zu gleichen Druckmesswerten gemessenen Zeiten $t_i$ mit der Formel für den Mittelwert
\[
\mu_t = \frac{1}{N}\sum_{i=1}^{N}t_i
\]
und dessen Standartabweichung
\[
\sigma_t = \sqrt{\frac{1}{N(N-1)}\sum_{i=1}^{N}(t_i-\mu_t)^2}
\]
gemittelt zu
\begin{equation}
\bar{t} = \mu_t\pm \sigma_t\text{.} \label{eq:tQuer}
\end{equation} 
Dabei entspricht $N$ der Anzahl der durchgeführten Messungen pro Messreihe.
In den Tabellen \ref{tab:TL1} bis \ref{tab:TL4} sind die Messwerte und die zugehörigen Mittelwerte der jeweiligen Messreihen aufgelistet. Der Fehler des Druckes ergibt sich dabei aus der Ungenauigkeit des Messgerätes und der verwendeten linearen Skala zu $10\%$ \cite{V70}.\\
In den Abbildungen \ref{fig:TL1} bis \ref{fig:TL4} ist der Druck $p$ gegen die gemittelte Zeit $\bar{t}$ aufgetragen.
Die linearen ungewichteten Ausgleichsrechnungen der Form
\[
p_i(t) = a_it+b_i
\]
ergeben für $p_g = \SI{2e-4}{\milli\bar}$ die Parameter
\begin{align*}
a_.{TL1} &= \SI{4.4(2)e-4}{\milli\bar\per\second} \text{,}\\
b_.{TL1} &= \SI{0.6(8)e-4}{\milli\bar} \text{,}\\
\end{align*}
für $p_g = \SI{1.4e-4}{\milli\bar}$
\begin{align*}
a_.{TL2} &= \SI{3.01(9)e-4}{\milli\bar\per\second} \text{,}\\
b_.{TL2} &= \SI{-1(1)e-4}{\milli\bar} \text{,}\\
\end{align*}
für $p_g = \SI{1e-4}{\milli\bar}$
\begin{align*}
a_.{TL3} &= \SI{2.26(6)e-4}{\milli\bar\per\second} \text{,}\\
b_.{TL3} &= \SI{-1(1)e-4}{\milli\bar} \\
\end{align*}
und für $p_g = \SI{0.5e-4}{\milli\bar}$ unter Auslassen des letzten Messwertes
\begin{align*}
a_.{TL4} &= \SI{0.676(4)e-4}{\milli\bar\per\second} \text{,}\\
b_.{TL4} &= \SI{0.43(4)e-4}{\milli\bar} \text{.}
\end{align*}
In den Graphen sind die nicht genutzten Messwerte Grau markiert.
Mit Formel \eqref{eq:S2} ergibt sich mit dem Volumen $V_.{TL}$ für das Saugvermögen $S$
\begin{align*}
S_.{TL1} &= \SI{22(3)}{\litre\per\second} \text{,}\\
S_.{TL2} &= \SI{22(3)}{\litre\per\second} \text{,}\\
S_.{TL3} &= \SI{23(3)}{\litre\per\second} \text{,}\\
S_.{TL4}   &= \SI{14(2)}{\litre\per\second} \text{.}
\end{align*}

\newpage
\begin{table}
\centering
\caption{Die Messwerte der Leckratenmessung bei der Turborpumpe mit einem Gleichgewichtsdruck von $p_.g = \SI{2e-4}{\milli\bar}$.}
\label{tab:tabTL1}
	\sisetup{table-format=1.2}
	\begin{tabular}{S[table-format=2.0]@{${}\pm{}$} S[table-format=1.1]S[table-format=2.2]S[table-format=2.2]S[table-format=2.2]S[table-format=2.2] @{${}\pm{}$} S[table-format=1.2]}
		\toprule
		\multicolumn{2}{c}{$p/10^{-4}\si{\milli\bar}$} & {$t_1/\si{\second}$} & {$t_2/\si{\second}$} & {$t_3/\si{\second}$} & \multicolumn{2}{c}{$\bar{t}/\si{\second}$} \\
		\midrule
		 6 & 0.6 & 1.20 & 1.01 & 0.99 & 1.07 & 0.07 \\
		10 & 1.0 & 2.29 & 2.29 & 2.09 & 2.22 & 0.07 \\
		20 & 2.0 & 4.75 & 4.87 & 4.64 & 4.75 & 0.07 \\
		30 & 3.0 & 7.08 & 7.15 & 7.00 & 7.08 & 0.04 \\
		40 & 4.0 & 9.31 & 9.31 & 9.13 & 9.25 & 0.06 \\
		50 & 5.0 & 11.24 & 11.27 & 11.23 & 11.25 & 0.01 \\
		60 & 6.0 & 13.17 & 13.37 & 13.11 & 13.22 & 0.08 \\
		\bottomrule
	\end{tabular}

\label{tab:TL1}
\end{table}

\begin{figure}
\centering
\includegraphics[width=\linewidth-70pt,height=\textheight-70pt,keepaspectratio]{content/images/TL1.png}
\caption{Der Druck $p$ in Abhängigkeit von der mittleren Zeit $\bar{t}$ bei der Leckratenmessung der Turbopumpe  mit $p_.g = \SI{2e-4}{\milli\bar}$.}
\label{fig:TL1}
\end{figure}

\newpage
\begin{table}
\centering
\caption{Die Messwerte der Leckratenmessung bei der Turborpumpe mit einem Gleichgewichtsdruck von $p_.g = \SI{1.4e-4}{\milli\bar}$.}
\label{tab:tabTL2}
	\sisetup{table-format=1.2}
	\begin{tabular}{S[table-format=2.0] @{${}\pm{}$} S[table-format=1.1]S[table-format=2.2]S[table-format=2.2]S[table-format=2.2]S[table-format=2.2] @{${}\pm{}$} S[table-format=1.2]}
		\toprule
		\multicolumn{2}{c}{$p/10^{-4}\si{\milli\bar}$} & {$t_1/\si{\second}$} & {$t_2/\si{\second}$} & {$t_3/\si{\second}$} & \multicolumn{2}{c}{$\bar{t}/\si{\second}$} \\
		\midrule
		 6 & 0.7 & 2.06 & 2.15 & 2.13 & 2.11 & 0.03 \\
		10 & 1.0 & 3.64 & 3.81 & 3.81 & 3.75 & 0.06 \\
		20 & 2.0 & 7.57 & 7.45 & 7.49 & 7.50 & 0.04 \\
		30 & 3.0 & 10.68 & 10.75 & 10.62 & 10.68 & 0.04 \\
		40 & 4.0 & 13.66 & 13.80 & 13.88 & 13.78 & 0.07 \\
		50 & 5.0 & 16.62 & 16.86 & 16.76 & 16.75 & 0.07 \\
		60 & 6.0 & 19.48 & 19.66 & 19.66 & 19.60 & 0.06 \\
		\bottomrule
	\end{tabular}

\label{tab:TL2}
\end{table}

\begin{figure}
\centering
\includegraphics[width=\linewidth-70pt,height=\textheight-70pt,keepaspectratio]{content/images/TL2.png}
\caption{Der Druck $p$ in Abhängigkeit von der mittleren Zeit $\bar{t}$ bei der Leckratenmessung der Turbopumpe  mit $p_.g = \SI{1.4e-4}{\milli\bar}$.}
\label{fig:TL2}
\end{figure}

\newpage
\begin{table}
\centering
\caption{Die Messwerte der Leckratenmessung bei der Turborpumpe mit einem Gleichgewichtsdruck von $p_.g = \SI{1e-4}{\milli\bar}$.}
\label{tab:tabTL3}
	\sisetup{table-format=1.2}
	\begin{tabular}{S[table-format=2.0] @{${}\pm{}$} S[table-format=1.1]S[table-format=2.2]S[table-format=2.2]S[table-format=2.2]S[table-format=2.2] @{${}\pm{}$} S[table-format=1.2]}
		\toprule
		\multicolumn{2}{c}{$p/10^{-4}\si{\milli\bar}$} & {$t_1/\si{\second}$} & {$t_2/\si{\second}$} & {$t_3/\si{\second}$} & \multicolumn{2}{c}{$\bar{t}/\si{\second}$} \\
		\midrule
		 4 & 0.4 & 1.43 & 1.78 & 1.55 & 1.59 & 0.11 \\
		 8 & 0.8 & 3.66 & 4.14 & 3.93 & 3.91 & 0.14 \\
		20 & 2.0 & 9.54 & 10.17 & 9.93 & 9.88 & 0.19 \\
		30 & 3.0 & 14.28 & 14.46 & 14.59 & 14.44 & 0.09 \\
		40 & 4.0 & 18.39 & 18.84 & 18.92 & 18.72 & 0.17 \\
		50 & 5.0 & 22.50 & 22.85 & 23.00 & 22.78 & 0.15 \\
		60 & 6.0 & 26.23 & 26.69 & 26.68 & 26.53 & 0.16 \\
		70 & 7.0 & 29.88 & 30.37 & 30.38 & 30.21 & 0.17 \\
		\bottomrule
	\end{tabular}

\label{tab:TL3}
\end{table}

\begin{figure}
\centering
\includegraphics[width=\linewidth-70pt,height=\textheight-70pt,keepaspectratio]{content/images/TL3.png}
\caption{Der Druck $p$ in Abhängigkeit von der mittleren Zeit $\bar{t}$ bei der Leckratenmessung der Turbopumpe  mit $p_.g = \SI{1e-4}{\milli\bar}$.}
\label{fig:TL3}
\end{figure}

%\newpage
\begin{table}
\centering
\caption{Die Messwerte der Leckratenmessung bei der Turborpumpe mit einem Gleichgewichtsdruck von $p_.g = \SI{0.5e-4}{\milli\bar}$.}
\label{tab:tabTL4}
	\sisetup{table-format=1.2}
	\begin{tabular}{S[table-format=2.0] @{${}\pm{}$} S[table-format=1.1]S[table-format=2.2]S[table-format=2.2]S[table-format=2.2]S[table-format=2.2] @{${}\pm{}$} S[table-format=1.2]}
		\toprule
		\multicolumn{2}{c}{$p/10^{-4}\si{\milli\bar}$} & {$t_1/\si{\second}$} & {$t_2/\si{\second}$} & {$t_3/\si{\second}$} & \multicolumn{2}{c}{$\bar{t}/\si{\second}$} \\
		\midrule
		 2 & 0.2 & 2.25 & 2.15 & 2.41 & 2.27 & 0.08 \\
		 3 & 0.3 & 3.83 & 3.66 & 4.00 & 3.83 & 0.10 \\
		 4 & 0.4 & 5.43 & 5.12 & 5.44 & 5.33 & 0.11 \\
		 5 & 0.5 & 6.94 & 6.61 & 6.92 & 6.82 & 0.11 \\
		 6 & 0.6 & 8.42 & 8.11 & 8.40 & 8.31 & 0.10 \\
		 7 & 0.7 & 9.95 & 9.56 & 9.85 & 9.79 & 0.12 \\
		 8 & 0.8 & 11.36 & 10.99 & 11.23 & 11.19 & 0.11 \\
		 9 & 0.9 & 12.85 & 12.38 & 12.62 & 12.62 & 0.14 \\
		10 & 1.0 & 14.34 & 13.78 & 14.07 & 14.06 & 0.16 \\
		20 & 2.0 & 28.07 & 26.68 & 27.37 & 27.37 & 0.40 \\
		\bottomrule
	\end{tabular}

\label{tab:TL4}
\end{table}

\begin{figure}
\centering
\includegraphics[width=\linewidth-70pt,height=\textheight-70pt,keepaspectratio]{content/images/TL4.png}
\caption{Der Druck $p$ in Abhängigkeit von der mittleren Zeit $\bar{t}$ bei der Leckratenmessung der Turbopumpe  mit $p_.g = \SI{0.5e-4}{\milli\bar}$.}
\label{fig:TL4}
\end{figure}

\subsubsection{Bestimmung des Saugvermögens $S$ über die Evakuierungskurve}

In Tabelle \ref{tab:TS} sind die Messwerte für die Evakuierungskurve, sowie die nach Formel \eqref{eq:tQuer} berechneten Mittelwerte $\bar{t}$ eingetragen.
Dabei entsprechen $p_0=\SI{500(50)e-5}{\milli\bar}$ dem Anfangsdruck und $p_.E=\SI{1.2(2)e-5}{\milli\bar}$ dem Enddruck der Kurve. Der Fehler des Druckes ergibt sich aus der Ungenauigkeit des Messgerätes und der verwendeten linearen Skala zu $10\%$ \cite{V70}. 
Der Fehler der logarithmischen Werte berechnet sich nach der Gaußschen Fehlerfortpflanzung:
\[
\sigma_f = \sqrt{\sum_i\left(\frac{\partial f}{\partial x_i}\sigma_{x_i}\right)^2}\text{.}
\]
Konkret:
\begin{equation}
\sigma_.{ln} = \sqrt{\left(\frac{\sigma_p}{p-p_.E}\right)^2+\left(\frac{\sigma_{p_0}}{p_0-p_.E}\right)^2+\left(\frac{(p-p_0)\cdot\sigma_{p_.E}}{(p-p_.E)(p_0-p_.E)}\right)^2}\text{.} \label{eq:ln}
\end{equation}
Dabei entsprechen die $\sigma_{p_i}$ den jeweiligen Abweichungen der $p_i$.
Die Evakuierungskurve ist in Abbildung \ref{fig:TSE} zu sehen, die logarithmische Darstellung in Abbildung \ref{fig:TSL}.
Lineare Ausgleichsrechnungen der Form
\[
\ln\left(\frac{p(t)-p_.E}{p_0-p_.E}\right) = a_it+b_i
\]
für die ersten vier Messwerte (Ausgleichsgerade 1), den fünften bis achten Messwert (Ausgleichsgerade 2) und den siebten bis zehnten Messwert (Ausgleichsgerade 3) ergeben die Parameter:
\begin{align*}
a_.{TE1} &= \SI{-0.89(2)}{\per\second} \text{,}\\
b_.{TE1} &= \SI{-0.05(5)}{} \text{,}\\
a_.{TE2} &= \SI{-0.51(4)}{\per\second} \text{,}\\
b_.{TE2} &= \SI{-1.9(3)}{} \text{,}\\
a_.{TE3} &= \SI{-0.21(3)}{\per\second} \text{,}\\
b_.{TE3} &= \SI{-4.6(3)}{} \text{.}\\
\end{align*} 
Daraus folgen durch Vergleich der Steigung mit Formel \eqref{eq:S} und dem Volumen $V_.{TE}$ für $S$ die Werte:
\begin{align*}
S_.{TE1} &= \SI{9.1(8)}{\litre\per\second} \text{,}\\
S_.{TE2} &= \SI{5.2(6)}{\litre\per\second} \text{,}\\
S_.{TE3} &= \SI{2.2(3)}{\litre\per\second} \text{.}\\
\end{align*} 

\begin{table}
\centering
\caption{Die Werte für die Evakuierungskurve der Turborpumpe.}
\label{tab:tabTS}
	\sisetup{table-format=1.2}
	\begin{tabular}{S[table-format=3.1] @{${}\pm{}$} S[table-format=2.1]S[table-format=2.1] @{${}\pm{}$} S[table-format=1.1]S[table-format=2.2]S[table-format=2.2]S[table-format=2.2]S[table-format=2.2]S[table-format=2.2]S[table-format=2.2]S[table-format=2.2] @{${}\pm{}$} S[table-format=1.2]}
		\toprule
		\multicolumn{2}{c}{$p/10^{-5}\si{\milli\bar}$} & \multicolumn{2}{c}{$\log\left(\frac{p-p_e}{p_0-p_e}\right)$} & {$t_1/\si{\second}$} & {$t_2/\si{\second}$} & {$t_3/\si{\second}$} & {$t_4/\si{\second}$} & {$t_5/\si{\second}$} & {$t_6/\si{\second}$} & \multicolumn{2}{c}{$\bar{t}/\si{\second}$} \\
		\midrule
		200.0 & 20.0 & -0.9 & 0.1 & 0.86 & 0.98 & 0.90 & 0.93 & 0.81 & 1.03 & 0.92 & 0.03 \\
		40.0 & 4.0 & -2.6 & 0.1 & 2.72 & 2.82 & 2.71 & 2.89 & 2.71 & 2.98 & 2.81 & 0.05 \\
		20.0 & 2.0 & -3.3 & 0.1 & 3.57 & 3.78 & 3.51 & 3.80 & 3.54 & 3.84 & 3.67 & 0.06 \\
		6.0 & 0.6 & -4.6 & 0.2 & 5.55 & 5.69 & 5.50 & 5.66 & 5.47 & 5.82 & 5.61 & 0.05 \\
		4.0 & 0.4 & -5.2 & 0.2 & 6.31 & 6.48 & 6.28 & 6.69 & 6.31 & 6.60 & 6.45 & 0.07 \\
		2.0 & 0.2 & -6.4 & 0.3 & 8.66 & 8.89 & 8.71 & 8.75 & 8.69 & 9.18 & 8.81 & 0.08 \\
		1.8 & 0.2 & -6.7 & 0.4 & 9.82 & 9.80 & 9.76 & 9.66 & 9.40 & 10.24 & 9.78 & 0.11 \\
		1.6 & 0.2 & -7.1 & 0.5 & 11.01 & 11.13 & 11.34 & 10.64 & 10.56 & 11.40 & 11.01 & 0.14 \\
		1.4 & 0.1 & -7.8 & 0.9 & 15.24 & 15.19 & 15.34 & 14.85 & 14.59 & 15.45 & 15.11 & 0.13 \\
		\bottomrule
	\end{tabular}

\label{tab:TS}
\end{table}

\begin{figure}
\centering
\includegraphics[width=\linewidth-70pt,height=\textheight-70pt,keepaspectratio]{content/images/TSE.png}
\caption{Die Evakuierungskurve der Turbopumpe.}
\label{fig:TSE}
\end{figure}

\begin{figure}
\centering
\includegraphics[width=\linewidth-70pt,height=\textheight-70pt,keepaspectratio]{content/images/TSL.png}
\caption{Die logarithmische Evakuierungskurve der Turbopumpe.}
\label{fig:TSL}
\end{figure}

\subsection{Drehschieberpumpe}

\subsubsection{Bestimmung des Saugvermögens $S$ über die Leckratenmessung}

Die Leckratenmessung wurde bei vier unterschiedlichen Gleichgewichtsdrücken $p_.g$ durchgeführt. Dabei werden die zu gleichen Druckmesswerten gemessenen Zeiten $t_i$ mit Formel \eqref{eq:tQuer} gemittelt.
In den Tabellen \ref{tab:DL1} bis \ref{tab:DL4} sind die Messwerte und die zugehörigen Mittelwerte der jeweiligen Messreihen aufgelistet. Der Fehler des Druckes ergibt sich dabei aus der Ungenauigkeit des Messgerätes und der verwendeten logarithmischen Skala zu $20\%$ \cite{V70}.\\
In den Abbildungen \ref{fig:DL1} bis \ref{fig:DL4} ist der Druck $p$ gegen die gemittelte Zeit $\bar{t}$ aufgetragen.
Die linearen ungewichteten Ausgleichsrechnungen der Form
\[
p_i(t) = a_it+b_i
\]
ergeben für $p_.g = \SI{1}{\milli\bar}$ die Parameter
\begin{align*}
a_.{DL1} &= \SI{0.122(2)}{\milli\bar\per\second} \text{,}\\
b_.{DL1} &= \SI{0.91(7)}{\milli\bar} \text{,}\\
\end{align*}
für $p_g = \SI{0.8}{\milli\bar}$ unter Auslassen des letzten Messwertes
\begin{align*}
a_.{DL2} &= \SI{0.098(2)}{\milli\bar\per\second} \text{,}\\
b_.{DL2} &= \SI{0.77(7)}{\milli\bar} \text{,}\\
\end{align*}
für $p_g = \SI{0.4}{\milli\bar}$
\begin{align*}
a_.{DL3} &= \SI{0.034(2)}{\milli\bar\per\second} \text{,}\\
b_.{DL3} &= \SI{0.25(9)}{\milli\bar} \text{,}\\
\end{align*}
und für $p_g = \SI{0.1}{\milli\bar}$
\begin{align*}
a_.{DL4} &= \SI{0.0048(2)}{\milli\bar\per\second} \text{,}\\
b_.{DL4} &= \SI{0.14(2)}{\milli\bar} \text{.}
\end{align*}
In den Graphen sind die nicht genutzten Messwerte Grau markiert.
Mit Formel \eqref{eq:S2}und dem Volumen $V_.{DL}$ ergibt sich für das Saugvermögen $S$:
\begin{align*}
S_.{DL1} &= \SI{1.4(4)}{\litre\per\second} \text{,}\\
S_.{DL2} &= \SI{1.4(4)}{\litre\per\second} \text{,}\\
S_.{DL3} &= \SI{1.0(3)}{\litre\per\second} \text{,}\\
S_.{DL4}   &= \SI{0.5(2)}{\litre\per\second} \text{.}
\end{align*}

\newpage
\begin{table}
\centering
\caption{Die Messwerte der Leckratenmessung bei der Drehschieberpumpe mit einem Gleichgewichtsdruck von $p_.g = \SI{1}{\milli\bar}$.}
\label{tab:tabDL1}
	\sisetup{table-format=1.2}
	\begin{tabular}{S[table-format=2.0] @{${}\pm{}$} S[table-format=1.1]S[table-format=2.2]S[table-format=2.2]S[table-format=2.2]S[table-format=2.2]S[table-format=2.2] @{${}\pm{}$} S[table-format=1.2]}
		\toprule
		\multicolumn{2}{c}{$p\si{\milli\bar}$} & {$t_1/\si{\second}$} & {$t_2/\si{\second}$} & {$t_3/\si{\second}$} & {$t_4/\si{\second}$} & \multicolumn{2}{c}{$\bar{t}/\si{\second}$} \\
		\midrule
		 2 & 0.2 & 8.91 & 9.81 & 10.36 & 9.55 & 9.66 & 0.30 \\
		 4 & 0.4 & 25.87 & 25.15 & 27.01 & 26.21 & 26.06 & 0.39 \\
		 6 & 0.6 & 40.78 & 41.33 & 42.06 & 40.74 & 41.23 & 0.31 \\
		 8 & 0.8 & 57.78 & 58.09 & 56.92 & 57.58 & 57.59 & 0.25 \\
		10 & 1.0 & 74.33 & 75.40 & 76.27 & 73.51 & 74.88 & 0.60 \\
		\bottomrule
	\end{tabular}

\label{tab:DL1}
\end{table}

\begin{figure}
\centering
\includegraphics[width=\linewidth-70pt,height=\textheight-70pt,keepaspectratio]{content/images/DL1.png}
\caption{Der Druck $p$ in Abhängigkeit von der mittleren Zeit $\bar{t}$ bei der Leckratenmessung der Drehschieberpumpe  mit $p_.g = \SI{1}{\milli\bar}$.}
\label{fig:DL1}
\end{figure}

\newpage
\begin{table}
\centering
\caption{Die Messwerte der Leckratenmessung bei der Drehschieberpumpe mit einem Gleichgewichtsdruck von $p_.g = \SI{0.8}{\milli\bar}$.}
\label{tab:tabDL2}
	\sisetup{table-format=1.2}
	\begin{tabular}{S[table-format=2.0] @{${}\pm{}$} S[table-format=1.1]S[table-format=2.2]S[table-format=2.2]S[table-format=2.2]S[table-format=2.2] @{${}\pm{}$} S[table-format=1.2]}
		\toprule
		\multicolumn{2}{c}{$p/\si{\milli\bar}$} & {$t_1/\si{\second}$} & {$t_2/\si{\second}$} & {$t_3/\si{\second}$} & \multicolumn{2}{c}{$\bar{t}/\si{\second}$} \\
		\midrule
		 2 & 0.4 & 12.28 & 11.88 & 12.29 & 12.15 & 0.14 \\
		 4 & 0.8 & 33.98 & 35.09 & 33.67 & 34.25 & 0.44 \\
		 6 & 1.3 & 52.83 & 52.47 & 52.65 & 52.65 & 0.11 \\
		 8 & 1.6 & 73.54 & 73.97 & 73.18 & 73.56 & 0.23 \\
		10 & 2.0 & 76.73 & 76.87 & 75.98 & 76.53 & 0.28 \\
		\bottomrule
	\end{tabular}

\label{tab:DL2}
\end{table}

\begin{figure}
\centering
\includegraphics[width=\linewidth-70pt,height=\textheight-70pt,keepaspectratio]{content/images/DL2.png}
\caption{Der Druck $p$ in Abhängigkeit von der mittleren Zeit $\bar{t}$ bei der Leckratenmessung der Drehschieberpumpe  mit $p_.g = \SI{0.8}{\milli\bar}$.}
\label{fig:DL2}
\end{figure}

\newpage
\begin{table}
\centering
\caption{Die Messwerte der Leckratenmessung bei der Drehschieberpumpe mit einem Gleichgewichtsdruck von $p_.g = \SI{0.4}{\milli\bar}$.}
\label{tab:tabDL3}
	\sisetup{table-format=1.2}
	\begin{tabular}{S[table-format=1.1] @{${}\pm{}$} S[table-format=1.1]S[table-format=2.2]S[table-format=2.2]S[table-format=2.2]S[table-format=2.2] @{${}\pm{}$} S[table-format=1.2]}
		\toprule
		\multicolumn{2}{c}{$p/\si{\milli\bar}$} & {$t_1/\si{\second}$} & {$t_2/\si{\second}$} & {$t_3/\si{\second}$} & \multicolumn{2}{c}{$\bar{t}/\si{\second}$} \\
		\midrule
		0.6 & 0.1 & 8.28 & 9.35 & 8.71 & 8.78 & 0.31 \\
		1.0 & 0.1 & 23.33 & 23.89 & 23.85 & 23.69 & 0.18 \\
		2.0 & 0.2 & 57.47 & 56.65 & 55.97 & 56.70 & 0.43 \\
		4.0 & 0.4 & 113.70 & 113.33 & 112.68 & 113.24 & 0.30 \\
		6.0 & 0.6 & 164.21 & 165.51 & 165.13 & 164.95 & 0.39 \\
		\bottomrule
	\end{tabular}

\label{tab:DL3}
\end{table}

\begin{figure}
\centering
\includegraphics[width=\linewidth-70pt,height=\textheight-70pt,keepaspectratio]{content/images/DL3.png}
\caption{Der Druck $p$ in Abhängigkeit von der mittleren Zeit $\bar{t}$ bei der Leckratenmessung der Drehschieberpumpe  mit $p_.g = \SI{0.4}{\milli\bar}$.}
\label{fig:DL3}
\end{figure}

\begin{table}
\centering
\caption{Die Messwerte der Leckratenmessung bei der Drehschieberpumpe mit einem Gleichgewichtsdruck von $p_.g = \SI{0.1}{\milli\bar}$.}
\label{tab:tabDL4}
	\sisetup{table-format=1.2}
	\begin{tabular}{S[table-format=1.1] @{${}\pm{}$} S[table-format=1.2]S[table-format=2.2]S[table-format=2.2]S[table-format=2.2]S[table-format=2.2] @{${}\pm{}$} S[table-format=1.2]}
		\toprule
		\multicolumn{2}{c}{$p/\si{\milli\bar}$} & {$t_1/\si{\second}$} & {$t_2/\si{\second}$} & {$t_3/\si{\second}$} & \multicolumn{2}{c}{$\bar{t}/\si{\second}$} \\
		\midrule
		0.2 & 0.04 & 9.95 & 10.08 & 10.66 & 10.23 & 0.22 \\
		0.4 & 0.08 & 46.08 & 46.54 & 46.48 & 46.37 & 0.14 \\
		0.6 & 0.12 & 93.01 & 95.29 & 95.11 & 94.47 & 0.73 \\
		0.8 & 0.16 & 141.51 & 141.29 & 142.19 & 141.66 & 0.27 \\
		1.0 & 0.20 & 179.62 & 180.72 & 180.07 & 180.14 & 0.32 \\
		\bottomrule
	\end{tabular}

\label{tab:DL4}
\end{table}

\begin{figure}
\centering
\includegraphics[width=\linewidth-70pt,height=\textheight-70pt,keepaspectratio]{content/images/DL4.png}
\caption{Der Druck $p$ in Abhängigkeit von der mittleren Zeit $\bar{t}$ bei der Leckratenmessung der Drehschieberpumpe  mit $p_.g = \SI{0.1}{\milli\bar}$.}
\label{fig:DL4}
\end{figure}

\subsubsection{Bestimmung des Saugvermögens $S$ über die Evakuierungskurve}

In Tabelle \ref{tab:DS} sind die Messwerte für die Evakuierungskurve, sowie die nach Formel \eqref{eq:tQuer} berechneten Mittelwerte $\bar{t}$ eingetragen.
Dabei entsprechen $p_0=\SI{1013(203)}{\milli\bar}$ dem Anfangsdruck und $p_.E=\SI{2.0(4)e-2}{\milli\bar}$ dem Enddruck der Kurve. Der Fehler des Druckes ergibt sich aus der Ungenauigkeit des Messgerätes und der verwendeten logarithmischen Skala zu $20\%$ \cite{V70}. 
Der Fehler der logarithmischen Werte berechnet sich nach Formel \eqref{eq:ln}.
Die Evakuierungskurve ist in Abbildung \ref{fig:DSE} zu sehen, die logarithmische Darstellung in Abbildung \ref{fig:DSL}.
Lineare Ausgleichsrechnungen der Form
\[
\ln\left(\frac{p(t)-p_.E}{p_0-p_.E}\right) = a_it+b_i
\]
für die ersten elf Messwerte (Ausgleichsgerade 1) und den zwölften bis siebzehnten Messwert (Ausgleichsgerade 2) ergeben die Parameter:
\begin{align*}
a_.{DE1} &= \SI{-0.093(2)}{\per\second} \text{,}\\
b_.{DE1} &= \SI{-0.08(8)}{} \text{,}\\
a_.{DE2} &= \SI{-0.0619(8)}{\per\second} \text{,}\\
b_.{DE2} &= \SI{-2.51(8)}{} \text{,}\\
\end{align*} 
Daraus folgen mit Formel \eqref{eq:S} und dem Volumen $V_.{DE}$ für $S$ die Werte:
\begin{align*}
S_.{DE1} &= \SI{1.0(1)}{\litre\per\second} \text{,}\\
S_.{DE2} &= \SI{0.69(6)}{\litre\per\second} \text{.}\\
\end{align*}

\begin{figure}
\centering
\includegraphics[width=\linewidth-70pt,height=\textheight-70pt,keepaspectratio]{content/images/DSE.png}
\caption{Die Evakuierungskurve der Drehschieberpumpe.}
\label{fig:DSE}
\end{figure}

\begin{figure}
\centering
\includegraphics[width=\linewidth-70pt,height=\textheight-70pt,keepaspectratio]{content/images/DSL.png}
\caption{Die logarithmische Evakuierungskurve der Drehschieberpumpe.}
\label{fig:DSL}
\end{figure}

\begin{table}
\centering
\caption{Die Werte für die Evakuierungskurve der Drehschieberpumpe.}
\label{tab:tabDS}
	\sisetup{table-format=1.2}
	\begin{tabular}{S[table-format=3.2] @{${}\pm{}$} S[table-format=2.2]S[table-format=3.1] @{${}\pm{}$} S[table-format=1.1]S[table-format=3.2]S[table-format=3.2]S[table-format=3.2]S[table-format=3.2]S[table-format=3.2]S[table-format=3.2]@{${}\pm{}$} S[table-format=1.2]}
		\toprule
		\multicolumn{2}{c}{$p/\si{\milli\bar}$} & \multicolumn{2}{c}{$\log\left(\frac{p-p_e}{p_0-p_e}\right)$} & {$t_1/\si{\second}$} & {$t_2/\si{\second}$} & {$t_3/\si{\second}$} & {$t_4/\si{\second}$} & {$t_5/\si{\second}$} & \multicolumn{2}{c}{$\bar{t}/\si{\second}$} \\
		\midrule
		100.00 & 20.00 & -2.3 & 0.3 & 18.81 & 21.62 & 21.50 & 21.49 & 22.65 & 21.21 & 0.64 \\
		60.00 & 12.00 & -2.8 & 0.3 & 28.25 & 30.40 & 29.56 & 29.04 & 29.66 & 29.38 & 0.36 \\
		40.00 & 8.00 & -3.2 & 0.3 & 34.20 & 35.60 & 34.94 & 34.60 & 35.09 & 34.89 & 0.24 \\
		20.00 & 4.00 & -3.9 & 0.3 & 41.58 & 42.91 & 42.67 & 42.14 & 42.74 & 42.41 & 0.24 \\
		10.00 & 2.00 & -4.6 & 0.3 & 48.64 & 50.71 & 50.13 & 49.57 & 50.15 & 49.84 & 0.35 \\
		8.00 & 1.60 & -4.8 & 0.3 & 51.02 & 52.75 & 52.53 & 51.62 & 52.26 & 52.04 & 0.32 \\
		6.00 & 1.20 & -5.1 & 0.3 & 54.26 & 55.95 & 55.67 & 55.18 & 55.54 & 55.32 & 0.29 \\
		4.00 & 0.80 & -5.5 & 0.3 & 57.89 & 59.43 & 59.56 & 58.78 & 59.44 & 59.02 & 0.31 \\
		2.00 & 0.40 & -6.2 & 0.3 & 64.57 & 66.21 & 66.08 & 65.64 & 66.19 & 65.74 & 0.31 \\
		1.00 & 0.20 & -6.9 & 0.3 & 71.90 & 73.67 & 73.49 & 73.12 & 73.60 & 73.16 & 0.33 \\
		0.80 & 0.16 & -7.2 & 0.3 & 74.34 & 75.85 & 76.00 & 75.51 & 75.98 & 75.54 & 0.31 \\
		0.60 & 0.12 & -7.5 & 0.3 & 78.49 & 80.16 & 80.20 & 79.46 & 80.18 & 79.70 & 0.33 \\
		0.40 & 0.08 & -7.9 & 0.3 & 85.43 & 87.42 & 86.81 & 86.62 & 87.16 & 86.69 & 0.34 \\
		0.20 & 0.04 & -8.6 & 0.3 & 98.56 & 99.93 & 99.71 & 99.35 & 99.56 & 99.42 & 0.24 \\
		0.10 & 0.02 & -9.4 & 0.3 & 110.69 & 111.87 & 111.74 & 110.95 & 111.57 & 111.36 & 0.23 \\
		0.08 & 0.02 & -9.7 & 0.3 & 116.74 & 117.48 & 116.94 & 116.99 & 116.93 & 117.02 & 0.12 \\
		0.06 & 0.01 & -10.1 & 0.4 & 129.17 & 129.04 & 128.40 & 127.59 & 128.66 & 128.57 & 0.28 \\
		\bottomrule
	\end{tabular}

\label{tab:DS}
\end{table}

\subsection{Zusammenfassung} 

In den Abbildungen \ref{fig:TGes} und \ref{fig:DGes} sind die Ergebnisse für das Saugvermögen der jeweiligen Pumpe gegen den Druck aufgetragen. Die Bezeichnungen in der Legende beziehen sich dabei auf die Indizes der Werte.\\
Die Werte aus der Leckratenmessung sind als Punkte dargestellt, die Werte aus der Evakuierungskurve als gestrichelte Linie zwischen dem Anfangs- und Endpunkt des jeweiligen Druckbereichs. Da der Druckbereich der TS1- und DS1-Linie vergleichsweise sehr groß ist, sind die Graphen rechts abgeschnitten. 

\begin{figure}
\centering
\includegraphics[width=\linewidth-70pt,height=\textheight-70pt,keepaspectratio]{content/images/TGes.png}
\caption{Das Saugvermögen $S$ der Turbomolekularpumpe in Abhängigkeit vom Druck $p$.}
\label{fig:TGes}
\end{figure}

\begin{figure}
\centering
\includegraphics[width=\linewidth-70pt,height=\textheight-70pt,keepaspectratio]{content/images/DGes.png}
\caption{Das Saugvermögen $S$ der Drehschieberpumpe in Abhängigkeit vom Druck $p$.}
\label{fig:DGes}
\end{figure}