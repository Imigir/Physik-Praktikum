
\section{Theorie}
\label{sec:Theorie}

\subsection{Das Vakuum}
Ein Vakuum ist ein Raum in dem keine oder sehr wenige Teilchen enthalten sind, sodass ein hindurch fliegendes Teilchen näherungsweise keine Wechselwirkungen (Stöße) mit seiner Umgebung erfährt. Das wird durch die mittlere freie Weglänge beschrieben, die angibt welche Strecke ein Teilchen in einem Medium zurücklegen kann ohne mit diesem zu Stoßen. Benötigt wurde ein Vakuum beispielsweise in frühen Glühlampen. In der Physik findet es viele Anwendungen vor allem in Teilchenbeschleunigern damit die dort beschleunigten Teilchenpakete nicht durch Kollisionen mit Gasteilchen verloren gehen.
Ein vollkommenes Vakuum zu erzeugen ist nicht möglich, doch es gibt Pumpen die verschiedene Arten von Vakuen erzeugen können.