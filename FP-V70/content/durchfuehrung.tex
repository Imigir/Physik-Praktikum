\section{Durchführung}
\label{sec:Durchführung}

Zur Entgasung des Rezipienten wird dieser bei laufender Turbopumpe mit einem Heißluftföhn erhitzt um mögliche adsorbierte Wasserdampfrückstände zu beseitigen.

\subsection{Die Drehschieberpumpe}

Zur Messung der p(t)-Kurve der Drehschieberpumpe wird mit einem Belüftungsventil der Druck im Inneren des Rezipienten auf Normaldruck erhöht. Bei laufendem Pumpenbetrieb wird das Ventil wieder geschlossen und mit dem Pirani-Vakuummeter der Druck in Abhängigkeit von der Zeit gemessen.
Die Messung wird fünf Mal durchgeführt und der Enddruck $p_.E$ bestimmt.\newline
Zur Leckratenmessung wird ein Nadelventil angebracht und über dieses ein Gleichgewichtsdruck $p_.g$ im $\SI{1}{\milli\bar}$-Bereich eingestellt. Die Drehschieberpumpe wird über ein Ventil abgeschiebert und der Druckanstieg in Abhängigkeit von der Zeit am Pirani-Vakuummeter abgelesen.
Die Messung wird für vier verschiedene $p_.g$ jeweils drei Mal durchgeführt.

\subsection{Turbomolekularpumpe}

Nach Erzeugen des Vorvakuums durch die Drehschieberpumpe wird die Turbopumpe bis zur vollen Drehzahl von $\SI{1350}{\hertz}$ hochgefahren.
Zur Bestimmung der p(t)-Kurve wird bei laufender Pumpe, mit einem Nadelventil ein Druck von $p_.0=\SI{5e-3}{\milli\bar}$ eingestellt und nach schließen des Ventils auf dem Glühkathoden-Vakuummeter der Druckabfall in abhängig von der Zeit abgelesen. Die Messung wird fünf Mal durchgeführt.\newline
Für die Leckratenmessung wird mit dem Nadelventil ein Gleichgewichtsdruck zwischen $5$ und $\SI{20e-5}{\milli\bar}$ eingestellt und nach Abschiebern der Pumpe der Druckanstieg in Abhängigkeit von der Zeit vom Glühkathoden-Vakuummeter abgelesen. Um Schäden am Glühkathoden-Vakuummeter zu vermeiden, wird dieses am Ende der Messung rechtzeitig ausgeschaltet und am weniger empfindlichen Kaltkathoden-Vakuummeter abgelesen, wann wieder ein ausreichend gutes Vakuum erreicht ist. Die Messung wird für 4 verschiedene Gleichgewichtsdrücke je drei Mal durchgeführt.


%☺☻☺
% ♥