\section{Diskussion}
\label{sec:Diskussion}

In Tabelle \ref{tab:Ergebnisse} finden sich die berechneten Ergebnisse und ihre Abweichung zu den Literaturwerten.\newline
Die Bragg-Bedingung, dass das Verhältnis von Zählrohrwinkel $\alpha$ zum Kristallwinkel $\theta$ 2:1 betragen muss, um die maximale Impulsrate aufzunehmen, konnte bestätigt werden. 
Der Grenzwinkel $\theta_.{gr}$, die minimale Wellenlänge $\lambda_.{min}$ und die maximale Energie $E_.{kin,max}$ der Elektronen konnten ebenfalls ziemlich genau bestimmt werden. Die hier auftretenden geringen Fehler sind darauf zurückzuführen, dass die Schrittweite von $\Delta\theta=\SI{0,1}{\degree}$ unzureichend klein gewählt wurde.\newline
Das Auflösungsvermögen ist mit $\Delta E=\SI{130,7}{\eV}$ bis $\SI{155}{\eV}$ ausreichend gut, um den statistischen Fehler zu vernachlässigen, da sich die im $\SI{10}{\kilo\eV}$-Bereich befinden.
Die Abweichungen bei den Abschirmkoeffizienten des Kupfers sind darauf zurückzuführen, dass die Grundzustandsenergie nur durch die $K_.{\beta}$-Energie genähert wurde. Der auffallend große Fehler von $\sigma_.{L_.{Cu}}$ von $\delta \sigma_.{L_.{Cu}}=\SI{-36,48}{\percent}$ lässt darauf schließen, dass $E_.{K_.{\alpha}}$ unzureichend bestimmt wurde.\newline
Die $K$-Kanten und $\sigma_.K$ der leichten Elemente ließen sich gut bestimmen.\newline
Der relativ große Fehler der bestimmten Rydberg-Energie $R_.{\infty}$ lässt sich dadurch erklären, dass beim Moseley-Gesetz $E_.K\propto z_.{eff}$ also proportional zur verminderten Kernladungszahl ist und nicht $E_.K\propto z$.
Der Fehler des Abschirmkoeffizienten $\sigma_.{L_.{Bi}}$ lässt sich ebenfalls durch eine zu große Schrittweite erklären.

\begin{table}
\centering
\caption{Die berechneten Werte und ihre Abweichungen von den Literaturwerten}
\label{tab:Ergebnisse}
	\sisetup{table-format=1.2}
	\begin{tabular}{c ccc}
		\toprule
		{Wert}&{gemessen}&{Referenzwert}&{Abweichung} \\
		\midrule
		$\alpha_{max}$ & \SI{28,1}\,\si{\degree} & \SI6{28}\,\si{\degree} & \SI{0,3}\,\si{\percent} \\
		$\theta_{gr}$ & \SI{5}\,\si{\degree} & \SI{5,04}\,\si{\degree} & \SI{-0,9}\,\si{\percent} \\
		$\lambda_{min}$ & \SI{35,1}\,\si{\pico\metre} & \SI{35,4}\,\si{\pico\metre}  & \SI{-1,13}\,\si{\percent} \\
		$E_{kin,max}$ & \SI{35316}\,\si{\eV} & \SI{35000}\,\si{\eV} & \SI{0,9}\,\si{\percent} \\
		$\Delta E_{alpha}$ & \SI{130,7}\,\si{\eV} & - & - \\
		$\Delta E_{beta}$ & \SI{155,0}\,\si{\eV} & - & - \\
		$\sigma_{Cu_K}$ & \SI{3,28} & \SI{3,31} & \SI{-0,76}\,\si{\percent}  \\
		$\sigma_{Cu_{L}}$ & \SI{13,16} & \SI{20,72} & \SI{-36,48}\,\si{\percent} \\
		$\sigma_{Cu_{M}}$ & \SI{29} & \SI{26,64} & \SI{8,87}\,\si{\percent} \\
		$E_{K_{Br}}$ & \SI{13282}\,\si{\eV} & \SI{13470}\,\si{\eV} & \SI{-1,40}\,\si{\percent} \\
		$E_{K_{Sr}}$ & \SI{15988}\,\si{\eV} & \SI{16090}\,\si{\eV} & \SI{-0,64}\,\si{\percent} \\
		$E_{K_{Zn}}$ & \SI{9650}\,\si{\eV} & \SI{9650}\,\si{\eV} & \SI{0,0}\,\si{\percent} \\
		$E_{K_{Zr}}$ & \SI{17903}\,\si{\eV} & \SI{17970}\,\si{\eV} & \SI{-0,37}\,\si{\percent} \\
		$\sigma_{K_{Br}}$ & \SI{3,75} & \SI{3,53} & \SI{6,23}\,\si{\percent} \\
		$\sigma_{K_{Sr}}$ & \SI{3,71} & \SI{3,66} & \SI{1,37}\,\si{\percent} \\
		$\sigma_{K_{Zn}}$ & \SI{3,36} & \SI{3,36} & \SI{0,0}\,\si{\percent} \\
		$\sigma_{K_{Zr}}$ & \SI{3,72} & \SI{3,65} & \SI{1,92}\,\si{\percent} \\
		$R_{\infty}$ & \SI{16,87\pm 0,32}\,\si{\eV} & \SI{13,6}\,\si{\eV} & \SI{24,04}\,\si{\percent} \\
		$\sigma_{L_{Bi}}$ & \SI{3,31} & \SI{3,58} & \SI{-7,54}\,\si{\percent} \\
		\bottomrule
	\end{tabular}

\label{tab:Ergebnisse}
\end{table}