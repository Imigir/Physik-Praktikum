\section{Auswertung}
\label{sec:Auswertung}


Die Graphen werden sowohl mit Matplotlib \cite{matplotlib} als auch NumPy \cite{numpy} erstellt. Die Fehlerrechnung wird mithilfe von Uncertainties \cite{uncertainties} durchgeführt.
Bei allen Messungen werden die Beschleunigungsspannung bei $U_.B=\SI{35}{\kilo\volt}$ und der Emissionsstrom bei $I_.E=\SI{1}{\milli\ampere}$ konstant eingestellt.

\subsection{Überprüfung der Bragg-Bedingung}
Der Kristallwinkel $\theta$ wird bei $\theta= \SI{14}{\degree}$ konstant gehalten, während der Winkel $\alpha$ des Geiger-Müller-Zählrohrs (GMZ) zwischen $\SI{26}{\degree}$ und $\SI{30}{\degree}$ mit einem Winkelzuwachs von $\Delta \alpha = \SI{0,1}{\degree}$ in $\Delta t = \SI{5}{\second}$ variiert wird.
Die aufgenommenen Messwerte sind in Tabelle \ref{tab:tab1} zu finden, der aufgezeichnete Graph ist in Abbildung \ref{fig:bragg}
Der Winkel bei dem das Maximum der Impulsrate $N$ erreicht ist, lässt sich ablesen zu $\alpha_.{max}=\SI{28,1}{\degree}$

\begin{table}
\centering
\caption{Die aufgenommenen Messwerte zur Überprüfung der Bragg-Bedingung}
\label{tab:tabBragg}
	\sisetup{table-format=1.2}
	\begin{tabular}{S[table-format=2.1]S[table-format=3.0]}
		\toprule
		{$\alpha/\si{\degree}$} & {$N/\si{1\per\second}$} \\
		\midrule
		26.0 &  28 \\
		26.1 &  38 \\
		26.2 &  37 \\
		26.3 &  33 \\
		26.4 &  34 \\
		26.5 &  35 \\
		26.6 &  43 \\
		26.7 &  45 \\
		26.8 &  43 \\
		26.9 &  58 \\
		27.0 &  57 \\
		27.1 &  68 \\
		27.2 &  71 \\
		27.3 &  86 \\
		27.4 & 100 \\
		27.5 &  97 \\
		27.6 &  99 \\
		27.7 & 126 \\
		27.8 & 119 \\
		27.9 & 131 \\
		28.0 & 130 \\
		28.1 & 132 \\
		28.2 & 130 \\
		28.3 & 115 \\
		28.4 & 122 \\
		28.5 & 105 \\
		28.6 & 106 \\
		28.7 &  94 \\
		28.8 &  84 \\
		28.9 &  75 \\
		29.0 &  67 \\
		29.1 &  55 \\
		29.2 &  52 \\
		29.3 &  45 \\
		29.4 &  45 \\
		29.5 &  34 \\
		29.6 &  33 \\
		29.7 &  34 \\
		29.8 &  32 \\
		29.9 &  30 \\
		30.0 &  29 \\
		\bottomrule
	\end{tabular}

\label{tab:tab1}
\end{table}

\subsection{Das Emissionsspektrum}
\subsubsection{Die minimale Wellenlänge}
In Abbildung \ref{fig:emission} ist das aufgenommene Emissionsspektrum der Röntgenröhre im Bereich von $\theta=\SI{4}{\degree}$ bis $\SI{26}{\degree}$ zu sehen. Die zugehörigen Werte finden sich in Tabelle \ref{tab:tab2}.
Der Grenzwinkel lässt sich aus Abbildung \ref{fig:brems} ablesen zu 
\[\theta_.{gr}=\SI{5}{\degree}\]
und damit die minimale Wellenlänge nach Formel \eqref{eq:} zu
\[
\lambda_.{min}=\SI{35,1}{\pico\metre}\text{.}
\]
und die maximale kinetische Energie 
\[
E_.{kin,max}=\SI{35316}{\eV}
\] 
Die Literaturwerte berechnet sich mit der Elektronenladung $e_.0$, dem Planckschen Wirkungsquantum $h$ und der Lichtgeschwindigkeit im Vakuum $c$ zu
\[
\lambda_.{min,theo}=\frac{hc}{e_.0U_.B}=\SI{35,4}{\pico\metre}
\]
und
\[
E_.{max,theo}=e_.0U_.B=\SI{35000}{\eV}
\]

\subsubsection{Das Auflösungsvermögen}

In Abbildung \ref{fig:Peak} ist eine Vergrößerung der Peaks der $K_.{\alpha}$- und $K_.{\beta}$-Linie zu sehen.
Zur Bestimmung der Halbwertsbreite werden die Werte für $\theta$ der Geraden
\begin{align*}
g_.1(\theta)&=\SI{6540}{\per\second\per\degree}\cdot\theta-\SI{129126}{\per\second}\text{,}\\
g_.2(\theta)&=\SI{-3990}{\per\second\per\degree}\cdot\theta+\SI{81082}{\per\second}
\end{align*}
gesucht bei denen gilt $g_.{1/2}(\theta)=\frac{N_.{\alpha,max}}{2}=\SI{540}{\per\second}$
und die bei denen für die Geraden
\begin{align*}
g_.3(\theta)&=\SI{20140}{\per\second\per\degree}\cdot\theta-\SI{442454}{\per\second}\text{,}\\
g_.4(\theta)&=\SI{-15600}{\per\second\per\degree}\cdot\theta+\SI{352019}{\per\second}
\end{align*}
gilt $g_.{3/4}(\theta)=\frac{N_.{\beta,max}}{2}=\SI{1929}{\per\second}$
Daraus ergeben sich die Werte
\begin{align*}
\theta_.3&=\theta_.{\alpha-}=\SI{22,06}{\degree}\text{,}\\
\theta_.4&=\theta_.{\alpha+}=\SI{22,44}{\degree}\text{,}\\
\theta_.1&=\theta_.{\beta-}=\SI{19,83}{\degree}\text{,}\\
\theta_.2&=\theta_.{\beta+}=\SI{20,19}{\degree}
\end{align*}
und mit Gleichung \ref{eq:E} für das Auflösungsvermögen
\begin{align*}
\Delta E_.{\alpha}&=\SI{130,7}{\eV}\\
\Delta E_.{\beta}&=\SI{155,0}{\eV}
\end{align*}

\subsubsection{Die Abschirmkoeffizienten von Cu}

Die Maxima der $K_.{\alpha}$- und $K_.{\beta}$-Linie lassen sich ablesen zu
\begin{align*}
\theta_.{\alpha,max}&=\SI{22,2}{\degree}\text{,}\\
\theta_.{\beta,max}&=\SI{20,0}{\degree}\text{.}
\end{align*}
Mit Gleichung \ref{eq:E} ergeben sich so
\begin{align*}
E_.{K_.{\alpha}}&=\SI{8641}{\eV}\text{,}\\
E_.{K_.{\beta}}&=\SI{9000}{\eV}\text{.}
\end{align*}
Damit lassen sich nun die Abschirmkoeffizienten $\sigma_.K$,$\sigma_.L$ und $\sigma_.M$ mithilfe von Gleichung \ref{eq:} und der Kernladungszahl des Kupfers $z=29$ bestimmen zu
\begin{align*}
\sigma_.K&=z-\sqrt{\frac{E_.{K_.{\beta}}}{R_.{\inf}}}=3,28\text{,}\\
\sigma_.L&=z-\sqrt{\left(z-\sigma_.K\right)^2-\frac{E_.{K_.{\alpha}}}{R_.{\inf}}}=13,16\text{,}\\
\sigma_.M&=z-\sqrt{\left(z-\sigma_.K\right)^2-\frac{E_.{K_.{\beta}}}{R_.{\inf}}}=29
\end{align*}

\subsection{Das Absorptionsspektrum}

