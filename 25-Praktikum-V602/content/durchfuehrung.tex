
\section{Durchführung}
\label{sec:Durchführung}

Die Messung wird mit dem Programm measure durchgeführt.
Bei allen Messungen wird die Beschleunigungsspannung der Röntgenröhre auf $U_.B = \SI{35}{\kilo\electronvolt}$ und der Emissionsstrom auf $I = \SI{1}{\milli\ampere}$ gestellt. Im Messprogramm werden die für den Versuchsteil benötigten Werte eingestellt und die Messung wird gestartet. Die Ergebnisse werden gespeichert und exportiert.

\subsection{Überprüfung der Bragg-Bedingung}

Zum Überprüfen der Bragg-Bedingung wird ein fester Kristallwinkel von $\theta = \SI{14}{\degree}$ eingestellt. Der Messbereich des Geiger-Müller-Zählrohrs wird auf einen Bereich von $\alpha_.{GM} = \SI{26}{\degree}$ bis $\alpha_.{GM} = \SI{30}{\degree}$ eingeschränkt. Der Winkelzuwachs soll $\Delta\alpha = \SI{1}{\degree}$ betragen.
Aus den Messdaten wird das Maximum der Kurve ermittelt und mit dem Theoriewert verglichen.

\subsection{Das Emissionsspektrum der Kupfer-Röntgenröhre}

Für die folgenden Messungen wird im Programm der $2:1$ Koppelmodus ausgewählt, sodass das Geiger-Müller-Zählrohr immer doppelt so weit ausgelenkt wird wie der LiF-Kristall (vergleiche Abbildung \ref{fig:Aufbau}).
Das Röntgenspektrum wird in $\Delta\theta = \SI{2}{\degree}$-Schritten im Bereich $\SI{4}{\degree} \leq \theta \leq \SI{26}{\degree}$ aufgenommen. Die Integrationszeit $\Delta t$ wird auf $\Delta t=\SI{5}{\second}$ eingestellt.\\ 
Es werden Die $K_\alpha$- und $K_\beta$-Linie, sowie der Bremsberg bezeichnet und die minimale Wellenlänge und die daraus folgende minimale Energie bestimmt.
Von der $K_\alpha$- und $K_\beta$-Linie wird die Halbwertsbreite berechnet und daraus das Auflösungsvermögen der Apparatur bestimmt.\\
Aus den Energien der $K_\alpha$- und $K_\beta$-Linie werden die Abschirmkonstanten $\sigma_.K$, $\sigma_.L$ und $\sigma_.M$ bestimmt. Dazu soll die Grundzustandsenergie zur Bestimmung von $\sigma_.K$ durch die Energie $E_{K_\beta}$ der $K_\beta$-Linie genähert werden. Mithilfe von $\sigma_.K$ können $\sigma_.L$ aus der Energiedifferenz der $K_\alpha$-Linie (Übergang von der L- in die K-Schale) und $\sigma_.M$ aus der Energiedifferenz der $K_\beta$-Linie (Übergang von der M- in die K-Schale) bestimmt werden.

\subsection{Das Absorptionsspektrum verschiedener Materialien}

Für vier verschiedene leichte Absorber mit Kernladungszahlen $Z$ zwischen $30$ und $50$ und einen schweren Absorber mit $Z \geq 70$ wird das Absorptionsspektrum in $\Delta\theta=\SI{0.1}{\degree}$-Schritten bei einer Messzeit von $\Delta t=\SI{20}{\second}$ aufgenommen. Der Messbereich wird ungefähr $\SI{1}{\degree}$ um den erwarteten Theoriewert der zu untersuchenden Kante eingestellt. Bei den leichten Absorbern werden die K-Kanten, bei dem schweren die L-Kante untersucht.\\
Für die leichten Absorber wird die Absorptionsenergie der jeweiligen K-Kanten berechnet und daraus die Abschirmzahl $\sigma_K$ des jeweiligen Absorbers bestimmt. Mit den Ergebnissen wird ein $\sqrt{E_K}$-$Z$-Diagramm erstellt und aus der Steigung die Rydbergkonstante berechnet.\\
Für den schweren Absorber wird mit den Messergebnissen aus der L-II- und L-III-Kante die Abschirmkonstante $\sigma_L$ bestimmt.
Alle Ergebnisse werden mit den Literaturwerten verglichen.