\section{Durchführung}
\label{sec:Durchführung}

Bei laufender Turbopumpe wird der Rezipient mit einem Heißluftföhn erhitzt um mögliche adsorbierte Wasserdampfrückstände zu beseitigen.

\subsection{Die Drehschieberpumpe}

Für die Messreihen mit der Drehschieberpumpe (Drehschieber Pfeiffer Duo 004A) sind die Ventile V1 und V5 geschlossen. Das Ventil V2 ist geöffnet.
Zur Messung der p(t)-Kurve der Drehschieberpumpe wird mit dem Belüftungsventil V3 der Druck im Inneren des Rezipienten auf Normaldruck erhöht. Das Nadelventil D1 ist für diesen Versuchsteil abgebaut. Bei laufendem Pumpenbetrieb wird das Ventil V3 wieder geschlossen und mit dem Pirani-Vakuummeter (Thermovac TR205 von Leybold-Heraeus) der Druck in Abhängigkeit von der Zeit gemessen.
Die Messung wird fünf Mal durchgeführt und der Enddruck $p_.E$ bestimmt.\newline
Zur Leckratenmessung wird ein Nadelventil D1 angebracht und über dieses ein Gleichgewichtsdruck $p_.g$ im $\SI{1}{\milli\bar}$-Bereich eingestellt. Die Drehschieberpumpe wird über ein Ventil direkt an der Pumpe abgeschiebert und der Druckanstieg in Abhängigkeit von der Zeit am Pirani-Vakuummeter abgelesen.
Die Messung wird für vier verschiedene $p_.g$ jeweils drei Mal durchgeführt.

\subsection{Turbomolekularpumpe}

Nach Erzeugen des Vorvakuums durch die Drehschieberpumpe wird die Turbopumpe (Turbo SST81 der Firma ILMVAC) bis zur vollen Drehzahl von $\SI{1350}{\hertz}$ hochgefahren.
Die Ventile V1 und V5 sind geöffnet. Das Ventil V2 ist geschlossen.
Zur Bestimmung der p(t)-Kurve wird bei laufender Pumpe, mit dem Nadelventil D1 ein Druck von $p_.0=\SI{5e-3}{\milli\bar}$ eingestellt und nach schließen des Ventils auf dem Glühkathoden-Vakuummeter (Glühkathode Ionivac IM210 von Leybold-Heraeus) der Druckabfall in abhängig von der Zeit abgelesen. Die Messung wird fünf Mal durchgeführt.\newline
Für die Leckratenmessung wird mit dem Nadelventil ein Gleichgewichtsdruck zwischen $5$ und $\SI{20e-5}{\milli\bar}$ eingestellt und nach Abschiebern der Pumpe mit V1 der Druckanstieg in Abhängigkeit von der Zeit vom Glühkathoden-Vakuummeter abgelesen. Um Schäden am Glühkathoden-Vakuummeter zu vermeiden, wird dieses am Ende der Messung rechtzeitig ausgeschaltet und am weniger empfindlichen Kaltkathoden-Vakuummeter abgelesen, wann wieder ein ausreichend gutes Vakuum erreicht ist. Die Messung wird für 4 verschiedene Gleichgewichtsdrücke je drei Mal durchgeführt.


%☺☻☺
% ♥