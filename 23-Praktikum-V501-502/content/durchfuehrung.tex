
\section{Durchführung}
\label{sec:Durchführung}

Mit dem ersten Versuchsaufbau für das elektrische Feld wird für fünf verschiedene Beschleunigungsspannungen $U_.B$ zwischen $\SI{180}{\volt}$ und $\SI{500}{\volt}$ die Proportionalität zwischen Verschiebung $D$ und Ablenkspannung $U_.d$ überprüft.
Dafür wird $U_.d$ so eingestellt, dass die neun äquidistanten Linien des Koordinatennetzes nacheinander abgetastet werden. $U_d$ kann am Voltmeter abgelesen werden.\\
Mit dem zweiten Versuchsaufbau für das elektrische Feld werden unter Variation der Sägezahnspannungsfrequenz $f_.S$ stehende Bilder der Sinusspannung mit Frequenz $f_.W$ erzeugt. Es werden die Fälle für 
\[
f_.S = nf_.W \text{ ;} n=0.5,1,2,3
\]
realisiert. $f_.S$ kann am Frequenzzähler abgelesen werden. 
Bei konstantem $U_.B$ wird die Amplitude der Sinusspannung vermessen.\\
Bei der Messung im magnetischen Feld wird die Helmholtzspule so ausgerichtet, dass das Erdmagnetfeld keinen Einfluss mehr hat. Die Richtung des Erdmagnetfeldes kann durch ein Deklinatorium-Inklinatorium bestimmt werden. Bei konstanten Beschleunigungsspannungen von $\SI{250}{\volt}$ und $\SI{350}{\volt}$ wird die Verschiebung $D$ des Strahls in Abhängigkeit von der Stromstärke $I$ gemessen. Dazu wird bei $I=\SI{0}{\ampere}$ der Leuchtpunkt auf die oberste Linie des Rasters gelenkt und anschließend werden unter Variation von $I$ die restlichen Linien nacheinander abgetastet. Die Flussdichte $B$ in der Mitte des Helmholtzspulenpaares beträgt:
\begin{equation}
B = \mu_0\frac{8}{\sqrt{125}}\frac{NI}{R}\text{.}
\end{equation}
Dabei ist $N$ die Windungszahl, $R$ der Spulenradius und $\mu_0=4\pi\cdot 10^{-7}\si{\volt\second\per\ampere\per\metre}$ die Vakuumpermeabilität.\\
Zuletzt wird die Stärke des Erdmagnetfeldes bestimmt, indem bei einer möglichst geringen Beschleunigungsspannung von $U_.B = \SI{150}{\volt}$ und bei $I=\SI{0}{\ampere}$ die Vorrichtung parallel zum Erdmagnetfeld ausgerichtet wird. Es wird sich die Position des Leuchtpunktes gemerkt. Anschließend wird die Vorrichtung senkrecht zum Erdmagnetfeld ausgerichtet und die Verschiebung des Leuchtpunktes durch das Erdmagnetfeld wird mit der Helmholtzspule kompensiert. Es wird die hierzu benötigte Stromstärke notiert.
Für die Bestimmung der Totalintensität wird mithilfe des Inklinatoriums der Inklinationswinkel $\phi$ bestimmt.    