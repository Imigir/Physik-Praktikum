\section{Diskussion}
\label{sec:Diskussion}

\begin{table}
\centering
\caption{Die Messwerte und ihre Abweichungen zum Literaturwert}
\label{tab:Ergebnisse}
	\sisetup{table-format=1.2}
	\begin{tabular}{c ccc}
		\toprule
		{Wert}&{gemessen}&{Referenzwert}&{Abweichung} \\
		\midrule
		$\alpha_{max}$ & \SI{28,1}\,\si{\degree} & \SI6{28}\,\si{\degree} & \SI{0,3}\,\si{\percent} \\
		$\theta_{gr}$ & \SI{5}\,\si{\degree} & \SI{5,04}\,\si{\degree} & \SI{-0,9}\,\si{\percent} \\
		$\lambda_{min}$ & \SI{35,1}\,\si{\pico\metre} & \SI{35,4}\,\si{\pico\metre}  & \SI{-1,13}\,\si{\percent} \\
		$E_{kin,max}$ & \SI{35316}\,\si{\eV} & \SI{35000}\,\si{\eV} & \SI{0,9}\,\si{\percent} \\
		$\Delta E_{alpha}$ & \SI{130,7}\,\si{\eV} & - & - \\
		$\Delta E_{beta}$ & \SI{155,0}\,\si{\eV} & - & - \\
		$\sigma_{Cu_K}$ & \SI{3,28} & \SI{3,31} & \SI{-0,76}\,\si{\percent}  \\
		$\sigma_{Cu_{L}}$ & \SI{13,16} & \SI{20,72} & \SI{-36,48}\,\si{\percent} \\
		$\sigma_{Cu_{M}}$ & \SI{29} & \SI{26,64} & \SI{8,87}\,\si{\percent} \\
		$E_{K_{Br}}$ & \SI{13282}\,\si{\eV} & \SI{13470}\,\si{\eV} & \SI{-1,40}\,\si{\percent} \\
		$E_{K_{Sr}}$ & \SI{15988}\,\si{\eV} & \SI{16090}\,\si{\eV} & \SI{-0,64}\,\si{\percent} \\
		$E_{K_{Zn}}$ & \SI{9650}\,\si{\eV} & \SI{9650}\,\si{\eV} & \SI{0,0}\,\si{\percent} \\
		$E_{K_{Zr}}$ & \SI{17903}\,\si{\eV} & \SI{17970}\,\si{\eV} & \SI{-0,37}\,\si{\percent} \\
		$\sigma_{K_{Br}}$ & \SI{3,75} & \SI{3,53} & \SI{6,23}\,\si{\percent} \\
		$\sigma_{K_{Sr}}$ & \SI{3,71} & \SI{3,66} & \SI{1,37}\,\si{\percent} \\
		$\sigma_{K_{Zn}}$ & \SI{3,36} & \SI{3,36} & \SI{0,0}\,\si{\percent} \\
		$\sigma_{K_{Zr}}$ & \SI{3,72} & \SI{3,65} & \SI{1,92}\,\si{\percent} \\
		$R_{\infty}$ & \SI{16,87\pm 0,32}\,\si{\eV} & \SI{13,6}\,\si{\eV} & \SI{24,04}\,\si{\percent} \\
		$\sigma_{L_{Bi}}$ & \SI{3,31} & \SI{3,58} & \SI{-7,54}\,\si{\percent} \\
		\bottomrule
	\end{tabular}

\label{tab:Fehler}
\end{table}

\noindent Die berechneten Werte und ihre Abweichung vom Literaturwert sind in Tabelle \ref{tab:Fehler} zu finden.
Die Abweichung der Steigung $a$ vom theoretischen Wert ist mit $\delta a=\SI{53,4}{\percent}$ ziemlich groß. Dies könnte daran liegen, dass der vorhandene Kondensator mit größer werdendem Plattenabstand $d$ durch einen vollständig parallelen Plattenkondensator angenähert wurde. Ein weiterer Faktor könnte gewesen sein, dass die Ablenkung $D$ per Augenmaß bestimmt wurde.
Für die Frequenz der Sinusspannung $f_.{sin}$ ist als Referenzwert nur der Bereich von $f_.{sin}=80$ bis $\SI{90}{\hertz}$ angegeben. Der gemessene Wert liegt mit einer Abweichung von $\delta f =\SI{-0,89}{\percent}$ ziemlich nah am unteren Bereichswert. 
Die Abweichungen der beiden bei den Beschleunigungsspannungen $U_.B=\SI{250}{\volt}$ und $\SI{350}{\volt}$ bestimmten spezifischen Ladungen der Elektronen könnten auch darauf zurückzuführen sein, dass die Ablenkung, die in die Berechnung von $\frac{e_.0}{m_.e}$ in der vierten Potenz eingeht, unzureichend bestimmt wurde.
\newline Die Abweichung des bestimmten Erdmagnetfelds lässt sich dadurch erklären, dass der Inklinationswinkel $\phi$ mit dem gegebenen Inklinatorium nur unzureichend genau bestimmt werden konnte. Außerdem ist es möglich das beim Drehen des Experiments nicht die exakte Nord-Süd- bzw. Ost-West-Richtung eingestellt wurde, sodass nicht das Maximum der Auslenkung vermessen wurde.