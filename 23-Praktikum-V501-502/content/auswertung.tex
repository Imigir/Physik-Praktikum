\section{Auswertung}
\label{sec:Auswertung}


Die Graphen werden sowohl mit Matplotlib \cite{matplotlib} als auch NumPy \cite{numpy} erstellt. Die Fehlerrechnung wird mithilfe von Uncertainties \cite{uncertainties} durchgeführt.

\subsection{Elektronen im E-Feld}
\subsubsection{Empfindlichkeiten}
\label{sec:empf}
In Tabelle \ref{tab:Elek} sind die Messwerte für die Ablenkspannungen $U_.d$ und Ablenkungen $D$ bei den Beschleunigungsspannungen
\begin{align*}
U_.B&=\SI{310}{\volt},\\
&=\SI{280}{\volt},\\
&=\SI{260}{\volt},\\
&=\SI{240}{\volt},\\
&=\SI{220}{\volt}
\end{align*}
zu finden. Die mittels der linearen Regression \[D(U_.d)=\alpha U_.d + \beta\] bestimmten Graphen sind in Abbildung \ref{fig:Elek} zu sehen.
Es ergeben sich die Parameter
\begin{align*}
\alpha_.{310}&=\SI{1,050(6)e-3}{\metre\per\volt},\\
\alpha_.{280}&=\SI{1,146(5)e-3}{\metre\per\volt},\\
\alpha_.{260}&=\SI{1,227(9)e-3}{\metre\per\volt},\\
\alpha_.{240}&=\SI{1,338(12)e-3}{\metre\per\volt},\\
\alpha_.{220}&=\SI{1,460(12)e-3}{\metre\per\volt},\\
\beta_.{310}&=\SI{-11,19(11)e-3}{\metre},\\
\beta_.{280}&=\SI{-10,45(9)e-3}{\metre},\\
\beta_.{260}&=\SI{-9,93(14)e-3}{\metre},\\
\beta_.{240}&=\SI{-10,11(16)e-3}{\metre},\\
\beta_.{220}&=\SI{-9,83(15)e-3}{\metre}
\end{align*}
In Abbildung \ref{fig:Elek2} sind die Empfindlichkeiten $\alpha$ gegen $\frac{1}{U_.B}$ aufgetragen.
Eine lineare Ausgleichsrechnung der Form
\[\alpha\left(\frac{1}{U_.B}\right)=a\cdot\frac{1}{U_.B}+b\]
ergibt die Parameter
\begin{align*}
a&=\SI{0,313(8)}{\metre},\\
b&=\SI{3,3(31)e-5}{\metre\per\volt}
\end{align*}
Der Theoriewert für die Steigung ergibt sich durch Koeffizientenvergleich zu
\[
a_.{theo}=\frac{pL}{2d}=\SI{0,204}{\metre},
\]
mit der Plattenlänge $p=\SI{19e-3}{\metre}$, dem Plattenabstand $d=\SI{6,65e-3}{\metre}$ und dem Abstand zum Leuchtschirm $L=\SI{143e-3}{\metre}$\cite{V501}.
\begin{table}
\centering
\caption{Messwerte der Ablenkspannungen bei verschiedenen Beschleunigungsspannungen $U_.B$}
\label{tab:tabElek}
	\sisetup{table-format=1.2}
	\begin{tabular}{S[table-format=2.2]S[table-format=2.1]S[table-format=2.1]S[table-format=2.1]S[table-format=2.1]S[table-format=2.1]}
		\toprule
		{$D/\si{\metre}$} & {$U_.{d,310}/\si{\volt}$} & {$U_.{d,280}/\si{\volt}$} & {$U_.{d,260}/\si{\volt}$} & {$U_.{d,240}/\si{\volt}$} & {$U_.{d,220}/\si{\volt}$} \\
		\midrule
		25.00 & 34.0 & 30.7 & 28.0 & 25.7 & 23.5 \\
		18.75 & 28.5 & 25.4 & 23.3 & 21.6 & 19.4 \\
		12.50 & 22.8 & 20.2 & 18.4 & 17.1 & 15.5 \\
		6.25 & 16.8 & 14.7 & 13.5 & 12.3 & 11.3 \\
		-0.00 & 10.9 & 9.2 & 8.4 & 8.0 & 6.9 \\
		-6.25 & 4.8 & 3.8 & 3.3 & 3.1 & 2.6 \\
		-12.50 & -1.2 & -1.6 & -2.2 & -1.8 & -1.6 \\
		-18.75 & -7.3 & -7.5 & -7.4 & -6.6 & -6.3 \\
		-25.00 & -13.4 & -12.8 & -12.5 & -11.4 & -10.7 \\
		\bottomrule
	\end{tabular}

\label{tab:Elek}
\end{table}
\begin{figure}
\centering
\includegraphics[width=\linewidth-70pt,height=\textheight-70pt,keepaspectratio]{content/images/GraphEle.pdf}
\caption{Das Verhältnis von Ablenkung $D$ zu Ablenkspannung $U_.d$ bei verschiedenen Beschleunigungsspannungen $U_.B$}\label{fig:Elek}
\end{figure}
\begin{figure}
\centering
\includegraphics[width=\linewidth-70pt,height=\textheight-70pt,keepaspectratio]{content/images/GraphEle6.pdf}
\caption{Das Verhältnis der Empfindlichkeit $\alpha$ zu $\frac{1}{U_.B}$ bei verschiedenen Beschleunigungsspannungen $U_.B$}\label{fig:Elek2}
\end{figure}

\subsubsection{Sinusfrequenz einer Kathodenstrahlröhre}
Für die Sinusspannung $f_.{sin}$ der Kathodenstrahlröhre gilt das Verhältnis zur Sägezahnspannung $f_.{Säge}$:
\[
f_.{sin} = n\cdot f_.{Säge}
\]
Die gemessenen Sägezahn- und berechneten Sinusspannungen für verschiedene Verhältnisse bei $U_.B=\SI{350}{\volt}$ sind in Tabelle \ref{tab:Freq} zu finden.
Mit der Formel für den Mittelwert
\[
\bar{f}_.{sin}=\frac{1}{N}\sum_i^Nf_.{sin,i}
\]
und der Standardabweichung 
\[
\sigma_.f=\sqrt{\frac{1}{N^2-N}\sum_i^N\left(f_.{sin,i}-\bar{f}_.{sin}\right)^2}
\]
Ergibt sich eine Sinusfrequenz von
\[
\bar{f}_.{sin}=\SI{79,29(1)}{\hertz}
\]
\noindent Die Amplitude $A$ der Sinusfrequenz wurde gemessen zu
\[
A=\SI{9,375e-3}{\metre}
\]
Da für die Ablenkung durch die Kathodenstrahlröhre gilt 
\[
A\approx\alpha U_.{d,max} = \left(\frac{a}{U_.B}\right)U_.{d,max}\text{,}
\]
lässt sich der Scheitelwert $S$ der Ablenkspannung bestimmen zu
\[
S=U_.{d,max}=\frac{AU_.B}{a}=\SI{10,5(3)}{\volt}\text{.}
\]
Der Fehler berechnet sich dabei nach der Gaußschen Fehlerfortpflanzung
\begin{align*}
\sigma_.S&=\sqrt{\left(\frac{\partial S}{\partial a}\right)^2\sigma^2_.a}\\
&=\frac{AU_.B}{a^2}\cdot \sigma_.a
\end{align*}
$a$ und $\sigma_.a$ sind dabei aus Abschnitt \ref{sec:empf} bekannt.
\begin{table}
\centering
\caption{Messwerte der Sägezahnspannungen}
\label{tab:tabFreq}
	\sisetup{table-format=1.2}
	\begin{tabular}{S[table-format=1.1]S[table-format=2.2]S[table-format=2.2]}
		\toprule
		{$n$} & {$f_.{Säge}/\si{\hertz}$} & {$f_.{sin}/\si{\hertz}$} \\
		\midrule
		2.0 & 39.66 & 79.32 \\
		1.0 & 79.27 & 79.27 \\
		0.5 & 158.55 & 79.28 \\
		0.3 & 237.84 & 79.28 \\
		\bottomrule
	\end{tabular}

\label{tab:Freq}
\end{table}

\subsection{Elektronen im B-Feld}
\subsubsection{Die spezifische Ladung der Elektronen}
In Tabelle \ref{tab:Mag} befinden sich die bei $U_.B=\SI{350}{\volt}$ und $U_.B=\SI{250}{\volt}$ aufgenommenen Messwerte für die zur Ablenkung $D$ benötigten Ströme $I$.
$L=\SI{0,175}{\metre}$\cite{V501} ist dabei die Strecke zwischen Beschleunigerelektrode und Leuchtschirm auf der die Elektronen dem Magnetfeld ausgesetzt sind.
Die magnetische Flussdichte $B$ der Helmholtz-Spulen wird über Gleichung \eqref{eq:B} bestimmt.
Eine lineare Regression der Form \[\frac{D}{D^2+L^2}(B)=a_.{mag}B+b_.{mag}\] ergibt die Parameter
\begin{align*}
a_.{mag,250}&=\SI{8,57(23)e3}{1\per\tesla\per\metre}\text{,}\\
a_.{mag,350}&=\SI{7,32(13)e3}{1\per\tesla\per\metre}\text{,}\\
b_.{mag,250}&=\SI{74(24)e-3}{1\per\metre}\text{,}\\
b_.{mag,350}&=\SI{33(16)e-3}{1\per\metre}
\end{align*}
In Abbildung \ref{fig:Mag} sind die zugehörigen Graphen zu sehen.
Durch Koeffizientenvergleich mit Formel \eqref{eq:e0/me} zeigt sich, dass sich die spezifische Ladung des Elektrons über
\[
\frac{e_.0}{m_.e}=8a_.{mag}^2U_.B
\]
und damit 
\begin{align*}
\left(\frac{e_.0}{m_.e}\right)_.{250}&=\SI{1,47(8)e11}{\coulomb\per\kilogram}\\
\left(\frac{e_.0}{m_.e}\right)_.{350}&=\SI{1,50(5)e11}{\coulomb\per\kilogram}
\end{align*}
Die Fehler berechnen sich dabei über die Gaußsche Fehlerfortpflanzung zu
\begin{align*}
\sigma_.{\frac{e_.0}{m_.e}}&=\sqrt{\left(\frac{\partial\left(\frac{e_.0}{m_.e}\right)}{\partial a_.{mag}}\right)^2\sigma^2_.a}\\
                           &=16a_.{mag}U_.B\sigma_.a
\end{align*}
Der Literaturwert ergibt sich mit der Ladung des Elektrons $e_.0\approx\SI{1,60e-19}{\coulomb}$ und der Masse des Elektrons $m_.e\approx\SI{9,11e-31}{\kilogram}$ zu 
\[
\frac{e_.0}{m_.e}\approx\SI{1,76e11}{\coulomb\per\kilogram}
\]
\begin{table}
\centering
\caption{Messwerte der Ablenkungen bei verschiedenen Beschleunigungsspannungen $U_.B$}
\label{tab:tabMag}
	\sisetup{table-format=1.2}
	\begin{tabular}{S[table-format=2.2]S[table-format=1.2]S[table-format=1.2]S[table-format=2.2]S[table-format=1.2]S[table-format=2.2]}
		\toprule
		{$D/10^{-3}\si{\metre}$} & {$\frac{D}{D^2+L^2}/\si{\metre}$} & {$I_.{250}/\si{\ampere}$} & {$B_.{250}/10^{-5}\si{\tesla}$} & {$I_.{350}/\si{\ampere}$} & {$B_.{350}/10^{-5}\si{\tesla}$} \\
		\midrule
		6.25 & 0.20 & 0.30 & 1.91 & 0.41 & 2.61 \\
		12.50 & 0.41 & 0.63 & 4.02 & 0.83 & 5.29 \\
		18.75 & 0.61 & 0.95 & 6.06 & 1.20 & 7.65 \\
		25.00 & 0.80 & 1.30 & 8.29 & 1.60 & 10.20 \\
		31.25 & 0.99 & 1.60 & 10.20 & 2.00 & 12.75 \\
		37.50 & 1.17 & 1.95 & 12.44 & 2.40 & 15.31 \\
		43.75 & 1.34 & 2.35 & 14.99 & 2.85 & 18.17 \\
		50.00 & 1.51 & 2.70 & 17.22 & 3.20 & 20.41 \\
		\bottomrule
	\end{tabular}

\label{tab:Mag}
\end{table}
\begin{figure}
\centering
\includegraphics[width=\linewidth-70pt,height=\textheight-70pt,keepaspectratio]{content/images/GraphMag1.pdf}
\caption{Das Verhältnis von $\frac{D}{D^2+L^2}$ zu magnetischer Flussdichte $B$ bei verschiedenen Beschleunigungsspannungen $U_.B$}\label{fig:Mag}
\end{figure}
\subsubsection{Bestimmung des Erdmagnetfelds}
Bei Aufstellung der Kathodenstrahlröhre in Ost-West-Richtung ergibt sich eine Verschiebung des Leuchtflecks um $D=\SI{6,25e-3}{\metre}$ im Vergleich zur Nord-Südrichtung.
Um diese zu kompensieren wird ein Strom von $I_.{hor}=\SI{0,235}{\ampere}$ benötigt. Der Inklinationswinkel beträgt $\phi=\SI{70}{\degree}$. Mit Gleichung \eqref{eq:B}
und dem Verhältnis von Gesamtflussdichte zum horizontalen Anteil
\[
B_.{ges}= \frac{B_.{hor}}{\cos(\phi)}
\]
ergibt sich eine magnetische Flussdichte des Erdmagnetfelds von
\[
B_.{Erde}=\SI{43,82e-6}{\tesla}
\]
