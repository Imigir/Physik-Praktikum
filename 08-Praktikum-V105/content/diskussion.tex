
\section{Diskussion}
\label{sec:Diskussion}

Die Werte des magnetischen Moments sind bei den ersten beiden Messreihen nahezu identisch, was darauf schließen lässt, dass diese sehr nah am wirklichen Wert liegen. Dies wird durch die geringen Fehler unterstützt. Dabei ist die zweite Methode leichter durchzuführen, da hier das Magnetfeld nicht so präzise wie bei der ersten Methode eingestellt werden muss. Der Fehler, der durch das ungenaue Messen der Periodendauer zustande kommt, wird dabei durch Messung mehrerer Periodendauern und anschließendem mitteln kompensiert.\newline 
Wie zu erwarten, weicht das Ergebnis der dritten Messung von den anderen beiden ab. Dies liegt daran, dass hier die meisten Messfehler gemacht werden können. Die Frequenz ist nur schwer auf den vorgegebenen Wert zu bringen und nimmt während der Präzession zu schnell ab. Zudem kann die Präzessionsdauer nur befriedigend genau bestimmt werden, welches an den teilweise größeren Abweichungen dieser Werte zu erkennen ist. Das Ergebnis dieser Messung könnte durch längere und häufigere Messungen verbessert werden. Zudem könnte durch ein besseres Luftkissen die Abnahme der Frequenz verringert werden. 