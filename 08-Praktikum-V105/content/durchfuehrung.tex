\section{Durchführung}
\label{sec:Durchführung}

Während allen Versuchen wird das Luftkissen eingeschaltet.
Nach jedem Messvorgang wird das externe Magnetfeld abgeschaltet, um die Spulen nicht zu überlasten.

\subsection{Messung über Gravitation} 

Die Stange wird mit der Masse in den Stiel an der Billardkugel gesteckt.
Es wird für zehn verschiedene Abstände der Masse zur Kugel die Stromstärke $I$ gemessen, bei der sich die Stange im Gleichgewicht befindet.

\subsection{Messung über Periodendauer}

Die Billardkugel wird am Stiel ausgelenkt und für zehn verschiedene Stromstärken wird die Periodendauer der resultieren Schwingung gemessen.

\subsection{Messung über Präzession}

Die Kugel wird in Rotation versetzt und aus der aufrechten Lage ausgelenkt.
Anschließend wird der weiße Punkt auf dem Stiel stroboskopisch mit einer Frequenz $\nu = \SI{5}{\hertz}$ beleuchtet. 
Sobald dieser stationär erscheint und damit die Frequenz des Stroboskops erreicht ist, wird das Magnetfeld eingeschaltet und für zehn verschiedene Stromstärken die Periodendauer der entstehenden Präzession gemessen.

