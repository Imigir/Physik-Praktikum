\section{Auswertung}
\label{sec:Auswertung}

Die Graphen werden sowohl mit Matplotlib \cite{matplotlib} als auch NumPy \cite{numpy} erstellt. Die Fehlerrechnung wird mithilfe von Uncertainties \cite{uncertainties} durchgeführt.

\subsection{Bestimmung der Reichweite von Alpha-Strahlung Messung 1}

Der Abstand der Probe wird auf $a=\SI{27}{\milli\metre}$ eingestellt.
In Abbildung \ref{fig:1N} ist mit den Werten aus Tabelle \ref{tab:1} die Impulsrate $N_1$ der ersten Messung gegen die effektive Laufzeit $x_.{eff1}$ aufgetragen, welche nach Formel \eqref{eq:x} bestimmt wird.
Mittels einer linearen Ausgleichsrechnung der Form $N(x_.{eff})=a x_.{eff} +b$ im Bereich von $x_.{eff1} = \SI{17.3}{\milli\metre}$ bis $x_.{eff1} = \SI{22.7}{\milli\metre}$ ergibt sich die mittlere Reichweite $R_m$:
\begin{align*}
a	&= \SI{-114(4)}{\becquerel\per\milli\per\metre}\text{,}\\
b	&= \SI{2.68(9)e3}{\becquerel}\text{,}\\
R_m	&= \frac{N_{1/2}-b}{a} = \SI{19(1)}{\milli\metre}\text{.}
\end{align*}
Dabei ist $N_{1/2} = \SI{530}{\becquerel}$ und der Fehler $\sigma_{R_m}$ von $R_m$ berechnet sich mit der Gaußschen Fehlerfortpflanzung nach:
\begin{equation*}
\sigma_{R_m} = \sqrt{\left(\frac{N_{1/2}-b}{a^2}\sigma_a\right)^2+\left(\frac{1}{a}\sigma_b\right)^2}\text{.}
\end{equation*} 
Mithilfe von $R_m$ lässt sich die Energie $E_\alpha$ der Alphateilchen mit Formel \eqref{eq:Rm} bestimmen zu:
\begin{equation*}
E_\alpha = \SI{3.3(1)}{\mega e\volt}\text{.}
\end{equation*} 
Der Fehler $\sigma_{E_\alpha}$ bestimmt sich dabei nach:
\begin{equation*}
\sigma_{E_\alpha} = \frac{2}{9,3}\left(\frac{R_m}{3,1}\right)^{-\frac{1}{3}}\sigma_{R_m}\text{.}
\end{equation*} 
In Abbildung \ref{fig:1E} ist mit den Werten aus Tabelle \ref{tab:1} die Energie $E_1$ der ersten Messung gegen die effektive Laufzeit $x_.{eff1}$ aufgetragen.
Mittels einer linearen Ausgleichsrechnung der Form $E(x_.{eff})=m_1 x_.{eff} +n_1$ ergibt sich für den Energieverlust $\frac{.dE_1}{.dx}$:
\begin{align*}
\frac{.dE_1}{.dx}	&= m_1 = \SI{-91(3)}{\mega e\volt\per\metre}\text{,}\\
n_1	&= \SI{3.74(4)e6}{\becquerel}\text{.}
\end{align*}

\begin{table}
	\centering
	\caption{Der Druck $p$ und die Impulsrate $N_1$, sowie die effektiven Längen $x_.{eff1}$ und die bestimmten Energien $E_1$ bei der ersten Messreihe mit einem Abstand zur Probe von $\SI{2.7}{\centi\metre}$.}
	\label{tab:tab1}
	\sisetup{table-format=1.2}
	\begin{tabular}{S[table-format=5.0] @{${}\pm{}$} S[table-format=3.0]S[table-format=3.0] @{${}\pm{}$} S[table-format=1.0]S[table-format=3.0]S[table-format=1.2]S[table-format=3.1] @{${}\pm{}$} S[table-format=1.1]}
		\toprule
		\multicolumn{2}{c}{$N_.{minute}/\si{1\per\minute}$} & \multicolumn{2}{c}{$N/\si{\becquerel}$} & {$U/\si{\volt}$} & {$I/10^{-6}\si{\ampere}$} & \multicolumn{2}{c}{$\Delta Q/10^{8}\mathrm{e}$} \\
		\midrule
		    0 &   0 &   0 & 0 & 300 & 0.00 & 0.0 & 0.0 \\
		15346 & 124 & 256 & 2 & 310 & 0.05 & 12.2 & 0.1 \\
		16073 & 127 & 268 & 2 & 320 & 0.10 & 23.3 & 0.2 \\
		16196 & 127 & 270 & 2 & 330 & 0.15 & 34.7 & 0.3 \\
		16190 & 127 & 270 & 2 & 340 & 0.20 & 46.3 & 0.4 \\
		16592 & 129 & 277 & 2 & 350 & 0.20 & 45.1 & 0.4 \\
		16365 & 128 & 273 & 2 & 360 & 0.25 & 57.2 & 0.4 \\
		16575 & 129 & 276 & 2 & 370 & 0.30 & 67.8 & 0.5 \\
		16543 & 129 & 276 & 2 & 380 & 0.30 & 67.9 & 0.5 \\
		16877 & 130 & 281 & 2 & 390 & 0.35 & 77.7 & 0.6 \\
		16663 & 129 & 278 & 2 & 400 & 0.40 & 89.9 & 0.7 \\
		16648 & 129 & 277 & 2 & 410 & 0.40 & 90.0 & 0.7 \\
		16491 & 128 & 275 & 2 & 420 & 0.45 & 102.2 & 0.8 \\
		16680 & 129 & 278 & 2 & 430 & 0.45 & 101.0 & 0.8 \\
		16783 & 130 & 280 & 2 & 440 & 0.50 & 111.6 & 0.9 \\
		16932 & 130 & 282 & 2 & 450 & 0.55 & 121.6 & 0.9 \\
		16901 & 130 & 282 & 2 & 460 & 0.60 & 132.9 & 1.0 \\
		16976 & 130 & 283 & 2 & 470 & 0.60 & 132.4 & 1.0 \\
		16856 & 130 & 281 & 2 & 480 & 0.65 & 144.4 & 1.1 \\
		16815 & 130 & 280 & 2 & 490 & 0.65 & 144.8 & 1.1 \\
		16936 & 130 & 282 & 2 & 500 & 0.70 & 154.8 & 1.2 \\
		16970 & 130 & 283 & 2 & 510 & 0.80 & 176.5 & 1.4 \\
		16704 & 129 & 278 & 2 & 520 & 0.80 & 179.4 & 1.4 \\
		16949 & 130 & 282 & 2 & 530 & 0.85 & 187.8 & 1.4 \\
		17096 & 131 & 285 & 2 & 540 & 0.90 & 197.1 & 1.5 \\
		17328 & 132 & 289 & 2 & 550 & 0.95 & 205.3 & 1.6 \\
		16883 & 130 & 281 & 2 & 560 & 0.95 & 210.7 & 1.6 \\
		17153 & 131 & 286 & 2 & 570 & 1.00 & 218.3 & 1.7 \\
		16992 & 130 & 283 & 2 & 580 & 1.05 & 231.4 & 1.8 \\
		17148 & 131 & 286 & 2 & 590 & 1.10 & 240.2 & 1.8 \\
		17111 & 131 & 285 & 2 & 600 & 1.10 & 240.7 & 1.8 \\
		16826 & 130 & 280 & 2 & 610 & 1.10 & 244.8 & 1.9 \\
		17411 & 132 & 290 & 2 & 620 & 1.15 & 247.4 & 1.9 \\
		17223 & 131 & 287 & 2 & 630 & 1.20 & 260.9 & 2.0 \\
		17259 & 131 & 288 & 2 & 640 & 1.25 & 271.2 & 2.1 \\
		17281 & 131 & 288 & 2 & 650 & 1.25 & 270.9 & 2.1 \\
		17549 & 132 & 292 & 2 & 660 & 1.30 & 277.4 & 2.1 \\
		17835 & 134 & 297 & 2 & 670 & 1.40 & 294.0 & 2.2 \\
		17923 & 134 & 299 & 2 & 680 & 1.40 & 292.5 & 2.2 \\
		17801 & 133 & 297 & 2 & 690 & 1.40 & 294.5 & 2.2 \\
		17950 & 134 & 299 & 2 & 700 & 1.50 & 312.9 & 2.3 \\
		\bottomrule
	\end{tabular}

	\label{tab:1}
\end{table}
\begin{figure}
	\centering
	\caption{Die Impulsrate $N_1$ aufgetragen gegen die effektive Länge $x_.{eff1}$.}
	\includegraphics[width=\linewidth-70pt,height=\textheight-70pt,keepaspectratio]{content/images/Graph1N.pdf}
	\label{fig:1N}
\end{figure}
\begin{figure}
	\centering
	\caption{Die Energie $E_1$ aufgetragen gegen die effektive Länge $x_.{eff1}$.}
	\includegraphics[width=\linewidth-70pt,height=\textheight-70pt,keepaspectratio]{content/images/Graph1E.pdf}
	\label{fig:1E}
\end{figure}

\subsection{Bestimmung der Reichweite von Alpha-Strahlung Messung 2}

Der Abstand der Probe wird auf $a=\SI{10}{\milli\metre}$ eingestellt.
In Abbildung \ref{fig:2N} ist mit den Werten aus Tabelle \ref{tab:2} die Impulsrate $N_2$ der zweiten Messung gegen die effektive Laufzeit $x_.{eff2}$ aufgetragen, welche nach Formel \eqref{eq:x} bestimmt wird. Eine Ausgleichsrechnung zur Bestimmung von $R_m$ und $E_\alpha$ wie in der ersten Messreihe kann aufgrund der Messdaten nicht durchgeführt werden.\\
In Abbildung \ref{fig:2E} ist mit den Werten aus Tabelle \ref{tab:2} die Energie $E_2$ der zweiten Messung gegen die effektive Laufzeit $x_.{eff2}$ aufgetragen.
Mittels einer linearen Ausgleichsrechnung der Form $E(x_.{eff})=m_2 x_.{eff} +n_2$ ergibt sich für den Energieverlust $\frac{.dE_2}{.dx}$:
\begin{align*}
\frac{.dE_2}{.dx}	&= m_2 = \SI{-118(1)}{\mega e\volt\per\metre}\text{,}\\
n_2	&= \SI{3.98(1)e6}{\becquerel}\text{.}
\end{align*}

\begin{table}
	\centering
	\caption{Der Druck $p$ und die Impulsrate $N_2$, sowie die effektiven Längen $x_.{eff2}$ und die bestimmten Energien $E_2$ bei der zweiten Messreihe mit einem Abstand zur Probe von $\SI{1}{\centi\metre}$.}
	\label{tab:tab2}
	\sisetup{table-format=1.2}
	\begin{tabular}{S[table-format=3.0]S[table-format=3.0]S[table-format=2.1]S[table-format=1.2]}
		\toprule
		{$p/\si{\milli\bar}$} & {$N_.2/\si{\becquerel}$} & {$x_.{eff2}/\si{\milli\metre}$} & {$E_2/\si{\mega e\volt}$} \\
		\midrule
		  0 & 3194 & 0.0 & 4.00 \\
		 50 & 3180 & 0.5 & 3.92 \\
		100 & 3169 & 1.0 & 3.86 \\
		150 & 3150 & 1.5 & 3.81 \\
		200 & 3116 & 2.0 & 3.73 \\
		250 & 3104 & 2.5 & 3.69 \\
		300 & 3089 & 3.0 & 3.65 \\
		350 & 3073 & 3.5 & 3.56 \\
		400 & 3061 & 3.9 & 3.49 \\
		450 & 3046 & 4.4 & 3.44 \\
		500 & 3016 & 4.9 & 3.39 \\
		550 & 3008 & 5.4 & 3.34 \\
		600 & 2994 & 5.9 & 3.29 \\
		650 & 2965 & 6.4 & 3.22 \\
		700 & 2930 & 6.9 & 3.16 \\
		750 & 2909 & 7.4 & 3.11 \\
		800 & 2886 & 7.9 & 3.05 \\
		850 & 2844 & 8.4 & 2.99 \\
		900 & 2818 & 8.9 & 2.93 \\
		950 & 2794 & 9.4 & 2.88 \\
		1000 & 2711 & 9.9 & 2.84 \\
		\bottomrule
	\end{tabular}

	\label{tab:2}
\end{table}
\begin{figure}
	\centering
	\caption{Die Impulsrate $N_2$ aufgetragen gegen die effektive Länge $x_.{eff2}$.}
	\includegraphics[width=\linewidth-70pt,height=\textheight-70pt,keepaspectratio]{content/images/Graph2N.pdf}
	\label{fig:2N}
\end{figure}
\begin{figure}
	\centering
	\caption{Die Energie $E_2$ aufgetragen gegen die effektive Länge $x_.{eff2}$.}
	\includegraphics[width=\linewidth-70pt,height=\textheight-70pt,keepaspectratio]{content/images/Graph2E.pdf}
	\label{fig:2E}
\end{figure}

\subsection{Statistik des Radioaktiven Zerfalls}

Aus den gemessenen Zählraten ergibt sich mit der Formel für den Mittelwert
\begin{equation*}
\mu_N= \frac{1}{n}\sum_{k=1}^n N_k = \SI{671}{\becquerel}
\end{equation*} 
und die Standartabweichung
\begin{equation*}
\sigma_N=\sqrt{\frac{1}{n^2-n}\sum_{k=1}^n (N_k-\bar{N})^2)} = \SI{19}{\becquerel}
\end{equation*} 
für die mittlere Zählrate ein Wert von:
\begin{equation*}
\bar{N}= \SI{671(19)}{\becquerel} \text{.}
\end{equation*} 
Mit diesen Werten ergibt sich die Gaußverteilung gemäß:
\begin{equation*}
G(x) = \frac{1}{\sqrt{2\pi\sigma_N^2}}\exp{\left(-\frac{(x-\mu_N)^2}{2\sigma_N^2}\right)}
\end{equation*} 
und die Poissonverteilung gemäß:
\begin{equation*}
P(x) = \frac{\mu_N^x}{x!}\exp{\left(-\mu_N\right)}\text{.}
\end{equation*} 
Die Häufigkeit eines Pulses ist in den Abbildungen \ref{fig:3_20} bis \ref{fig:3_scott} für verschiedene $\Delta N$ gegen die Impulsrate $N$ aufgetragen. 

\begin{figure}
	\centering
	\caption{Die normierte Häufigkeit aufgetragen gegen die Impulsrate $N$, sowie die zugehörige Poisson- und Gaußverteilung bei einem $\Delta N$ von 20.}
	\includegraphics[width=\linewidth-70pt,height=\textheight-70pt,keepaspectratio]{content/images/Graph3_20.pdf}
	\label{fig:3_20}
\end{figure}
\begin{figure}
	\centering
	\caption{Die normierte Häufigkeit aufgetragen gegen die Impulsrate $N$, sowie die zugehörige Poisson- und Gaußverteilung bei einem $\Delta N$ von 10.}
	\includegraphics[width=\linewidth-70pt,height=\textheight-70pt,keepaspectratio]{content/images/Graph3_10.pdf}
	\label{fig:3_10}
\end{figure}
\begin{figure}
	\centering
	\caption{Die normierte Häufigkeit aufgetragen gegen die Impulsrate $N$, sowie die zugehörige Poisson- und Gaußverteilung bei einem weiteren $\Delta N$.}
	\includegraphics[width=\linewidth-70pt,height=\textheight-70pt,keepaspectratio]{content/images/Graph3_scott.pdf}
	\label{fig:3_scott}
\end{figure}
