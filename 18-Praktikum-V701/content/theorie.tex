\section{Theorie}
\label{sec:Theorie}
\subsection{Entstehung von $\alpha$-Strahlung}

$\alpha$-Teilchen sind Helium-Kerne und entstehen beim Zerfall radioaktiver Kerne.\newline
Im vorliegenden Fall ist dies Americium und seine Zerfallsgleichung ist gegeben durch
\[
\overset{241}{95}Am\rightarrow\overset{237}{95}Np + \overset{4}{2} He^{++}
\]

\subsection{Eigenschaften von $\alpha$-Strahlung}
Beim $\alpha$-Strahlung Durchlaufen eines Mediums verlieren die $\alpha$-Teilchen durch elastische Stöße, sowie durch Ionisations- und Anregungsprozesse Energie.
Für große Teilchenenergien kann dieser Energieverlust dargestellt werden durch die Bethe-Bloch-Gleichung
\begin{equation}
-\frac{\mathrm{d}E_.{\alpha}}{\mathrm{d}x}=\frac{z^2e^4nZ}{4\pi\epsilon_.0m_.ev^2}\ln\left(\frac{2m_.ev^2}{I}\right)\label{eq:dE},
\end{equation}
mit der Ladung $z$ und Geschwindigkeit $v$ der $\alpha$-Strahlung, der Ordnungszahl $Z$, Teilchendichte n und der zur Ionisation benötigten Energie $I$ des Mediums.
Die Reichweite $R$ eines $\alpha$-Teilchens ist gegeben durch
\begin{equation}
R=\int_0^{E_.{\alpha}}\frac{\mathrm{d}E_.{\alpha}}{-\frac{\mathrm{d}E_.{\alpha}}{\mathrm{d}x}}\text{.}
\end{equation}
Da die Anzahl der Stöße, die die $\alpha$-Teilchen auf einer bestimmten Weglänge ausführen, nicht für jedes Teilchen gleich ist, wird die mittlere Reichweite $R_.m$ definiert.
Dies ist der Abstand den $50\%$ der ausgesendeten $\alpha$-Teilchen noch erreichen und lässt sich für Strahlungsenergien $E_.{\alpha}\leqq\SI{2,5e6}{\electronvolt}$ quantitativ berechnen über
\begin{equation}
R_.m=3,1\cdot\sqrt{E^3_.{\alpha}}\label{eq:Rm}\text{.}
\end{equation}
Je mehr Teilchen die $\alpha$-Strahlung durchqueren muss, desto schneller ist ihre Energie durch Stöße aufgebraucht. Es lässt sich daher ein Zusammenhang zwischen dem herrschenden Druck und der Reichweite der Teilchen aufstellen. Die effektive Weglänge $x$ die ein Teilchen zurücklegt, wenn es sich bei Druck $p$ um $x_.0$ fortbewegt, beträgt deshalb
\begin{equation}
x=x_.0\cdot\frac{p}{p_.0}\label{eq:x},
\end{equation}
mit dem Normaldruck $p_.0=\SI{1,013}{\bar}$.

\subsection{Der Halbleiterrsperschichtzähler}
Ein Halbleitersperrschichtzähler besteht aus zwei Schichten - einer Schicht eines p- und eines n-dotierten Leiters. Am Übergang dieser Schichten wandern die Elektronen des n-Leiters zu den positiven Ionen des p-Leiters, wodurch eine ladungsfreie Zone entsteht. Wenn ein ionisierendes Teilchen - hier $\alpha$-Strahlung - in diese Zone erzeugt es Elektronen-Loch-Paare. Diese werden jedoch durch das vorherrschende Raumladungsfeld getrennt. Die dabei abgegebene Energie kann gemessen und über einen Verstärker in Impulse umgewandelt werden. Aufgrund der geringen benötigten Energie die zur Erzeugung der Paare benötigt wird, ist eine sehr genaue Bestimmung der Energie des ankommenden Teilchens möglich.

\subsection{Der Diskriminator}
Der Zähldiskriminator zählt ankommende Impulse und wandelt sie in Spannungen um.
Um Störsignale durch elektrisches Grundrauschen zu verhindern, wird die Diskriminatorschwelle so eingestellt, dass solche Signale ignoriert werden.