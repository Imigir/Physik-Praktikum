
\section{Diskussion}
\label{sec:Diskussion}

Die Ergebnisse der Auswertung sind:
\begin{align*}
R_m			&= \SI{19(1)}{\milli\metre}\text{,}\\
E_\alpha 	&= \SI{3.3(1)}{\mega e\volt}\text{,}\\
\frac{.dE_1}{.dx}	&= \SI{-91(3)}{\mega e\volt\per\metre}\text{,}\\
\frac{.dE_2}{.dx}	&= \SI{-118(1)}{\mega e\volt\per\metre}\text{,}\\
\bar{N}		&= \SI{671(19)}{\becquerel} \text{.}
\end{align*}

Der für die Energie der $\alpha$-Teilchen $E_.{\alpha}$ bestimmte Wert weich vom Literaturwert für die verwendete Probe $\ce{^{241}_{95}Am}$
$E_.{Am}=\SI{5.5}{\megae\volt}\cite{Americium-241}$ um $\delta E = 39,8 \%$ ab.
Bei der zweiten Messung konnte keine mittlere Reichweite und damit auch keine Teilchenenergie bestimmt werden, da kein signifikanter Abfall der Impulsrate zu erkennen war. Dies lässt sich dadurch erklären, dass der bei dieser Messung gewählte Abstand $a=\SI{10}{\milli\metre}$ bereits wesentlich kleiner ist, als der in Messung 1 bestimmte Wert für $R_.m$.
Der in Messung 2 bestimmte Wert für den Energieabfall $\frac{.dE_1}{.dx}$ weicht von dem in Messung 1 berechneten um $\delta \frac{.dE}{.dx}= 22,8\%$ ab.
Zur bestimmten durchschnittlichen Aktivität $\bar{N}$ lässt sich kein Vergleich ziehen, da nicht bekannt ist, wie groß die verwendete Probe war.