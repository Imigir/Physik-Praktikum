\section{Durchführung}
\label{sec:Durchführung}
Die Messzeit $t$ ist bereits um die Totzeit bereinigt.
Zur Bestimmung der Energiekalibrierung des Detektors wird über eine Messzeit von $t=\SI{2723}{\second}$ das Spektrum einer $^{152}.{Eu}$-Probe aufgenommen.
Es wird der Peaklage und -inhalt und daraus die Effizienz de Detektors bestimmt.
Das Spektrum eines $^{137}.{Cs}$-Strahlers wird über eine Messzeit von $t=\SI{4543}{\second}$ aufgenommen und die Energieauflösung des Detektors bestimmt.
Das Spektrum einer unbekannten Quelle wird über einen Zeitraum von $t=\SI{6312}{\second}$ aufgenommen, um sie  als $^{125}.{Sb}$ oder $^{133}.{Ba}$ zu identifizieren und ihre Aktivität zu bestimmen.
Das Spektrum eines unbekannten Minerals wird für $t=\SI{4489}{\second} $ aufgenommen, um die aktiven Nuklide einer Zerfallsreihe über die Verteilung der Peaks zu bestimmen.