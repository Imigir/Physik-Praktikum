\section{Durchführung}
\label{sec:Durchführung}

Zur Bestimmung der Energieeichung des Detektors wird das Spektrum einer über eine Messzeit von $t=$ $^{152}.{Eu}$-Probe aufgenommen.
Es wird der Peaklage und -inhalt und daraus die Effizienz de Detektors bestimmt.
Das Spektrum einer $^{137}.{Cs}$-Strahlers wird über eine Messzeit von $t=$ aufgenommen und die Energieauflösung des Detektors bestimmt.
Das Spektrum einer unbekannten Quelle wird über einen Zeitraum von $t=$ aufgenommen, um sie über die Aktivität als $^{125}.{Sb}$ oder $^{133}.{Ba}$ zu identifizieren.
Das Spektrum eines unbekannten Minerals wird für $t= $ aufgenommen, um die aktiven Nuklide einer Zerfallsreihe über die Verteilung der Peaks zu bestimmen.


%☺♥