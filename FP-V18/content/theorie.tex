\section{Theorie}
\label{sec:Theorie}

Nach dem $\alpha$- oder $\beta$-Zerfall eines instabilen Nuklids, kommt es vor, dass der neu entstehende Kern in einem angeregten Zustand ist. Nach einer kurzen Zeit fällt er unter Aussendung eines Photons, eines $\gamma$-Quants, in seinen Grundzustand zurück.

\subsection{Wechselwirkung von $\gamma$-Strahlung mit Materie}

Trifft ein solches $\gamma$-Quant auf Materie wechselwirkt es, im Gegensatz zu Teilchenstrahlung, im Wesentlichen einmal. Drei Arten von Wechselwirkungen dominieren dabei für unterschiedliche Energien. Die Wahrscheinlichkeit für jede Art von Wechselwirkung lässt sich durch einen Wirkungsquerschnitt $\sigma$ beschreiben, der anschaulich darstellt, dass die eingestrahlten Projektilteilchen (Photonen)auf ein ausgedehntes Ziel, also ein Targetteilchen(Absorberatom), treffen.


\subsubsection{Der Photoeffekt}

Beim Photoeffekt trifft ein Photon auf ein Hüllenelektron und schlägt dieses aus der Bindung des Kerns. Das Photon, das bei diesem Prozess vollständig absorbiert wird, muss dafür mindestens die Bindungsenergie $E_.B$ besitzen, sodass gilt:
\[
E_.{\gamma}>E_.B
\]
Findet die Reaktion in einer inneren Schale statt, fällt ein Elektron aus einem höheren Zustand unter Aussendung eines charakteristischen Photons in das freigewordene Energieniveau. Die freigewordene Energie verbleibt im Medium. Somit kann angenommen werden, dass die vollständige Energie des $\gamma$-Quants deponiert wird, was bei einer Messung des Energiespektrums zu einem scharfen Peak führt.
Für den Wirkungsquerschnitt gilt dabei mit der Kernladungszahl $Z$ des Absorbers:
\[
\sigma_.{Ph}\propto\frac{Z^4}{E^{\delta}_.{\gamma}}
\]
Der Exponent $\delta$ ist abhängig von der Photonenenergie. Für natürlich vorkommende $\gamma$-Strahlung gilt $\delta\approx 3,5$, für Energien $E_.{\gamma}\geq \SI{5}{\mega\electronvolt}$ sinkt er auf $\delta\approx 1$.
Der Wirkungsquerschnitt steigt außerdem immer sprunghaft an, wenn $E_.{\gamma}$ einen Wert erreicht, der ausreicht um ein Elektron aus dem Atom herauszulösen.

\subsubsection{Der Compton-Effekt}

Der Compton-Effekt beschreibt die elastische Streuung eines $\gamma$-Quants an einem als in Ruhe angenommenen Hüllenelektron. Das Photon gibt dabei nur einen Teil der Energie ab und fliegt danach mit veränderter Frequenz abgelenkt weiter. Geschieht dies mit einem Elektron auf einem äußeren Energieniveau, so kann dieses einen großen Teil der Photonenenergie aufnehmen und aus der Bindung herausgeschlagen werden. Bei stärker gebundenen Elektronen wird nur ein kleiner Teil der Energie abgegeben und das Elektron verbleibt im gebundenen Zustand.
Da Energie und Impuls erhalten sind, müssen die Gleichungen
\begin{align*}
h \nu + m_.0 c^2 &= h \nu '+ \gamma m_.0 c^2\\
\frac{h \nu}{c}&= \frac{h \nu '}{c}\cos(\psi_.{\gamma})+\gamma m_.0 v\cos(\psi_.e)\\
0&= \frac{h\nu '}{c}\sin(\psi_.{\gamma})+\gamma m_.0 v\sin(\psi_.e)
\end{align*}
erfüllt sein. Dabei ist $h$ das Plancksche Wirkungsquantum, $c$ die Lichtgeschwindigkeit, $\gamma$ der Lorentzfaktor, $\nu$ und $\nu '$ die Frequenz des Photons vor und nach dem Compton-Effekt und $\psi_.{\gamma}$ und $\psi_.e$ die Winkel des ausfallenden Photons und des Elektrons bezüglich der Einfallsrichtung des Photons.
Umformen und Gleichsetzen ergibt für die Energie des einfallenden und des ausfallenden Photons $E_.{\gamma}=h\nu$ und $E_.{\gamma}'=h\nu '$ mit \[\epsilon=\frac{E_.{\gamma}}{m_.0c^2}\] die Beziehung
\begin{equation}
E_.{\gamma}'= E_.{\gamma}\frac{1}{1+\epsilon\cdot\left(1-\cos(\psi_.{\gamma})\right)}\text{.}
\end{equation}
Die Energiedifferenz
\begin{equation}
E_.e=E_.{\gamma}-E_.{\gamma}' =\frac{\epsilon\cdot\left(1-\cos(\psi_.{\gamma})\right)}{1+\epsilon\cdot\left(1-\cos(\psi_.{\gamma})\right)}E_.{\gamma}
\end{equation}
ist somit die an das Elektron abgegebene Energie.
Für $\psi_.{\gamma}=\SI{180}{\degree}$ ist sie maximal und beträgt
\begin{equation}
E_.{e,max}=\frac{\epsilon}{1+\epsilon}E_.{\gamma}\text{.}
\end{equation}
Dieser Wert ist kleiner als $E_.{\gamma}$, somit wird nie die gesamte Photonenenergie auf das Elektron übertragen. Im Energiespektrum äußert sich das durch ein breites Kontinuum an Energien, welches an der Compton-Kante bei
$E_.e=E_.{e,max}$ vor dem Photopeak endet.
Der Wirkungsquerschnitt kann für geringe Energien, $\epsilon << 1$, mit dem klassischen Elektronenradius $r_.e$ genähert werden zu 
\begin{equation}
\sigma_.C=2\pi r^2_.e\cdot\left(1-2\epsilon\right)\label{eq:s_C_klein}
\end{equation}
und für sehr hohe Energien zu
\begin{equation}
\sigma_.C=\frac{\pi r^2_.e}{\epsilon}\cdot\left(\ln(2\epsilon)+\frac{1}{2}\right)\text{.}\label{eq:s_C_gross}
\end{equation}

\subsubsection{Paarerzeugung}

Eine weitere Möglichkeit ist, dass sich das Photon im Coulombfeld eines Kerns in ein Elektron und ein Positron aufspaltet. Auf Grund der Impulserhaltung muss es einen Stoßpartner geben. Ist dieser ein Atomkern, so ist die zusätzlich benötigte Rückstoßenergie 
\[
E_.=\frac{p^2}{2M}
\]
mit der Atommasse $M$ gering, sodass die Photonenenergie nur die Ruheenergie von Elektron und Positron übersteigen muss:
\[
E_.{\gamma}>2m_.0c^2
\]
Ist der Stoßpartner ein Hüllenelektron muss die Atommasse durch die Elektronenmasse ersetzt werden, wodurch $E_.R$ wesentlich größer ist und nicht mehr vernachlässigt werden kann.
In diesem Fall muss für die Photonenenergie gelten:
\[
E_.{\gamma}>4m_.0c^2
\]
Da Elektron und Positron dieselbe Masse besitzen, wird die verbleibende Energie gleichmäßig aufgeteilt.
Während das Elektron im Detektor durch Spannung abgesaugt wird, kann das Positron mit einem vorhandenen Hüllenelektron wieder zu zwei Photonen annihilieren.
Da diese den Detektor verlassen können ohne zu wechselwirken, wird im Energiespektrum nicht nur eine Linie bei $E_.{\gamma}$, sondern auch Linien bei $E_.{\gamma}-m_.0c^2$ und bei $E_.{\gamma}-2m_.0c^2$ beobachtet.
Der Wirkungsquerschnitt ist abhängig davon wo im Coulombfeld der Prozess stattfindet, da in höheren Bereichen Abschirmungseffekte durch die Hüllenelektronen auftreten.
Der wichtigste Fall für die $\gamma$-Spektroskopie ist die Paarbildung in Kernnähe.
Hier gilt für den Wirkungsquerschnitt bei Energien $\SI{10}{MeV}<E_.{\gamma}<\SI{25}{MeV}$ näherungsweise mit der Sommerfeldschen Feinstrukturkonstante $\alpha$
\begin{equation}
\sigma_.P=\alpha r^2_.ez^2\left(\frac{28}{9}\ln(2\epsilon)-\frac{218}{27}\right)\text{.}\label{eq:s_P}
\end{equation}

\subsection{Der Reinst-Germanium-Detektor}



%☺☺┤☻♣•
