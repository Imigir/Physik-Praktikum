\section{Theorie}
\label{sec:Theorie}

Nach dem $\alpha$- oder $\beta$-Zerfall eines instabilen Nuklids, kommt es vor, dass der neu entstehende Kern in einem angeregten Zustand ist. Nach einer kurzen Zeit fällt er unter Aussendung eines Photons, eines $\gamma$-Quants, in seinen Grundzustand zurück.

\subsection{Wechselwirkung von $\gamma$-Strahlung mit Materie}

Trifft ein solches $\gamma$-Quant auf Materie wechselwirkt es, im Gegensatz zu Teilchenstrahlung, im Wesentlichen einmal. Drei Arten von Wechselwirkungen dominieren dabei für unterschiedliche Energien. Die Wahrscheinlichkeit für jede Art von Wechselwirkung lässt sich durch einen Wirkungsquerschnitt $\sigma$ beschreiben, der anschaulich darstellt, dass die eingestrahlten Projektilteilchen (Photonen)auf ein ausgedehntes Ziel, also ein Targetteilchen(Absorberatom), treffen.


\subsubsection{Der Photoeffekt}

Beim Photoeffekt trifft ein Photon auf ein Hüllenelektron und schlägt dieses aus der Bindung des Kerns. Das Photon, das bei diesem Prozess vollständig absorbiert wird, muss dafür mindestens die Bindungsenergie $E_.B$ besitzen, sodass gilt:
\[
E_.{\gamma}>E_.B
\]
Findet die Reaktion in einer inneren Schale statt, fällt ein Elektron aus einem höheren Zustand unter Aussendung eines charakteristischen Photons in das freigewordene Energieniveau. Die freigewordene Energie verbleibt im Medium. Somit kann angenommen werden, dass die vollständige Energie des $\gamma$-Quants deponiert wird, was bei einer Messung zu einem scharfen Peak führt.
Für den Wirkungsquerschnitt gilt dabei mit der Kernladungszahl $Z$ des Absorbers:
\[
\sigma_.{Ph}\propto\frac{Z^4}{E^{\delta}_.{\gamma}}
\]
Der Exponent $\delta$ ist abhängig von der Photonenenergie. Für natürlich vorkommende $\gamma$-Strahlung gilt $\delta\approx 3,5$, für Energien $E_.{\gamma}\geq \SI{5}{\mega\electronvolt}$ sinkt er auf $\delta\approx 1$.
Der Wirkungsquerschnitt steigt außerdem immer sprunghaft an, wenn $E_.{\gamma}$ einen Wert erreicht, der ausreicht um ein Elektron aus dem Atom herauszulösen.

\subsubsection{Der Compton-Effekt}

Der Compton-Effekt beschreibt die elastische Streuung eines $\gamma$-Quants an einem Hüllenelektron. Das Photon gibt dabei nur einen Teil der Energie ab und fliegt danach mit veränderter Wellenlänge abgelenkt weiter. Geschieht dies mit einem Elektron auf einem äußeren Energieniveau, kann dieses einen großen Teil der Photonenenergie aufnehmen und aus der Bindung herausgeschlagen werden, während bei stärker gebundenen Elektronen nur ein kleiner Teil der Energie abgegeben wird und das Elektron im gebundenen Zustand verbleibt.

%☺
