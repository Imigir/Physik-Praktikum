
\section{Diskussion}
\label{sec:Diskussion}

Der bei der Kalibrierung bestimmte lineare Zusammenhang zwischen den Kanälen und Energien zeigt nur geringe Unsicherheiten. Die Standardabweichungen der bestimmten Peaks in Tabelle \ref{tab:parameterEu} fallen mit maximal $\num{6.1(8)}$ gering aus. Daraus folgt aus der Gaußförmigkeit der Kurven, dass die Peaks nur geringe Halbwertsbreiten besitzen. Dies lässt auf die hohe Auflösung des Reinst-Germaniumdetektors schließen.\\
Um Unsicherheiten zu minimieren, war darauf zu achten, einen Strahler mit möglichst reichhaltigem Spektrum zur Kalibrierung zu verwenden. Die Unsicherheiten, welche nach der Poisson-Statistik eigentlich auf die Höhe der einzelnen Kanäle genommen werden würden, wurden vernachlässigt, da sie häufig geringer als der Fehler durch die Gaußsche Fehlerfortpflanzung sind, oder zumindest in der selben Größenordnung liegen.\\
Die Untersuchung des Spektrums von $^{137}.{Cs}$ ergab die erwarteten Werte.
Die experimentell bestimmten Werte liegen bei allen untersuchten Größen in der Nähe der theoretisch bestimmten Werte.
Beim Vergleich des bestimmten Verhältnis von Zehntel und Halbwertsbreite mit dem Verhältnis, welches eine Gaußkurve besitzt, ist zu erkennen, dass diese nah beieinander liegen. Daher können die vermessenen Peaks gut durch Gaußkurven angenähert werden.\\
Es ist jedoch zu erkennen, dass die experimentelle Halbwertsbreite doppelt so groß ausfällt wie die theoretisch bestimmte. Die Ursache hierfür sind die in Kapitel \ref{subsubsec:Effizienz} beschriebenen Halbwertsbreiten aufgrund von Rauscheffekten $H_.R$, $H_.I$ und $H_.E$, aus welchen sich die Energieauflösung $H_.{Ges}$ gemäß \eqref{eq:HGes} ergibt. Da der Detektor gekühlt wird, ist die Halbwertsbreite des Rauschens infolge von Leckströmen $H_.R$ vernachlässigbar klein. Daher wird die zusätzliche Halbwertsbreite des Rauschens entweder durch den Verstärker ($H_.E$) oder durch Inhomogenitäten im Feld ($H_.I$) verursacht.\\
Anhand der bestimmten Verhältnisse zwischen experimentellem und theoretischem Auftreten von Photo- und Comptoneffekt ist zudem zu erkennen, dass sich die Wahrscheinlichkeit eines Photoeffekts bei höheren Energien nicht so schnell verringert wie theoretisch angenommen.
Dies lässt sich dadurch erklären, dass das $\gamma$-Quant zunächst durch den Compton-Effekt Energie verliert und dann ein Photoeffekt stattfindet, dessen Wahrscheinlichkeit bei geringeren Energien steigt.\\
Die Spektren der beiden untersuchten unbekannten Proben stimmen gut mit den Spektren der zugeordneten Elemente überein. 
Die bestimmten Aktivitäten der Barium-Probe liegen wie in Tabelle \ref{tab:ABa} zu sehen abgesehen vom ersten Wert sehr nah beisammen. Dass der erste Wert von den Anderen abweicht, liegt daran, dass die verwendete Formel zur Bestimmung der Aktivität in diesem Energiebereich noch nicht gültig ist. Der Fehler auf den gemittelten Wert der Aktivität $A_.{Ba}=\SI{1263(9)}{\becquerel}$ ist somit gering, weshalb es sich um eine geeignete Methode zur Aktivitätsbestimmung zu handeln scheint. 