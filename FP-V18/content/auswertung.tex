\section{Auswertung}
\label{sec:Auswertung}

Die Graphen werden sowohl mit Matplotlib \cite{matplotlib} als auch NumPy \cite{numpy} erstellt. Die Fehlerrechnung wird mithilfe von Uncertainties \cite{uncertainties} durchgeführt.

\subsection{Kalibrierung des Detektors}

\begin{figure}
	\centering
	\includegraphics[width=\linewidth-70pt,height=\textheight-70pt,keepaspectratio]{build/EU152.png}
	\caption{Das Spektrum eines $^{152}.{Eu}$-Strahlers, bei einer Messzeit von $\SI{2723}{\second}$, mit der Kanalnummer $K$ und der Anzahl der im Kanal nachgewiesenen Ereignisse $N$.}
	\label{fig:SpektrumEu}
\end{figure}

\begin{table}
	\centering
	\caption{Die Parameter der gefitteten Peaks des Spektrums von $^{152}.{Eu}$ mit den zugeordneten Energien.}
	\label{tab:a}
	\sisetup{table-format=1.2}
	\begin{tabular}{S[table-format=4.4]@{${}\pm{}$}S[table-format=1.4]S[table-format=2.3]@{${}\pm{}$}S[table-format=1.3]S[table-format=4.3]@{${}\pm{}$}S[table-format=1.3]S[table-format=1.3]@{${}\pm{}$}S[table-format=1.3]S[table-format=4.1]@{${}\pm{}$}S[table-format=2.1]S[table-format=2.2]@{${}\pm{}$}S[table-format=1.2]}
		\toprule
		\multicolumn{2}{c}{$E_\gamma^{\text{lit}}/\si{\kilo\electronvolt}$} & \multicolumn{2}{c}{$W/\si{\percent}$} & \multicolumn{2}{c}{$b$} & \multicolumn{2}{c}{$\sigma$} & \multicolumn{2}{c}{$a$} & \multicolumn{2}{c}{$c$} \\
		\midrule
		121.7817 & 0.0003 & 28.531 & 0.176 & 309.883 & 0.008 & 1.121 & 0.008 & 3213.3 & 17.8 & 68.06 & 5.17 \\
		244.6974 & 0.0008 & 7.549 & 0.046 & 614.957 & 0.020 & 1.353 & 0.022 & 454.7 & 6.0 & 24.54 & 1.98 \\
		295.9387 & 0.0017 & 0.440 & 0.005 & 741.733 & 0.160 & 0.920 & 0.172 & 18.8 & 2.9 & 18.86 & 0.88 \\
		344.2785 & 0.0012 & 26.590 & 0.206 & 862.023 & 0.023 & 1.554 & 0.026 & 970.3 & 12.7 & 14.79 & 5.26 \\
		367.7891 & 0.0020 & 0.859 & 0.006 & 920.444 & 0.218 & 1.836 & 0.248 & 28.3 & 3.1 & 11.24 & 1.28 \\
		411.1165 & 0.0012 & 2.237 & 0.013 & 1027.803 & 0.118 & 1.738 & 0.142 & 59.5 & 3.8 & 12.18 & 1.92 \\
		443.9606 & 0.0016 & 2.827 & 0.016 & 1109.509 & 0.087 & 1.720 & 0.100 & 70.6 & 3.3 & 10.76 & 1.76 \\
		688.6700 & 0.0050 & 0.856 & 0.007 & 1716.484 & 0.534 & 2.977 & 0.620 &  8.8 & 1.5 & 8.79 & 0.66 \\
		778.9045 & 0.0024 & 12.928 & 0.087 & 1940.474 & 0.054 & 2.094 & 0.064 & 138.5 & 3.3 & 10.09 & 1.55 \\
		867.3800 & 0.0030 & 4.228 & 0.032 & 2160.298 & 0.241 & 2.620 & 0.272 & 32.5 & 2.8 & 9.12 & 1.11 \\
		964.0570 & 0.0050 & 14.510 & 0.079 & 2400.014 & 0.099 & 3.037 & 0.105 & 90.3 & 2.6 & 4.60 & 0.77 \\
		1085.8370 & 0.0100 & 10.115 & 0.057 & 2702.631 & 0.241 & 3.372 & 0.270 & 49.3 & 3.2 & 5.84 & 1.27 \\
		1112.0760 & 0.0030 & 13.667 & 0.091 & 2767.474 & 0.121 & 3.448 & 0.132 & 69.1 & 2.2 & 2.73 & 0.74 \\
		1212.9480 & 0.0110 & 1.415 & 0.009 & 3017.728 & 0.454 & 3.373 & 0.597 &  7.2 & 1.0 & 1.79 & 0.59 \\
		1299.1420 & 0.0080 & 1.633 & 0.012 & 3231.990 & 0.582 & 4.397 & 0.639 &  4.6 & 0.6 & 0.78 & 0.20 \\
		1408.0130 & 0.0030 & 20.868 & 0.108 & 3502.362 & 0.091 & 3.998 & 0.096 & 74.3 & 1.5 & 0.57 & 0.41 \\
		1457.6430 & 0.0110 & 0.497 & 0.005 & 3629.673 & 0.678 & 6.137 & 0.758 &  3.3 & 0.4 & 0.27 & 0.13 \\
		\bottomrule
	\end{tabular}

	\label{tab:parameterEu}
\end{table}

\begin{figure}
	\centering
	\includegraphics[width=\linewidth-70pt,height=\textheight-70pt,keepaspectratio]{build/EnergieKali.png}
	\caption{Die Energie $E_\gamma$ gegen die Kanalnummer $K$ aufgetragen, wobei die Unsicherheit des einzelnen Wertepaares sich durch die Unsicherheit der Position der Gaußkurve und der Unsicherheit des Literaturwertes für $E_\gamma$ ergibt.}
	\label{fig:Kalibrierungi}
\end{figure}

\subsection{Bestimmung der Effizienz des Detektors}

\begin{figure}
	\centering
	\includegraphics[width=\linewidth-70pt,height=\textheight-70pt,keepaspectratio]{build/Q.png}
	\caption{Die Vollenergienachweiswahrscheinlichkeit $Q$ gegen die Energie $E_\gamma$ aufgetragen.}
	\label{fig:Q}
\end{figure}

\begin{table}
	\centering
	\caption{Die berechneten Peakinhalte $Z$, die berechneten Vollenergienachweiswahrscheinlichkeiten $Q$, sowie die berechneten Energien $E_\gamma$. Zudem die aus der Literatur entnommenen Energien $E_\gamma^.{lit}$ und Emissions-Wahrscheinlichkeiten $W$.}
	\label{tab:a2}
	\sisetup{table-format=1.2}
	\begin{tabular}{S[table-format=4.4]@{${}\pm{}$}S[table-format=1.4]S[table-format=4.2]@{${}\pm{}$}S[table-format=1.2]S[table-format=2.2]@{${}\pm{}$}S[table-format=1.2]S[table-format=4.0]@{${}\pm{}$}S[table-format=2.0]S[table-format=1.3]@{${}\pm{}$}S[table-format=1.3]}
		\toprule
		\multicolumn{2}{c}{$E_\gamma^{\text{lit,\cite{MARTIN20131497}}}/\si{\kilo\electronvolt}$} & \multicolumn{2}{c}{$E_\gamma/\si{\kilo\electronvolt}$} & \multicolumn{2}{c}{$W^\text{\cite{MARTIN20131497}}/\si{\percent}$} & \multicolumn{2}{c}{$Z$} & \multicolumn{2}{c}{$Q$} \\
		\midrule
		121.7817 & 0.0003 & 121.82 & 0.04 & 28.53 & 0.18 & 9030 & 58 & 0.458 & 0.008 \\
		244.6974 & 0.0008 & 244.74 & 0.03 & 7.55 & 0.05 & 1543 & 24 & 0.296 & 0.007 \\
		295.9387 & 0.0017 & 295.82 & 0.07 & 0.44 & 0.00 &   43 &  8 & 0.142 & 0.026 \\
		344.2785 & 0.0012 & 344.28 & 0.03 & 26.59 & 0.21 & 3779 & 64 & 0.206 & 0.005 \\
		367.7891 & 0.0020 & 367.82 & 0.09 & 0.86 & 0.01 &  130 & 18 & 0.219 & 0.031 \\
		411.1165 & 0.0012 & 411.08 & 0.05 & 2.24 & 0.01 &  259 & 23 & 0.168 & 0.015 \\
		443.9606 & 0.0016 & 444.00 & 0.04 & 2.83 & 0.02 &  304 & 19 & 0.156 & 0.010 \\
		688.6700 & 0.0050 & 688.56 & 0.22 & 0.86 & 0.01 &   66 & 15 & 0.111 & 0.025 \\
		778.9045 & 0.0024 & 778.81 & 0.03 & 12.93 & 0.09 &  727 & 24 & 0.081 & 0.003 \\
		867.3800 & 0.0030 & 867.38 & 0.10 & 4.23 & 0.03 &  213 & 23 & 0.073 & 0.008 \\
		964.0570 & 0.0050 & 963.96 & 0.05 & 14.51 & 0.08 &  687 & 23 & 0.069 & 0.003 \\
		1085.8370 & 0.0100 & 1085.89 & 0.10 & 10.11 & 0.06 &  417 & 34 & 0.060 & 0.005 \\
		1112.0760 & 0.0030 & 1112.02 & 0.06 & 13.67 & 0.09 &  597 & 23 & 0.063 & 0.003 \\
		1212.9480 & 0.0110 & 1212.85 & 0.19 & 1.41 & 0.01 &   61 & 13 & 0.062 & 0.013 \\
		1299.1420 & 0.0080 & 1299.18 & 0.24 & 1.63 & 0.01 &   51 &  7 & 0.045 & 0.007 \\
		1408.0130 & 0.0030 & 1408.12 & 0.05 & 20.87 & 0.11 &  744 & 17 & 0.052 & 0.001 \\
		1457.6430 & 0.0110 & 1459.41 & 0.28 & 0.50 & 0.00 &   51 &  6 & 0.148 & 0.019 \\
		\bottomrule
	\end{tabular}

	\label{tab:Q}
\end{table}

\subsection{Untersuchung des Spektrums von $^{137}.{Cs}$}

\begin{figure}
	\centering
	\includegraphics[width=\linewidth-70pt,height=\textheight-70pt,keepaspectratio]{build/Cs137.png}
	\caption{Das Spektrum eines $^{137}.{Cs}$-Strahlers, bei einer Messzeit von $\SI{4543}{\second}$.}
	\label{fig:SpektrumCs}
\end{figure}

\begin{table}
	\centering
	\caption{Die Parameter der gefitteten Peaks des Spektrums von $^{137}.{Cs}$ mit den ermittelten Energien, wobei es sich beim zweiten Peak um den Rückstreupeak handelt.}
	\label{tab:b}
	\sisetup{table-format=1.2}
	\begin{tabular}{S[table-format=3.1]S[table-format=4.1]S[table-format=2.1]S[table-format=4.0]S[table-format=2.0]}
		\toprule
		{$E_\gamma/\si{\kilo\electronvolt}$} & {$b$} & {$\sigma$} & {$a$} & {$c$} \\
		\midrule
		\SI{661.5\pm 0.0} & \SI{1649.2\pm  0.0} & \SI{2.2\pm0.0} & \SI{2983\pm  39} & \SI{20\pm16} \\
		\SI{190.8\pm 0.3} & \SI{481.2\pm  0.6} & \SI{14.5\pm0.8} & \SI{  63\pm   3} & \SI{78\pm 1} \\
		\bottomrule
	\end{tabular}

	\label{tab:parameterCs}
\end{table}

\begin{figure}
	\centering
	\includegraphics[width=\linewidth-70pt,height=\textheight-70pt,keepaspectratio]{build/Cs137Zehntel.png}
	\caption{Die Bestimmung der Zehntelwertsbreite des $^{137}.{Cs}$-Strahlers.}
	\label{fig:10tel}
\end{figure}

\begin{figure}
	\centering
	\includegraphics[width=\linewidth-70pt,height=\textheight-70pt,keepaspectratio]{build/Cs137Halb.png}
	\caption{Die Bestimmung der Halbwertsbreite des $^{137}.{Cs}$-Strahlers.}
	\label{fig:2tel}
\end{figure}

\begin{table}
	\centering
	\caption{Die Parameter der gefitteten Geraden zur Bestimmung der Halbwertsbreite und Zehntelbreite des Vollenergiepeaks des Spektrums von $^{137}.{Cs}$.}
	\label{tab:geraden1}
	\sisetup{table-format=1.2}
	\begin{tabular}{S[table-format=4.2]S[table-format=4.2]}
		\toprule
		{$a$} & {$b$} \\
		\midrule
		\SI{185\pm51} & \SI{-303003\pm83052} \\
		\SI{664\pm31} & \SI{-1092626\pm50811} \\
		\SI{-789\pm51} & \SI{1305453\pm83424} \\
		\SI{-285\pm79} & \SI{471675\pm130827} \\
		\bottomrule
	\end{tabular}

\end{table}

\begin{figure}
	\centering
	\includegraphics[width=\linewidth-70pt,height=\textheight-70pt,keepaspectratio]{build/Cs137Kon.png}
	\caption{Die Approximation des Compton-Kontinuums des $^{137}.{Cs}$-Strahlers.}
	\label{fig:Comptonkontinuum}
\end{figure}

\begin{figure}
	\centering
	\includegraphics[width=\linewidth-70pt,height=\textheight-70pt,keepaspectratio]{build/Cs137Emax.png}
	\caption{Die Bestimmung der Position der Compton-Kante des $^{137}.{Cs}$-Strahlers.}
	\label{fig:Emax}
\end{figure}

\begin{table}
	\centering
	\caption{Die Parameter der gefitteten Geraden zur Bestimmung der Position der Compton-Kante des Spektrums von $^{137}.{Cs}$.}
	\label{tab:geraden2}
	\sisetup{table-format=1.2}
	\begin{tabular}{S[table-format=4.2]S[table-format=4.2]}
		\toprule
		{$a$} & {$b$} \\
		\midrule
		\SI{0.33\pm0.03} & \SI{-74\pm14} \\
		\SI{-2.51\pm0.32} & \SI{1257\pm151} \\
		\bottomrule
	\end{tabular}

\end{table}

\subsection{Bestimmung der Aktivität einer nicht eindeutig bestimmten Probe}

\begin{figure}
	\centering
	\includegraphics[width=\linewidth-70pt,height=\textheight-70pt,keepaspectratio]{build/D.pdf}
	\caption{Das Spektrum des Strahlers, bei einer Messzeit von $\SI{6312}{\second}$.}
	\label{fig:Ba}
\end{figure}

\begin{table}
	\centering
	\caption{Die Parameter der gefitteten Peaks des Spektrums mit den ermittelten Energien.}
	\label{tab:D}
	\sisetup{table-format=1.2}
	\begin{tabular}{S[table-format=2.0]S[table-format=2.0]S[table-format=2.0]S[table-format=2.0]S[table-format=2.0]}
		\toprule
		{$E_\gamma/\si{\kilo\electronvolt}$} & {$b$} & {$\sigma$} & {$a$} & {$c$} \\
		\midrule
		\SI{81.0\pm0.0} & \SI{208.63\pm0.03} & \SI{1.06\pm0.04} & \SI{5596\pm163} & \SI{222.0\pm55.2} \\
		\SI{276.4\pm0.0} & \SI{693.56\pm0.01} & \SI{1.31\pm0.01} & \SI{709\pm5} & \SI{27.0\pm0.7} \\
		\SI{302.9\pm0.0} & \SI{759.25\pm0.01} & \SI{1.41\pm0.01} & \SI{1501\pm10} & \SI{22.7\pm3.4} \\
		\SI{356.0\pm0.0} & \SI{891.11\pm0.01} & \SI{1.51\pm0.01} & \SI{4000\pm17} & \SI{19.4\pm6.0} \\
		\SI{383.8\pm0.0} & \SI{960.13\pm0.02} & \SI{1.64\pm0.02} & \SI{493\pm5} & \SI{5.8\pm1.9} \\
		\bottomrule
	\end{tabular}

	\label{tab:parameterBa}
\end{table}

\begin{table}
	\centering
	\caption{Die berechneten Peakinhalte $Z$, die mit den Vollenergienachweiswahrscheinlichkeiten $Q$ berechneten Aktivitäten $A$,  sowie die berechneten Energien $E_\gamma$.  Zudem die aus der Literatur entnommenen Energien $E_\gamma^\text{lit}$ und Emissions-Wahrscheinlichkeiten $W$.}
	\label{tab:D2}
	\sisetup{table-format=1.2}
	\begin{tabular}{S[table-format=4.2]S[table-format=4.2]S[table-format=2.2]S[table-format=5.0]S[table-format=0.3]S[table-format=4.0]}
		\toprule
		{$E_\gamma^{\text{lit,\cite{KHAZOV2011855}}}/\si{\kilo\electronvolt}$} & {$E_\gamma$} & {$W^\text{\cite{KHAZOV2011855}}/\si{\percent}$} & {$Z$} & {$Q$} & {$A$} \\
		\midrule
		\SI{81.00\pm0.00} & \SI{81.02\pm0.04} & \SI{32.95\pm0.33} & \SI{14932\pm  526} & \SI{0.513\pm0.031} & \SI{1549\pm 109} \\
		\SI{276.40\pm0.00} & \SI{276.41\pm0.03} & \SI{7.16\pm0.05} & \SI{ 2329\pm   17} & \SI{0.145\pm0.004} & \SI{3929\pm 118} \\
		\SI{302.85\pm0.00} & \SI{302.87\pm0.03} & \SI{18.34\pm0.13} & \SI{ 5294\pm   43} & \SI{0.132\pm0.003} & \SI{3831\pm 109} \\
		\SI{356.01\pm0.00} & \SI{356.00\pm0.03} & \SI{62.05\pm0.19} & \SI{15093\pm   77} & \SI{0.112\pm0.003} & \SI{3812\pm  92} \\
		\SI{383.85\pm0.00} & \SI{383.81\pm0.03} & \SI{8.94\pm0.06} & \SI{ 2024\pm   25} & \SI{0.104\pm0.002} & \SI{3833\pm 101} \\
		\bottomrule
	\end{tabular}

	\label{tab:ABa}
\end{table}

\subsection{Identifizierung der aktiven Nuklide in einer Zerfallsreihe}

\begin{figure}
	\centering
	\includegraphics[width=\linewidth-70pt,height=\textheight-70pt,keepaspectratio]{build/unbekannt.png}
	\caption{Das aufgenommene Spektrum einer unbekannten Probe, bei einer Messzeit von $\SI{4489}{\second}$.}
	\label{fig:SpektrumUnbekannt}
\end{figure}

\begin{table}
	\centering
	\caption{Die Parameter der gefitteten Peaks des unbekannten Spektrums mit den ermittelten Energien.}
	\label{tab:unbekannt}
	\sisetup{table-format=1.2}
	\begin{tabular}{S[table-format=2.0]S[table-format=2.0]S[table-format=2.0]S[table-format=2.0]S[table-format=2.0]}
		\toprule
		{$E_\gamma/\si{\kilo\electronvolt}$} & {$b$} & {$\sigma$} & {$a$} & {$c$} \\
		\midrule
		\SI{92.6\pm0.1} & \SI{237.41\pm0.13} & \SI{1.20\pm0.14} & \SI{810\pm79} & \SI{589\pm24} \\
		\SI{186.1\pm0.0} & \SI{469.35\pm0.02} & \SI{1.37\pm0.03} & \SI{1449\pm23} & \SI{423\pm8} \\
		\SI{242.0\pm0.0} & \SI{608.18\pm0.03} & \SI{1.31\pm0.03} & \SI{1478\pm25} & \SI{273\pm7} \\
		\SI{295.2\pm0.0} & \SI{740.28\pm0.01} & \SI{1.37\pm0.01} & \SI{2982\pm17} & \SI{194\pm6} \\
		\SI{351.9\pm0.0} & \SI{880.92\pm0.01} & \SI{1.50\pm0.01} & \SI{4620\pm32} & \SI{150\pm9} \\
		\SI{609.1\pm0.0} & \SI{1519.30\pm0.02} & \SI{2.09\pm0.02} & \SI{2478\pm21} & \SI{52\pm6} \\
		\SI{768.1\pm0.0} & \SI{1913.96\pm0.07} & \SI{2.54\pm0.08} & \SI{188\pm5} & \SI{36\pm2} \\
		\SI{933.9\pm0.1} & \SI{2325.39\pm0.15} & \SI{3.04\pm0.16} & \SI{81\pm3} & \SI{28\pm1} \\
		\SI{1120.1\pm0.0} & \SI{2787.46\pm0.05} & \SI{3.09\pm0.05} & \SI{334\pm4} & \SI{21\pm1} \\
		\SI{1377.5\pm0.1} & \SI{3426.36\pm0.25} & \SI{3.63\pm0.27} & \SI{59\pm4} & \SI{17\pm1} \\
		\SI{1401.4\pm0.1} & \SI{3485.81\pm0.32} & \SI{3.45\pm0.52} & \SI{19\pm2} & \SI{16\pm2} \\
		\SI{1508.9\pm0.1} & \SI{3752.41\pm0.28} & \SI{3.88\pm0.31} & \SI{34\pm2} & \SI{14\pm1} \\
		\SI{1729.5\pm0.1} & \SI{4299.91\pm0.23} & \SI{4.74\pm0.26} & \SI{32\pm1} & \SI{4\pm1} \\
		\SI{1764.3\pm0.1} & \SI{4386.30\pm0.11} & \SI{4.71\pm0.12} & \SI{163\pm3} & \SI{4\pm1} \\
		\SI{2204.0\pm0.1} & \SI{5477.59\pm0.19} & \SI{5.72\pm0.21} & \SI{34\pm1} & \SI{1\pm0} \\
		\bottomrule
	\end{tabular}

	\label{tab:parameterUnbekannt}
\end{table}

\begin{table}
	\centering
	\caption{Die berechneten Peakinhalte $Z$, die mit den Vollenergienachweiswahrscheinlichkeiten $Q$ berechneten Aktivitäten $A$, sowie die berechneten Energien $E_\gamma$. Zudem die aus der Literatur entnommenen Energien $E_\gamma^\text{lit}$ und Emissions-Wahrscheinlichkeiten $W$.}
	\label{tab:unbekannt2}
	\sisetup{table-format=1.2}
	\begin{tabular}{S[table-format=4.0]S[table-format=4.2]@{${}\pm{}$}S[table-format=1.2]S[table-format=2.0]S[table-format=5.0]@{${}\pm{}$}S[table-format=3.0]S[table-format=1.3]@{${}\pm{}$}S[table-format=1.3]S[table-format=4.0]@{${}\pm{}$}S[table-format=3.0]}
		\toprule
		{$E_\gamma^{\text{lit}}/\si{\kilo\electronvolt}$} & \multicolumn{2}{c}{$E_\gamma/\si{\kilo\electronvolt}$} & {$W/\si{\percent}$} & \multicolumn{2}{c}{$Z$} & \multicolumn{2}{c}{$Q$} & \multicolumn{2}{c}{$A/\si{\becquerel}$} \\
		\midrule
		  93 & 92.62 & 0.07 &  4 &  2432 & 277 & 0.787 & 0.035 & 1106 & 135 \\
		 186 & 186.07 & 0.04 &  4 &  4992 &  98 & 0.385 & 0.012 & 4640 & 164 \\
		 242 & 242.01 & 0.04 &  4 &  4853 &  96 & 0.294 & 0.008 & 5905 & 185 \\
		 295 & 295.23 & 0.03 & 19 & 10245 &  73 & 0.240 & 0.006 & 3218 &  72 \\
		 352 & 351.90 & 0.03 & 36 & 17373 & 136 & 0.201 & 0.004 & 3447 &  70 \\
		 609 & 609.11 & 0.03 & 47 & 12977 & 130 & 0.114 & 0.002 & 3460 &  65 \\
		 769 & 768.13 & 0.04 &  5 &  1195 &  38 & 0.090 & 0.002 & 3798 & 136 \\
		 935 & 933.90 & 0.07 &  3 &   617 &  32 & 0.074 & 0.002 & 3992 & 218 \\
		1120 & 1120.07 & 0.04 & 17 &  2591 &  40 & 0.061 & 0.002 & 3565 &  98 \\
		1378 & 1377.49 & 0.11 &  5 &   535 &  41 & 0.050 & 0.002 & 3094 & 246 \\
		1400 & 1401.45 & 0.14 &  4 &   166 &  34 & 0.049 & 0.002 & 1218 & 246 \\
		1509 & 1508.87 & 0.13 &  2 &   331 &  27 & 0.045 & 0.002 & 5252 & 454 \\
		1728 & 1729.46 & 0.11 &  3 &   383 &  22 & 0.039 & 0.002 & 4662 & 297 \\
		1764 & 1764.27 & 0.08 & 17 &  1920 &  50 & 0.038 & 0.002 & 4207 & 171 \\
		2204 & 2203.97 & 0.12 &  5 &   485 &  18 & 0.031 & 0.002 & 4539 & 234 \\
		\bottomrule
	\end{tabular}

	\label{tab:AUnbekannt}
\end{table}

