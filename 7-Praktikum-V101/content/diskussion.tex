\section{Diskussion}
\label{sec:Diskussion}
Alle berechneten Trägheitsmomente der verschiedenen Körper sind negativ.
Die Differenz zwischen theoretischem und experimentellem Wert liegen bei
\[
I_.{Kugel,Theorie}-I_.{Kugel}=\SI{2,9(2)e-3}{\kilo\gram\metre\squared}
I_.{Zylinder,Theorie}-I_.{Zylinder}=\SI{3,12(2)e-3}{\kilo\gram\metre\squared}
I_.{Puppe,aus,theo}-I_.{Puppe,ausg}=\SI{3,25(10)e-3}{\kilo\gram\metre\squared}
I_.{Puppe,an,theo}-I_.{Puppe,ang}=\SI{2,82(1)e-3}{\kilo\gram\metre\squared}\text{.}
\]
Das Eigenträgheit $I_.D$ wurde bestimmt zu $I_.D=\SI{3,2(1)e-3}{\kilo\gram\metre\squared}$
Die Ähnlichkeit dieser Werte lässt darauf schließen, dass $I_.D$ zu groß bestimmt wurde, da es nach Formel \eqref{eq:I_K} jedes Mal von den durch die Periodendauer bestimmten Werten abgezogen wird.\newline
Dies könnte daran liegen, dass die Periodendauer $T_0^2$ zu groß bestimmt wurde, oder der Stab, an denen die Zylinder befestigt sind doch hätte mit berücksichtigt werden müssen. Zudem könnte es eine Rolle spielen, dass das System nur für kleine Auslenkungen harmonisch schwingt, die Drillachse jedoch um große Winkel ausgelenkt wurde.\newline
Außerdem ist bei der Puppe auf Grund des Näherns der Körperteile zu einfachen geometrischen Gebilden und des inhomogenen Materials nur beschränkt ein exakter Theoriewert zu ermitteln, was zu Abweichungen vom durch die Periodendauer bestimmten Wert führen kann.
Wenn man den Quotienten der Theoriewerte $\frac{I_.{Puppe,aus,theo}}{I_.{Puppe,an,theo}}\approx 9,466$
mit dem Quotienten der experimentellen $\frac{I_.{Puppe,ausg}}{I_.{Puppe,ang}}\approx 0,80$
vergleicht fällt auf, dass das Verhältnis der Theoriewerte sich um einen Faktor 10 von dem der Messwerten unterscheidet, was sich unter anderem darauf zurückführen lässt, dass $I_.{Puppe,ang}$ im Betrag größer als $I_.{Puppe,ausg}$ ist, wodurch der Quotient kleiner als 1 wird. Dies ist auch physikalisch sinnvoll, da mit ausgebreiteten Armen ein Teil der Masse weiter von der Drehachse entfernt und damit das Trägheitsmoment auf Grund der $R^2$-Proportionalität größer ist.