\section{Diskussion}
\label{sec:Diskussion}
Alle berechneten Trägheitsmomente der verschiedenen Körper sind negativ.
Das lässt darauf schließen,dass das Eigenträgheitsmoment $I_.D $zu groß bestimmt wurde, da nach Formel \eqref{eq:I_K} dies jedes Mal von den durch die Periodendauer bestimmten Werten abgezogen wird.\newline
Weiterhin ist es möglich das die bestimmte Winkelrichtgröße $D$ zu klein ist.
Beides könnte auf Ungenauigkeiten beim Messen der Kraft bzw. der Periodendauer zurückzuführen sein, da diese von Hand gemessen wurden.\newline
Außerdem ist bei der Puppe auf Grund des Näherns der Körperteile zu einfachen geometrischen Gebilden nur beschränkt ein exakter Theoriewert zu ermitteln, was zu Abweichungen vom durch die Periodendauer bestimmten Wert führen kann.