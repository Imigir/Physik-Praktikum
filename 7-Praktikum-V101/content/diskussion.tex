\section{Diskussion}
\label{sec:Diskussion}
Alle berechneten Trägheitsmomente der verschiedenen Körper sind negativ.
Das lässt darauf schließen,dass das Eigenträgheitsmoment $I_.D$ zu groß bestimmt wurde, da nach Formel \eqref{eq:I_K} dies jedes Mal von den durch die Periodendauer bestimmten Werten abgezogen wird.\newline
Dies könnte daran liegen, dass die Periodendauer $T_0^2$ zu groß bestimmt wurde, oder der Stab, an denen die Zylinder befestigt sind doch hätte mit berücksichtigt werden müssen. Das sich bei dem $T_0^2$ zu große Werte ergeben, könnte an der Reaktionszeit beim starten und stoppen der Messung liegen. Zudem könnte es eine Rolle spielen, dass das System nur für kleine Auslenkungen harmonisch schwingt, die Drillachse jedoch um große Winkel ausgelenkt wurde.\newline
Außerdem ist bei der Puppe auf Grund des Näherns der Körperteile zu einfachen geometrischen Gebilden nur beschränkt ein exakter Theoriewert zu ermitteln, was zu Abweichungen vom durch die Periodendauer bestimmten Wert führen kann.