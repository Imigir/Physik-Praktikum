
\section{Theorie}
\label{sec:Theorie}

Das Trägheitsmoment einer punktförmigen Masse $m$ im Abstand $r$ von einer Drehachse $\vec{\omega}$ lässt sich berechnen durch:
\begin{equation}
I_.S = m\cdot r^2 
\end{equation}
Für mehrere Punktmassen in einem starren Körper gilt dann:
\begin{equation}
I_.S = \sum_i m_i\cdot r_i^2
\end{equation}
Verallgemeinert auf eine kontinuierliche Massenverteilung ergibt sich:
\begin{equation}
I_.S = \int \rho r^2 dV \label{eq:I_S}
\end{equation}
Dabei bezieht sich $I_.S$ immer auf die Drehachse bezüglich des Schwerpunktes des Körpers.  
Ist die Drehachse um den Abstand $\vec{a}$ vom Schwerpunkt verschoben, so lässt sich das Trägheitsmoment nach dem Satz von Steiner berechnen:
\begin{equation}
I = I_.S+m a^2 \label{eq:Satz von Steiner}
\end{equation}
Dabei ist $a$ die Länge von $\vec{a}$. Das Drehmoment eines Körpers lässt sich berechnen nach:
\begin{equation}
\vec{M} = \vec{F}x\vec{r} \label{eq;M}
\end{equation}
Dabei ist $\vec{F}$ die Kraft, die im Abstand $\vec{r}$ von der Drehachse auf den Körper wirkt. Wirkt $\vec{F}$ als rückwirkende Kraft, so schwingt das System bei kleinen Auslenkungen mit:
\begin{equation}
T = 2\pi\sqrt{\frac{I}{D}} \label{eq;M}
\end{equation}
Dabei ist $D$ die Winkelrichtgröße und berechnet sich bei senkrecht wirkender Kraft durch: 
\begin{align}
D 	&= \frac{M}{\phi}\nonumber\\ 
	&= \frac{F\cdot r}{\phi} \label{eq:D}
\end{align}
Sind $\vec{a}$ und $T$ eines schwingenden Systems bekannt, so lässt sich das Eigenträgheitsmoment $I_.D$ nach der in Kapitel\ref{sec:Durchführung} beschriebenen Methode bestimmen durch:
\begin{equation}
I_.D = \frac{nD}{4\pi^2}-\sum I_.S \label{eq:I_D}
\end{equation}
Dabei entspricht $n$ dem y-Achsenabschnitt, wenn $T^2$ gegen $a^2$ aufgetragen wird.