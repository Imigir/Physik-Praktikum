\section{Auswertung}
\label{sec:Auswertung}

Mithilfe der Messdaten soll die Molwärme von Blei, Kupfer und Graphit berechnet werden. Dazu müssen die Wärmekapazität des Kalorimeters, sowie die spezifischen Wärmekapazitäten der einzelnen Materialien bestimmt werden.
Alle Mittelwerte werden berechnet mit:
\begin{equation}
\bar{x} = \sum_{i=1}^{n}x_i/n \label{eq:quer}
\end{equation}
Die Abweichungen ergeben sich nach:
\begin{equation}
\bar{\sigma} = \sqrt{\frac{1}{n(n-1)}\sum_{i=1}^{n}(\bar{x}-x_i)^2}  \label{eq:sigma}
\end{equation}

\subsection{Die Wärmekapazität des Kalorimeters}

Die Wärmekapazität $c_\text{g}m_\text{g}$ des Kalorimeters lässt sich gemäß Formel \eqref{eq:cgmg} bestimmen. Die spezifische Wärmekapazität ist als $c_\text{w} = \SI{4.18}{\joule\per\gram\per\kelvin}$ \cite{V201} gegeben. Die zugehörigen Messwerte befinden sich in Tabelle \ref{tab:tab1}.
\begin{table}
	\centering
	\caption{Die gemessenen Daten für die Massen und Temperaturen.}
	\label{tab:tab1}
	\sisetup{table-format=1.2}
	\begin{tabular}{S[table-format=1.2]S[table-format=4.0]}
		\toprule
		{$\Delta s/\si{\milli\meter}$} & {$N$} \\
		\midrule
		5.00 & 3144 \\
		5.00 & 3105 \\
		5.00 & 3076 \\
		5.00 & 2973 \\
		5.00 & 3183 \\
		\bottomrule
	\end{tabular}

	\label{tab:tab1}
\end{table}
Es folgt:
\begin{displaymath}
	c_\text{g}m_\text{g} = \SI{1842,47}{\joule\per\kelvin}
\end{displaymath}
\subsection{Bestimmung der Molwärme von verschiedenen Matearialien}

Die Molwärme der Materialien wird über die spezifische Wärmekapazität berechnet. Diese wird aus den Messergebnissen bestimmt.
Die Werte für $\rho$, $M$, $\alpha$ und $\kappa$ werden aus der Tabelle der Versuchsanleitung\cite{V201} übernommen und befinden sich in Tabelle \ref{tab:tabWerte}.
\begin{table}
	\centering
	\caption{Die Werte für $\rho$, $M$, $\alpha$ und $\kappa$ }
	\input{build/tabWerte.tex}
	\label{tab:tabWerte}
\end{table}

\subsubsection{Kupfer}

Mit den Messwerten aus Tabelle \ref{tab:tab2} und den Formeln \eqref{eq:ck} und \eqref{eq:ckCV} folgen die jeweiligen Werte $c_\text{k}$ und $C_\text{V}$ aus Tabelle \ref{tab:tab3}. Dabei wird in Formel \eqref{eq:ckCV} für die Temperatur $T$ die Mischtemperatur $T_\text{m}$ eingesetzt. Die Masse des verwendeten Kupfers beträgt \[m_k = \SI{235,55}{\gram}.\]
Für die Mittelwerte gilt:
\begin{align*}
	c_\text{k} = \SI{0,69(7)}{\joule\per\gram\per\kelvin} \\
	C_\text{V} = (43,00\pm5)\si{\joule\per\mol\per\kelvin}
\end{align*}
\begin{table}
	\centering
	\caption{Die Messwerte für Kupfer.}
	\input{build/tab2-1.tex}
	\label{tab:tab2}
\end{table}
\begin{table}
	\centering
	\caption{Die berechneten Werte für $c_\text{k}$ und $C_\text{V}$ von Kupfer.}
	\input{build/tab2-2.tex}
	\label{tab:tab3}
\end{table}

\subsubsection{Blei}

Mit den Messwerten aus Tabelle \ref{tab:tab4} und den Formeln \eqref{eq:ck} und \eqref{eq:ckCV} folgen die jeweiligen Werte $c_\text{k}$ und $C_\text{V}$ aus Tabelle \ref{tab:tab5}. Dabei wird in Formel \eqref{eq:ckCV} für die Temperatur $T$ die Mischtemperatur $T_\text{m}$ eingesetzt. Die Masse des verwendeten Bleis beträgt \[m_k = \SI{541,89}{\gram}.\]
Für die Mittelwerte gilt:
\begin{align*}
	c_\text{k} = \SI{0,30(1)}{\joule\per\gram\per\kelvin}\\
	C_\text{V} =(61,00\pm1)\si{\joule\per\mol\per\kelvin}
\end{align*}
\begin{table}
	\centering
	\caption{Die Messwerte für Blei.}
	\input{build/tab3-1.tex}
	\label{tab:tab4}
\end{table}
\begin{table}
	\centering
	\caption{Die berechneten Werte für $c_\text{k}$ und $C_\text{V}$ von Blei.}
	\input{build/tab3-2.tex}
	\label{tab:tab5}
\end{table}

\subsubsection{Graphit}

Mit den Messwerten aus Tabelle \ref{tab:tab6} und den Formeln \eqref{eq:ck} und \eqref{eq:ckCV} folgen die jeweiligen Werte $c_\text{k}$ und $C_\text{V}$ aus Tabelle \ref{tab:tab7}. Dabei wird in Formel \eqref{eq:ckCV} für die Temperatur $T$ die Mischtemperatur $T_\text{m}$ eingesetzt. Die Masse des verwendeten Graphits beträgt \[m_k = \SI{106,43}{\gram}.\]
Für die Mittelwerte gilt:
\begin{align*}
	c_\text{k} = (1,60\pm0,1)\si{\joule\per\gram\per\kelvin} \\
	C_\text{V} = (19,00\pm1)\si{\joule\per\mol\per\kelvin}
\end{align*}
\begin{table}
	\centering
	\caption{Die Messwerte für Graphit.}
	\input{build/tab4-1.tex}
	\label{tab:tab6}
\end{table}
\begin{table}
	\centering
	\caption{Die berechneten Werte für $c_\text{k}$ und $C_\text{V}$ von Graphit.}
	\input{build/tab4-2.tex}
	\label{tab:tab7}
\end{table}


