
\section{Diskussion}
\label{sec:Diskussion}

\begin{table}
	\centering
	\caption{Die Ergebnisse für die Halbwertszeiten $\tau$ und deren Abweichungen zu den Literaturwerten\cite{Halbwertszeiten} .}
	\label{tab:Ergebnisse}
	\sisetup{table-format=1.2}
	\begin{tabular}{c ccc}
		\toprule
		{Wert}&{gemessen}&{Referenzwert}&{Abweichung} \\
		\midrule
		$\alpha_{max}$ & \SI{28,1}\,\si{\degree} & \SI6{28}\,\si{\degree} & \SI{0,3}\,\si{\percent} \\
		$\theta_{gr}$ & \SI{5}\,\si{\degree} & \SI{5,04}\,\si{\degree} & \SI{-0,9}\,\si{\percent} \\
		$\lambda_{min}$ & \SI{35,1}\,\si{\pico\metre} & \SI{35,4}\,\si{\pico\metre}  & \SI{-1,13}\,\si{\percent} \\
		$E_{kin,max}$ & \SI{35316}\,\si{\eV} & \SI{35000}\,\si{\eV} & \SI{0,9}\,\si{\percent} \\
		$\Delta E_{alpha}$ & \SI{130,7}\,\si{\eV} & - & - \\
		$\Delta E_{beta}$ & \SI{155,0}\,\si{\eV} & - & - \\
		$\sigma_{Cu_K}$ & \SI{3,28} & \SI{3,31} & \SI{-0,76}\,\si{\percent}  \\
		$\sigma_{Cu_{L}}$ & \SI{13,16} & \SI{20,72} & \SI{-36,48}\,\si{\percent} \\
		$\sigma_{Cu_{M}}$ & \SI{29} & \SI{26,64} & \SI{8,87}\,\si{\percent} \\
		$E_{K_{Br}}$ & \SI{13282}\,\si{\eV} & \SI{13470}\,\si{\eV} & \SI{-1,40}\,\si{\percent} \\
		$E_{K_{Sr}}$ & \SI{15988}\,\si{\eV} & \SI{16090}\,\si{\eV} & \SI{-0,64}\,\si{\percent} \\
		$E_{K_{Zn}}$ & \SI{9650}\,\si{\eV} & \SI{9650}\,\si{\eV} & \SI{0,0}\,\si{\percent} \\
		$E_{K_{Zr}}$ & \SI{17903}\,\si{\eV} & \SI{17970}\,\si{\eV} & \SI{-0,37}\,\si{\percent} \\
		$\sigma_{K_{Br}}$ & \SI{3,75} & \SI{3,53} & \SI{6,23}\,\si{\percent} \\
		$\sigma_{K_{Sr}}$ & \SI{3,71} & \SI{3,66} & \SI{1,37}\,\si{\percent} \\
		$\sigma_{K_{Zn}}$ & \SI{3,36} & \SI{3,36} & \SI{0,0}\,\si{\percent} \\
		$\sigma_{K_{Zr}}$ & \SI{3,72} & \SI{3,65} & \SI{1,92}\,\si{\percent} \\
		$R_{\infty}$ & \SI{16,87\pm 0,32}\,\si{\eV} & \SI{13,6}\,\si{\eV} & \SI{24,04}\,\si{\percent} \\
		$\sigma_{L_{Bi}}$ & \SI{3,31} & \SI{3,58} & \SI{-7,54}\,\si{\percent} \\
		\bottomrule
	\end{tabular}

	\label{tab:tabFehler}
\end{table}

\noindent In Tabelle \ref{tab:tabFehler} ist zu erkennen, dass die Abweichungen zu den Literaturwerten vor allem beim Rhodium sehr gering sind. Allerdings ist der große Fehler von $\tau_.{Rh_{104i}}$ zu beachten. Dieser kommt wahrscheinlich aufgrund der geringen Impulsrate des Präparats in den größeren Zeiten zustande, da hier die Hintergrundstrahlung die Werte verfälscht. Aufgrund dieser werden auch die letzten 15 Messwerte der zweiten Messung vernachlässigt, da diese das Ergebnis zu stark verfälscht hätten. Bei Vanadium muss zudem der Wert bei $t=\SI{660}{\second}$ vernachlässigt werden, da dieser zu einem undefinierten Wert im Logarithmus führen würde. Die Messwerte mit Standardabweichungen liegen in beiden Messungen leicht außerhalb der Ausgleichsgeraden, beziehungsweise Ausgleichskurve. Dies lässt auf eine größere Sigmaumgebung schließen. Die größeren Abweichungen bei Vanadium sind ebenfalls auf die geringe Impulsrate und die Hintergrundstrahlung zurückzuführen.       
