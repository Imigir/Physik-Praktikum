\section{Durchführung}
\label{sec:Durchführung}

Es wird eine Nulleffektmessung durchgeführt, um die natürliche Untergrundstrahlung zu bestimmen. Dazu wird in einem Zeitraum $\Delta t= \SI{900}{\second}$ ohne Probe eine Messung durchgeführt um einen möglichst geringen Fehler zu erhalten.
Zur Messung der Halbwertszeit einer einfachen Zerfallsreihe wird eine $\ce{^{51}_{23}V}$-Probe nach Aktivierung in die Bleiabschirmung gesteckt und $30$ Messungen der Aktivität einer Zeitspanne $\Delta t= \SI{30}{\second}$ durchgeführt.
Zur Messung einer doppelten Zerfallsreihe wird eine $\ce{^{103}_{45}Rh}$- Probe nach Aktivierung in die Bleiabschirmung gesteckt und $60$ Messungen a $\SI{10}{\second}$
durchgeführt.