
\section{Diskussion}
\label{sec:Diskussion}

\begin{table}
\centering
\caption{Die berechneten Werte im Vergleich zu ihren Referenzwerten .}
\sisetup{table-format=1.2}
	\begin{tabular}{cccS[table-format=2.1]}
		\toprule
		{} & {berechneter Wert} & {Referenzwert} & {Abweichung$/\%$} \\
		\midrule
		a & 1,14  $\pm$ 0,01 & 1,5 \cite{V504} & -24,0\\
		T & \SI{2,7(1)e3}{\kelvin} & \SI{2,3e3}{\kelvin} & 17,4\\
		$\phi$ & \SI{4,799(8)}{\electronvolt} & \SI{4,5}{\electronvolt} \cite{wolfaus} & 6,6\\
		\bottomrule
	\end{tabular}
\label{tab:Diskussion}
\end{table}

\noindent In Abbildung \ref{fig:Kennlinien} ist zu erkennen, dass die Kennlinien für die Heitzspannungen von $I_.H=\SI{2.3}{\ampere}$ bis  $I_.H=\SI{2.5}{\ampere}$ nicht weit genug bis zum Plateau gemessen wurden, was die Abschätzung des Sättigungsstromes in Tabelle \ref{tab:STA} schwierig und fehleranfällig macht. Ansonsten folgen sie dem aus der Theorie erwateten Verlauf.
Der in Abschnitt \ref{subsec:Exponent} bestimmte Wert für den Exponenten $a$ ist im Vergleich zum Theoriewet ziemlich klein (siehe Tabelle \ref{tab:Diskussion}). Dabei liegen die Werte in der logarithmischen Darstellung in Abbildung \ref{fig:Kennlinie25_log} alle gut auf der Ausgleichsgeraden.
Die über den Anlaufstrom bestimmte Temperatur $T$ ist wie in Tabelle \ref{tab:Diskussion} zu sehen im Vergleich zu den später bestimmten Temperaturen etwas zu groß. Dies liegt vermutlich an der Empfindlichkeit der Messanordnung, da sich die gemessenen Ströme im $\si{\nano\ampere}$ Bereich befinden. Zudem wird das Ergebnis durch den Innenwiederstand der Kabel leicht verfälscht.
Die Austrittsarbeit $\phi$ fällt ebenfalls zu groß aus, liegt aber in der nähe des Literaturwertes. Fehlerquelle ist hier der Sättigungsstrom, welcher nur grob abgeschätzt ist. Zudem könnten die Temperaturen zu hoch sein.