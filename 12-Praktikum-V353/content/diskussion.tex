
\section{Diskussion}
\label{sec:Diskussion}

Die Fehler der einzelnen Messwerte sind in allen Messungen gering und auch die Messwerte in Abschnitt \ref{subsec:4} entsprechen bis auf einen kleinen Ausreißer der Theorie. Aus Abbildung \ref{fig:U} ist zu erkennen, dass der RC-Kreis als Integrator geeignet ist, da alle Messungen durch eine theoretische Funktion dargestellt werden können. Die Frequenzabhängigkeiten der Amplitude (Tiefpass) und der Phasenverschiebung sind in Abbildung \ref{fig:Graph2} und \ref{fig:Graph3} gut zu erkennen.\newline
Es lässt sich jedoch zwischen den Werten von $RC$ aus Tabelle \ref{tab:tabRC} eine Abweichung von dem Ersten zu den beiden anderen Werten erkennen. Dies könnte daran liegen, dass der Innenwiederstand $R_.i$ des Generators im Vergleich zu $R$ relativ groß ist. Dies könnte durch eine entsprechende Messung von $R$ überprüft werden.

\begin{table}
	\centering
	\caption{Die in den verschiedenen Versuchsteilen ermittelten Werte für $RC$.}
	\sisetup{table-format=1.2}
	\begin{tabular}{| c | S |}
		\hline
		{Abschnitt} & {$RC$ (Messung)} \\
		\hline
		\ref{subsec:1} & \SI{8.27(7)e-4}{\second} \\
		\hline
		\ref{subsec:2} & \SI{5.25(8)e-3}{\second} \\
		\hline
		\ref{subsec:3} & \SI{4.5(3)e-3}{\second} \\
		\hline
	\end{tabular}
	\label{tab:tabRC}
\end{table}