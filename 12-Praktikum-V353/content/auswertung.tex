\section{Auswertung}
\label{sec:Auswertung}

Die Graphen wurden sowohl mit Matplotlib \cite{matplotlib} als auch NumPy \cite{numpy} erstellt.

\subsection{Bestimmung von RC mithilfe des Entladevorgangs des Kondensators}
\begin{figure}
	\centering
	\includegraphics[width=\linewidth-70pt,height=\textheight-70pt,keepaspectratio]{content/images/Graph1.pdf}
	\caption{Die normierte logarithmische Kondensatorspannung $\mathrm{log}\left(\frac{U_.C}{U_.0}\right)$ in Abhängigkeit von der Zeit $t$.}
	\label{fig:Graph1}
\end{figure}

\noindent Nach Formel \eqref{eq:Q} ergibt sich mit $U_.C=\frac{Q}{C}$:
\[
\mathrm{log}\left(\frac{U_.C}{U_.0}\right) = -\frac{1}{\tau}t 
\]
Dabei beträgt die Spannung $U_.0 = \SI{49.6}{\volt}$.
Eine lineare Ausgleichsrechnung dieser Form liefert mit den Werten aus Tabelle \ref{tab:a} für die Zeitkonstante $\tau$:
\[
\tau = RC = \SI{8.27(7)e-4}{\second}\text{.}
\]
Der Graph ist in Abbildung \ref{fig:Graph1} dargestellt.

\begin{table}
	\centering
	\caption{Die Messwerte für die Zeit $t$, die Spannung $U_.C$, sowie die berechneten Werte für $\mathrm{log}\left(\frac{U_.C}{U_.0}\right)$.}
	\label{tab:taba}
	\sisetup{table-format=1.2}
	\begin{tabular}{S[table-format=2.0]S[table-format=2.3] @{${}\pm{}$} S[table-format=1.3]S[table-format=2.3] @{${}\pm{}$} S[table-format=1.3]}
		\toprule
		{t/\si{\second}} & \multicolumn{2}{c}{$\frac{\text{d}T_1}{\text{d}t}/\si[per-mode=reciprocal]{\kelvin\per\second}$} & \multicolumn{2}{c}{$\frac{\text{d}T_2}{\text{d}t}/\si[per-mode=reciprocal]{\kelvin\per\second}$} \\
		\midrule
		240 & 0.028 & 0.002 & -0.028 & 0.002 \\
		480 & 0.028 & 0.002 & -0.025 & 0.003 \\
		720 & 0.028 & 0.003 & -0.022 & 0.004 \\
		960 & 0.028 & 0.003 & -0.019 & 0.006 \\
		\bottomrule
	\end{tabular}

	\label{tab:a}
\end{table}

\subsection{Bestimmung von RC mithilfe der Amplitude}

\begin{figure}
	\centering
	\includegraphics[width=\linewidth-70pt,height=\textheight-70pt,keepaspectratio]{content/images/Graph2.pdf}
	\caption{Die Amplitude $U$ in Abhängigkeit von der Frequenz $f$.}
	\label{fig:Graph2}
\end{figure}

\noindent Eine nichtlineare Ausgleichsrechnung nach Formel \eqref{eq:U(w)} ergibt mit $U_.0 = \SI{96.0}{\volt}$ und den Werten aus Tabelle \ref{tab:b}:
\[
\tau = RC = \SI{5.25(8)e-3}{\second}\text{.}
\]
Der Graph ist in Abbildung \ref{fig:Graph2} dargestellt.

\begin{table}
	\centering
	\caption{Die Messwerte für die Frequenz $f$, die Amplitude $U$ und die Verschiebung $a$, sowie die berechneten Werte für $\frac{U}{U_.0}$ und die Phasenverschiebung $\phi$.}
	\label{tab:tabb}
	\sisetup{table-format=1.2}
	\begin{tabular}{S[table-format=5.0]S[table-format=2.2]S[table-format=2.2]S[table-format=1.2]S[table-format=1.2]}
		\toprule
		{$f/\si{\hertz}$} & {$A/\si{\volt}$} & {$\frac{A}{U_0}$} & {$a/10^{-4}\si{\second}$} & {$\phi/\si{\radian}$} \\
		\midrule
		   10 & 96.00 & 1.00 & 6.00 & 0.04 \\
		   30 & 94.00 & 0.98 & 6.00 & 0.11 \\
		   70 & 90.00 & 0.94 & 6.00 & 0.26 \\
		  100 & 86.00 & 0.90 & 7.60 & 0.48 \\
		  200 & 66.40 & 0.69 & 6.20 & 0.78 \\
		  300 & 52.00 & 0.54 & 5.20 & 0.98 \\
		  500 & 34.40 & 0.36 & 3.00 & 0.94 \\
		  700 & 25.20 & 0.26 & 3.00 & 1.32 \\
		 1000 & 17.40 & 0.18 & 2.30 & 1.45 \\
		 2000 & 8.00 & 0.08 & 1.20 & 1.51 \\
		 3000 & 4.88 & 0.05 & 0.78 & 1.47 \\
		 4000 & 3.28 & 0.03 & 0.59 & 1.48 \\
		 5000 & 2.24 & 0.02 & 0.47 & 1.48 \\
		 6000 & 1.64 & 0.02 & 0.40 & 1.51 \\
		 7000 & 1.20 & 0.01 & 0.34 & 1.50 \\
		 8000 & 0.84 & 0.01 & 0.30 & 1.51 \\
		10000 & 0.34 & 0.00 & 0.25 & 1.57 \\
		\bottomrule
	\end{tabular}

	\label{tab:b}
\end{table}

\subsection{Bestimmung von RC mithilfe der Phasenverschiebung}
\begin{figure}
	\centering
	\includegraphics[width=\linewidth-70pt,height=\textheight-70pt,keepaspectratio]{content/images/Graph3.pdf}
	\caption{Die Phasenverschiebung $\phi$ in Abhängigkeit von der Frequenz $f$.}
	\label{fig:Graph3}
\end{figure}

\noindent Eine nichtlineare Ausgleichsrechnung nach Formel \eqref{eq:phi(w)} ergibt mit den Werten aus Tabelle \ref{tab:b}:
\[
\tau = RC = \SI{4.5(3)e-3}{\second}\text{.}
\]
Dabei werden die Werte für $\phi$ in der Tabelle nach Formel \eqref{eq:phi} berechnet. Der Graph ist in Abbildung \ref{fig:Graph3} dargestellt.

\subsection{Die RC-Kreis Relativamplitude in Abhängigkeit von der Phase}
\begin{figure}
	\centering
	\includegraphics[width=\linewidth-70pt,height=\textheight-70pt,keepaspectratio]{content/images/Graph4.pdf}
	\caption{Die Relativamplitude $\frac{U}{U_.0}$ in Abhängigkeit von der Phasenverschiebung $\phi$.}
	\label{fig:Graph4}
\end{figure}

\noindent Nach Formel \eqref{eq:U(phi)} und \eqref{eq:phi(w)} ergibt sich:
\begin{equation}
\frac{U(\phi)}{U_0}=\frac{\mathrm{sin}(\phi)}{\mathrm{tan}(\phi)}=\mathrm{cos}(\phi) \label{eq:theo}
\end{equation}
In Abbildung \ref{fig:Graph4} ist die Relativamplitude $\frac{U}{U_.0}$ in einem Polardiagramm gegen die Phasenverschiebung $\phi$ aufgetragen. Die Messwerte werden Tabelle \ref{tab:b} entnommen, und die Theoriekurve wird nach Formel \eqref{eq:theo} berechnet. 

\subsection{Der RC-Kreis als Integrator}
\begin{figure}
\centering
%\center{Dreieck}
\begin{minipage}{0.48\textwidth}
\centering
\includegraphics[width=\linewidth-10pt,height=\textheight-10pt,keepaspectratio]{content/images/Graph5.pdf}
\end{minipage}
\begin{minipage}{0.48\textwidth}
\centering
\includegraphics[width=\linewidth-10pt,height=\textheight-10pt,keepaspectratio]{content/images/5.jpg}
\end{minipage}

\vspace{2em}
%\center{Sinus}
\begin{minipage}{0.48\textwidth}
\centering
\includegraphics[width=\linewidth-10pt,height=\textheight-10pt,keepaspectratio]{content/images/Graph6.pdf}
\end{minipage}
\begin{minipage}{0.48\textwidth}
\centering
\includegraphics[width=\linewidth-10pt,height=\textheight-10pt,keepaspectratio]{content/images/6.jpg}
\end{minipage}

\vspace{2em}
%\center{Rechteck}
\begin{minipage}{.48\textwidth}
\centering
\includegraphics[width=\linewidth-10pt,height=\textheight-10pt,keepaspectratio]{content/images/Graph7.pdf}
\end{minipage}
\begin{minipage}{0.48\textwidth}
\centering
\includegraphics[width=\linewidth-10pt,height=\textheight-10pt,keepaspectratio]{content/images/7.jpg}
\end{minipage}
\caption{Beispiele für eine Dreieck-, Sinus- und Rechteckfunktion, sowie ihre Stammfunktionen im Vergleich mit den Messergebnissen}
\label{fig:U}
\end{figure}

In Abbildung \ref{fig:U} sind die Dreieck-, Sinus- und Rechteckspannung, sowie ihre Stammfunktionen im Vergleich mit theoretischen Beispiel-Funktionen abgebildet.