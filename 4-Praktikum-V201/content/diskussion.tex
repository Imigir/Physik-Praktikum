
\section{Diskussion}
\label{sec:Diskussion}

Es ist zu beobachten, dass die ermittelten Werte von $C_\text{V}$ sehr stark von den zu erwartenden $3R$ abweichen. Dies liegt an den schon stark von den Literaturwerten abweichenden $c_\text{k}$. Aufgrund der starken Abweichungen kann keine Aussage über die Richtigkeit der klassischen Methode getroffen werden. 

Die starken Abweichungen der $c_\text{k}$ nach oben könnte von verschiedenen Faktoren abhängen. Zum einen erscheint der ermittelte Wert für $c_\text{g}m\text{g}$ extrem groß, da bei einer Abschätzung etwa ein Wert von $\SI{250}{\joule\per\kelvin}$ zu erwarten wäre.

Andererseits könnte der falsche Wert durch falsche beziehungsweise unpräzise Messung der Temperaturen zustande gekommen sein, da hier eine Abweichung von $\SI{1}{\kelvin}$ bereits beinahe zu einer Verdoppelung der Ergebnisse führen kann.

Ebenso können die Massen leicht von ihrem wahren Gewicht abweichen, da sie zusammen mit ihrer Halterung gewogen wurden und entsprechend für das Gewicht eine Differenz gebildet werden musste. Die Waage könnte zusätzlich falsch geeicht sein. 

  