
\section{Auswertung}
\label{sec:Auswertung}

Mithilfe der Messdaten soll die Molwärme von Blei, Kupfer und Graphit berechnet werden. Dazu müssen die Wärmekapazität des Kalorimeters, sowie die spezifischen Wärmekapazitäten der einzelnen Materialien bestimmt werden.
Die Fehlerrechnung wurde mithilfe von Uncertainties \cite{uncertainties} durchgeführt.

\subsection{Die Wärmekapazität des Kalorimeters}

Die Wärmekapazität $c_\text{g}m_\text{g}$ des Kalorimeters lässt sich gemäß Formel \eqref{eq:cgmg} bestimmen. Die spezifische Wärmekapazität ist als $c_\text{w} = \SI{4.18}{\joule\per\gram\per\kelvin}$ \cite{V201} gegeben. Die zugehörigen Messwerte befinden sich in Tabelle \ref{tab:tab1}.

\begin{table}
	\centering
	\caption{Die gemessenen Daten für die Massen und Temperaturen.}
	\label{tab:tab1}
	\sisetup{table-format=1.2}
	\begin{tabular}{S[table-format=1.2]S[table-format=4.0]}
		\toprule
		{$\Delta s/\si{\milli\meter}$} & {$N$} \\
		\midrule
		5.00 & 3144 \\
		5.00 & 3105 \\
		5.00 & 3076 \\
		5.00 & 2973 \\
		5.00 & 3183 \\
		\bottomrule
	\end{tabular}

	\label{tab:tab1}
\end{table}

Es folgt:
\begin{displaymath}
	c_\text{g}m_\text{g} = \SI{1842,47}{\joule\per\kelvin}
\end{displaymath}

\subsection{Bestimmung der Molwärme von verschiedenen Matearialien}

Die Molwärme der Materialien wird über die spezifische Wärmekapazität berechnet. Diese wird aus den Messergebnissen bestimmt.
Die Werte für $\alpha$, $\kappa$, $M$ und $\rho$ werden aus derTabelle der Versuchsanleitung übernommen\cite{V201}.

\subsubsection{Kupfer}

Mit den Messwerten aus Tabelle \ref{tab:tab2} und den Formeln \eqref{eq:ck} und \eqref{eq:ckCV} folgen die jeweiligen Werte $c_\text{k}$ und $C_\text{V}$ aus Tabelle \ref{tab:tab3}.
\begin{table}
	\centering
	\caption{Die Messwerte für Kupfer.}
	\input{build/tab2-1.tex}
	\label{tab:tab2}
\end{table}

\begin{table}
	\centering
	\caption{Die berechneten Werte für $c_\text{k}$ und $C_\text{V}$ von Kupfer.}
	\input{build/tab2-2.tex}
	\label{tab:tab3}
\end{table}

Für die Mittelwerte gilt:
\begin{align*}
	c_\text{k} = \SI{0,69(7)}{\joule\per\gram\per\kelvin} \\
	C_\text{V} = \SI{43(5)}{\joule\per\mol\per\kelvin}
\end{align*}
Verglichen mit dem Literaturwert $c_\text{l}=\SI{0,381}{\joule\per\gram\per\kelvin}$ \cite{clit} ergibt sich für $c_\text{k}$ eine Abweichung von $\SI{80,6}{\percent}$ und für $C_\text{V}$ im Vergleich zu den erwarteten $3R$ eine Abweichung von $\SI{72,4}{\percent}$ .

\subsubsection{Blei}

Mit den Messwerten aus Tabelle \ref{tab:tab4} und den Formeln \eqref{eq:ck} und \eqref{eq:ckCV} folgen die jeweiligen Werte $c_\text{k}$ und $C_\text{V}$ aus Tabelle \ref{tab:tab5}.\\
\begin{table}
	\centering
	\caption{Die Messwerte für Blei.}
	\input{build/tab3-1.tex}
	\label{tab:tab4}
\end{table}

\begin{table}
	\centering
	\caption{Die berechneten Werte für $c_\text{k}$ und $C_\text{V}$ von Blei.}
	\input{build/tab3-2.tex}
	\label{tab:tab5}
\end{table}

Für die Mittelwerte gilt:
\begin{align*}
	c_\text{k} = \SI{0,304(5)}{\joule\per\gram\per\kelvin}\\
	C_\text{V} = \SI{61(1)}{\joule\per\mol\per\kelvin}
\end{align*}
Verglichen mit dem Literaturwert $c_\text{l}=\SI{0,129}{\joule\per\gram\per\kelvin}$ \cite{clit} ergibt sich für $c_\text{k}$ eine Abweichung von $\SI{133,8}{\percent}$ und für $C_\text{V}$ im Vergleich zu den erwarteten $3R$ eine Abweichung von $\SI{145,4}{\percent}$ .

\subsubsection{Graphit}

Mit den Messwerten aus Tabelle \ref{tab:tab6} und den Formeln \eqref{eq:ck} und \eqref{eq:ckCV} folgen die jeweiligen Werte $c_\text{k}$ und $C_\text{V}$ aus Tabelle \ref{tab:tab7}.\\

Für die Mittelwerte gilt:
\begin{align*}
	c_\text{k} = \SI{1,6(1)}{\joule\per\gram\per\kelvin} \\
	C_\text{V} = \SI{19(1)}{\joule\per\mol\per\kelvin}
\end{align*}
Verglichen mit dem Literaturwert $c_\text{l}=\SI{0,715}{\joule\per\gram\per\kelvin}$ \cite{clit} ergibt sich für $c_\text{k}$ eine Abweichung von $\SI{127,1}{\percent}$ und für $C_\text{V}$ im Vergleich zu den erwarteten $3R$ eine Abweichung von $\SI{-22,6}{\percent}$ .

\begin{table}
	\centering
	\caption{Die Messwerte für Graphit.}
	\input{build/tab4-1.tex}
	\label{tab:tab6}
\end{table}

\begin{table}
	\centering
	\caption{Die berechneten Werte für $c_\text{k}$ und $C_\text{V}$ von Grphit.}
	\input{build/tab4-2.tex}
	\label{tab:tab7}
\end{table}

