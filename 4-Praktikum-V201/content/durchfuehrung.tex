
\section{Durchführung}
\label{sec:Durchführung}

Zum Befüllen des Kalorimeters wir immer dasselbe Becherglas verwendet, dessen Masse zuvor bestimmt wurde. Nach jeder Bestimmung der jeweiligen Mischtemperatur wird das Kalorimeter geleert, neu befüllt und seine Temperatur notiert. Weiterhin wird auch jedes Mal die Füllmasse des Kalorimeters bestimmt.
\subsection{Wärmekapazität des Kalorimeters}
Zur Bestimmung der Wärmekapazität des Kalorimeters wird das Becherglas gefüllt, die Masse des Wassers bestimmt und das Kalorimeter mit einem Teil des Wassers befüllt. Es wird die Temperatur gemessen und der Rest des Wassers im Glas auf einer Herdplatte erhitzt. Die Temperatur des heißen Wassers wird gemessen und der restlich Inhalt des Glases in das Kalorimeter gegeben. Die sich daraus ergebende Mischtemperatur wird notiert.
\subsection{Wärmekapazität der Proben}
Es wird je drei Mal eine Bleiprobe, eine Kupferprobe und eine Graphitprobe in einem Wasserbad erhitzt, die Temperatur notiert und in das kalte Wasser des Kalorimeters getaucht. Die entstehende Mischtemperatur wird notiert.