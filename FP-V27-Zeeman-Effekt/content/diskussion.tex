
\section{Diskussion}
\label{sec:Diskussion}

Die Ergebnisse der Messungen sind in Tabelle \ref{tab:diskussion} zusammengefasst. Es ist zu erkennen, dass die Ergebnisse mit Abweichungen unter $\SI{3}{\%}$ gut mit den theoretischen Werten übereinstimmen. Dies lässt auf eine sehr gute Kalibrierung des B-Feldes schließen, welches durch die geringen Unsicherheiten auf die Parameter unterstützt wird. Es ist ebenfalls gut, dass die Ausgleichsgerade der Kalibrierung sehr Nahe am Ursprung beginnt. Das Ergebnis ist besser als erwartet, da die B-Feldkalibrierung und das Ablesen der Abstände aus den Bildern nicht exakt sind und dadurch stärkere Abweichungen auf die Werte möglich gewesen wären. Die gemessenen Werte bestätigen die Theorie. 

\begin{table}
	\centering
	\caption{Vergleich der gemessenen und der theoretischen Werte für die verschiedenen $g_{ij}$, sowie die Abweichungen in Prozent.}
	\label{tab:tabrotsigma}
	\sisetup{table-format=1.2}
	\begin{tabular}{cS[table-format=1.3]S[table-format=1.2]S[table-format=2.1]}
		\toprule
		{Messung} & {$|g_{ij_.{gemessen}}|$} & {$|g_{ij_.{theorie}}|$} & {Abweichung/\%} \\
		\midrule
		{rot, \sigma} & \num{0.976(10)} & 1 & -2,4 \\
		{blau, \sigma} & \num{1.714(13)} & 1,75 & -2,1 \\
		{blau, \pi} & \num{0.502(5)} & 0,5 & 0,4 \\
		\bottomrule
	\end{tabular}

	\label{tab:diskussion}
\end{table}
