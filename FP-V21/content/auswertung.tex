\section{Auswertung}
\label{sec:Auswertung}

Die Graphen werden sowohl mit Matplotlib \cite{matplotlib} als auch NumPy \cite{numpy} erstellt. Die Fehlerrechnung wird mithilfe von Uncertainties \cite{uncertainties} durchgeführt.

\subsection{Bestimmung des vertikalen Erdmagnetfelds}

Das vertikale Magnetfeld wird durch ein vertikales Helmholtzspulenpaar kompensiert. Es kann also bei einem angelegten Strom von $\SI{229}{\milli\ampere}$ und Formel \ref{eq:Helmholtz} bestimmt werden zu:
\[
B_.{Erde,vertikal} = \SI{35.1}{\micro\tesla}
\]
Die die Helmholtzspule besitzt dabei den Radius $R=\SI{11,735}{\centi\metre}$ und eine Windungszahl von $N=20$ \cite{V21}.

\subsection{Bestimmung des horizontalen Erdmagnetfelds und der Landé-Faktoren}

Es wird in Abhängigkeit der Modulationsfrequenz $\nu$ das benötigte horizontale B-Feld für die Resonanzen der beiden Rb-Isotope vermessen. Das gesamte B-Feld setzt sich zusammen aus dem der Sweep- und der Horizontalfeld-Spule. Die gemessenen Werte, sowie die nach Formel \ref{eq:Helmholtz} berechneten Werte der B-Felder sind in Tabelle \ref{tab:messung1} eingetragen. Hier sind die Werte für die Sweep-Spule gegeben als $R=\SI{16.39}{\centi\metre}$ und $N=11$, sowie für die Horizontalfeld-Spule $R=\SI{15.79}{\centi\metre}$ und $N=154$ \cite{V21}.
Es wird das gesamte horizontale B-Feld $B_.{Ges,i}$ der beiden Rb-Isotope gegen $\nu$ aufgetragen und eine lineare Ausgleichsrechnung der Form
\[
B_.{Ges,i}(\nu) = a_i\nu + b_i
\]
durchgeführt (vergleiche Abbildung \ref{fig:messung1}).
Es ergeben sich die Parameter:
\begin{align*}
a_1 &= \SI{1.475(17)e-10}{\tesla\per\hertz}\\
a_2 &= \SI{2.185(12)e-10}{\tesla\per\hertz}\\
b_1 &= \SI{14.3(11)}{\micro\tesla}\\
b_2 &= \SI{15.3(8)}{\micro\tesla}\text{.}\\
\end{align*}
Der Parameter $b$ entspricht hier dem horizontalen Erdmagnetfeld. Gemittelt ergibt sich so:
\[
B_.{Erde,horizontal} = \SI{14.8(5)}{\micro\tesla}\text{.}
\]
Der Fehler ist hier der Fehler auf den Mittelwert.
Die Landé-Faktoren sind mit dem Parameter $a$ mithilfe von Gleichung \eqref{eq:} verknüpft über
\begin{equation*}
g_{F,i}=\frac{4\pi m_e}{e a_i} \text{.}
\end{equation*}
Damit ergeben sich die Werte
\begin{align*}
g_{F,1} &= \num{0.484(6)}\\
g_{F,2} &= \num{0.327(18)}\text{.}
\end{align*}
Der Fehler stammt aus der Gaußschen Fehlerfortpflanzung.

\begin{figure}
	\centering
	\includegraphics[width=\linewidth-60pt,height=\textheight-60pt,keepaspectratio]{build/messung1.pdf}
	\caption{Das gesamte horizontale B-Feld $B_.{Ges,i}$ aufgetragen gegen die Modulationsfrequenz $\nu$ für die beiden Rb-Isotope A und B.}
	\label{fig:messung1}
\end{figure}

\begin{table}
	\centering
	\caption{Messwerte der Ströme $I$ der Sweep(S)-Spule und Horizontalfeld(H)-Spule, sowie die daraus berechneten Magnetfelder $B$ für die beiden Rb-Isotope A und B.}
	\label{tab:messung1A}
	\sisetup{table-format=1.2}
	\begin{tabular}{S[table-format=4.0]S[table-format=3.0]S[table-format=3.0]S[table-format=3.2]S[table-format=3.2]S[table-format=3.2]}
		\toprule
		{$\nu/\si{\kilo\hertz}$} & {$I_\text{S,A}/\si{\milli\ampere}$} & {$I_\text{H,A}/\si{\milli\ampere}$} & {$B_\text{S,A}/\si{\micro\tesla}$} & {$B_\text{H,A}/\si{\micro\tesla}$} & {$B_\text{Ges,A}/\si{\micro\tesla}$} \\
		\midrule
		 100 & 483 &   0 & 29.15 & 0.00 & 29.15 \\
		 200 & 718 &   0 & 43.33 & 0.00 & 43.33 \\
		 300 & 382 &  42 & 23.05 & 36.83 & 59.89 \\
		 400 & 112 &  72 & 6.76 & 63.14 & 69.90 \\
		 500 & 271 &  84 & 16.35 & 73.67 & 90.02 \\
		 600 & 237 & 102 & 14.30 & 89.45 & 103.75 \\
		 700 & 129 & 126 & 7.78 & 110.50 & 118.28 \\
		 800 & 368 & 126 & 22.21 & 110.50 & 132.71 \\
		 900 &  69 & 162 & 4.16 & 142.07 & 146.23 \\
		1000 & 582 & 144 & 35.12 & 126.28 & 161.41 \\
		\bottomrule
	\end{tabular}

	\label{tab:messung1B}
	\sisetup{table-format=1.2}
	\begin{tabular}{S[table-format=4.0]S[table-format=3.0]S[table-format=3.0]S[table-format=3.2]S[table-format=3.2]S[table-format=3.2]}
		\toprule
		{$\nu/\si{\kilo\hertz}$} & {$I_\text{S,B}/\si{\milli\ampere}$} & {$I_\text{H,B}/\si{\milli\ampere}$} & {$B_\text{S,B}/\si{\micro\tesla}$} & {$B_\text{H,B}/\si{\micro\tesla}$} & {$B_\text{Ges,B}/\si{\micro\tesla}$} \\
		\midrule
		 100 & 601 &   0 & 36.27 & 0.00 & 36.27 \\
		 200 & 952 &   0 & 57.45 & 0.00 & 57.45 \\
		 300 & 736 &  42 & 44.42 & 36.83 & 81.25 \\
		 400 & 688 &  72 & 41.52 & 63.14 & 104.66 \\
		 500 & 863 &  84 & 52.08 & 73.67 & 125.75 \\
		 600 & 945 & 102 & 57.03 & 89.45 & 146.48 \\
		 700 & 955 & 126 & 57.63 & 110.50 & 168.13 \\
		 800 & 781 & 162 & 47.13 & 142.07 & 189.20 \\
		 900 & 720 & 192 & 43.45 & 168.38 & 211.83 \\
		1000 & 640 & 222 & 38.62 & 194.69 & 233.31 \\
		\bottomrule
	\end{tabular}

	\label{tab:messung1}
\end{table}

\subsection{Bestimmung des Kernspins}

Der Kernspin wird mit Formel \eqref{eq:} berechnet. Dabei ist $L=0$, $S=\frac{1}{2}$, $J=\frac{1}{2}$ und $g_J=2,0023$. 
Somit ergeben sich die Werte:
\begin{align*}
I_.A &= \num{1.567(24)}\\
I_.B &= \num{2.561(17)}
\end{align*}