\section{Theorie}
\label{sec:Theorie}
Bei einer Besetzungsinversion wird die induzierte Abkehr von der thermischen Besetzungswahrscheinlichkeit der Energieniveaus eines Atoms herbeigeführt.
Um dies zu verstehen ist es zunächst nötig, die elektronische Struktur des hier untersuchten Elements Rubidium zu erläutern.

\subsection{Beschreibung der Energieniveaus in Rubidium}
\label{subsec:energieniveaus}

Bei Rubidium handelt es sich um ein Alkalimetall, das nur über ein ungepaartes Elektron in der fünften Schale verfügt, sodass nur dieses zu den Quantenzahlen der Elektronenhülle beiträgt.
Ohne Betrachtung von Korrekturen befindet es sich als Spin-1/2-Teilchen in dem Zustand mit $n=5$, $L=0$ und $M_L=0$, wobei $n$ die Hauptquantenzahl, $L$ die Neben- oder auch Drehimpulsquantenzahl und $M_L$ die zu ihr gehörige magnetische Quantenzahl bezeichnet. In niedrigster Ordnung verfügt das Niveau mit $L=1$ und $M_L=0,\pm1$ über die gleiche Energie wie der besetzte Zustand. Sie sind entartet.\\
Die Berücksichtigung der Spin-Bahn-Kopplung zeigt allerdings, dass dies nicht der Fall ist: Der Bahndrehimpuls $\vec{L}$ und der Spin $\vec{S}$ des Elektrons addieren sich vektoriell zu einem Gesamtdrehimpuls $\vec{J}=\vec{L}+\vec{S}$, sodass dieser eine Erhaltungsgröße darstellt. Die Quantenzahlen $J$ und $M_J$ definieren nun statt $L$ und $M_L$ einen gegebenen Zustand, wobei $J$ von $\lvert L - S \rvert$ bis $\lvert L + S \rvert$ reicht. Zustände mit verschiedenem $J$ sind in dieser Feinstruktur nicht mehr energetisch entartet, während Zustände mit gleichem $J$ und verschiedenem $M_J$ ohne äußeres Magnetfeld energetisch entartet bleiben.
Die Feinstruktur ist als gröbste Aufspaltung links in Abbildung \ref{fig:energieniveaus} zu sehen.\\
Die Niveaus werden in eine weitere Hyperfeinstruktur aufgespalten, wenn die Spin-Spin-Kopplung des Kernspins an den Gesamtdrehimpuls der Elektronenhüllen berücksichtigt wird. Dazu wird der Gesamtdrehimpuls $\vec{F}$ des Atoms wieder durch Vektoraddition gebildet: $\vec{F} = \vec{J} + \vec{I}$. Zu diesem gehört die Quantenzahl $F$, die Werte von $\lvert J - I \rvert$ bis $\lvert J + I \rvert$ annimmt, mit der magnetischen Quantenzahl $M_F$. Im Beispiel in Abbildung \ref{fig:energieniveaus} werden die Feinstrukturniveaus dementsprechend jeweils in zwei Niveaus aufgespalten, wobei wieder Zustände mit gleichem $F$ und verschiedenem $M_F$ ohne äußeres Magnetfeld energetisch entartet bleiben.\\
Die Aufhebung der Entartung dieser Zustände durch ein Magnetfeld geschieht durch den sogenannten Zeeman-Effekt. Dieser spaltet einen $F$-Zustand wiederum in $2F$+1-Zustände auf. Die Energiedifferenz zweier benachbarter Zeeman-Niveaus beträgt
\begin{equation}
\Delta E_Z = g_F \mu_{\text{B}} B
\label{eqn:zeemanDifferenz}
\end{equation}
und ist somit proportional zur magnetischen Flussdichte $B$. Der Faktor $\mu_{\text{B}} = \frac{e \hbar}{2m_e}$ ist das Bohrsche Magneton, das kleinstmögliche magnetische Moment, das ein Elektron tragen kann.
\newpage
\begin{figure}
  	\centering
  	\includegraphics[width=\textwidth]{content/images/energieniveaus.png}
  	\caption{Energieniveaus in \ce{^{87}Rb} mit $I=3/2$ (nicht maßstäblich) \cite{energieniveaus}.}
  	\label{fig:energieniveaus}
\end{figure}

\noindent Der Landé-Faktor des Gesamtdrehimpulses des Atoms $g_F$ ergibt sich aus Kopplungsdiagrammen der beteiligten Drehimpulse näherungsweise zu
\begin{equation}
g_F = g_J \frac{F(F+1)+J(J+1)-I(I+1)}{2F(F+1)}\,,
\label{eqn:g_F_Theorie}
\end{equation}
wobei der Landé-Faktor des Gesamtdrehimpulses des Elektrons
\begin{equation}
g_J = \frac{3{,}0023J(J+1)+1{,}0023(S(S+1)-L(L+1))}{2J(J+1)}
\label{eqn:g_J_Theorie}
\end{equation}
beträgt.

\subsection{Prinzip des optischen Pumpens und der Hochfrequenzspektroskopie}
\label{subsec:prinzipOptischesPumpen} 

Es ist üblich, die Feinstrukturniveaus als Termsymbole ${}^{2S+1}L_J$ mit der Multiplizität $2S+1$ und dem Kennbuchstaben $L$ für den elektronischen Drehimpuls zu schreiben, wobei für $L=0$ ein S und für $L=1$ ein P geschrieben wird.
Für das optische Pumpen an Rubidium ist die Aufspaltung der Hyperfeinstrukturniveaus durch den Zeeman-Effekt nötig. Ohne äußere, zeitabhängige Störung folgt die Besetzung dann näherungsweise einer thermischen Boltzmann-Verteilung, sodass die Elektronen größtenteils im Grundzustand mit niedrigstem $m_F$ angereichert sind.\\
Es wird nun rechtszirkular polarisiertes Licht der Frequenz des $D_1$-Übergangs eingestrahlt, dessen Energie dem Übergang von ${}^{2}P_{1/2}$ nach ${}^{2}S_{1/2}$ entspricht. Übergänge, die durch Absorption dieser Photonen entstehen, gehorchen der Auswahlregel $M_F=+1$, während die spontane Emission keine bestimmten Übergänge bevorzugt. Es gelingt also, durch dieses Prinzip den niedrigsten Zustand nahezu leer zu pumpen und den S-Zustand mit $F=2$, $M_F=2$ anzureichern und eine Besetzungsinversion herbeizuführen. Eine schematische Darstellung der Übergänge findet sich in Abbildung \ref{fig:pumpschema}.

\begin{figure}
	\centering
	\includegraphics[width=\textwidth]{content/images/pumpschema.png}
	\caption{Pumpschema für \ce{^{87}Rb} ($I=3/2$) zur Herstellung einer Besetzungsinversion \cite{caltech}.}
    \label{fig:pumpschema}
\end{figure}

\noindent Ein geeignetes Maß für die Besetzungsinversion stellt die Transparenz der Dampfzelle gegenüber dem einstrahlenden $D_1$-Licht dar. Diese wird mit einer ansteigenden Exponentialfunktion parametrisiert, welche sich bei vollständiger Inversion sättigt, da das Licht dann aufgrund der Auswahlregel keinen Übergang anregen kann.\\
Optischen Pumpen wird oft als spektroskopisches Verfahren eingesetzt, um die aufgespaltenen Energieniveaus mit hoher Genauigkeit zu vermessen. Dieses Messverfahren bedient sich eines zweiten, hochfrequenten magnetischen Feldes (RF-Feld), welches stimulierte Emission aus dem angereicherten Niveau heraus anregt. Für die dafür benötigte Flussdichte $B_m$ gilt die Beziehung
\begin{equation}
h f = g_F \, \mu_{\text{B}} B_m \Delta M_F \implies B_m = \frac{4 \pi m_e}{e g_F} f\,,
\label{eqn:B_M_Theorie}
\end{equation}
sodass ein linearer Zusammenhang zu der Frequenz des Feldes besteht.
Da mit der stimulierten Emission eine Entleerung des zuvor angereicherten Niveaus verbunden wird, wird das Erreichen der Feldstärke $B_m$ mit einer deutlichen Abnahme der Transparenz des Gases verbunden sein, weil der konkurrierende Prozess der Besetzungsinversion durch optisches Pumpen wieder aufnehmen kann. Grafisch ist dies in Abbildung \ref{fig:transparenz} dargestellt.\\
Um 0 herum sinkt die Transparenz deutlich ab, dort kann keine Besetzungsinversion hergestellt werden. Dies liegt daran, dass dies in einem Zweiniveausystem nicht möglich ist, da die Prozesse der Absorption und der stimulierten Emission miteinander konkurrieren, wobei die zusätzliche spontane Emission die Besetzung des Grundzustandes noch weiter bevorzugt. Experimentell wird dies ausgenutzt, um den Einfluss des Erdmagnetfelds zu minimieren.

\begin{figure}
    \centering
    \includegraphics[width=\textwidth]{content/images/transparenz.png}
  	\caption{Transparenz der Dampfzelle in Abhängigkeit eines äußeren Magnetfelds \cite{V21}.}
    \label{fig:transparenz}
\end{figure}

\newpage
\subsection{Abschätzung des quadratischen Zeeman-Effekts}
\label{subsec:quadratischerZeeman}

Die obigen Ausführungen gelten nur für schwache äußere magnetische Felder $B$.
Starke Felder brechen die Drehimpulskopplungen auf, was hier aber nicht betrachtet werden soll.
Um Felder mittlerer Stärke zu betrachten, werden weitere Ordnungen der Störungstheorie bei der Berechnung des Zeeman-Effekts berücksichtigt. In niedrigster Ordnung ist dies dann der quadratische Zeeman-Effekt, sodass die Energiedifferenz aus Gleichung \eqref{eqn:zeemanDifferenz} zu
\begin{equation}
\Delta E_Z = g_F \, \mu_{\text{B}} B + g_F^2 \, \mu_{\text{B}}^2 B^2 \frac{1-2M_F}{\Delta E_{\text{Hy}}}
\label{eqn:zeemanDifferenzQuadratisch}
\end{equation}
korrigiert wird, wobei $E_\text{Hy}$ die Hyperfeinstrukturaufspaltung zwischen den Niveaus zwischen $F$ und $F+1$ bezeichnet.

\subsection{Transiente Effekte}
\label{subsec:transient}

Beim schnellen Ein- und Ausschalten des RF-Feldes zeigt sich, dass der Spin $\vec{F}$ im rotierenden Koordinatensystem um die Achse des RF-Feldes präzediert, wenn das statische Magnetfeld resonant eingestellt ist. Die Lamor-Frequenz beträgt dabei $\gamma B_{\text{RF}}$ mit dem gyromagnetischen Verhältnis $\gamma = g_F \mu_{\text{B}}/\hbar$, sodass die Periodendauer der sogenannten Rabi-Oszillation $T=1/(\gamma B_\text{RF})$ amplitudenabhängig ist.
Es gilt:
\begin{equation}
\frac{T_{87}}{T_{85}} = \frac{\gamma_{85}}{\gamma_{87}}
\label{eqn:transient}
\end{equation}