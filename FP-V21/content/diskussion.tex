
\section{Diskussion}
\label{sec:Diskussion}

\begin{table}
	\centering
	\caption{Die bestimmten Messgrößen, sowie ihre Abweichungen von den Theoriewerten \cite{erde,verhaeltnis,V21}.}
	\label{tab:Ergebnisse}
	\sisetup{table-format=1.2}
	\begin{tabular}{c ccc}
		\toprule
		{Wert}&{gemessen}&{Referenzwert}&{Abweichung} \\
		\midrule
		$\alpha_{max}$ & \SI{28,1}\,\si{\degree} & \SI6{28}\,\si{\degree} & \SI{0,3}\,\si{\percent} \\
		$\theta_{gr}$ & \SI{5}\,\si{\degree} & \SI{5,04}\,\si{\degree} & \SI{-0,9}\,\si{\percent} \\
		$\lambda_{min}$ & \SI{35,1}\,\si{\pico\metre} & \SI{35,4}\,\si{\pico\metre}  & \SI{-1,13}\,\si{\percent} \\
		$E_{kin,max}$ & \SI{35316}\,\si{\eV} & \SI{35000}\,\si{\eV} & \SI{0,9}\,\si{\percent} \\
		$\Delta E_{alpha}$ & \SI{130,7}\,\si{\eV} & - & - \\
		$\Delta E_{beta}$ & \SI{155,0}\,\si{\eV} & - & - \\
		$\sigma_{Cu_K}$ & \SI{3,28} & \SI{3,31} & \SI{-0,76}\,\si{\percent}  \\
		$\sigma_{Cu_{L}}$ & \SI{13,16} & \SI{20,72} & \SI{-36,48}\,\si{\percent} \\
		$\sigma_{Cu_{M}}$ & \SI{29} & \SI{26,64} & \SI{8,87}\,\si{\percent} \\
		$E_{K_{Br}}$ & \SI{13282}\,\si{\eV} & \SI{13470}\,\si{\eV} & \SI{-1,40}\,\si{\percent} \\
		$E_{K_{Sr}}$ & \SI{15988}\,\si{\eV} & \SI{16090}\,\si{\eV} & \SI{-0,64}\,\si{\percent} \\
		$E_{K_{Zn}}$ & \SI{9650}\,\si{\eV} & \SI{9650}\,\si{\eV} & \SI{0,0}\,\si{\percent} \\
		$E_{K_{Zr}}$ & \SI{17903}\,\si{\eV} & \SI{17970}\,\si{\eV} & \SI{-0,37}\,\si{\percent} \\
		$\sigma_{K_{Br}}$ & \SI{3,75} & \SI{3,53} & \SI{6,23}\,\si{\percent} \\
		$\sigma_{K_{Sr}}$ & \SI{3,71} & \SI{3,66} & \SI{1,37}\,\si{\percent} \\
		$\sigma_{K_{Zn}}$ & \SI{3,36} & \SI{3,36} & \SI{0,0}\,\si{\percent} \\
		$\sigma_{K_{Zr}}$ & \SI{3,72} & \SI{3,65} & \SI{1,92}\,\si{\percent} \\
		$R_{\infty}$ & \SI{16,87\pm 0,32}\,\si{\eV} & \SI{13,6}\,\si{\eV} & \SI{24,04}\,\si{\percent} \\
		$\sigma_{L_{Bi}}$ & \SI{3,31} & \SI{3,58} & \SI{-7,54}\,\si{\percent} \\
		\bottomrule
	\end{tabular}

	\label{tab:Fehler}
\end{table}

\noindent Die bestimmten Messgrößen, sowie ihre Abweichungen von den Theoriewerten stehen in Tabelle \ref{tab:Fehler}. 
Die Erdmagnetfelder $B_.{Erde,v} = \SI{35,1}{\micro\tesla}$ und $B_.{Erde,h}=\SI{14,8(5)}{\micro\tesla}$ liegen mit Abweichungen von $\SI{-20,2}{\%}$ und $\SI{-26}{\%}$ nur in der richtigen Größenordnung. Dabei sind die statistischen Fehler auf die Parameter der Ausgleichsrechnungen gering. Die Abweichung der Werte von der Theorie kommt also vermutlich durch systematische Abweichungen bei der Ausrichtung der Messapparatur zum Erdmagnetfeld und dem Kompensieren des vertikalen Magnetfelds zustande.\\
Die Landé-Faktoren wurden bestimmt zu:
\begin{align*}
g_{F,1} &= \num{0.484(6)}\\
g_{F,2} &= \num{0.327(18)}\text{.}
\end{align*}
Die daraus bestimmten Kernspins sind $I_.A= \num{1,567(24)}$ und $I_.B=\num{2,561(17)}$. Somit kann das Isotop A $\text{Rb}^{87}$ zugeordnet werden und Isotop B $\text{Rb}^{85}$. Die Kernspins besitzen Abweichungen von den Theoriewerten von $\SI{4,5}{\%}$ und $\SI{2,4}{\%}$. Damit stimmen sie gut mit der Theorie überein. Da sie im direkten Verhältnis mit den Landé-Faktoren stehen und die Werte bei der Ausgleichsrechnung alle nahe der Ausgleichsgeraden liegen, sind auch diese gut bestimmt worden.\\
Das bestimmte Isotopenverhältnis beträgt $\xi=2,263$ und weicht damit um $\SI{-12.7}{\%}$ von der Theorie ab.
Bei der Untersuchung des quadratischen Zeeman-Effekts ist zu erkennen, dass der Beitrag des quadratischen Zeeman-Effekts um drei Größenordnungen kleiner ist, als der des linearen Zeeman-Effekts. Er kann also vernachlässigt werden.\\
Die Zunahme der Transparenz mit der Zeit kann gut durch eine Exponentialfunktion angenähert werden. Die Messwerte liegen alle nahe der Ausgleichsfunktion, was durch die geringen Unsicherheiten auf die bestimmten Parameter unterstützt wird. 
Über die Rabi-Oszillationen wurde das Verhältnis der Landé-Faktoren bestimmt zu $\eta=\num{1,30(16)}$ und weicht damit um $\SI{-13.3}{\%}$ von der Theorie ab. Die Werte können dabei gut durch die Hyperbelfunktionen beschrieben werden. Die statistischen Fehler auf die Parameter der Ausgleichsrechnung sind gering. Die Abweichungen bei den Verhältnissen $\xi$ und $\eta$ sind also vermutlich ebenfalls auf die nicht optimale Ausrichtung der Messapparatur und der damit eingehenden Unterschätzung des B-Feldes zurückzuführen. 
