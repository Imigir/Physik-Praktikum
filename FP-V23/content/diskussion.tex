
\section{Diskussion}
\label{sec:Diskussion}

Bei der Untersuchung der Reproduzierbarkeit der Messungen sind wie zu erwarten keine Abweichungen der Frequenzen zu beobachten. Die geringen Abweichungen der Amplitude liegen im Rahmen der statistischen Ungenauigkeit.\\
Bei der Modellierung des Wasserstoffatoms konnten die ersten vier Peaks gut den ersten Kugelflächenfunktionen mit $m=0$ zugeordnet werden. Der letzte Peak lies sich nicht eindeutig zuordnen. Werden die vorherigen Peaks betrachtet, so würden die Quantenzahlen $m=0$ und $l=5$ erwartet werden. Der Plot sieht jedoch am ehesten nach einer Kugelflächenfunktion mit $m=1$ und $l=2$ aus. \\
Die Schallgeschwindigkeit $c$ konnte sowohl über die Frequenzabstände der Resonanzen in Abhängigkeit von der Länge, als auch über die Dispersionskurve gut bestimmt werden. Der Vergleich mit der Literatur ist in Tabelle \ref{tab:Diskussion} zu sehen. Die lineare Dispersion stimmt mit der Theorie sehr gut überein.\\
Die Breiten der Bänder und Bandlücken aus Tabelle \ref{tab:band} verhalten sich ebenfalls wie erwartet, da ein schmalerer Durchmesser von der Iris eine schwächere Kopplung darstellt.\\
Beim Untersuchen von zusammenhängen, wurden direkte lineare Zusammenhänge zwischen der Anzahl der Peaks pro Band und der Anzahl der Segmente, sowie zwischen der Länge der Segmente und der Anzahl der Bänder entdeckt. Die Zustandsdichte steigt ebenfalls wie erwartet an den Rändern der Bänder stark an. Da es sich um einen endlichen Aufbau handelt ist sie dort nicht, wie in der Theorie, unendlich, sondern nimmt einen endlichen Wert an.\\
Bei periodischen Strukturen konnte die Ausbildung von Substrukturen beobachtet werden. Das einführen einer Fehlstelle hingegen führte zu einem Defekt-Peak in der Bandlücke. Dies kann nützlich sein, um zum Beispiel Elektronen leichter in ein höheres Band anzuregen.

\begin{table}
\centering
	\caption{Die Werte für die Schallgeschwindigkeit $c$ im Vergleich mit dem Literaturwert \cite{cLuft}.}
	\label{tab:Diskussion}
	\sisetup{table-format=1.2}
	\begin{tabular}{ccc}
		\toprule
		{gemessen}&{Referenzwert\cite{DipolW}}&{Abweichung} \\
		\midrule
		\SI{339,6(4)}{\metre\per\second}   & \SI{343,6}{\metre\per\second} & -1,16\%\\
		\SI{343,77(16)}{\metre\per\second} & \SI{343,6}{\metre\per\second} & 0,05\%\\
		\bottomrule
	\end{tabular}
\end{table}