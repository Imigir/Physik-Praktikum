\section{Durchführung}
\label{sec:Durchführung}

Bei allen Versuchen wird eine Schrittdauer von $\SI{50}{\milli\second}$ verwendet.

\subsection{Modellierung: Teilchen im Potentialtopf}

Es werden acht $\SI{75}{\milli\metre}$-Rohrstücke zwischen Lautsprecher und Mikrofon befestigt. Am Computer wird mit dem Programm SpektrumSLC.exe ein Spektrum von $100$ bis $\SI{10000}{\hertz}$ mit einer Schrittweite von $\SI{10}{\hertz}$ aufgenommen.\\
Es werden zwei $\SI{75}{\milli\metre}$-Rohrstücke zwischen Lautsprecher und Mikrofon befestigt. Mit einer Schrittweite von $\SI{5}{\hertz}$ wird ein Spektrum von $5000$ bis $\SI{14000}{\hertz}$ aufgenommen.

\subsection{Modellierung: Ein eindimensionaler periodischer Festkörper}
Im Bereich von  $5000$ bis $\SI{14000}{\hertz}$ wird in $\SI{5}{\hertz}$-Schritten das Spektrum für ein bis acht $\SI{75}{\milli\metre}$-Stücke aufgenommen.\\
Von einer Konstellation aus zwölf $\SI{50}{\milli\metre}$-Rohren wird ein Spektrum von $100$ bis $\SI{12000}{\hertz}$.\\
Dasselbe wird für acht $\SI{50}{\milli\metre}$-Rohre die durch Irisse mit einem Innendurchmesser von $\SI{16}{\milli\metre}$ gekoppelt sind und für denselben Aufbau mit Irissen mit $13$ und $\SI{10}{\milli\metre}$ Innendurchmesser wiederholt.\\
Die Messung wird für einen Aufbau aus zehn und zwölf $\SI{50}{\milli\metre}$-Röhren mit $\SI{16}{\milli\metre}$-Iris-Kopplung wiederholt. \\
Dasselbe Spektrum wird für acht $\SI{75}{\milli\metre}$-Rohre mit $\SI{16}{\milli\metre}$-Iris-Kopplung aufgenommen.\\
Für ein einzelnes $\SI{50}{\milli\metre}$-Röhrenstück wird ein Spektrum von $100$ bis $\SI{22000}{\hertz}$ aufgenommen.\\
Die Messung wird für ein einzelnes ein $\SI{75}{\milli\metre}$-Röhrenstück wiederholt.\\
Als Einheitszelle werden zwei $\SI{50}{\milli\metre}$-Rohre über $10$-, $13$- und $\SI{16}{\milli\metre}$-Irisse gekoppelt.
Für einen Verbund von zwei, drei, vier und sechs solcher Zellen, wird ein Spektrum von $100$ bis $\SI{12000}{\hertz}$ aufgenommen.\\
Für zwölf abwechselnd über $13$- und $\SI{16}{\milli\metre}$-Irisse gekoppelte $\SI{50}{\milli\metre}$-Rohre wird dasselbe Spektrum aufgenommen.\\
Als Einheitszelle wird ein ein $\SI{50}{\milli\metre}$-Röhrenstück mit einem ein $\SI{75}{\milli\metre}$-Röhrenstück über eine $\SI{16}{\milli\metre}$-Iris gekoppelt und mit einer weiteren $\SI{16}{\milli\metre}$-Iris abgeschlossen.
Für einen Verbund aus fünf solchen Zellen wird ein Spektrum von $400$ bis $\SI{12000}{\hertz}$ aufgezeichnet.\\
Zwölf $\SI{50}{\milli\metre}$-Röhren werden über $\SI{16}{\milli\metre}$-Irisse gekoppelt.
Als Defekt wird das dritte Segment durch ein $\SI{75}{\milli\metre}$-Röhrenstück ersetzt und ein Spektrum von $100$ bis $\SI{6000}{\hertz}$ aufgenommen.
Dasselbe wird für $\SI{75}{\milli\metre}$-Defekte am sechsten und am neunten Segment, sowie für $\SI{25}{\milli\metre}$-Defekte an allen drei Stellen wiederholt.

\subsection{Modellierung: Das Wasserstoffatom}

Der Skalenwinkel $\alpha$, unter dem die Messungen durchgeführt werden, kann in den Polarwinkel $\theta$ überführt werden durch
\begin{equation}
\theta=\arccos{\left(\frac{1}{2}\cos{\alpha}-\frac{1}{2}\right)}\text{.}\label{eq:theta}
\end{equation}
\newline
Für $\alpha=\SI{180}{\degree}$ zwischen den Hemisphären wird bei einer Schrittweite von $\SI{10}{\hertz}$ ein Spektrum von $100$ bis $\SI{10000}{\hertz}$ aufgenommen.\\
Die Messung wird für $\alpha=\SI{0}{\degree}$ und weitere Winkel wiederholt.\\
Für $\alpha=\SI{0}{\degree}$, $\alpha=\SI{20}{\degree}$ und $\alpha=\SI{40}{\degree}$ wird mit einer Schrittweite von $\SI{0,1}{\hertz}$ der Bereich um $\SI{5000}{\hertz}$ aufgelöst.\\
Bei $\alpha=\SI{180}{\degree}$ wird ein Spektrum von $2000$ bis $\SI{7000}{\hertz}$ aufgenommen um die Resonanzfrequenzen zu identifizieren.
Für diese werden Polarplots angefertigt, indem $\alpha$ in $\SI{10}{\degree}$-Schritten variiert wird.

%◙☻♠◘♥☺
