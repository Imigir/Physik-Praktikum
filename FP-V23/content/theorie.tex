\section{Theorie}
\label{sec:Theorie}

\subsection{Akkustische stehende Wellen in einer Röhre - ein Teilchen im Potentialtopf}
In einem Gas, hier der Luft, ist Schall eine Longitudinal-Welle. Sie kann sowohl über die Geschwindigkeit der der bewegten Luft, als auch über die sich fortpflanzenden Schwankungen des Drucks $p$ beschrieben werden. Im folgenden wird letztere Beschreibung verwendet.
Wird in einer abgeschlossenen Röhre von einem Ende eine Schallwelle ausgesandt und an der gegenüberliegenden Wand kann es zu Resonanzeffekten kommen. Falls gilt
\begin{align}
2L&=n\lambda=n\frac{c}{f}\\
\Leftrightarrow f&=\frac{n c}{2L},\\\label{eq:f}
\end{align}
mit der Länge der Röhre $L$, $n\in\mathbb{N}$, der Frequenz und der Wellenlänge der Schallwelle $f$ und $\lambda$ und der Schallgeschwindigkeit in Luft $c$, kann sich eine stehende Welle ausbilden.
Die Ausbreitung der Schallwelle lässt sich dabei über die Wellengleichung
\[
\frac{\partial^2 p}{\partial t^2}=\frac{1}{c^2}\Delta p
\]
beschreiben. Aufgrund der Longitudinalität der Welle kann jedoch angenommen werden, dass $p$ nur von einer Dimension abhängt - der hier als $x$ bezeichneten Symmetrieachse des Rohrs. Dadurch vereinfacht sich die Wellengleichung zu
\[
\frac{\partial^2 p}{\partial t^2}=\frac{1}{c^2}\frac{\partial^2 p}{\partial x^2},
\]
sodass ein Lösungsansatz durch
\[
p(x,t)=p_.0cos(k x +\alpha)cos(\omega t)
\]
gegeben ist, mit der Winkelgeschwindigkeit $\omega=2\pi f$,dem Wellenvektor $k$ und der Phase $\alpha$. Aus den Randbedingungen 
\[
\frac{\partial p}{\partial x}(0)=\frac{\partial p}{\partial x}(L)=0
\]
ergibt sich $\alpha=0$ und $k=\frac{m\pi}{L}$, mit $m\in\mathbb{N}$.\\\newline


\subsection{Der Kugelresonator - das Wasserstoff-Atom}


\subsection{Symmetriebrechung am Kugelresonator - molekulare Bindung}


\subsection{Ein eindimensionaler Festkörper}