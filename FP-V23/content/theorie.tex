\section{Theorie}
\label{sec:Theorie}

\subsection{Akkustische stehende Wellen in einer Röhre - ein Teilchen im Potentialtopf}
In einem Gas, hier der Luft, ist Schall eine Longitudinal-Welle. Sie kann sowohl über die Geschwindigkeit der der bewegten Luft, als auch über die sich fortpflanzenden Schwankungen des Drucks $p$ beschrieben werden. Im folgenden wird letztere Beschreibung verwendet.
Wird in einer abgeschlossenen Röhre von einem Ende eine Schallwelle ausgesandt und an der gegenüberliegenden Wand kann es zu Resonanzeffekten kommen. Falls gilt
\begin{align}
2L&=n\lambda=n\frac{c}{f}\\
\Leftrightarrow f&=\frac{n c}{2L},\\\label{eq:f}
\end{align}
mit der Länge der Röhre $L$, $n\in\mathbb{N}$, der Frequenz und der Wellenlänge der Schallwelle $f$ und $\lambda$ und der Schallgeschwindigkeit in Luft $c$, kann sich eine stehende Welle ausbilden.
Die Ausbreitung der Schallwelle lässt sich dabei über die Wellengleichung
\begin{equation}
\frac{\partial^2 p}{\partial t^2}=c^2\Delta p\label{eq:WGL}
\end{equation}
beschreiben. Auf Grund der Longitudinalität der Welle kann jedoch angenommen werden, dass $p$ nur von einer Dimension abhängt - der hier als $x$ bezeichneten Symmetrieachse des Rohrs. Dadurch vereinfacht sich die Wellengleichung zu
\[
\frac{\partial^2 p}{\partial t^2}=\frac{1}{c^2}\frac{\partial^2 p}{\partial x^2},
\]
sodass ein Lösungsansatz durch
\[
p(x,t)=p_.0\cos(k x +\alpha)cos(\omega t)
\]
gegeben ist, mit der Winkelgeschwindigkeit $\omega=2\pi f$,dem Wellenvektor $k$ und der Phase $\alpha$. Aus den Randbedingungen 
\[
\frac{\partial p}{\partial x}(0)=\frac{\partial p}{\partial x}(L)=0
\]
ergibt sich $\alpha=0$ und $k=\frac{m\pi}{L}$, mit $m\in\mathbb{N}$.
Aus dem multiplizieren der Gleichung \eqref{eq:f} mit $2\pi$ ergibt sich die Dispersionsrelation
\begin{equation}
\omega(k)=c k\text{.}\label{eq:omega_k}
\end{equation}
Im klassischen Fall haben die verschiedenen Eigenzustände eine Lebensdauer, da es auf Grund von Reibung zu Energieverlust kommt.
Die Darstellung dessen wird über einen mit der Zeit abfallenden Term $.e^{-\gamma t}$ erreicht.
Da aber die stehende Welle auch durch den Lautsprecher betrieben wird, lässt sich die zeitliche Abhängigkeit des Drucks als getriebener harmonischer Oszillator betrachten:
\[
\frac{\partial^2p(t)}{\partial t^2}+2\gamma\frac{\partial p(t)}{\partial t}+\omega_.0^2p(t)=K .e^{\omega t}
\]
Damit ergibt sich eine Lösung 
\[
p(t)=A_.1 .e^{-\gamma t+i(\omega t+\phi_.1)}+A .e^{i(\omega t + \phi)},
\]
mit den Phasen $\phi_.1$ und $\phi$, wobei letzterer gegeben ist durch 
\[
\phi=\arctan{\frac{2\gamma\omega}{\omega^2_.0-\omega^2}}
\]
gegeben ist.
Für große Zeiten geht der 1. Term schnell gegen $0$, sodass nur noch die komplexe Amplitude $A$ des stabilen Zustand gemessen wird.
Für eine einzelne Resonanzstelle kann diese geschrieben werden als
\[
A=\frac{K .e^{i\phi}}{\omega^2_.0-\omega^2+2i\gamma\omega}
\]
und ihr Betrag als
\[
|A|=\frac{K}{\sqrt{\left(\omega^2_.0-\omega^2\right)^2+4\gamma^2\omega^2}}\text{.}
\]
Bei der Untersuchung eines Resonanzpeaks im Spektrum kann angenommen werden, dass $\omega\approx const$ und die Reibung sehr klein gegen die Resonanzfrequenz ist: $\gamma\ll\omega_.0$. Damit vereinfacht sich der Amplitudenbetrag zu
\[
|A|=\frac{2\omega_.0}{\sqrt{\left(\omega^2_.0-\omega^2\right)^2+4\gamma^2\omega^2}}\text{.}
\]
Nach Bedingung \eqref{eq:f} sind mehrere Resonanzfrequenzen $\omega_.{0,n}$ und damit auch andere Reibungskoeffizienten $\gamma_.n$ und Phasen $\phi_.n$ möglich, sodass $|A|$ geschrieben werden muss als
\begin{equation}
|A|=\biggl|\sum_.{n=0}^N\frac{K .e^{i\phi_.n}}{\omega^2_.{0,n}-\omega^2+2 i\gamma_.n\omega}\biggl|\text{.}\label{eq:absA}
\end{equation}
\\
\newline
Das quantenmechanische Analogon zur stehenden Welle in einer Röhre ist ein Teilchen im Potentialtopf.
Dies kann über die Schrödingergleichung mit der Wellenfunktion $\psi(x,t)$ beschrieben werden:
\[
i\hbar\frac{\partial\psi(r,t)}{\partial t}=-\frac{\hbar^2}{2m}\Delta\psi(r,t) + V(r)\psi(r,t)
\]
mit dem Potential 
\[V(r) = \left\{
\begin{array}{ll}
0 & |r| \geq L \\
\infty & \, \textrm{sonst} \\
\end{array}
\right.
\]
und dem reduzierten Planck'schen Wirkungsquantum $\hbar$.
Unter Berücksichtigung, dass nur ein eindimensionales Problem betrachtet wird und durch einsetzen von $V(r)$ vereinfacht sich die Schrödingergleichung zu
\[
i\hbar\frac{\partial\psi(x,t)}{\partial t}=-\frac{\hbar^2}{2m}\Delta\psi(x,t)\text{.}
\]
Ein Separationsansatz $\psi(x,t)=\phi(x)T(t)$, mit der Lösung der zeitunabhängigen Schrödingergleichung $\phi(x)$, liefert
$\psi(x,t)=A\sin(kx+\alpha) .e^{-i\omega t}$. Aus den Randbedingungen 
$\psi(0,t)=\psi(L,t)=0$ folgt wie im klassischen Fall $k=\frac{m\pi}{L}$ und $\alpha=0$.
Beim quantenmechanischen Teilchen ist die Frequenz mit der Energie verknüpft:
\[
\hbar\omega=E=\frac{\hbar^2k^2}{2m}\text{.}
\]
Damit ergibt sich eine Dispersionsrelation
\begin{equation}
\omega(k)=\frac{\hbar k^2}{2m}\text{.}\label{eq:omega_q}
\end{equation}
Auch im quantenmechanischen System gilt das prinzip der minimalen Energie, sodass alle angeregten Zustände mit der Zeit in den Grundzustand zurückkehren. Damit lässt sich die Zeitabhängigkeit der Wellenfunktion schreiben, als
\[
\psi(t)=.e^{-(\lambda+i\omega_.0)t}
\]
Aus einer Fouriertransformation ergibt sich die Amplitude
\[
A(\omega)=\frac{1}{\sqrt{2\pi}}\frac{1}{\lambda+i(\omega_.0-\omega)}
\]
und ihr Betrag
\begin{equation}
|A(\omega)|=\frac{1}{\sqrt{2\pi}}\frac{1}{\sqrt{(\omega_.0-\omega)^2+\lambda^2}}
\end{equation}
\subsection{Der Kugelresonator - das Wasserstoff-Atom}
Für die Resonanzen in einem sphärischen Resonator muss zunächst wieder die Wellengleichung \eqref{eq:WGL} betrachtet werden. Abseparieren der Zeit mithilfe des Ansatzes $p(\vec{r},t)=p(\vec{r})\cos{(\omega t)}$ liefert die Eigenwert-Gleichung
\begin{equation*}
-\frac{\omega^2}{c^2}p(\vec{r})=\Delta p(\vec{r})\text{.}
\end{equation*}
In Kugelkoordinaten und mit dem Separationsansatz $p(r,\theta,\phi)=Y^m_l(\theta,\phi)R(r)$ lässt sich diese aufteilen in einen Winkel-
\begin{equation}
-\left[\frac{1}{\sin{\theta}}\frac{\partial}{\partial\theta}\left(\sin{\theta}\frac{\partial}{\partial\theta}\right)+\frac{1}{\sin^2{\theta}}\frac{\partial^2}{\partial\theta^2}\right]Y^m_l(\theta,\phi)=l(l+1)Y^m_l(\theta,\phi)\label{eq:Y_lm}
\end{equation}
und einen Radialanteil 
\begin{equation}
-\frac{\partial^2 R(r)}{\partial r^2}-\frac{2}{r}\frac{\partial R(r)}{\partial r}+\frac{l(l+1)}{r^2}R(r)=\frac{\omega^2}{c^2}R(r)\text{.}\label{eq:R}
\end{equation}
Die Eigenwerte von Gleichung \eqref{eq:Y_lm} werden durch die Winkelquantenzahl $l$ charakterisiert, die von Gleichung \eqref{eq:R} durch die radiale Quantenzahl $k$. Die Resonanzfrequenzen $\omega_.{k l}$, die eine stehende Welle im sphärischen Resonator hervorrufen, sind die Eigenwerte dieser radialen Eigenwertgleichung und damit $(2l+1)$-fach entartet in der magnetischen Quantenzahl $m$.\\
\newline
Für das quantenmechanische Analogon - das Wasserstoffatom - muss wieder die Schrödingergleichung gelöst werden, wobei hier $V(\vec{r})=-\frac{e^2}{r}$ das Coulomb-Potential ist:
\[
E\psi(\vec{r})=-\frac{\hbar^2}{2m}\Delta\psi(\vec{r})-\frac{e^2}{r}\psi(\vec{r})\text{.}
\]
Mit dem selben Ansatz wie beim sphärischen Resonator lässt sich diese DGL aufteilen, wobei der Winkelanteil äquivalent zu Gleichung \eqref{eq:Y_lm} ist. Der radiale Anteil ist gegeben durch 
\begin{equation}
-\frac{\hbar^2}{2m}\left(\frac{1}{r}\frac{\partial^2}{\partial r^2}r R(r)+\frac{l(l+1)}{r^2}R(r)\right)-\frac{e^2}{r}R(r)=E R(r)\text{.}\label{eq:R2}
\end{equation}
Die Energieeigenwerte dieser Gleichung können mit der Hauptquantenzahl
$n=l+1+k$ und der Rydberg-Energie $E_.{Ryd}$ geschrieben werden als
\[
E_{n l}=-E_.{Ryd}\frac{1}{n^2}\text{.}
\]
Auch sie sind auf Grund der Kugelsymmetrie des Coulomb-Potentials entartet in $m$.
%\subsection{Symmetriebrechung am Kugelresonator - molekulare Bindung}
%Wird die Kugelsymmetrie des sphärischen Resonators durch das Einbringen von Metallringen zwischen seine Hemisphären gebrochen, wird auch die Entartung in der Magnetquantenzahl $m$ aufgehoben. Die Quantisierungsachse ist nicht mehr durch die Verbindungslinie zwischen Lautsprecher und Mikrofon gegeben, sondern durch die Symmetrieachse des Resonators. Da $-l\leq m\leq l$ gilt, sind Moden mit $\pm m$ weiterhin entartet, überlagern sich und es kommt zu einer stehenden Welle. Da $Y^m_l\propto .e^{i m\phi}$, ergibt eine Superposition
%\[
%.e^{-i m\phi}+.e^{i m\phi}=\cos{m\phi}
%\]
\subsection{Ein eindimensionaler Festkörper}



%☻☻☺○☻☻♦☺