\section{Theorie}
\label{sec:Theorie}

\subsection{Akkustische stehende Wellen in einer Röhre - ein Teilchen im Potentialtopf}
In einem Gas, hier der Luft, ist Schall eine Longitudinal-Welle. Sie kann sowohl über die Geschwindigkeit der der bewegten Luft, als auch über die sich fortpflanzenden Schwankungen des Drucks $p$ beschrieben werden. Im folgenden wird letztere Beschreibung verwendet.
Wird in einer abgeschlossenen Röhre von einem Ende eine Schallwelle ausgesandt und an der gegenüberliegenden Wand kann es zu Resonanzeffekten kommen. Falls gilt
\begin{align}
2L&=n\lambda=n\frac{c}{f}\\
\Leftrightarrow f&=\frac{n c}{2L},\\\label{eq:f}
\end{align}
mit der Länge der Röhre $L$, $n\in\mathbb{N}$, der Frequenz und der Wellenlänge der Schallwelle $f$ und $\lambda$ und der Schallgeschwindigkeit in Luft $c$, kann sich eine stehende Welle ausbilden.
Die Ausbreitung der Schallwelle lässt sich dabei über die Wellengleichung
\[
\frac{\partial^2 p}{\partial t^2}=\frac{1}{c^2}\Delta p
\]
beschreiben. Auf Grund der Longitudinalität der Welle kann jedoch angenommen werden, dass $p$ nur von einer Dimension abhängt - der hier als $x$ bezeichneten Symmetrieachse des Rohrs. Dadurch vereinfacht sich die Wellengleichung zu
\[
\frac{\partial^2 p}{\partial t^2}=\frac{1}{c^2}\frac{\partial^2 p}{\partial x^2},
\]
sodass ein Lösungsansatz durch
\[
p(x,t)=p_.0\cos(k x +\alpha)cos(\omega t)
\]
gegeben ist, mit der Winkelgeschwindigkeit $\omega=2\pi f$,dem Wellenvektor $k$ und der Phase $\alpha$. Aus den Randbedingungen 
\[
\frac{\partial p}{\partial x}(0)=\frac{\partial p}{\partial x}(L)=0
\]
ergibt sich $\alpha=0$ und $k=\frac{m\pi}{L}$, mit $m\in\mathbb{N}$.
Aus dem multiplizieren der Gleichung \eqref{eq:f} mit $2\pi$ ergibt sich die Dispersionsrelation
\begin{equation}
\omega(k)=c k\text{.}\label{eq:omega_k}
\end{equation}
Im klassischen Fall haben die verschiedenen Eigenzustände eine Lebensdauer, da es auf Grund von Reibung zu Energieverlust kommt.
Die Darstellung dessen wird über einen mit der Zeit abfallenden Term $.e^{-\gamma t}$ erreicht.
Da aber die stehende Welle auch durch den Lautsprecher betrieben wird, lässt sich die zeitliche Abhängigkeit des Drucks als getriebener harmonischer Oszillator betrachten:
\[
\frac{\partial^2p}{\partial t^2}+2\gamma\frac{\partial p}{\partial t}+\omega_.0^2p=K\cos{(\omega t)}
\]
\\
\newline
Das quantenmechanische Analogon zur stehenden Welle in einer Röhre ist ein Teilchen im Potentialtopf.
Dies kann über die Schrödingergleichung mit der Wellenfunktion $\psi(x,t)$ beschrieben werden:
\[
i\hbar\frac{\partial\psi(r,t)}{\partial t}=-\frac{\hbar^2}{2m}\Delta\psi(r,t) + V(r)\psi(r,t)
\]
mit dem Potential 
\[
V(r) = \left\{
\begin{array}{ll}
0 & \|r\| \geq L \\
\infty & \, \textrm{sonst} \\
\end{array}
\right.
\]
und dem reduzierten Planck'schen Wirkungsquantum $\hbar$.
Unter Berücksichtigung, dass nur ein eindimensionales Problem betrachtet wird und durch einsetzen von $V(r)$ vereinfacht sich die Schrödingergleichung zu
\[
i\hbar\frac{\partial\psi(x,t)}{\partial t}=-\frac{\hbar^2}{2m}\Delta\psi(x,t)\text{.}
\]
Ein Separationsansatz $\psi(x,t)=\phi(x)T(t)$, mit der Lösung der zeitunabhängigen Schrödingergleichung $\phi(x)$, liefert
$\psi(x,t)=A\sin(kx+\alpha) .e^{-i\omega t}$. Aus den Randbedingungen 
$\psi(0,t)=\psi(L,t)=0$ folgt wie im klassischen Fall $k=\frac{m\pi}{L}$ und $\alpha=0$.
Beim quantenmechanischen Teilchen ist die Frequenz mit der Energie verknüpft:
\[
\hbar\omega=E=\frac{\hbar^2k^2}{2m}\text{.}
\]
Damit ergibt sich eine Dispersionsrelation
\begin{equation}
\omega(k)=\frac{\hbar k^2}{2m}\text{.}\label{eq:omega_q}
\end{equation}
\subsection{Der Kugelresonator - das Wasserstoff-Atom}


\subsection{Symmetriebrechung am Kugelresonator - molekulare Bindung}


\subsection{Ein eindimensionaler Festkörper}



%☻☻