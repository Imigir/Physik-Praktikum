\section{Durchführung}
\label{sec:Durchführung}

Mit Hilfe des Justierlasers werden die Resonatorspiegel und das Laserrohr auf der optischen Schiene montiert und mit Hilfe je einer Lochblende mit Fadenkreuz direkt am Laser und im größtmöglichen Abstand justiert. Die Stromquelle wird auf $\SI{6.5}{\milli\ampere}$ eingestellt und die Resonatorspiegel nachjustiert, bis ein roter Laserstrahl zwischen Brewsterfenster und Resonatorspiegel zu sehen.
Für eine Kombination aus einem flachen und einem konkaven Spiegel und zwei konkaven Spiegeln wird zur Überprüfung der Stabilitätsbedingung die maximale Resonatorlänge durch das Bewegen der Spiegel auf der optischen Schiene bestimmt. Dabei wird darauf so nachjustiert, dass der Laserbetrieb nicht abbricht.\\
Zur Bestimmung der Polarisation des Laserlichts wird hinter dem Auskopplungsspiegel ein Polarisationsfilter angebracht. Der Filter wird in mehreren Schritten gedreht um verschiedene Polarisationsrichtungen einzustellen und mit einer Photodiode die Intensität des ankommenden Strahls gemessen.\\
Bei möglichst geringer Resonatorlänge wird ein Draht in den Strahlengang zwischen Laserrohr und Resonatorspiegel eingebracht. Hinter diesem wird eine Streulinse und eine Photodiode montiert und die Intensitätsverteilung der $\text{TEM}_.{00}$- und der $\text{TEM}_.{01}$-Mode gemessen.\\
Um die Wellenlänge des Lasers zu bestimmen, wird ein Gitter hinter dem zweitem Auskopplungsspiegel montiert und hinter diesem eine Photodiode. Der Abstand zwischen Gitter und Photodiode wird notiert, sowie die Positionen einiger Maxima.
Für drei verschiedene Abstände werden die longitudinalen Moden vermessen. Dazu wird mit einer schnellen Photodiode und einem Spektrumanalysator das Frequenzspektrum des Lasers aufgenommen.