
\section{Diskussion}
\label{sec:Diskussion}

Die Stabilitätsbedingung konnte für beide Resonatorkonfigurationen bestätigt werden. Bei zwei konkaven Spiegeln ist dabei die mögliche Resonatorlänge durch den Versuchsaufbau auf $\SI{192}{\centi\metre}$ beschränkt.\\
Die Vermessung der $.{TEM}_{00}$-Mode zeigt mit geringen Abweichungen den theoretisch erwarteten Verlauf einer Gaußkurve. Das Maximum ist dabei um $\SI{6.38(6)}{\milli\meter}$ zur Nulllage der Mikrometerschiene verschoben.
Die Vermessung der $.{TEM}_{01}$-Mode zeigt Ähnlichkeiten zur vorhergesagten Form, jedoch sind Abweichungen in der Symmetrie zu erkennen. Es sind zwei gleich hohe Maxima zu erwarten, die Messung zeigt jedoch unterschiedlich starke Maxima. Dies liegt vermutlich daran, dass der Draht zur Unterdrückung der Nullmode aufgrund seiner Dicke nicht exakt mittig platziert ist. Dies liegt daran, dass sonst der gesamte Laserstrahl durch den Draht geblockt wird. Dadurch wird eines der Maxima stärker unterdrückt und die Symmetrie gebrochen. Zudem ist die $.{TEM}_{01}$-Mode anfälliger gegenüber Schwankungen. 
Die Nullstelle ist hier um $\SI{6.43(20)}{\milli\meter}$ zur Nulllage der Mikrometerschiene verschoben. Dies entspricht wie erwartet der Verschiebung des Maximums der Nullmode.\\
Die Untersuchung der Polarisation des Lasers zeigt den theoretisch erwarteten Verlauf. Die maximale Intensität tritt bei einem Winkel von $\varphi=0.62\pi$ auf. Bei optimaler Ausrichtung des Lasers und des Polarisationsfilters ist das Maximum bei $\varphi=0.5\pi$ zu erwarten. Dies spricht zusammen mit den Parametern $b$ und $c$ der Ausgleichsrechnung dafür, dass die Komponenten leicht gegeneinander verdreht sind.\\
Bei der Vermessung der longitudinalen Moden konnte die theoretische Relation bestätigt werden. Dabei ist darauf zu achten, dass $\Delta\lambda=\frac{c_0}{\Delta f}$ definiert wurde und nicht $\Delta\lambda\prime=\lambda_{n+1}-\lambda_n\neq\Delta\lambda$.\\
Die Wellenlänge des Lasers wurde zu $\SI{640(17)}{\nano\metre}$ bestimmt. Die Schwankungen der einzelnen Messungen sind auf Ungenauigkeiten der Messung der $x_n$ zurückzuführen. Die bestimmte Wellenlänge des Lasers liegt im oberen Bereich der Theoriewerte von $611-\SI{640}{\nano\meter}$ \cite{VHeNeGoettingen}. 
%Die Abweichungen sind auf die Ungenauigkeiten des Versuchsaufbaus und der Schwankungen der Intensität des Lasers zurückzuführen. Aufgrund der kleinen Skalen können geringe Unsicherheiten bereits zu größeren Abweichungen führen.  
