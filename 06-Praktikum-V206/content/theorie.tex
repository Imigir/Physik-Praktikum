
\section{Theorie}
\label{sec:Theorie}

\subsection{Die Wärmepumpe}
Dem 2. Hauptsatz der Thermodynamik nach fließt eine Wärmemenge nur von  einem  wärmeren in ein kälteres Wärmereservoir, solange keine mechanische Arbeit $W$ verrichtet wird.
Bei der Wärmepumpe wird nun mithilfe dieser Arbeit aus einem kälteren Reservoir die Wärmemenge $Q_.{2}$ entnommen und an das wärmere abgegeben. Dem 1. Hauptsatz der Thermodynamik zufolge gilt für diese vom wärmeren Reservoir aufgenommene Wärmemenge:
\begin{equation}
Q_.1 = Q_.2 + W \label{eq:q2}
\end{equation}
Das Verhältnis 
\begin{equation}
\nu = \frac{Q_.{1}}{W} \label{eq:ny1}
\end{equation}
bezeichnet die Güteziffer der Wärmepumpe.
Wenn der Prozess der Wärmeübertragung ohne Verluste, also umkehrbar, verläuft, gilt außerdem die Beziehung\cite{V206}:
\begin{equation}
\frac{Q_.{1}}{T_.{1}}-\frac{Q_.{2}}{T_.{2}} = 0 
\end{equation}
Mit \eqref{eq:q2} folgt:
\begin{align*}
Q_.{1} &= \frac{T_.{2}}{T_.{1}}Q_.{1} + W\\
	   &= W \frac{T_.{1}}{T_.{1}-T_.{2}} 
\end{align*}
und damit für die Güteziffer einer idealen Wärmepumpe\cite{V206}:
\begin{equation}
\nu_.{ideal} = \frac{T_.{1}}{T_.{1}-T_.{2}} \label{eq:nyideal}
\end{equation}
Da bei einer realen Wärmepumpe die Wärmeübertragung irreversibel verläuft, gilt in diesem Fall
\begin{equation}
\frac{Q_.{1}}{T_.{1}}-\frac{Q_.{2}}{T_.{2}} > 0 \label{eq:diffreal}
\end{equation}
und mit \eqref{eq:q2} und \eqref{eq:ny1}:
\begin{equation}
\nu_.{real} < \frac{T_.{1}}{T_.{1}-T_.{2}} \label{eq:nyreal} 
\end{equation}
Für ähnliche Temperaturen $T_.{1}$ und $T_.{2}$ ist dieser Prozess energetisch wesentlich günstiger, als die direkte Umwandlung von Arbeit in Wärme.\cite{V206}
\subsubsection{Die reale Güteziffer}
Die pro Zeiteinheit gewonnene Wärmemenge lässt sich mit der Temperaturänderung $\frac{\Delta T_.{1}}{\Delta t}$ im wärmeren Reservoir und aus den Wärmekapazitäten des in diesem enthaltenen Wassers und der Kupferschlange errechnen als\cite{V206}:
\begin{equation}
\frac{\Delta Q_.{1}}{\Delta t}=(m_.{1}c_.{W} +m_.{K}c_.{K})\frac{\Delta T_.{1}}{\Delta t}\label{eq:Q2/dt}
\end{equation}
Daraus folgt für die reale Güteziffer:
\begin{equation}
\nu = \frac{\Delta Q_.{1}}{\Delta t P}\label{eq:ny}
\end{equation}
$P$ ist dabei die gemittelte Leistungsaufnahme des Kompressors \cite{V206}.
\subsubsection{Der Massendurchsatz}
Durch den Differenzenquotienten $\frac{\Delta T_.{2}}{\Delta t}$
des kälteren Reservoirs und den Wärmekapazitäten lässt sich die pro Zeiteinheit entnommene Wärmemenge bestimmen als\cite{V206}:
\[\frac{\Delta Q_.{2}}{\Delta t}=(m_.{2}c_.{W} +m_.{K}c_.{K})\frac{\Delta T_.{2}}{\Delta t}\]
Ist die Verdampfungswärme $L$ des zum Wärmetransport benutzten Mediums bekannt, kann der Massendurchsatz berechnet werden durch
\begin{equation*}
\frac{\Delta m}{\Delta t} = \frac{\Delta Q_.{2}}{\Delta t L},\label{eq:Md1}
\end{equation*}
also
\begin{equation}
\frac{\Delta m}{\Delta t} = (m_.{2}c_.{W} +m_.{K}c_.{K})\frac{\Delta T_.{2}}{\Delta t L} \label{eq:Md2}
\end{equation}
\subsubsection{Die mechanische Kompressorleistung}
Bei der Verringerung eines Volumens durch einen Kompressor gilt für die geleistete Arbeit:
\[W_.{mech}= - \int_{V_.{a}}^{V_.{b}} p\,\mathrm{d}V\]
Ist diese Kompression adiabatisch, so folgt mit der Poisson-Gleichung
\[p_.{a}V_.{a}^{\frac{C_.{p}}{C_.{V}}} = p_.{b}V_.{b}^{\frac{C_.{p}}{C_.{V}}} = pV^{\frac{C_.{p}}{C_.{V}}} =pV^{\kappa}\]
und
\begin{align}
P_.{mech} 	&= \frac{\mathrm{d}W_.{mech}}{\mathrm{d}t} \nonumber\\
			&= \frac{1}{\kappa - 1}\left(p_.{b}\sqrt[\kappa]{\frac{p_.{b}}{p_.{a}}} - p_.{a}\right)\frac{1}{\rho}\frac{\Delta m}{\Delta t},\label{eq:P}
\end{align}
wobei $\rho$ die Gasdichte des Transportmediums bei Druck $p_.{a}$ ist und sich nach Umformen der idealen Gasgleichung berechnen lässt durch:
\begin{equation}
	\rho = \frac{p\cdot \rho_0\cdot T_0}{p_0\cdot T}\label{eq:rho}
\end{equation}