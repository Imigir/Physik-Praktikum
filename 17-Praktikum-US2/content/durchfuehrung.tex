
\section{Durchführung}
\label{sec:Durchführung}

\subsection{Untersuchung des Acrylblocks auf Fehlstellen}

Die Abmessungen des Acrylblocks, sowie die Lage der Fehlstellen werden als Referenzwerte mit einer Schieblehre bestimmt. Mit einem A-Scan wird die Höhe des Acrylblocks bestimmt.\\
Mit dem Impuls-Echo-Verfahren wird die Lage der Bohrungen sowohl mit einem A-Scan, als auch mit einem B-Scan bestimmt. Dabei wird die Probe einmal von unten und einmal von oben vermessen und aus der Differenz der erhaltenen Werte mit der Höhe des Acrylblocks der Durchmesser der Fehlstellen bestimmt. Es wird eine $\SI{2}{\mega\hertz}$ Sonde und destilliertes Wasser als Koppelmittel genutzt. Bei dem B-Scan ist darauf zu achten, die Sonde langsam und möglichst gleichmäßig über den Acrylblock zu bewegen.\\
Die beiden Fehlstellen, die am nächsten beisammen liegen werden zusätzlich mit einer $\SI{4}{\mega\hertz}$ Sonde vermessen, um ihr Auflösungsvermögen zu untersuchen.

\subsection{Untersuchung eines Herzmodells mit dem TM-Scan}

Das Herzmodell wird zu einem Drittel mit Wasser gefüllt und eine $\SI{2}{\mega\hertz}$ Sonde wird so eingestellt, dass sie gerade das Wasser berührt. Mit einem A-Scan wird der Abstand von der Sonde bis zur Membran über die Laufzeit des Echos bestimmt. Es wird überprüft, ob auch bei gewölbter Membran ein A-Scan durchgeführt werden kann.\\  
Mithilfe des Gummiballs wird die Membran periodisch gewölbt und ein TM-Scan durchgeführt. Anhand der Messkurve wird die Herzfrequenz und das Herzvolumen bestimmt. Dabei wird angenommen, dass das Volumen die Form eines Kugelsegments besitzt.