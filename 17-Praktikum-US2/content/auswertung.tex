\section{Auswertung}
\label{sec:Auswertung}

Die Graphen werden sowohl mit Matplotlib \cite{matplotlib} als auch NumPy \cite{numpy} erstellt. Die Fehlerrechnung wird mithilfe von Uncertainties \cite{uncertainties} durchgeführt.

\begin{table}
	\centering
	\caption{Mit der Schieblehre bestimmte Referenzwerte für den Acrylblock.}
	\label{tab:tabAScan2MHzRef}
	\sisetup{table-format=1.2}
	\begin{tabular}{S[table-format=1.0]S[table-format=2.1]S[table-format=2.1]S[table-format=2.2]}
		\toprule
		{$n$} & {$l_.{oben}/10^{-3}\si{\metre}$} & {$l_.{unten}/10^{-3}\si{\metre}$} & {$d/10^{-3}\si{\metre}$} \\
		\midrule
		1 & 19.2 & 59.7 & 1.40 \\
		2 & 17.5 & 61.4 & 1.40 \\
		3 & 61.2 & 13.3 & 5.80 \\
		4 & 53.9 & 21.7 & 4.70 \\
		5 & 46.4 & 30.2 & 3.70 \\
		6 & 38.9 & 38.7 & 2.70 \\
		7 & 30.9 & 46.7 & 2.70 \\
		8 & 22.9 & 54.8 & 2.60 \\
		9 & 14.9 & 62.8 & 2.60 \\
		10 & 7.0 & 71.7 & 1.60 \\
		11 & 55.3 & 15.2 & 9.80 \\
		\bottomrule
	\end{tabular}

	\label{tab:AScanRef}
\end{table}

\noindent Mittels Schieblehre wird die Höhe $h$ des Acrylblocks gemessen zu $h=\SI{80,3e-3}{\metre}$. Die gemessenen Abstände der Löcher zum oberen bzw. unteren Rand  $l_.{oben}$ und $l_.{unten}$ sind in Tabelle \ref{tab:AScanRef} zu sehen.
Der Durchmesser bestimmt sich dabei über
\begin{equation}
d = h-l_.{oben}-l_.{unten}\text{.}\label{eq:d}
\end{equation}

\subsection{A-Scan}
Eine Messung der Laufzeit durch den gesamten Block liefert $\Delta t = \SI{58,69e-6}{\second}$. Mit der Schallgeschwindigkeit in Acryl $c_.{Acryl}=\SI{2730}{\metre\per\second}$ ergibt sich über Gleichung \eqref{eq:l} eine experimentelle Höhe des Quaders von $h_.{exp}\approx\SI{80,11e-3}{\metre}$.\newline
Die Messwerte zur Bestimmung des Durchmessers $d_.A$ der Löcher mit der $\SI{2}{\mega\hertz}$-Sonde befinden sich in Tabelle \ref{tab:AScan2MHz}. Die Abstände $l_.{oben_.A}$ und $l_.{unten_.A}$ berechnen sich ebenfalls über Formel \eqref{eq:l}.
$d_.A$ lässt sich über Gleichung \eqref{eq:d} bestimmen. \newline

\begin{table}
	\centering
	\caption{A-Scan Messung $\SI{2}{\mega\hertz}$}
	\label{tab:tabAScan2MHz}
	\sisetup{table-format=1.2}
	\begin{tabular}{S[table-format=1.0]S[table-format=2.2]S[table-format=2.2]S[table-format=2.2]S[table-format=2.2]S[table-format=2.2]}
		\toprule
		{$n$} & {$\Delta t_.{oben_.{A}}/10^{-6}\si{\second}$} & {$l_.{oben_.{A}}/10^{-3}\si{\metre}$} & {$\Delta t_.{unten_.{A}}/10^{-6}\si{\second}$} & {$l_.{unten_.{A}}/10^{-3}\si{\metre}$} & {$d_.{A}/10^{-3}\si{\metre}$} \\
		\midrule
		1 & 14.16 & 19.33 & 42.92 & 58.59 & 2.20 \\
		2 & 12.97 & 17.70 & 44.22 & 60.36 & 2.05 \\
		3 & 44.32 & 60.50 & 9.63 & 13.14 & 6.47 \\
		4 & 38.91 & 53.11 & 15.68 & 21.40 & 5.60 \\
		5 & 33.51 & 45.74 & 21.19 & 28.92 & 5.45 \\
		6 & 27.99 & 38.21 & 27.68 & 37.78 & 4.12 \\
		7 & 21.94 & 29.95 & 33.41 & 45.60 & 4.56 \\
		8 & 16.21 & 22.13 & 39.35 & 53.71 & 4.27 \\
		9 & 10.27 & 14.02 & 44.97 & 61.38 & 4.71 \\
		10 & 6.49 & 8.86 & 51.35 & 70.09 & 1.16 \\
		11 & 40.21 & 54.89 & 10.93 & 14.92 & 10.31 \\
		\bottomrule
	\end{tabular}

	\label{tab:AScan2MHz}
\end{table}

\noindent Die Messergebnisse zur genaueren Untersuchung der Löcher 1 und 2 mittels einer $\SI{4}{\mega\hertz}$-Sonde sind in Tabelle \ref{tab:AScan4MHz} aufgetragen.

\begin{table}
	\centering
	\caption{A-Scan Messung $\SI{4}{\mega\hertz}$}
	\label{tab:tabAScan4MHz}
	\sisetup{table-format=1.2}
	\begin{tabular}{S[table-format=1.0]S[table-format=2.2]S[table-format=2.2]S[table-format=2.2]S[table-format=2.2]S[table-format=2.2]}
		\toprule
		{$n$} & {$\Delta t_.{o_.{4\si{\mega\hertz}}}/10^{-6}\si{\second}$} & {$l_.{o_.{4\si{\mega\hertz}}}/10^{-3}\si{\metre}$} & {$\Delta t_.{u_.{4\si{\mega\hertz}}}/10^{-6}\si{\second}$} & {$l_.{u_.{4\si{\mega\hertz}}}/10^{-3}\si{\metre}$} & {$d_.{4\si{\mega\hertz}}/10^{-3}\si{\metre}$} \\
		\midrule
		1 & 14.05 & 19.18 & 43.13 & 58.87 & 2.06 \\
		2 & 12.54 & 17.12 & 44.64 & 60.93 & 2.06 \\
		\bottomrule
	\end{tabular}

	\label{tab:AScan4MHz}
\end{table}

\subsection{B-Scan}

Aus den B-Scan-Aufnahmen \ref{fig:B-Scan-oben} und \ref{fig:B-Scan-unten} werden die Laufzeiten $\Delta t_.{oben_.B}$ und $\Delta t_.{unten_.B}$ abgelesen und in Tabelle \ref{tab:BScan} aufgetragen. Die Abstände $l_.{oben_.B}$ und $l_.{unten_.B}$ werden wieder über Gleichung \eqref{eq:l}, sowie der Durchmesser der Löcher $d_.B$ nach Gleichung \eqref{eq:d} bestimmt.
\begin{table}
	\centering
	\caption{B-Scan Messung}
	\label{tab:tabBScan}
	\sisetup{table-format=1.2}
	\begin{tabular}{S[table-format=1.0]S[table-format=2.2]S[table-format=2.2]S[table-format=2.2]S[table-format=2.2]S[table-format=2.2]}
		\toprule
		{$n$} & {$\Delta t_.{oben_.{B} }/10^{-6}\si{\second}$} & {$l_.{oben_.{B}}/10^{-3}\si{\metre}$} & {$\Delta t_.{unten_.{B}}/10^{-6}\si{\second}$} & {$l_.{unten_.{B}}/10^{-3}\si{\metre}$} & {$d_.{B}/10^{-3}\si{\metre}$} \\
		\midrule
		1 & 14.40 & 19.66 & 43.70 & 59.65 & 0.81 \\
		2 & 13.10 & 17.88 & 45.50 & 62.11 & 0.12 \\
		3 & 44.60 & 60.88 & 10.00 & 13.65 & 5.58 \\
		4 & 39.40 & 53.78 & 16.00 & 21.84 & 4.49 \\
		5 & 33.90 & 46.27 & 21.90 & 29.89 & 3.94 \\
		6 & 28.60 & 39.04 & 28.40 & 38.77 & 2.31 \\
		7 & 22.70 & 30.99 & 34.80 & 47.50 & 1.62 \\
		8 & 16.80 & 22.93 & 40.30 & 55.01 & 2.17 \\
		9 & 10.90 & 14.88 & 46.10 & 62.93 & 2.31 \\
		10 & 8.70 & 11.88 & 51.90 & 70.84 & -2.61 \\
		11 & 40.30 & 55.01 & 11.30 & 15.42 & 9.68 \\
		\bottomrule
	\end{tabular}

	\label{tab:BScan}
\end{table}

\begin{figure}
	\centering
	\caption{B-Scan von oben}
	\includegraphics[width=\linewidth-40pt,height=\textheight-40pt,keepaspectratio]{content/images/B-Scan-oben.jpg}
	\label{fig:B-Scan-oben}
\end{figure}

\begin{figure}
	\centering
	\caption{B-Scan von unten}
	\includegraphics[width=\linewidth-40pt,height=\textheight-40pt,keepaspectratio]{content/images/B-Scan-unten.jpg}
	\label{fig:B-Scan-unten}
\end{figure}

\subsection{TM-Scan}
Mit einer Schieblehre wird der Durchmesser des Herzmodells gemessen zu $d=\SI{49,4e-3}{\metre}$ und damit sein Radius zu $a=\frac{d}{2}=\SI{24,7e-3}{\metre}$ bestimmt.
Mit einem A-Scan wird die Laufzeit des Schalls von der Sonde zur Membran im entspannten Zustand bestimmt zu $\Delta t_.{entspannt}= \SI{46,48e-6}{\second}$ und im gewölbten Zustand zu $\Delta t_.{gewölbt}= \SI{15,89e-6}{\second}$.\newline
Mit der Schallgeschwindigkeit in Wasser $c_.{Wasser}=\SI{1485}{\metre\per\second}$ und Gleichung \eqref{eq:l} ergibt sich für den Abstand bei ruhendem Herzen $h_.{Ruhe} = \SI{34,51e-3}{\metre}$.
Mit einem TM-Scan werden $n=25$ Herzschläge aufgezeichnet. Sowohl der zeitliche Abstand $\Delta t_.{n\rightarrow n+1}$ zwischen ihnen, als auch ihre Höhe $\Delta t_.h$ werden aus Abbildung \ref{fig:TM-Scan} abgelesen und in Tabelle \ref{tab:TMScan} aufgetragen.
Die absolute Höhe $h$ wird mit Hilfe von Gleichung \ref{eq:l} bestimmt.\\
Mit Hilfe der Formel für den Mittelwert ergibt sich für den mittleren zeitlichen Abstand
\[
\Delta\bar{t}= \sum_.{n=1}^N \frac{\Delta t_.{n\rightarrow n+1}}{N} = \SI{1,79(4)}{\second} \text{.}
\]
Der Fehler berechnet sich dabei aus der Formel für die Standardabweichung
\[
\sigma_.t=\sqrt{\frac{1}{N^2-N}\sum_.n=1^N ((\Delta t_.{n\rightarrow n+1})^2-\Delta\bar{t}^2)} \text{.}
\]
Daraus lässt sich die Frequenz $f$ bestimmen zu
\[
f=\frac{1}{\Delta\bar{t}}= \SI{0,56(1)}{\hertz}=\SI{33,6(6)}{1\per\minute} \text{.}
\]
Der Fehler bestimmt sich dabei nach der Gauß'schen Fehlerfortpflanzung aus
\[
\sigma_.f=\frac{\partial f}{\partial \Delta\bar{t}}\cdot \sigma_.t \text{.}
\]
Die mittlere Höhe der Pulse und ihr Fehler lassen sich analog über die Formel für den Mittelwert und die Standardabweichung bestimmen:
\[
\bar{h}=\SI{23,5(1)e-3}{\metre}\text{.}
\]
Mit der Formel für das Volumen eines Kugelsegments
\[
V=\frac{h\cdot\pi}{6}\cdot\left(3a^2+h^2\right)
\]
lässt sich das Schlagvolumen des Herzbeutels bestimmen:
\[
V_.{Schlag}=\SI{29,4(3)e-6}{\cubic\metre}=\SI{29,4(3)e-3}{\litre}\text{.}
\]
Der Fehler berechnet sich dabei über die Gauß'sche Fehlerfortpflanzung
\[
\sigma_.V=\frac{\partial V}{\partial h}\cdot\sigma_.h \text{.}
\]
Das Herzzeitvolumen berechnet sich über
\[
V_.{Zeit} = V_.{Schlag} \cdot f =\SI{16,5(4)e-6}{\cubic\metre\per\second} = \SI{0,99(2)}{\litre\per\minute}\text{.}
\]
Der Fehler ist dabei gegeben durch:
\[
\sigma_.{V_.Z}= \sqrt{\left(\frac{\partial V_.{Zeit}}{\partial V_.{Schlag}}\cdot \sigma_.V\right)^2+\left(\frac{\partial V_.{Zeit}}{\partial f}\cdot \sigma_.f\right)^2}\text{.}
\]
\begin{table}
	\centering
	\caption{TM-Scan Messung}
	\label{tab:tabHerz}
	\sisetup{table-format=1.2}
	\begin{tabular}{S[table-format=1.0]S[table-format=1.2]S[table-format=2.2]S[table-format=2.2]}
		\toprule
		{$n$} & {$\Delta t_.{n\rightarrow n+1}/\si{\second}$} &{$h_.{mess}/10^{-6}\si{\second}$} & {$h/10^{-3}\si{\metre}$} \\
		\midrule
		1 & 2.28 & 30.98 & 46.01 \\
		2 & 2.03 & 32.74 & 48.62 \\
		3 & 2.03 & 31.18 & 46.30 \\
		4 & 2.03 & 32.16 & 47.76 \\
		5 & 1.97 & 32.16 & 47.76 \\
		6 & 1.91 & 33.32 & 49.48 \\
		7 & 1.91 & 33.32 & 49.48 \\
		8 & 1.73 & 31.56 & 46.87 \\
		9 & 1.79 & 32.16 & 47.76 \\
		10 & 1.85 & 32.16 & 47.76 \\
		11 & 1.66 & 30.60 & 45.44 \\
		12 & 1.79 & 30.98 & 46.01 \\
		13 & 1.73 & 31.76 & 47.16 \\
		14 & 1.79 & 31.96 & 47.46 \\
		15 & 1.73 & 31.76 & 47.16 \\
		16 & 1.60 & 30.98 & 46.01 \\
		17 & 1.66 & 30.78 & 45.71 \\
		18 & 1.73 & 31.76 & 47.16 \\
		19 & 1.60 & 32.34 & 48.02 \\
		20 & 1.60 & 32.34 & 48.02 \\
		21 & 1.60 & 33.12 & 49.18 \\
		22 & 1.66 & 30.98 & 46.01 \\
		23 & 1.60 & 31.56 & 46.87 \\
		24 & 1.60 & 30.00 & 44.55 \\
		25 & nan & 30.00 & 44.55 \\
		\bottomrule
	\end{tabular}

	\label{tab:TMScan}
\end{table}

\begin{figure}
	\centering
	\caption{TM-Scan}
	\includegraphics[width=\linewidth-40pt,height=\textheight-40pt,keepaspectratio]{content/images/TM-Scan.jpg}
	\label{fig:TM-Scan}
\end{figure}