
\section{Diskussion}
\label{sec:Diskussion}

\begin{table}
	\centering
	\caption{Die Abweichungen der Durchmesser des A- und B-Scans mit der $\SI{2}{\mega\hertz}$ Sonde.}
	\label{tab:tabAScan2MHzFehler}
	\sisetup{table-format=1.2}
	\begin{tabular}{S[table-format=1.0]S[table-format=2.2]S[table-format=2.2]S[table-format=1.2]S[table-format=2.1]S[table-format=3.1]}
		\toprule
		{$n$} & {$d/10^{-3}\si{\metre}$} & {$d_.{A}/10^{-3}\si{\metre}$} & {$d_.{B}/10^{-3}\si{\metre}$} & {$\sigma_{d_.{A}}/\%$} & {$\sigma_{d_.{B}}/\%$} \\
		\midrule
		1 & 1.40 & 2.20 & 0.81 & 57.0 & -42.5 \\
		2 & 1.40 & 2.05 & 0.12 & 46.3 & -91.2 \\
		3 & 5.80 & 6.47 & 5.58 & 11.6 & -3.7 \\
		4 & 4.70 & 5.60 & 4.49 & 19.1 & -4.5 \\
		5 & 3.70 & 5.45 & 3.94 & 47.2 & 6.6 \\
		6 & 2.70 & 4.12 & 2.31 & 52.7 & -14.6 \\
		7 & 2.70 & 4.56 & 1.62 & 68.9 & -39.8 \\
		8 & 2.60 & 4.27 & 2.17 & 64.3 & -16.5 \\
		9 & 2.60 & 4.71 & 2.31 & 81.1 & -11.3 \\
		10 & 1.60 & 1.16 & -2.61 & -27.5 & -262.9 \\
		11 & 9.80 & 10.31 & 9.68 & 5.2 & -1.2 \\
		\bottomrule
	\end{tabular}

	\label{tab:AScan2MHzFehler}
\end{table}

\begin{table}
	\centering
	\caption{Die Abweichungen der Durchmesser des A-Scans mit der $\SI{4}{\mega\hertz}$ Sonde.}
	\label{tab:tabAScan4MHzFehler}
	\sisetup{table-format=1.2}
	\begin{tabular}{S[table-format=1.0]S[table-format=2.2]S[table-format=2.2]S[table-format=2.1]}
		\toprule
		{$n$} & {$d/10^{-3}\si{\metre}$} & {$d_.{4\si{\mega\hertz}}/10^{-3}\si{\metre}$} & {$\sigma_{d_.{4\si{\mega\hertz}}}/\%$} \\
		\midrule
		1 & 1.40 & 2.06 & 47.2 \\
		2 & 1.40 & 2.06 & 47.2 \\
		\bottomrule
	\end{tabular}

	\label{tab:AScan4MHzFehler}
\end{table}

\noindent Die experimentelle Messung der Blocktiefe mittels A-Scan $h_.{exp}=\SI{80.11e-3}{\metre}$ hat nur eine geringe Abweichung $\Delta h = -0,3 \%$ vom gemessenen Wert $h=\SI{80.3e-3}{\metre}$.
Aus Tabelle \ref{tab:AScan2MHzFehler} lässt sich erkennen, dass bis auf die letzten beiden Werte deutlich zu große Ergebnisse durch den A-Scan geliefert wurden. Dies könnte an einem Ablesefehler liegen, da möglicherweise der Impuls, der beim Eintritt des Schalls in den Acrylblock entsteht, falsch lokalisiert wurde. 
Die Messung mit der $\SI{4}{\mega\hertz}$-Sonde bringt keine Vorteile im Vergleich zur
$\SI{2}{\mega\hertz}$-Messung.\\
Beim B-Scan wurden bessere Ergebnisse erzielt, jedoch fällt auf, dass bis auf den Durchmesser von Loch 5 alle Durchmesser zu klein sind. Das könnte damit zu tun haben, dass aus den Abbildungen \ref{fig:B-Scan-oben} und \ref{fig:B-Scan-unten} die obere Kante der Löcher als Abstand abgelesen wurde. Besonders auffällig ist außerdem der negative Durchmesser von Loch 10, welches in den Aufnahmen auf der rechten Seite zu sehen sein sollte. Da es allerdings von Loch 11 überlagert wird und die Aufnahme an den Rändern sehr undeutlich ist, wurde möglicherweise am falschen Punkt die Tiefe abgelesen.\\

