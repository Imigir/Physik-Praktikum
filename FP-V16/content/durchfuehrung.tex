\section{Durchführung}
\label{sec:Durchführung}

Die Vakuumpumpe bleibt während des gesamten Versuchs angeschaltet.
Die Folien bestehen aus Gold.\\
\newline
Während keine Folie im Strahl steht, werden am Oszilloskop das nicht verstärkte und das verstärkte Signal verglichen. Währenddessen bleibt das Dosierventil geschlossen, sodass sich ein Druck von etwa $p=\SI{0,03}{\milli\bar}$ einstellt.\\
Über das Dosierventil wird für 20 Drücke zwischen $0$ und $\SI{200}{\milli\bar}$ die Impulshöhe des verstärkten Signals bestimmt.
Dieselbe Messung wird mit einer $\SI{2}{\micro\meter}$-dicken Folie im Strahlengang wiederholt.\\
Das Dosierventil wird geschlossen und der Druck auf $p=\SI{0,03}{\milli\bar}$ gesenkt.
Der an den Zähler angeschlossene Detektor wird im Kreis um eine $\SI{4}{\micro\meter}$-Folie herumgefahren und für verschiedene Winkel $\theta$ die Anzahl der Ereignisse gezählt. Als Messzeit wird für $\theta\leq\SI{8,4}{\degree}$ $t=\SI{300}{\second}$ gewählt. Für größere Winkel beträgt, bis auf den Winkel $\theta=\SI{13}{\degree}$, die Messzeit $t=\SI{600}{\second}$.


%♥☻♦