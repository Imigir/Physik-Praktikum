\section{Theorie}
\label{sec:Theorie}

Beim Rutherford'schen Streuversuch werden $\alpha$-Teilchen auf eine dünne Goldfolie geschossen und die Anzahl der gestreuten Teilchen in Abhängigkeit vom Streuwinkel $\theta$ gemessen.\\
Die $\alpha$-Teilchen werden aus dem Zerfall eines radioaktiven Isotops, hier $^{241}.{Am}$, gewonnen:
\[
^{241}_{95}.{Am}\rightarrow ^{237}_{93}.{Np} + ^{4}_{2}.{\alpha}
\]
Wechselwirkt ein geladenes Teilchen mit einer Masse $m_.{\alpha}\ll m_.e$, wobei $m_.e$ die Elektronenmasse ist, in Materie, so gibt es zwei Möglichkeiten der Wechselwirkung.
Entweder das Teilchen wechselwirkt mit den Hüllenelektron, auf Grund des großen Massenunterschieds wird dabei zwar Energie abgegeben, jedoch ändert sich seine Flugrichtung nicht. Die andere Möglichkeit ist eine Wechselwirkung mit dem Coulombpotential des Kerns. Da ein $\alpha$-Teilchen zweifach positiv geladen ist, wird es stark aus seiner Bahn ausgelenkt und gibt nur einen kleinen Teil seiner Energie in Form von Bremsstrahlung an das Medium ab.
Auf Grund der Ausdehnung der Materie kommt es zu sehr vielen Wechselwirkungen, wobei der Energieverlust pro Wegstrecke des einfallenden $\alpha$-Teilchens gegeben ist durch die Bethe-Bloch-Formel für niedrige Energien
\begin{equation}
\frac{\mathrm{d}E}{\mathrm{d}x}=\frac{4\pi e^4z^2N Z}{m_.ev^2(4\pi\epsilon_.0)^2}\log{\left(\frac{2m_.ev^2}{I}\right)}\text{.}\label{eq:BBF}
\end{equation} 
Dabei ist $e$ die Elementarladung, $z=2$ die Ladungszahl des $\alpha$-Teilchens, $Z$ die Kernladungszahl der Atome des Mediums, $N$ die Anzahl der Atome pro $\si{\centi\meter^3}$, $v$ die Geschwindigkeit des $\alpha$-Teilchens und $I$ die mittlere Ionisationsenergie.
Der Wirkungsquerschnitt $\sigma$ ist ein Maß für die Wahrscheinlichkeit der Wechselwirkung des $\alpha$-Teilchens im Medium.
Der differentielle Wirkungsquerschnitt pro Raumwinkelelement $\frac{\mathrm{d}\sigma}{\mathrm{d}\Omega}$ ist gegeben durch die Rutherford-Streuformel
\begin{equation}
\frac{\mathrm{d}\sigma}{\mathrm{d}\Omega}(\theta)=\frac{1}{(4\pi\epsilon)^2}\left(\frac{z Z e^2}{4 E_.{\alpha}}\right)^2\frac{1}{\sin^4{\left(\frac{\theta}{2}\right)}},\label{eq:RSF}
\end{equation}
wobei $E_.{\alpha}$ die mittlere Energie des $\alpha$-Teilchens ist. Auch diese Formel gilt nur für niedrige Energie, da sonst die Form der Kerne über Formfaktoren berücksichtigt werden müssen.
Der Gesamtwirkungsquerschnitt ergibt sich durch Aufsummieren der differentiellen Wirkungsquerschnitte und infinitesimal ergibt sich damit
\[
\sigma = \int \frac{\mathrm{d}\sigma}{\mathrm{d}\Omega}\mathrm{d}\Omega\text{.}
\]
Auf Grund der geringen Reichweite von $\alpha$-Strahlung in Luft, wird eine Vakuumkammer um den Versuch herum benötigt.






%±
