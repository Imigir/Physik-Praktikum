\section{Diskussion}
\label{sec:Diskussion}

\begin{table}
	\centering
	\caption{Die experimentellen und theoretischen Werte der Aktivität $A$ und der Dicke $d$ mit ihrem Fehler.}
	\label{tab:Ergebnisse}
	\sisetup{table-format=1.2}
	\begin{tabular}{c ccc}
		\toprule
		{Wert}&{gemessen}&{Referenzwert}&{Abweichung} \\
		\midrule
		$\alpha_{max}$ & \SI{28,1}\,\si{\degree} & \SI6{28}\,\si{\degree} & \SI{0,3}\,\si{\percent} \\
		$\theta_{gr}$ & \SI{5}\,\si{\degree} & \SI{5,04}\,\si{\degree} & \SI{-0,9}\,\si{\percent} \\
		$\lambda_{min}$ & \SI{35,1}\,\si{\pico\metre} & \SI{35,4}\,\si{\pico\metre}  & \SI{-1,13}\,\si{\percent} \\
		$E_{kin,max}$ & \SI{35316}\,\si{\eV} & \SI{35000}\,\si{\eV} & \SI{0,9}\,\si{\percent} \\
		$\Delta E_{alpha}$ & \SI{130,7}\,\si{\eV} & - & - \\
		$\Delta E_{beta}$ & \SI{155,0}\,\si{\eV} & - & - \\
		$\sigma_{Cu_K}$ & \SI{3,28} & \SI{3,31} & \SI{-0,76}\,\si{\percent}  \\
		$\sigma_{Cu_{L}}$ & \SI{13,16} & \SI{20,72} & \SI{-36,48}\,\si{\percent} \\
		$\sigma_{Cu_{M}}$ & \SI{29} & \SI{26,64} & \SI{8,87}\,\si{\percent} \\
		$E_{K_{Br}}$ & \SI{13282}\,\si{\eV} & \SI{13470}\,\si{\eV} & \SI{-1,40}\,\si{\percent} \\
		$E_{K_{Sr}}$ & \SI{15988}\,\si{\eV} & \SI{16090}\,\si{\eV} & \SI{-0,64}\,\si{\percent} \\
		$E_{K_{Zn}}$ & \SI{9650}\,\si{\eV} & \SI{9650}\,\si{\eV} & \SI{0,0}\,\si{\percent} \\
		$E_{K_{Zr}}$ & \SI{17903}\,\si{\eV} & \SI{17970}\,\si{\eV} & \SI{-0,37}\,\si{\percent} \\
		$\sigma_{K_{Br}}$ & \SI{3,75} & \SI{3,53} & \SI{6,23}\,\si{\percent} \\
		$\sigma_{K_{Sr}}$ & \SI{3,71} & \SI{3,66} & \SI{1,37}\,\si{\percent} \\
		$\sigma_{K_{Zn}}$ & \SI{3,36} & \SI{3,36} & \SI{0,0}\,\si{\percent} \\
		$\sigma_{K_{Zr}}$ & \SI{3,72} & \SI{3,65} & \SI{1,92}\,\si{\percent} \\
		$R_{\infty}$ & \SI{16,87\pm 0,32}\,\si{\eV} & \SI{13,6}\,\si{\eV} & \SI{24,04}\,\si{\percent} \\
		$\sigma_{L_{Bi}}$ & \SI{3,31} & \SI{3,58} & \SI{-7,54}\,\si{\percent} \\
		\bottomrule
	\end{tabular}

	\label{tab:Fehler}
\end{table}

\noindent Die experimentell bestimmte Aktivität besitzt wie in Tabelle \ref{tab:Fehler} zu sehen eine Abweichung von $\SI{53}{\%}$. Dies ist wie schon in der Auswertung angemerkt auf die zu geringe abgedeckte Fläche von $\SI{4}{\milli\metre}$ zurückzuführen. Diese ist zu gering, da bei der Kollimation mit den Blenden parallele Strahlen angenommen werden. Wäre der Abstand der Quelle zu den Blenden bekannt, könnte ein genauerer Wert berechnet werden. Durch Rückrechnen ist er im Bereich um die $\SI{6}{\milli\metre}$ anzunehmen.\\
Werden die Abbildungen \ref{fig:ohneAmplifier} und \ref{fig:mitAmplifier} verglichen, so ist zu erkennen, dass durch das Verstärken des Signals eine stetige Funktion entsteht und der Peak abgerundet wird. Der generelle Verlauf bleibt jedoch erhalten.\\
Die Dicke der Folie ist zu $d=\SI{2.56(17)}{\micro\metre}$ bestimmt worden. Dies ergibt eine Abweichung von $\SI{28}{\%}$ zu der tatsächlichen Dicke von $\SI{2}{\micro\metre}$. Dies ist dadurch zu erklären, dass die Peakhöhe der Impulse, besonders bei der Messung mit Folie, stark fluktuiert sind und nur ein ungenauer Wert am Oszilloskop abgelesen werden konnte. Zudem entspricht die Energie der $\alpha$-Teilchen, wie Abbildung \ref{fig:uebergang} zu entnehmen, nur in $\SI{85,2}{\%}$ der Fälle der angenommenen Energie.\\
Wird der differentielle Wirkungsquerschnitt in Abbildung \ref{fig:Rutherford} betrachtet, so scheinen die Werte außer bei der Singularität um $\theta=\SI{0}{\degree}$ gut mit der Theorie übereinzustimmen. In Abbildung \ref{fig:Rutherford2} ist jedoch zu erkennen, dass die Werte erst für deutlich größere Winkel gegen Null laufen. Dies liegt möglicherweise am Einfluss der Mehrfachstreuung, deren Auftreten in Kapitel \ref{subsec:Mehrfachstreuung} bestätigt wird, da bei der dickeren Folie eine deutlich geringere Intensität zu beobachten ist.\\
Wie in Abbildung \ref{fig:ZAbh} zu sehen, konnte die eigentlich quadratische $Z$-Abhängigkeit nicht bestätigt werden. Da die Messdaten zur Verfügung gestellt wurden, kann hier keine Aussage über die Abweichung gemacht werden.
