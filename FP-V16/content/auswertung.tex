\section{Auswertung}
\label{sec:Auswertung}

Die Graphen werden sowohl mit Matplotlib \cite{matplotlib} als auch NumPy \cite{numpy} erstellt. Die Fehlerrechnung wird mithilfe von Uncertainties \cite{uncertainties} durchgeführt.

\subsection{Bestimmung der Aktivität}

Die Aktivität der Probe wird über das Zerfallsgesetz, sowie über eine Nullmessung bestimmt.
Mit dem Zerfallsgesetz bestimmt sich die Aktivität über: 
\[
A = A_0 .e^{(-\frac{\ln(2)}{\tau}t)}\text{.}
\]
Dabei entspricht $A_0 = \SI{330(1)}{\kilo\becquerel}$ der Aktivität der Probe im Oktober 1994, $\tau = \SI{432.6(6)}{\second}$ der Halbwertszeit für $^{241}.{Am}$ und $t = \SI{7,665(13)e8}{\second}$ der seit Oktober 1994 vergangenen Zeit. Es ergibt sich:
\[
A_.{theo} = \SI{317.4(10)}{\kilo\becquerel} \text{.}
\]
Bei der $t = \SI{300}{\second}$ langen Nullmessung im Vakuum von $\SI{0,033}{\milli\bar}$ und ohne Folie wurden $N = 4539$ Teilchen registriert.
Damit ergibt sich die Aktivität pro Raumwinkelelement zu:
\[
\frac{\Omega A_.{exp}}{4\pi} = \frac{N}{t} = \SI{15,13(22)}{\becquerel} \text{.}
\] 
Das Raumwinkelelement $\frac{\Omega}{4\pi}$ bestimmt sich aus dem Verhältnis der vom Detektor abgedeckten Fläche zur Oberfläche einer Kugel, deren Radius $r = \SI{101}{\milli\metre}$ \cite{V16} dem Abstand der Quelle zum Detektor entspricht. Die abgedeckte Fläche wird wegen der Kollimation der $\SI{2}{\milli\metre}$ Schlitzblenden zu $\SI{4}{\milli\metre\squared}$ genähert. Der tatsächliche Wert ist als größer anzunehmen, weswegen eine zu große Aktivität zu erwarten ist. Es ergibt sich:
\[
\frac{\Omega}{4\pi} = \num{3,12e-5}\text{.}
\]   
Daraus folgt für die Aktivität:
\[
A_.{exp} = \SI{485(7)}{\kilo\becquerel} \text{.}
\]

\subsection{Form der vorverstärkten Impulse auf dem Oszilloskop}

Es werden die vorverstärkten Impulse einmal mit und einmal ohne Amplifier betrachtet. Die Bilder sind in den Abbildungen \ref{fig:ohneAmplifier} und \ref{fig:mitAmplifier} zu sehen. Dabei ist die Spannung gegen die Zeit aufgetragen. In beiden Fällen ist ein Untergrundrauschen erkennbar.\\
Ohne Amplifier steigt die Spannung sprunghaft auf einen Maximalwert an und fällt dann exponentiell in die Ruhelage zurück.
Mit Amplifier ist eine Anstiegszeit von einigen $\si{\micro\second}$ zu erkennen. Die Kurve nimmt einen Gaußförmigen verlauf an, wobei die linke Flanke steiler ist als die Rechte.\\

\begin{figure}
	\centering
	\includegraphics[width=\linewidth-60pt,keepaspectratio]{content/images/ohneVerstaerker02.jpg}
	\caption{Der Impulsverlauf am Oszilloskop ohne Amplifier.}
	\label{fig:ohneAmplifier}
\end{figure}

\begin{figure}
	\centering
	\includegraphics[width=\linewidth-60pt,keepaspectratio]{content/images/mitVerstaerker01.jpg}
	\caption{Der Impulsverlauf am Oszilloskop mit Amplifier.}
	\label{fig:mitAmplifier}
\end{figure}

\subsection{Bestimmung der Foliendicke mittels einer Energieverlustmessung}

Durch eine Energieverlustmessung wird die Dicke einer Goldfolie bestimmt. Dazu wird einmal mit und einmal ohne Folie die Spannung $U$ der Impulse bei verschiedenen Drücken $p$ gemessen. Da die Impulshöhe schwankt, wird der Mittelwert aus der unteren Grenze, der oberen Grenze und dem vermuteten Mittel mit der Formel für den Mittelwert
\[
\mu_U = \frac{1}{N}\sum_{i=1}^N U_i
\]
und dessen Standardabweichung
\[
\sigma_U = \sqrt{\frac{1}{N(N-1)}\sum_{i=1}^N (U_i-\mu_U)^2}
\]
gebildet zu
\[
U = \mu_U \pm \sigma_U \text{.}
\]
Die Werte befinden sich in den Tabellen \ref{tab:dataOhne} und \ref{tab:dataMit}. Die Graphen sind in Abbildung \ref{fig:Foliendicke} zu sehen.
Es werden lineare Ausgleichsrechnungen der Form
\[
U = ap + b
\]
durchgeführt. Dies ergibt die Parameter:
\begin{align*}
a_.{ohne} &= \SI{-1.213(20)e-2}{\volt\per\milli\per\bar}\\
b_.{ohne} &= \SI{4.372(18)}{\milli\bar}\\
a_.{mit}  &= \SI{-1.38(7)e-2}{\volt\per\milli\per\bar}\\
b_.{mit}  &= \SI{3.47(6)}{\milli\bar}\text{.}
\end{align*}
Die Parameter $a_.{ohne}$ und $b_.{ohne}$ gehören dabei zur Messung ohne Folie, die Parameter $a_.{mit}$ und $b_.{mit}$ zu der mit Folie.
Der Energieverlust lässt sich mit Formel \eqref{eq:dx} aus den Achsenabschnitten $b_.{ohne}$ und $b_.{mit}$ bestimmen. Dazu muss ein Umrechnungsfaktor $\kappa$ von der Spannung in die Energie bestimmt werden. Aus der Theorie ist mit $E_\alpha=\SI{5,486}{\mega\electronvolt}$ die Energie der $\alpha$-Teilchen im Vakuum bekannt. Es ergibt sich:
\[
\kappa = \frac{E_\alpha}{b_.{ohne}} = \SI{1.255(5)}{\mega\electronvolt\per\volt}\text{.}
\]
Der Fehler $\sigma_\kappa$ bestimmt sich nach der Gaußschen Fehlerfortpflanzung:
\[
\sigma_\kappa = \frac{E_\alpha}{b_.{ohne}}\sigma_{b_.{ohne}}\text{.}
\]
%Der Energieverlust lässt sich über die Spannungsdifferenz der y-Achsenabschnitte bestimmen, indem die Proportionalität zur Energie genutzt wird.
%Aus der Theorie ist mit $E_\alpha=\SI{5,48}{\mega\electronvolt}$ die Energie der $\alpha$-Teilchen im Vakuum bekannt.
Somit beträgt der Energieverlust:
\[
\Delta E = \kappa(b_.{ohne}-b_.{mit}) = \SI{1.13(8)}{\mega\electronvolt} \text{.}
\]
Der Fehler $\sigma_E$ berechnet sich über die Gaußsche Fehlerfortpflanzung:
\begin{align*}
\sigma_E &= \sqrt{(b_.{ohne}-b_.{mit})^2\sigma_\kappa^2+\kappa^2\left(\sigma_{b_.{ohne}}^2+\sigma_{b_.{mit}}^2\right)}\text{.}
\end{align*}
Es folgt mit Formel \eqref{eq:dx} und den Werten für Gold aus Tabelle \ref{tab:values} für die Dicke $d$ der Folie:
\[
d = \SI{2.62(18)}{\micro\metre}\text{.}
\]
Der Fehler $\sigma_d$ von $d$ berechnet sich mit der Gaußschen Fehlerfortpflanzung über:
\[
\sigma_d = \sigma_E\log\left(\frac{4m_eE_\alpha}{m_\alpha I}\right)^{-1}\frac{8\pi m_eE_\alpha\epsilon_0^2}{m_\alpha e^4z^2Zn}\text{.}
\]

\begin{figure}
\centering
\includegraphics[width=\linewidth-70pt,keepaspectratio]{build/Energieverlust.pdf}
\caption{Die Spannung $U$ aufgetragen gegen den Druck $p$ mit und ohne Folie.}
\label{fig:Foliendicke}
\end{figure}

\begin{table}
	\centering
	\caption{Die Spannungen $U_i$ zu den Drücken $p$ bei der Messung ohne Folie.}
	\label{tab:tabDataOhne}
	\sisetup{table-format=1.2}
	\begin{tabular}{S[table-format=3.2]S[table-format=1.2]S[table-format=1.2]S[table-format=1.2]S[table-format=1.2]@{${}\pm{}$}S[table-format=1.2]}
		\toprule
		{$p_\text{ohne}/(\si{\milli\bar})$} & {$U_\text{high,ohne}/\si{\volt}$} & {$U_\text{low,ohne}/\si{\volt}$} & {$U_\text{mid,ohne}/\si{\volt}$} & \multicolumn{2}{c}{$\bar{U}_\text{ohne}/(\si{\volt})$} \\
		\midrule
		0.05 & 4.84 & 3.92 & 4.24 & 4.33 & 0.27 \\
		5.00 & 4.72 & 3.84 & 4.20 & 4.25 & 0.26 \\
		15.00 & 4.64 & 3.80 & 4.12 & 4.19 & 0.25 \\
		25.00 & 4.60 & 3.72 & 4.08 & 4.13 & 0.26 \\
		30.00 & 4.52 & 3.52 & 3.96 & 4.00 & 0.29 \\
		40.00 & 4.40 & 3.48 & 3.76 & 3.88 & 0.28 \\
		50.00 & 4.24 & 3.32 & 3.68 & 3.75 & 0.27 \\
		60.00 & 4.20 & 3.24 & 3.56 & 3.67 & 0.29 \\
		70.00 & 4.08 & 3.12 & 3.48 & 3.56 & 0.28 \\
		80.00 & 3.96 & 3.00 & 3.40 & 3.45 & 0.28 \\
		90.00 & 3.72 & 2.80 & 3.28 & 3.27 & 0.27 \\
		100.00 & 3.72 & 2.72 & 3.20 & 3.21 & 0.29 \\
		110.00 & 3.68 & 2.62 & 2.92 & 3.07 & 0.32 \\
		120.00 & 3.40 & 2.30 & 2.92 & 2.87 & 0.32 \\
		130.00 & 3.24 & 2.32 & 2.72 & 2.76 & 0.27 \\
		140.00 & 3.06 & 2.30 & 2.68 & 2.68 & 0.22 \\
		150.00 & 2.94 & 2.26 & 2.56 & 2.59 & 0.20 \\
		160.00 & 2.78 & 2.00 & 2.30 & 2.36 & 0.23 \\
		\bottomrule
	\end{tabular}

	\label{tab:dataOhne}
\end{table}

\begin{table}
	\centering
	\caption{Die Spannungen $U_i$ zu den Drücken $p$ bei der Messung mit Folie.}
	\label{tab:tabDataMit}
	\sisetup{table-format=1.2}
	\begin{tabular}{S[table-format=3.2]S[table-format=1.2]S[table-format=1.2]S[table-format=1.2]S[table-format=1.2]@{${}\pm{}$}S[table-format=1.2]}
		\toprule
		{$p_\text{mit}/(\si{\milli\bar})$} & {$U_\text{high,mit}/\si{\volt}$} & {$U_\text{low,mit}/\si{\volt}$} & {$U_\text{mid,mit}/\si{\volt}$} & \multicolumn{2}{c}{$\bar{U}_\text{mit}/(\si{\volt})$} \\
		\midrule
		0.05 & 3.92 & 3.04 & 3.44 & 3.47 & 0.26 \\
		5.00 & 3.82 & 3.04 & 3.40 & 3.42 & 0.23 \\
		10.00 & 3.72 & 2.88 & 3.28 & 3.29 & 0.25 \\
		15.00 & 3.56 & 2.80 & 3.16 & 3.17 & 0.22 \\
		20.00 & 3.60 & 2.68 & 3.16 & 3.15 & 0.27 \\
		25.00 & 3.40 & 2.56 & 3.08 & 3.01 & 0.25 \\
		30.00 & 3.44 & 2.52 & 2.92 & 2.96 & 0.27 \\
		40.00 & 3.44 & 2.40 & 2.88 & 2.91 & 0.31 \\
		50.00 & 3.24 & 2.40 & 2.84 & 2.83 & 0.25 \\
		60.00 & 3.08 & 2.28 & 2.66 & 2.67 & 0.24 \\
		70.00 & 2.92 & 2.24 & 2.52 & 2.56 & 0.20 \\
		80.00 & 2.72 & 2.28 & 2.52 & 2.51 & 0.13 \\
		90.00 & 2.64 & 2.18 & 2.40 & 2.41 & 0.14 \\
		100.00 & 2.50 & 2.22 & 2.36 & 2.36 & 0.09 \\
		110.00 & 2.48 & 2.20 & 2.32 & 2.33 & 0.09 \\
		120.00 & 1.86 & 1.38 & 1.54 & 1.59 & 0.15 \\
		130.00 & 1.78 & 1.16 & 1.46 & 1.47 & 0.18 \\
		140.00 & 1.66 & 1.12 & 1.42 & 1.40 & 0.16 \\
		150.00 & 1.56 & 1.10 & 1.32 & 1.33 & 0.14 \\
		160.00 & 1.46 & 0.96 & 1.20 & 1.21 & 0.15 \\
		\bottomrule
	\end{tabular}

	\label{tab:dataMit}
\end{table}

\subsection{Untersuchung des differentiellen Streuquerschnitts für eine dünne Goldfolie}

\begin{figure}
\centering
\includegraphics[width=\linewidth-70pt,keepaspectratio]{build/Rutherford.pdf}
\caption{Der experimentelle und  theoretische differentielle Wirkungsquerschnitt $\frac{.d\sigma}{.d\Omega}$ aufgetragen gegen den Winkel $\theta$.}
\label{fig:Rutherford}
\end{figure}

\begin{figure}
\centering
\includegraphics[width=\linewidth-70pt,keepaspectratio]{build/Rutherford2.pdf}
\caption{Der experimentelle und  theoretische differentielle Wirkungsquerschnitt $\frac{.d\sigma}{.d\Omega}$ aufgetragen gegen den Winkel $\theta$ ohne den ersten Messwert.}
\label{fig:Rutherford2}
\end{figure}

\begin{table}
	\centering
	\caption{Die Anzahl der Counts $N$ und die gemessene Zeit $t$ in Abhängigkeit vom Winkel $\theta$.}
	<<<<<<< HEAD
\label{tab:tabDataDeg}
	\sisetup{table-format=1.2}
	\begin{tabular}{S[table-format=2.1]S[table-format=4.0]@{${}\pm{}$}S[table-format=2.0]S[table-format=3.0]r@{${}\pm{}$}lc}
		\toprule
		{$\theta/\si{\degree}$} & \multicolumn{2}{c}{$N$} & {$t/\si{\second}$} & \multicolumn{2}{c}{$\left(\frac{\mathrm{d}\sigma}{\mathrm{d}\Omega}\right)_\text{exp}/10^{-24}\si{\meter^2}$} & {$\left(\frac{\mathrm{d}\sigma}{\mathrm{d}\Omega}\right)_\text{theo}/10^{-24}\si{\meter^2}$} \\
		\midrule
		0.7  & 3310 & 60 & 300 &  132 & 4  & 77200 \\
		1.4  & 3490 & 60 & 300 &  133 & 4  & 4825 \\
		2.1  & 3600 & 60 & 300 &  144 & 4  & 953,3 \\
		2.8  & 3500 & 60 & 300 &  140 & 4  & 301,7 \\
		3.5  & 3430 & 60 & 300 &  137 & 4  & 123,6 \\
		4.2  & 3150 & 60 & 300 &  126 & 3  & 59,62 \\
		4.9  & 2720 & 60 & 300 &  108 & 3 & 32,19 \\
		5.6  & 2484 & 50 & 300 &  99 & 3  & 18,88 \\
		6.3  & 2160 & 50 & 300 &  86 & 3  & 11,79 \\
		7.0  & 1880 & 50 & 300 &  75 & 3  & 7,739 \\
		7.7  & 1470 & 40 & 300 &  59 & 2  & 5,289 \\
		8.4  & 2090 & 50 & 600 &  42 & 2  & 3,736 \\
		9.1  & 1710 & 50 & 600 &  34 & 1  & 2,714 \\
		9.8  & 1420 & 40 & 600 &  28 & 1  & 2,019 \\
		10.7 &  940 & 40 & 600 &  19 & 1  & 1,422 \\
		13.0 &  161 & 13 & 300 &  6,4 & 0,6  & 0,6546 \\
		15.5 &  121 & 11 & 600 &  2,4 & 0,3  & 0,3251 \\
		20.0 &   26 &  6 & 600 &  0,5 & 0,2  & 0,1182 \\
		\bottomrule
	\end{tabular}
||||||| merged common ancestors
\label{tab:tabDataDeg}
	\sisetup{table-format=1.2}
	\begin{tabular}{S[table-format=2.1]S[table-format=4.0]@{${}\pm{}$}S[table-format=2.0]S[table-format=3.0]r@{${}\pm{}$}lc}
		\toprule
		{$\theta/\si{\degree}$} & \multicolumn{2}{c}{$N$} & {$t/\si{\second}$} & \multicolumn{2}{c}{$\left(\frac{\mathrm{d}\sigma}{\mathrm{d}\Omega}\right)_\text{exp}/10^{-24}\si{\meter^2}$} & {$\left(\frac{\mathrm{d}\sigma}{\mathrm{d}\Omega}\right)_\text{theo}/10^{-24}\si{\meter^2}$} \\
		\midrule
		0.7  & 3310 & 60 & 300 &  800 & 19  & 77200 \\
		1.4  & 3490 & 60 & 300 &  846 & 20  & 4825 \\
		2.1  & 3600 & 60 & 300 &  871 & 20  & 953,3 \\
		2.8  & 3500 & 60 & 300 &  848 & 20  & 301,7 \\
		3.5  & 3430 & 60 & 300 &  830 & 19  & 123,6 \\
		4.2  & 3150 & 60 & 300 &  763 & 18  & 59,62 \\
		4.9  & 2720 & 60 & 300 &  658 & 16  & 32,19 \\
		5.6  & 2484 & 50 & 300 &  602 & 16  & 18,88 \\
		6.3  & 2160 & 50 & 300 &  522 & 14  & 11,79 \\
		7.0  & 1880 & 50 & 300 &  454 & 13  & 7,739 \\
		7.7  & 1470 & 40 & 300 &  356 & 11  & 5,289 \\
		8.4  & 2090 & 50 & 600 &  253 & 7   & 3,736 \\
		9.1  & 1710 & 50 & 600 &  207 & 6   & 2,714 \\
		9.8  & 1420 & 40 & 600 &  172 & 6   & 2,019 \\
		10.7 &  940 & 40 & 600 &  114 & 5   & 1,422 \\
		13.0 &  161 & 13 & 300 &   39 & 4   & 0,6546 \\
		15.5 &  121 & 11 & 600 & 14,7 & 1,4 & 0,3251 \\
		20.0 &   26 &  6 & 600 &  3,2 & 0,7 & 0,1182 \\
		\bottomrule
	\end{tabular}
=======
\label{tab:tabDataDeg}
	\sisetup{table-format=1.2}
	\begin{tabular}{S[table-format=2.1]S[table-format=4.0]@{${}\pm{}$}S[table-format=2.0]S[table-format=3.0]r@{${}\pm{}$}lc}
		\toprule
		{$\theta/\si{\degree}$} & \multicolumn{2}{c}{$N$} & {$t/\si{\second}$} & \multicolumn{2}{c}{$\left(\frac{\mathrm{d}\sigma}{\mathrm{d}\Omega}\right)_\text{exp}/10^{-24}\si{\meter^2}$} & {$\left(\frac{\mathrm{d}\sigma}{\mathrm{d}\Omega}\right)_\text{theo}/10^{-24}\si{\meter^2}$} \\
		\midrule
		0.7  & 3310 & 60 & 300 &  800 & 19  & 77200 \\
		1.4  & 3490 & 60 & 300 &  846 & 20  & 4825 \\
		2.1  & 3600 & 60 & 300 &  871 & 20  & 953,3 \\
		2.8  & 3500 & 60 & 300 &  848 & 20  & 301,7 \\
		3.5  & 3430 & 60 & 300 &  830 & 19  & 123,6 \\
		4.2  & 3150 & 60 & 300 &  763 & 18  & 59,62 \\
		4.9  & 2720 & 60 & 300 &  658 & 16  & 32,19 \\
		5.6  & 2484 & 50 & 300 &  602 & 16  & 18,88 \\
		6.3  & 2160 & 50 & 300 &  522 & 14  & 11,79 \\
		7.0  & 1880 & 50 & 300 &  454 & 13  & 7,739 \\
		7.7  & 1470 & 40 & 300 &  356 & 11  & 5,289 \\
		8.4  & 2090 & 50 & 600 &  253 & 7   & 3,736 \\
		9.1  & 1710 & 50 & 600 &  207 & 6   & 2,714 \\
		9.8  & 1420 & 40 & 600 &  172 & 6   & 2,019 \\
		10.7 &  940 & 40 & 600 &  114 & 5   & 1,422 \\
		13.0 &  161 & 13 & 300 &   39 & 4   & 0,6546 \\
		15.5 &  121 & 11 & 600 & 14,7 & 1,4 & 0,3251 \\
		20.0 &   26 &  6 & 600 &  3,2 & 0,7 & 0,1182 \\
		\bottomrule
	\end{tabular}
>>>>>>> ?

	\label{tab:dataDeg}
\end{table}

\subsection{Untersuchung des Einflusses von Merfachstreuung}



\subsection{$Z$-Abhängigkeit des Wirkungsquerschnitts}

\begin{figure}
\centering
\includegraphics[width=\linewidth-70pt,keepaspectratio]{build/Rutherford2.pdf}
\caption{Der experimentelle und  theoretischen differentielle Wirkungsquerschnitte $\frac{.d\sigma}{.d\Omega}$ aufgetragen gegen die Kernladungszahl $Z$.}
\label{fig:Rutherford2}
\end{figure}

\begin{table}
	\centering
	\caption{der experimentelle differentielle Wirkungsquerschnitt, sowie die theoretischen differentiellen Wirkungsquerschnitte für $\theta=\SI{3}{\degree}$, $\theta=\SI{3,5}{\degree}$ und $\theta=\SI{4}{\degree}$ in Abhängigkeit von der Kernladungszahl $Z$.}
	\label{tab:tabZAbh}
	\sisetup{table-format=1.2}
	\begin{tabular}{S[table-format=2.0]S[table-format=1.2]@{${}\pm{}$}S[table-format=1.2]S[table-format=2.0]@{${}\pm{}$}lS[table-format=3.2]}
		\toprule
		{$Z$} & \multicolumn{2}{c}{$I_\theta/\si{\second^{-1}}$} & \multicolumn{2}{c}{$\left(\frac{\mathrm{d}\sigma}{\mathrm{d}\Omega}\right)_\text{exp}/10^{-24}\si{\meter^2}$} & {$\left(\frac{\mathrm{d}\sigma}{\mathrm{d}\Omega}\right)_\text{theo}/10^{-24}\si{\meter^2}$} \\
		\midrule
		79 & 2.85 & 0.03 &  34 &  1 & 1.86 \\
		13 & 0.68 & 0.06 &   5 &  1 & 0.05 \\
		83 & 0.35 & 0.05 &  17 &  3 & 2.06 \\
		\bottomrule
	\end{tabular}

	\label{tab:ZAbh}
\end{table}

\begin{table}
	\centering
	\caption{Die Werte zur Berechnung des differentiellen Wirkungsquerschnitts.}
	\label{tab:tabZWerte}
	\sisetup{table-format=1.2}
	\begin{tabular}{S[table-format=2.0]S[table-format=1.0]S[table-format=1.1]S[table-format=1.2]@{${}\pm{}$}S[table-format=1.2]}
		\toprule
		{$Z$} & {$dx/\si{\micro\metre}$} & {$n/10^{28}\si{\metre^{-3}}$} & \multicolumn{2}{c}{$I_\alpha/\si{\second^{-1}}$} \\
		\midrule
		79 & 2 & 5.9 & 2.85 & 0.03 \\
		13 & 3 & 6.2 & 0.68 & 0.06 \\
		83 & 1 & 2.9 & 0.35 & 0.05 \\
		\bottomrule
	\end{tabular}

	\label{tab:ZWerte}
\end{table}